\chapter{The Gamma Function} 

12"1. Definitions of the Gamma- function. The Weierstrassian product.

Historically, the Gamma-function* V z) was first defined by Euler as
the limit of a product (§ 12*11) from which can be derived the
infinite integral

1 i ~ e~ dt; but in developing the theory of the function, it is more
con- Jo

venient to define it by means of an infinite product of Weierstrass'
canonical

form.

Consider the product ze fl -I ( 1 -H -

e

where 7= lim ]-+--f-...-l- log ?/i[ =0-5772157....

[The constant y is known as Euler's or Mascheroni's constant; to
prove that it exists we observe tliat, if

u = I -. .a<= - log - -,

[ dt \ "

u IS positive and less than / -, = -,; therefore 2 w converges, and Jo
n 'i n=l

,. n 1 1, 1,. ( "', m + l\ *

hm J- + -+...+ logwl= hm < 2 ?< + log \= 2 .

The value of y has been calculated by J. C. Adams to 260 places of
decimals.]

The product under consideration represents an analytic function of z,
for

all values of z; for, if N be an integer such that \ z\ \ iV, we
havef, if n > N,

1 2- 1 Z'

1

2n'-' Sn' '"

  

z z z n- 1 n n-

-

IN \ \ 1 1 1 1 '

 4 r' 2 + 2' +---) 2 n '

00

Since the series 2 [N' l 2n?)] converges, it follows that, when \ z\ \
N,

n = N+l

* The notation V (z) was introduced by Legendre iu 1814. t Taking the
principal value of log (1 + z/?().

236 THE TRANSCENDENTAL FUNCTIONS [CHAP. XII

S ]log(l + - is an absolutely and uniformly convergent series

of analytic functions, and so it is an analytic function (§ 5'3);
con- sequently its exponential 11 ( l+-]e~nl is an analytic function,
and

so zey n ! ( 1 + - J e ~ [ is an analytic function when \ z\ \ N,
where N is any integer; that is to say, the product is analytic for
all finite values of z. The Gamma-function was defined by Weierstrass*
by the equation

,= -T- = zey U ( l+-]e ~n[-

from this equation it is apparent that T (z) is analytic except at the
points z=0, - 1, - 2, . . ., where it has simple poles.

Proofs have been published by Holder t, Moore J, and Bai-nes§ of a
theorem known to Weierstrass that the Gamma-function does not satisfy
any differential equation with rational coefficients.

Example 1. Prove that

r(i) = i, r'(i)=-y,

where y is Euler's constant.

[Justify differentiating logarithmically the equation

~ = zey''n ( l+ -)e~ri\ Viz) 1 IV nJ I

by § 4-7, and put z = l after the differentiations have been
performed.]

Example 2. Shew that

,11 1 P 1

2 3 n Jo

t and hence that Euler's constant y is given by||

-(1-""*,

lim

[/: '-04)" ?-/:(-r.)"T']'

Example 3. Shew that

e> r(g + l)

T z-x + ) '

n = \ \ \ Z + n) J

* Journal filr Math. li. (1856). This formula for F z) had been
obtained from Euler's formula (§ 12-11) in 1848 by F. W. Newman,
Cambridge and Dublin Math. Journal, iii. (1848), p. 60. t Math. Ann.
sxvm. (1887), pp. 1-13. + Math. Ann. xlviii. (1897), pp. 70-74. §
Messenger of Math. xxix. (1900), pp. 122-128. li The reader will see
later (§ 12*2 example 4) that this limit may be written

12-11, 12-12] THE GAMMA FUNCTION

12-11. Elder s formula for the Gamma-function. By the definition of an
infinite product we have

TODO

Hence T (.) = - H j(l + -j (l +,;j | .

This formula is due to Euler*; it is valid except when = 0, - 1, - 2,
....

Example. Prove that

,,,. 1.2...(n-l)

r (2) = hm, .,. 7- - - . n\

(Euler.)

VlVl. The difference equation satisfied hy the Gamma-function.

We shall now shew that the function F (z) satisfies the difference
equation

T z -l) = zT z). For, by Euler's formula, if z is not a negative
integer,

FU + 1)/F(.) = -

lim n

;h -*• X = 1

(1 +

V n

ly+r

1 +

z+ 1

1 lim n ii

m-*---c ? =1

1 +

1 J

= - lim II,

  + 1, -*x = 1 I + ?i +

, . m + 1

= z hm = z.

 i x 2: + Wi + 1

This is one of the most important properties of the Gamma-function.
Since F (1) = 1, it follows that, if 2: is a positive integer, F (2 )
= (2 - 1) !.

It was given in 1729 in a letter to Goldbacb, printed in Fuss'
Corresp. Math.

238 THE TRANSCENDENTAL FUNCTIONS [CHAP. XII

Example. Prove that

[Consider the expression

111 1

z z z- ) 2(s + l)(2 + 2) z z + l)... z+m)

in d

It can be expressed in partial fractions in the form 2 -, where

\ (-) / \ \ 1 1 ]\ (-)" f °° 1

"=1 e "'( -)"!( =" 11

Noting that 2 -,<-, itt-t, prove that 2 - .- \ 2 -;h- 0 as

r=m-ri+\ r\ m-n + l)\ n=o n\ z + n [r=m-n+irl]

m- <X) when z is not a negative integer.]

12'13. The evaluation of a general class of infinite products.

By means of the Gamma-function, it is possible to evaluate the general
class of infinite products of the form

n Un,

7t = l

where Un is any rational function of the index n.

For, resolving m into its factors, we can Avrite the product in the
form

* f (n -a )(n-a2) ...(n- ak) ]

n=i[ (n-b,).: n-bi) I'

and it is supposed that no factor in the denominator vanishes.

In order that this product may converge, the number of factors in the
numerator must clearly be the same as the number of factors in the
denominator, and also A 1; for, otherwise, the general factor of the
product would not tend to the value unity as n tends to infinity.

We have therefore k= I, and, denoting the product by P, we may \ vrite

 =i ] n-b ) ... n-bk)i ' The general term in this product can be
written

(- •••(- )(-r-(-r

a + a + ... +ak-b,- ...-bk . = 1 - - h Jin

n

where 4 is (h~-) when n is large.

In order that the infinite product may be absolutely convergent, it is
therefore necessary further (§ 2*7) that

tti + . . . + Ojk - 1 - • • - /fc = 0.

12 13, 12-14] THE GAMMA FUNCTION 239

We can therefore introduce the factor

exp /i-i ( ! + ... +ak-b,- ... - hk)

into the general factor of the product, without altering its value;
and thus we have

p= n

(i\ ), (i\ )e"...(i- *)."

"=N (- )--(- )-

But it is obvious from the Weierstrassian definition of the Gamma-
function that

,"|('" )''"]"- r(- )e-T''

and so P = r(- 6,)6,r(-6 (- .) TJ l-K) . a T -CH)...akV -ak), =
ir(l-a )

a formula which expresses the general infinite product P in terms of
the Gamma-function.

Example 1. Prove that

• " (a-i-6 + ) \ r(a + l)r (6-1-1) s=i (a+ ) (6-l-s) T a + b + ) '

Example 2. Shew that, if a = cos (27r/n) + i sin 27r jn\ then

/ \ / \ - -

 (i- j(i-|;)... = -r(-x )r(-aa- )...r(-a"-i:r ) -'.

1214. Connexion betiveen the Gamma-function and the circular
functions.

We now proceed to establish another most important property of the
Gamma-function, expressed by the equation

r( )r(i-. )= .- .

sm TTZ We have, by the definition of Weierstrass (§ 12"1),

r(.)r(-.)=-in (i+i)r rnj(i-i);"

Z Sin TTZ

by § 7-5 example 1. Since, by § 12-12,

r(i-z) = -zr(-z)

we have the result stated.

240 THE TRANSCENDENTAL FUNCTIONS [CHAP. XII

Corollary 1. If we assign to z the value \, this formula gives r( ) 2
= 7r; since, by the formula of Weierstrass, r ( ) is positive, we
have

Corollary "l. If r/.(s) = r'(s)/r(2), then i/ (1 -z)-- /r (0) = 7r cot
tts. 12-15. The multiplication-theorem of Gauss* and Legendre. We
shall next obtain the result

r(.)r(. + )r(. + )...r(. + 'i ) = (2.)n -. - -rM.

n) \ n For let ( (.) = iT;; \ .

Then we have, by Euler's formnla (§ 12-11 example),

 - 1 .2...(m- l).m - /'*

n'"- n lim

 (Z)

,. 1 . 2 ... (m - 1) . (nm)'*

n lim

\ im - 1)! m

nz (nz + 1 ) . . . (71Z + n7n - 1)

 n'ln - 1) ! nm)'" i(m- 1) !j wi" 'n

m oo (wm-1)!

It is evident from this last equation that <p (z) is independent of z.
Thus (j) (z) is equal to the value which it has when z = -; and so

Therefore < (z)Y = n| | T ( -) T (l -;:)|

tt' -i (27r)"-

. TT . 27r . (?i- l)7r

sm - sm - ... sm

n n n

Thus, since cj) (n~ ) is positive,

<,W = (2,r)4<"-'' -*,

i.e. r (.) r (. + ) . . . r (. + '-i i) = - (2,r)4 < - '> r ( .).

Corollary. Taking n = 2, we have

2 -ir(.)r( +i)=,r4r(24

This is called the duplication formula.

* Werke, iii. p. 149. The case in which n - 2 was given by Legendre.

12-15-12-2] THE GAMMA FUNCTION 241

Example. If (, o) = EMlM

shew that

D(p,,)b p + \ q)...B(p ',q) B (np, no) = a ~ i \ n /

1216. Expansions for the logarithmic derivates of the Gamma-function.

We have r (2+l) -J = '>' n /("l + -") e~ ]- .

DifiFerentiating logarithmically (§ 4"7), this gives

l gJ iiiLi)= \ + !\ + --,,

rfs ''' 1 (2 + 1) '*"2(2 + 2) " 3(2 + 3) "•"

Therefore, since log r (s + 1 ) = log 2 + r (2), we have

Iogr(2)=-y--.+ 2 2

dz ° ' z n=in(2 + n)'

d-. \,,. (f

Differentiating again,,logr(.+ l) = + 2 (2 + 2) + -

1 1

 (2+l>' (2 + 2)2 "•"

These expansions are occasionally used in ai)plications of the theory.
122. Elder s expression of T z) as an injinite integral.

The infinite integral I e~H ~ dt represents an analytic function of z
when*

Jo

the real part of z is positive (§ 5-32); it is called the Eulerian
Integral of the Second Kindf. It will now be shewn that, when li z)>0,
the integral is equal to F (z). Denoting the real part of z by x, we
have x > 0. Now, if J

n(z,n)=j'Ul-yJ\'- dt,

we have U (z, n) = n l (1 - TyW-'hlr,

J )

if we write t= nr; it is easily shewn by repeated integrations by
parts that, when a; > and n is a positive integer,

f n 1 n; !

 1 - t)"t - dr = - T- (1 - Tf + - (I - Tf-'T'dr Jo \ J Jo S'.

n(n-l)...l fi,

z(5 + l)...( + n-l)Jo '* '

J n / \ 1 . 2 . . . ?i

and so 11 (2, n) = -. - n

z z ) ... z -n)

Hence, by the example of § 12*11, 11 z, n) r( ) as /i -* x .

* If the real part of z is not positive the integral does not converge
on account of the singu- larity of the integrand at f = 0.

t The name was given by Legendre; see § 12-4 for the Eulerian
Integral of the First Kind.

t The many-valued function < ~i is made precise by the equation *~i =
c(*-i)'°s', log r being purely real.

W. M. A. 16'

242 THE TRANSCENDENTAL FUNCTIONS [CHAP. XII

Consequently F ( ) = lim (1 ) t-~ dt.

And so, if T, (2) = I e-H'-'dt,

J (\ we have

ri( )-r( )= lim r ie-'-(l- y\ t'-'dt+j e-H'-'dt .

Now lim I e'U'-'dt = 0,

since I e~H ~ dt converges.

.'0

To shew that zero is the limit of the first of the two integrals in
the formula for Fj z) - F z) we observe that

[To establish these inequalities, we proceed as follows : when < y <
1, from the series for e and (1 - 3/)" Writing tjn for y, we have

1 + - e-'

n

HJ-

and so 0 e- -fl--

71

= e-Ml-eM 1

<e-Ml- 1

-1)1

Now, if O-Sa l, (1 - a)" l - na by induction when ?ia<l and obviously
when 7ia 1; and, writing t /n for a, we get

and so* 0 e-*-(l--) %e-H ln,

which is the required result.]

From the inequalities, it follows at once that

r' - f 1 - - j ) t'-'dt ' I n-'e-ff'- 'dt

<n-'j e-H' - 'dt O, Jo

as n - X, since the last integral converges.

* This analysis is a modification of that given by Schloinilch,
Compendium der hoheren Analysis, ii. p. 243. A simple method of
obtaining a less precise inequality (which is sufficient for the
object required) is given by Bromwich, Infinite Series, p. 459.

12-21] THE GAMMA FUNCTION 243

Consequently Fi 2)= V (z) when the integral, by which Tj (z) is
defined, converges; that is to say that, when the real part of z is
positive,

J n And so, when the real part of z is positive, F (z) may be defined
either by this integral or by the Weierstrassian product. Example 1,
Prove that, when R (z) is positive,

Example 2. Prove that, if li z) > and R (s) > 0,

/ e-"=x -'cla.-= . Jo z

Example 3. Prove that, if R z)>Q and R is) > 1,

Example 4. From :; 12'1 example 2, by using the inequality

0%e- -( X fie- ln,

deduce that

l\ e- \ e-i/<

-at.

<

t

12*21. Extension of the infinite integral to the case in which the
argument of the Gamma-function is negative.

The formula of the last article is ni> longer applicable when the real
part of z is negative. Cauchy* and Saalschiitzt have shewn, however,
that, for negative arguments, an analogous theorem exists. This can be
obtained in the following way.

Consider the function

r, z)=j\'- (e-'-\ + t-, + ... + -)>'- dt,

where k is the integer so chosen that -k>x>-k-l, x being the real part
of z. By partial integration we have, when s < - 1,

r,(.)=[f(.-.-n.<-il + ...+(-r.,)];

+.'j7''( -'-i+'--+(-''(j )*

The integrated part tends to zero at each limit, since x+k is negative
and x + k + l is positive : so we have

r, z) = lT, z+i).

The same proof applies when x lies between and -1, and leads to the
result

r z+i)=zr2 z) (0>.r>-i).

The last equation shews that, between the values and -1 of a*,

To z) = r(z).

* Exercices de Math. u. (1827), pp. 01-92.

t Zeitschrift fur Math, und Phys. xxxii. (1887), xxxiii. (1888).

16-2

244 THE TRANSCENDENTAL FUNCTIONS [CHAP. XII

The preceding equation then shews that To z) is the same as T (2) for
all negative values of Ji z) less than -1. Thus, for all negative
values of R(z), we have the result of Cauchy and Saalschiitz

where k is the integer next less than - B (z).

Example. If a function P (ft) be such that for positive values of /x
we have

J

- 1 -x

e~'-' dx,

and if for negative values of /x we define Pj (fi) by the equation

dx.

Pi(;u)=|\ i'--i(e- -l + .r-... + (-)*- + i,)

where k is the integer next less than - ju, shew that

/',W./'W-l + -... + (-)'--.-,. (Saalschiitz.)

12"22. Hankers expression of V z) as a contour integral.

The integrals obtained for T z) in §§ 12-2, 12-21 are members of a
large class of definite integrals by which the Gamma-function can be
defined. The most general integral of the class in question is due to
Hankel*; this integral will now be investigated.

Let D be a contour which starts from a point p on the real axis,
encircles the origin once counter-clockwise and returns to p.

Consider 1 (-t)~~ e~hlt, when the real part of z is positive and z is
not an integer.

The many-valued function (- ty~ is to be made definite by the
convention that (- ty~ = e' ~ ' ° '"*' and log (- t) is purely real
when t is on the negative part of the real axis, so that, on D, - ir %
arg (- t) % it.

The integrand is not analytic inside D, but, by § 5'2 corollary 1, the
path of integration may be deformed (without affecting the value of
the integral) into the path of integration which starts from p,
proceeds along the real axis to h, describes a circle of radius h
counter-clockwise round the origin and returns to p along the real
axis.

On the real axis in the first part of this new path we have arg (- ) =
- tt, so that - ty~' - e~ '' ~ H ~ (where log i is purely real); and
on the last part of the new path (- ty- = e " ' - ' t ' .

On the circle we write - i = 8e*; then we get

f -ty-'e-Ht-= \ e-'-(--i) '-ie-'f? +f"(8e' )--ie (cose+;sm< )gg,ej 7

+ ''e' ' -' H'-' e' dt = - 2i sin -jTz) j t'-' e-hit + ih' I " e<>e+
(cose+tsine) \

* Zeitschrift filr Math, und Phys. ix. (1864), p. 7.

12 22] THE GAMMA FUNCTION 245

This is true for all positive values of S p; now make S; then 8' and
I gize+s (COS 0+isin e 0 j ize iQ [ \ \ q integrand tends to its limit

J - TT J -IT

uniformly.

We consequently infer that

[ (-ty-'e-'dt = -21 sin 7r2)[''t'-'e-*dt.

J B Jo

This is true for all positive values of p; make p x, and let C be
the limit of the contour D.

Then f (- ty-'e- dt = - 2i sin ttz) f t'-'e- dt.

J c Jo

Therefore T ( ) = - r 4 I (- tf-'e-'dt.

'2i sm TTzJ c

Now, since the contour C does not pass through the point t = 0, there
is no need longer to stipulate that the real part of 2 is positive;
and

I (-ty~ e~ dt is a one-valued analytic function of z for all values of
2.

J c

Hence, by § 5"5, the equation, just proved when the real part 0/2 is
positive, persists for all values of z with the exception of the
values 0, ±1, +2

Consequently, for all except integer values of,

r(0) = -\ J - I -ty-'e-'dt.

This is Hankel's formula; if we write 1 - 2 for z and make use of §
12"14, we get the further result that

We shall write / for \, meaning thereby that the path of inte-

•1 re C

gration starts at 'infinity' on the real axis, encircles the origin in
the positive direction and returns to the starting point.

Example 1. Shew that, if the real part of z be positive and if a be
any positive

constant, l - t)~ e~ dt tends to zero as p -ao, when the path of
integration is either of

the quadrants of circles of radius p + a with centres at - a, the end
points of one quadrant being p and - a + 1 (p -|- a), and of the other
p and -a - i p- a).

V

' (0 + )

24G THE TRANSCENDENTAL FUNCTIONS [CHAP. XII

Deduce that lim I " ' t)-' e- dt= lim I -t)-'e- dt,

p-*xy-a + 'P p- y J C

and hence, by writing t= -a-iu, shew that

\ \ ['

--- = -- I e "*- " (a + <)"* f M- r (s) ztt y \ x

[This formula was given by Laplace, Theorie Analytique des
Prohahilites (1812), p. 134, and it is substantially equivalent to
Hankel's formula involving a contour integral.]

Example 2, By taking a = 1, and putting t= -\ + i tan B in example 1,
shew that -J- = - ( "" cos (tan (9 - 2(9) cos - 2 (9c?<9.

r (2) 77 Jo

Example 3. By taking as contour of integration a parabola whose focus
is the origin, shew that, if a > 0, then

Y z)=-. - - e- ''(l + -) -Jcos 2a + (22-l)arctani; (/i!. sin nz J

(Bourguet, Acta Math, i.)

Example 4. Investigate the values of x for which the integral

2 r*

 - sin tdt

TT J

converges; for such values of x express it in terms of
Gamma-functions, and thence shew that it is equal to

(St John's, 1902.)

Example 5. Prove that I (log t)" ' dt converges when m > 0, and, bv
means

Jot

of example 4, evaluate it when m=l and when m = 2. (St John's, 1902.)

12'3. Gauss' expression for the logarithmic derivate of the
Gamma-function as an infinite integral*.

We shall now express the function -7- log V (z) = as an infinite

integral when the real part of z is positive; the function in
question is frequently written yjr (z). We first need a new formula
for 7.

Take the formula (§ 12'2 example 4)

  Jo i J\ i 5 0 Us J t / a olJA J\& t J'

where A = l-e- since | - = log .- O as 8-*-0.

./a l-e"

 Yriting = 1 -e~" in the first of these integrals and then replacing u
by t we have

y= lim I r,dt- r '- dt] = r \ j~,-]] e- dt.

5-*o Us l-e Js t J Jo U-e ' t)

This is the formula for y which was required.

* Wtr'ke, HI, p. 159.

12-3] THE GAMMA FUNCTION 247

To get Gauss' formula, take the equation (§ 12"16)

r'( ) 1 .. /I 1 \

T z) ' z acm=iVm z-vm)

1 r

and write = e-'( +'">rf;

z - m Jo

this is permissible when m = 0, 1, 2, ... if the real part of z is
positive.

It follows that

 '=-7- I e- ? + lim S (e-""-e-(' + )')f <

.0 rt- >ooJowi = l

r( )

-7+ lim = -I dt

1-

V 1 - e-'

W - lim 1 -e-(''+ '(/t

1-e-

Now, when < < 1, i is a bounded function of t whose limit as - [0 is
6nite;

I 1 - e 'I

and when t 1, - | < - J- < - .

Therefore we can find a number A' independent of t such that, on the
path of integration,

I 1 - ' I

andso I r\ zl~,-K ' Vdt\ < K[ e-i"*')' rf = /r(H + l)-' 0 as /i- -x .

I jo 1-e-' ./o

We have thus proved the formula

t< ) = s'" ' *'->=.C(T-r ')'"-

which is Gauss' expression of - z) as an infinite integral. It may be
remarked that this is the first integral which we have encountered
connected with the Gamma-function in which the integrand is a
single-valued function.

Writing <=log (1 -f-.i') in Gauss' result, we get, if A=(J - 1,

 -if=limf t'-,,U.

since < I - dt < j y =log - g- 0 as 8 0.

T'(z),. / " f 11 rf.r

Hence \ = uaj |,-.\ \ \ \ |\,

sothat ''<=)=rM/J e-'-(, §',

an equation due to Dirichlet*.

Werke, i. p. 275.

248 THE TRANSCENDENTAL FUNCTIONS [CHAP. XII

Example 1 . Prove that, if the real part of z is positive,

Example ± Shew that y=\ l -t)-' -e- ]t-' dt. (Dirichlet.)

12'31. Binet's first expression for log T z) in terms of an infinite
integral. Binet* has given two expressions for logr( ) which are of
great

importance as shewing the way in which log V z) behaves as ] 2 j -* oo
. To

obtain the first of these expressions, we observe that, when the real
part of

z is positive,

r' /'-r \ L 1 \ r* (o-t o-tz ]

dt,

r( + i) Jo [ e -i

writing z + \ for 2 in § 12"3.

Now, by § 6'222 example 6, we have

log2=) - r~ '

f" 1 and so, since (22 )" = e~ dt,

'Jo

we have

dz

logr(. + l) = l + log.-/"g-;+ - Je-*.

The integrand in the last integral is continuous as i -; and since

- - - + -f - z. is bounded as - 00, it follows without difficulty
that the

integral converges uniformly when the real part of z is posijiive; we
may consequently integrate from 1 to under the sign of integration (§
4*44) and we get-f-

iogr(.+i) = (. + |)iog.-.H-i+/;g- + p-- -

dt.

Since - - + - - i 7 continuous as i - by § 7 '2, and since

log r (z + 1) = log + log r (z),

have

/ 1\ (" CI 1 1 ) e-*

iogr(.) = (.-2)iog.-. + i.-j |--- + - \ -- |- .

J (2 t e' - l)t

* Journal de I'Ecole Poly technique, xvi. (1839), pp. 123-143.

t Logr(2 + l) meaus the sum of the principal values of the logarithms
in the factors of the Weierstrassian product.

12 -31] THE GAMMA FU XTION 249

To evaluate the second of these integrals, let*

so that, taking z = in the last expression for logr(2:), we get

i log 7r = i + t/-/.

( + - ) - - dt, we have

t

r ' 1 \ dt

-jo V t e'- ) t •

=/:r- '-HT

~jo 1 c/ V t ) t ' '2tj

* - rf<

t

= 2 + Uogi Consequently /=1- log(27r).

We therefore have Binet's result that, when the real part of z is
positive,

3-tZ

 -dt.

log r( ) = [z- \ ogz-z + \ log(27r) + J G ~ 7 + "'-t)

li z = x- iy, we see that, if the upper bound of ( ~ 7 " t \ i ) 7 ' *
' values of is K, then

logr( )-( -i)log + -|log(27r)'<i(:| "'"

so that, when x is large, the terms (z - \ og z - z \ o (2'jr) furnish
an

approximate expression for log V (z).

Example I. Prove that, when A' (a) > 0,

logr(.-)= r *i4.1-r+( -l) "j 7- (Malmsten.)

Example 2. Prove that, when R (z) > 0,

* This artifice is due to Priagsheim, Math. Ann. xxxi. (1888), p. 473.

250 THE TRANSCENDENTAL FUNCTIONS [CHAP. XII

Example 3. From the formula of § 12" 14, shew that, if <;r < 1,

2logrW-log.+logsin..-=/; |-' Ui=fl-'-(l-2.).-. f .

(Kummer.)

Example 4. By expanding sinh (| - .v) and 1 - 2x' in Fourier sine
series, shew from example 3 that, if < .v < 1,

00

log r (.p) = i log TT - - log sin -nx + 2 2 a,i sin 2?i7r,

, / " r 2?i7r e-n dt

" jo V + 4wV2 2%7rJ i! Deduce from example 2 of § 12 "3 that

"*" "" 2 " "*" 2"" + * g ''* )•

(Kummer, Journal fiir Math. xxxv. (1847), p. 1.)

12*32. Binet's second expression for log V z) in terms of an infinite
integral.

Consider the application of example 7 of Chapter vii (p. 145) to the
equation (§ 12-16)

The conditions there stated as sufficient for the transformation of a
series into integrals are obviously satisfied by the function ( )= -
r-, if the real part of z be positive; and we have

Since !5'(, + /i) I is easily seen to be less than K t/n, where Ki is
inde- pendent of t and n, it follows that the limit of the last
integral is 'zero.

Hence V;: log F ( ) = 2,- -I i - -, - -.

dz 2z- z Jo z + t'f e- - 1

I 2 I Since ~ - - does not exceed K (where K depends only on 8) when
the

I I*

real part of z exceeds 8, the integral converges uniformly and we may
integrate under the integral sign (§ 4'44) from 1 to z.

We get

-I log r (.) = -;+ log . + - 2 /; . JL,

where C is a constant. Integrating again,

log r (.) = (. - 1) log . + (c - 1) . + c + 2/; '- dt.

where 6" is a constant.

12-32, 12-33] THE GAMMA FUNCTION 251

Now, if z is real, arc tan tjz \$ tlz,

and so

logrW-( -i)log -(C-l).-C"|<?/" d(.

But it has been shewn in § 12"31 that

\ \ % z)-[z-- \ ogz- z- \ og 1'K) -0,

as 2 -* X through real values. Comparing these results we see that C =
0, 6"=ilog(27r).

Hence for all values of z whose real part is positive,

logrW = (.-l)log.-. + llog(2.) + 2|; -I rf*.

where arc tan u is defined by the equation

arc tan xi = |,

.'o l+<-

in which the path of integration is a straight line.

This is Binet's second expression for log V z). •

Example. Justify differentiating with regard to z under the sign of
integration, so as to get the equation

12 33. The asymptotic expansion of the logarithm of the Gamma-
function (Stirling's series).

We can now obtain an expansion which represents the function log F z)
asymptotically (§ 8-2) for large values of \ z\, and which is used in
the calculation of the Gamma-function.

Let us assume that, if = x -f iy, then a-' S >; and we have, by
Binet's second formula,

log r ( ) = ( - ) log - + - log (27r) + </) ( ),

where (.) = 2/; c/.

Now

(\ )n-i 2,1-1 (\ y rt u n

arc

,,,,, t It' It' (\ )n-i 2n-i (\ )H rt

u + z' Substituting and remembering (§ 7-2) that

Jo e' ' -l ~4w'

252 THE TRANSCENDENTAL FUNCTIONS [CHAP. XII

where Bi, Bo, ... are Bernoulli's numbers, we have

  / X V (-)' "' Br 2 (-)'* r " f f U' ' du \ dt

.=1 2r (2?- - 1) '--l 271-1 j \ [ ! () 2 2 + 2 j g2,r< \ I

Let the upper bound * of - 1 for positive values of u be K,

j w + j

Then

ftl

f u' 'du] dt 1 r, I, r" 1 r, ) (

uow + ' j e-' '-li" ' ' Jo [Jo \ e'-'-l

Ji-z n+1

 4(n+l)(2M+l)| |2" Hence

2(-y r ( f* u l \ d KA+

z "-' Jo Vou' + z e -l 2 n + l) 2n + l)\ z p +i '

and it is obvious that this tends to zero uniformly as | | - oo if I
arg | tt - A, where tt > A > 0, so that K cosec 2 A.

Also it is clear that if ] arg r | Itt (so that Kz = l) the error in
taking the first 71 terms of the series

I (-y-'Br 1

rti 2r (2r - 1) z -' as an approximation to <f) (z) is numerically
less than the (n + l)th term. Since, if | arg z\ \ 7r - A,

-,211-1

!</> ( ) - i Z ' \ I < cosec 2 A .,

r r=i 2r(2r-l)l; 2(n + l)(2,

2(n + l)(2n + l)

-0,

as 00, it is clear that

B, B, B,

1.2.2 3. 4. 2=* ' 5.6.2 "* is the asymptotic expansion f (§ 8-2) of
(2). We see therefore that the series

is the asymptotic expansion of log V z) when | arg | tt - A.

-Kj 18 the lower bound of -!, .j and is consequentlj' equal to

4 2y2,

, 9, .n o or 1 as x2<m2 or x >y . t The development is asymptotic;
for if it converged when | 2 | p, by § 2-6 we could find K, such that
B <(2;i-l)2nA>2"; and then the series 2 ~ "~' " " would define an
integral function; this is contrary to § 7-2.

12 4] THE GAMMA FUNCTION 253

This is generally known as Stirling s series. In § 13'6 it will be
estab- lished over the extended range | arg tt - A.

In particular when z is positive (= x), we have

r u? du'] dt Bn+.

Jo Uo w + j e ' '-l

w + je ' '-l 2 n + l) 2n + l)af

Hence, when x>0, the value of < )(oc) always lies between the sm7i of
n terms and the sum ofn + 1 terms of the series for all values of n.

D g

In particular < (f> (x) < - -, so that <f> (./•) = y where < < 1.

Hence T x) = x'- - e'' i2'rrf e ' 'l

Also, taking the exponential of Stirling's series, we get

\ x .r-, a f 1 1 139 571 /'I

r ix) = e X ~ (27r) -,1 - 288 ~ 51840 ~ 2488320 "*" V

This is an asymptotic formula for the Gamma-function. In conjunction
with the formula T x + l) = xr x), it is very useful for the purpose
of com- puting the numerical value of the function for real values of
x.

Tables of the function logioT (. ), correct to 12 decimal places, for
values of .v between 1 and 2, were constructed in this way by
Legendre, and published in his Exercices de Calcul Integral, ii. p.
85, in 1817, and his Traite des fonctions elliptiques (1826), p. 489.

It may be observed that V (x) has one minimum for positive values of
.>;, when .r= 1-4616321..., the value of log,or(.r) then being
1-9472391....

Example. Obtain the expansion, convergent when R z) > 0,

\ og,T z) = z-l)\ og, z- z+ \ og, 27r) + J (z),

where in which

- W- 5 |, + | + 2 (- +1) (. + 2) 3 (.-+1) (.- + 2) z + S) -

and generally

c = r (x+l) x + 2) ... x + n- ) 2x - ) xdx. (Binet.)

J

12 "4. The Eulenan Integral of the First Kind.

The name Eulerian Integral of the First Kind was given by Legendre to
the integral

B (p, q) = f xP-' (1 - x)'i-' dec,

J

which was first studied by Euler and Legendre*. In this integral, the
real parts of p and q are supposed to be positive; and xp~, (1 - x) ~
are to be understood to mean those values of e( ~ ) °°* and
e('v-i)'°e(i- =) which correspond to the real determinations of the
logarithms.

* Euler, Nov. Comvi. Petrop. xvi. (1772); Legendre, Exercices, i. p.
221.

254 THE TRANSCENDENTAL FUNCTIONS [CHAP. XII

With these stipulations, it is easily seen that B (p, q) exists, as a
(possibly improper) integral (§ 4*5 example 2).

We have, on writing (1 - *•) for w,

B p,q) = B q,p).

Also, integrating by parts,

Jo L i Jo PJo

so that B p,q + l) = B p+l,q).

Example 1. Shew that

B p,q)= £ip + l,q) + B p,q + l).

Example 2. Deduce from example 1 that

B P,<l + )= B p,q).

Example 3. Prove that if n is a positive integei*,

,,, 1 . 2 ... M

B p,n + ) = -

p p + l)... p + n)\ Example 4. Prove that

Example 5. Prove that

r (2) = lim n B z, n).

12'41. Expression of the Eulerian Integral of the First Kind in terms
of the Gammaf unction.

We shall now establish the important theorem that

R/.,, x\ r(m)r(n)

First let the real parts of m and n exceed |; then

r (m) r (n) = e-* '"-i c a; x g-?' if'- dy. Jo Jo

On writing a;'- for .t, and y for y, this gives

fR rR

V (m) r (n) = 4 lim e"*' x'-""-' dx x g-?/' /-"-i dy

= 4 lim e-( '+2'') 2m-iy2n-i( ( \

Now, for the values of m and n under consideration, the integrand is
continuous over the range of integration, and so the integral may be
con- sidered as a double integral taken over a square Sji. Calling the
integrand

12-41] THE GAMMA FUNCTION 255

f x, y), and calling Qg the quadrant with centre at the origin and
radius R, we have, if Tji be the part of S, outside Qb,,

I f(x, y) dxdy - f oc,y) dxdy JJsr JJqr I

= 1 f(a:,y)dxdy

 JJ Tr

i \ f >y)ida dy-\ \ \ f,y)dxdy\

JJSr JJ Sir

  . ., .,

- 0 as R X,

since 1 1 |/(;, y) \ dxdy converges to a limit, namely

JJ Sr

Therefore

Imi

2 I e- '-, x""-' idxx2\ e' \ y-''-' \ dy. Jo' Jo

I f(x,y)dxdy= lim // f x,y)dxdy.

Changing to polar* coordinates x = rcos 0, y = r sin 6), we have f x,
y) dxdy = |

Oh

Hence

1 1 f x, y) dxdy = ( \ e-"" (r cos 0) '"-' (r sin 0)-''-' rdrd0.

J J Or J J

r (w) r ( ) = 4 e-r' r-('"+")-i dr cos=" -i sin ' - c? Jo Jo

= 2r ( i + ?z) cos '"-' sin ''- <9c? .

Jo

Writing cos- = u we at once get

r (m) r (?i) = r (?u + ) . z (w, ?i).

This has only been proved when the real parts of ni and n exceed |;
but it can obviously be deduced when these are less than | by § 124
example 2.

This result, discovered by Euler, connects the Eulerian Integral of
the First Kind with the Gamma-function.

Example 1. Shew that

[' (l-f-A-)' -i(l-.r) -i( =2P + 9-i Mli2).

  -I ' ' ip+q)

* It is easily proved by the methods of § 411 that the areas A /x of §
4-3 need not be rect- angles provided only that their greatest
diameters can be made arbitrarily small by taking the number of areas
sufficiently large; so the areas may be taken to be the regions
bounded by radii vectores and circular arcs.

256 THE TRANSCENDENTAL FUNCTIONS [CHAP. XII

Example 2. Shew that, if

j\ isj .v x + 1 2! .r + 2 3! .r + d

then

f x,y)=f y + \,x-\ \

where and y have such vakies that the series are convergent. (Jesus,
1901.)

Example 3. Prove that

j'J'j xy) l-xT- y> i yr- dxdy = - > 'j

(Math. Trip. 1894.)

1242. Evaluation of trigonometrical integrals in terms of the Gamma-
function.

We can now evaluate the integral cos" ~ x sm' ~ xdx, where m and n

Jo

are not restricted to be integers, but have their real parts positive.
For, writing cos o; = t, we have, as in § 12'41,

I'*" .,, IF am) ran)

cos' - cc sin" -1 xdx = f. ' - .

The well-known elementary formulae for the cases in which m and n are
integers can be at once derived from this result.

Example. Prove that, when \ k\ < .\,

fh CDS'" e sin' edd \ T (lm+ )T /i+ ) fh cos'" -'"Odd J \ l ksm 6)i ~
Tlh-a + hi+l) 7r J (l - k sin2 )i + *

(Trinity, 1898.)

12'43. Pochhammers* extension of the Ealerian Integral of the First
Kind.

We have seen in § 12'22 that it is possible to replace the second
Eulerian integral for F z) by a contour integral which converges for
all values of z. A similar process has been carried out by Pochhammer
for Eulerian integrals of the first kind.

Let P be any point on the real axis between and 1; consider the

integral

r(i+,o+, 1-, 0-) e-'"'<' + ) t - (1 - 0 -' dt = e a, /3).

J F

The notation employed is that introduced at the end of § 12"22 and
means that the path of integration starts from P, encircles the point
1 in the positive (counter-clockwise) direction and returns to P, then
encircles the origin in the positive direction and returns to P, and
so on.

, * Math. Ann. xxxv. (1890), p. 495.

12-42, 12-43]

THE GAMMA FUNCTION

257

At the starting-point the arguments of t and - t are both zero; after
the circuit (1 +) they are and l-ir; after the circuit (0 +) they are
Itt and 27r; after the circuit (1 - ) they are lir and and after the
circuit (0 - ) they are both zero, so that the final vahie of the
integrand is the same as the initial value.

It is easily seen that, since the path of integration may be deformed
in any way so long as it does not pass over the branch points 0, 1 of
the integrand, the path may be taken to be that shewn in the figure,
wherein the four parallel lines are supposed to coincide with the real
axis.

>

// the real parts of a. and y9 are positive the integrals round the
circles tend to zero as the radii of the circles tend to zero*; the
integrands on the paths marked a, h, c, d are

 a-Ig-'W(a-l) (1 \ )3-lg2.r<(8-l) a-l g2,r.(a-]) ( \ )fl-i

respectively, the arguments of t and 1 - nuiu being zero in each case.

Hence we may write e (a, ) as the sum of four (possibly improper)
integrals, thus :

e (a, /3) = e- '(''+ )

I t -' (1 - 0 "' + I t"-' (1 - 0 ~'e-''

 dt

C 1 /""

+ t"-' (1 - tf-' e -'t"* ) dt + -' (1 - tf-'e- ' dt . .' 1

Hence

€ (a, /3) = e- '(' - ) (1 - e---) (1 - e ' ) f -' (1 - tf-' dt

JO

, ., ., l (a)r(/9) = - -isin (a7r)sin (ott) ~

- 47r* "r(l-a)r(l-/: )r(a + )-

Now e (a, /S) and this last expression are analytic functions of a and
of /3 for all values of a and /S. So, by the theory of analytic
continuation, this equality, proved when the real parts of a and are
positive, holds for all values of a and l3. Hence for all values of a
and /3 lue have proved that

/ J -47r-

r i-cc)l\ l-id)l\ a+id)- ' The reader ought to have uo difficulty in
proving this.

AV. M. A.

17

258 THE TRANSCENDENTAL FUNCTIONS [CHAP. XII

12"5. Dirichlet's integral*.

We shall now shew how the repeated integral

/ = f f . . . I f t, + t,+ ...+ tn) i"'- o" - . . . tn' n-idt dt, . .
. dt

may be reduced to a simple integral, where/is continuous, a > (?* = 1,
2, ... n) and the integration is extended over all positive values of
the variables such that 1 + 4+ ... + tn \$1.

To simplify p'M f\ t + T+ ) t' -'T -'dtdT

Jo Jo

(where we have written t, T, a, j3 for t, t, cui, Og and X for 3 + 4
+ ... +tn), put t = T \ - v)/v; the integral becomes (if X. 0)

r~U' /(x + Tlv) l - vy-' V-''-' r-+3-i dvdT.

Jo J T/(l- )

Changing the order of integration (§ 4*51), the integral becomes

( [ V( + T/v) l- vy-'v''-' T- -'dTdv. J J

Putting T = VTo, the integral becomes

I i /( + " 2) (1 - vy-' v -' r. + -' dr., dv J J

r (a)r(8) r -

r( + /3) Jo

Hence

  J|... j/(T. + 3+ ... +Qt/.+ -i 3" -i ... tn ' -'dr.dt, ... dtn,

r(,)r(,) r(e

the integration being extended over all positive values of the
variables such that T2 + ts + ... +tn- .

Continually reducing in this way we get

r((x, + a,+ ...+an) Jo- ' which is Dirichlet's result. Example 1.
Reduce

to a simple integral; the range of integration being extended over
all positive values of the variables such that

it being assumed that a, 6, c. a,, y, p, q, r are positive.
(Dirichlet.)

* Werkt, I. pp. 375, 391.

12o] THE GAMMA FUNCTION 259

Example 2. Evaluate / / x yi dxdy,

m and n being positive and

••' >0, 2/ 0, x>" + y " 1 . ( Pembroke, 1 907 . )

Example 3. Shew that the moment of inertia of a homogeneous ellipsoid
of unit density, taken about the axis of z, is

where a, 6, c are the semi-axes.

Example 4. Shew that the area of the epicycloid x-' +y =P is fn-r-.

REFERENCES. N. Nielsen, Handhnch der Theorie aer (Jamma-funktion*.
(Leipzig, 1906.) 0. ScHLOMiLCH, Compendium der ho her en Analysis, ll.
(Brun.swick, 1874,) E. L. LiNDELOF, Le Calcul des Residus, Ch. iv.
(Paris, 1905.) A. Pringsheim, Math. Ann. xxxi. (1888), pp. 455-481.

Miscellaneous Examples.

1. Shew that

(Trinity, 1897.)

2. Shew that

.,'i"i vr. rrp rTr.-i7i;""='"<"+"- '™'"' ' ' '>

3. Prove that

r'(i) T' (is)

f(i) ~ r(|) = S - (' ® " ' 2-)

4. Shew that

 r(i)! 32 52-1 72 92-1 IP . .

"16;; " = 3 • 5 • r T • 92 • i -ZTi •••• (Trmity, 1891.)

5. Shew that

- f ( -a)(M+ + y ) a \ \ 1 .,, .,\,,

n \ - - --7-7 -; - r 1+, H- = - - sm (an) B (/3, y).

 =o I ( +/:<) ( +y) \ + l/j TT ' " '

(Trinity, 1905.)

6. Shew that ( ) ('Isf ' (Peterhouse, 1906.)

7. Shew that, if z = iC where is real, then

l l=\/(fSVc)- (Trinity, 1904.)

8. When x is positive, shew thatt

r (x) r (h) °° 2n ! 1

 . ., Tx = 2 -; - - - . (Math. Trip. 1897.)

* This work contains a complete bibliography.

t This and some other examples are most easily proved by the result of
§ 14-11.

17-2

260 THE TRANSCENDENTAL FUNCTIONS [cHAP. XII

9. If a is positive, shew that

r( )r( +i) i ( -)" ( - 1) ((f - 2) ...(g-H) 1 r(2+ ) =o ! s+ '

10. If .V > and

JO

shew that

and

11. Shew that if X > 0,,c> 0, - tt <a <\ tt, then

/ -le-• cosacos(X;siua)c? = X- (. ') cosa. •,

/ V ~ e-' i'fosasin (Xisin a) dt==\~ r (.2;) sin a. ". (Euler.)

12. Prove that, if 6 > 0, then, when < s < 2,

/ - 7" dx = \ u\ f- cosec (J7rs)/r (s), y - and, when < s < 1,

/• * cosfto; \ J (a. .)/p (5), (Euler.)

\ / .

13. If < /i < 1, prove that

jj(l+.t'-cos -rf.r=r(,0 cos (f -1) - +,

(Peterhouse, 1895.)

14. By taking as contour of integration a parabola with its vertex at
the origin, derive from the formula

1 r** "")

2 sin an \ the result

1 f

r(a)= - -. / e- ''x"--' l+:c''-) [Ziim x + a?(.YCcot(-Jc)

2 sui utt J )

+ sin x + a - 2) arc cot ( - x) ] dx, the arc cot denoting an obtuse
angle.

(Bourguet, Acta Math. i. p. 367.)

15. Shew that, if the real part of a-n is positive and 2 l/a is
convergent, then

n = l

d' is convergent when m > 2, where \//-( ) (2) = -r log r (z). (Math.
Trip. 1907.)

16. Prove that

"jo 1'

d\ ogr z)\ /•"e- g-e-g'

< 2

= r (l+a)-i-(l+a)- ---), Jo a

/ 1 .2 - 1 \ ] = / dx-y. (Legendre.)

Jo x-l

THE GAMMA FUNCTION 261

17. Prove that, when R z)>0,

\ ogr z)=r i- -xiz-l)] f- . (Bmet.)

18. Prove that, for all values of z except negative real values,

log r (2) = (3 - 1) log 2 - 2 + Uog (27r)

J 1 1 2 J; \ 3\ = J ]

19. Prove that, when (2) > 0,

-rlogr(2) = log2- / r j l-a.-+log.r .

dz ° ° Jo (l-.r)log.r

20. Prove that, when R (2) > 0,

c/2' °

21. If

shew that

/2+1 logr(Oo? = ?',

and deduce from § 12-33 that, for all values of 2 except negative real
value.s,

u = z log z - z + i log (2n-).

(Raabe, Jovnial fur Math, xxv.)

22. Prove that, for all wilues of 2 except negative real values,

sin 2nTrx

00 f dx' logr(2) = (2-i)log2-2 + Uog(27r)+ 2 / --

n=lj •*+'

23. Prove that

(Bourguet*.

Bip,p)B p+h,p+h)= y (Binet.)

24. Prove that, when -(</•<(,

r,,,1 /""" cosh (2n*) c?M

25. Prove that, when q>\,

B p, q) + Bip + \, q) + B p + 2, q) + ...=B p, q- ) .

26. Prove that, when p-a>0,

B p-ci,q) aq a(a + l) g(g + l)

 (i, ?) ' it;4-?'*' 1. 2. (/> + (/) (io + ? + l) • ••

27. Prove that

B p, q)B p q, r) = B q, r)B q+r, p). (Euler.)

28. Shew that

n,,, .,, d.r r(a)r(b) 1

Jo " '' (.r+?>) + '' r(a + 6) (l+jo) jo"'

if a > 0, 6 > 0, p > 0. (Trinity,. 1908.)

* This result is attributed to Bourguet by Stieltjes, Journal de Math,
(i), v. p. 432.

•262

THE TRANSCENDENTAL FUNCTIONS

[chap. XII

29. Shew that, if m > 0, n>0, then

n ( i+. )2 >-i (1 - -y n-i, \ r(m)r n)

and deduce that, when a is real and not an integer multiple of ijr,

'i /cos 6 + sin \ cos 2a

/'

\ i,r veos C - SUl

d6=.

and

30. Shew that, if a > 0, > 0,

/:

2 sin (tt cos a) '

(St John's, 1904.)

(Kummer.)

31 . Shew that, if a > 0, + 6 > 0,

fr(a)r(S) r(a + b)r 8)

r..r--(i- ) \,, lrw

Jo I-''*'' 5-0 I (a

l= / (a+6)- /'(a).

' + 8) T a + b + 8) Deduce that, if in addition a + c>0, a- b + c>0,

/•l a-l (l\ b)(l\ . .c) r (t ).! ( ± \ +\ ' )

jo "(i-.r)(-log.r) * ~ *r(a + 6)r(a+c)" 32. Shew that, if a, b, c be
such that the integral converges,

fUl-x")(l-afi)(l-x'),, T b + c+l)T c + a + l)T a+b + l) ' a.v=los

JO (i-A-)(-log.r) '" '' °r(a + ' ) r b + l)r c + l)r a + h + c+l)' 33.
By the substitution cos = 1-2 tan <, shew that

(3-cos )5 4v/7r

(St John's, 1896.)

f sinP.v

34. Evaluate in terms of Gamma-functions the integral / ' dx, when /)
is a

J '*-"

fraction greater than unity whose numerator and denominator are both
odd integers.

[Shew that the integral is h I sin'' x\~+ 2 ( - )" ( 1 ) [ d.v.]

'- Jo l n=i \ x+mr x-mrj)

35. Shew that

(Clare, 1898.)

/:

93?-

2" + 2 r=o2/'!(n-r)

-a-mi-

36. Prove that

log Bip,g) log ( +i) 4- I ' \: '"\ \ l '" dv. (Euler.)

s v/', -/; °\ pq J j(, (1-V)l0g 7

37. Prove that, if p>0, p + s>0, then

B(p,p)

 '

s s~l), s(s-l)(s-2)(s-3)

 (P>P + )=- f7 ' l+ ( ) + 2.4.(2 +

fd i,-- - <--

38. The curve r"*=2'"~i a'"cos ??i is composed of ?n equal closed
loops. Shew that the length of the arc of half of one of the loops is

i~ a I (i cos x) ' dx, Jo

and hence that the total perimeter of the curve is

a iV

lm)\

THE GAMMA FUNCTION 263

39. Draw the straight line joiniug the points ±i, and the semicircle
of \ z\ = \ which lies on the right of this line. Let C be the contour
formed by indenting this figure at

- ?', 0, i. By considering / 2P- -i z + z- )p "-- dz, shew that, if p
+ q>l, q <l,

I " cosP*i--6 cos ip-q) 6 d6 =,, "!,, r.

Jo ip + q-l)2P i- B(p,q)

Prove that the result is true for all values of p and q such that p +
q>'l.

(Cauchy.)

40. If s is positive (not necessarily integral), and - in .r hn, shew
that

Mild draw graphs of the series and of the function cos*.i'.

41. Obtain the expansion

cos .f-2, r (*+i)[r(| +ia+i)r(i5- a+i) " r(h+?,a+i)raj- a+iy-]'

and find the values of x for which it is applicable. (Cauchy.)

42. Prove that, if /> > A,

22p-i r 2/>"- f 12 12.32 1 "|2

'"'''"=-A ""Wi'U + l i' 2(2yT3-) + 2.4.(2p+3)(2f+5)+ . '

(Binet.)

43. Shew that, if .c < 0, ./• + -- > 0, then

r(-.f) [-.f,(-..-)(l-.r) (-.r)(l- )(2-f ) 1 r(5) 1 5 - s(l+2) " z
l+z) 2 + z) j

and deduce that, when x + r> 0,

  Ina m +•* ) \ f \ 4 • • -1), X •'•(• •-l)(- -2) \ f/-' r>) z z(z +
l) - z z + l) z + 2) •• •

44. Using the result of example 43, prove that

logr(2 + a) = logr(0 + rtlogs

2z

fa

dt

a [\ \ { t)(2-t) ...(n-t)dt- f" t l-t) 2-t) ...(n-t)'

\ 5 .' .'o

nti n + l)z z+l) z + 2) ... z + n)

investigating the region of convergence of the .series.

(Binet, Journal de V Ecole polytechnique, xvi. (1839), p. 256.)

45. Prove that, if /> > 0, > 0, then

-. P-

B ip, q) = V- i (2 )* * "' '

264 THE TRANSCENDENTAL FUNCTIONS [CHAP, XII

where

II (p, g) = 2p - - T arc tan - - -. - -, -,

and p2 = 2 + q +pq-

46. If 6 =2*-'7r(l-i'r), F=2 - 7r(i-iA'),

and if the function F (,r) be defined by the equation

shew (1) that F o:) satisfies the equation

F x+ ) =xF x) +

r(l-A-)'

(2) that, for all positive integral values of x,

F x) = rix\

(3) that F(x) is analytic for all finite values of x,

1 (7 \ 2

(4) that "- "* '

47. Expand

F(x)= ~- r -r- log- 7

as a series of ascending powei's of a.

(Various evaluations of the coefficients in this expansion have been
given by Bourguet, Bull des Set. Math. v. (1881), p. 43; Bourguet,
Acta Math. ll. (1883), p. 261; Schlomilch,' Zeitschrift fiir Math,
und Phys. xxv. (1880), pp. 35, 351.)

48. Prove that the G -function, defined by the equation

G(z + ) = (2nf'e- ' + - n |(i+i)%-'+-'- /(2")|,

is an integral function which satisfies the relations

0 z + ) = V z)G z), (?(1) = 1,

(n !)'V6-' (n + 1) = 11 . 22 . 33 ... w". (Alexeiewsky.)

(The most important properties of the G -function are discussed in
Barnes' memoir, Quarterly Journal, xxxi.)

49. Shew that and deduce that

log Q J = / s cot Tvzdz-z log (27r).

50. Shew that

logrri + l)c <=i3log(2rr)-|3(2+i;+3logr(s + l)-logG'(s + l).
