\chapter{Mathieu Functions} 

19"1. The differential equation of Mathieu.

The preceding five chapters have been occupied with the discussion of
functions which belong to what may be generally described as the
hyper- geometric type, and many simple properties of these functions
are now well known.

In the present chapter we enter upon a region of Analysis which lies
beyond this, and which is, as yet, only very imperfectly explored.

The functions which occur in Mathematical Physics and which come next
in order of complication to functions of hypergeometric type are
called Mathieu functions; these functions are also known as the
functions associated ivith the elliptic cylinder. They arise from the
equation of two- dimensional wave motion, namely

dx dy c- dt'

This partial difterential equation occurs in the theory of the
propagation of electro- magnetic waves; if the electric vector in the
wave-front is parallel to OZ and if E denotes the electric force,
while Hx, ffy, 0) are the components of magnetic force, Maxwell's
fundamental equations are

lo \ 8 \ a c\ H \ \ dE dHy\ dE <? dt ~ dx ly ' at ~ dij ct ~ dx '

c denoting the velocity of light; and these equations give at once

c ct cx' cy' '

In the case of the scattering of waves, propagated parallel to OX,
incident on an elliptic cylinder for which OX and OY are axes of a
pi-incipal section, the boundary condition is that E should vanish at
the surface of the cylinder.

The same partial differential equation occurs in connexion with the
vibrations of a uniform plane membrane, the dependent variable being
the displacement perpendiculai- to the membrane; if the membrane be
in the shape of an ellipse with a rigid boundary, the boundary
condition is the same as in the electromagnetic problem just
discussed.

The differential equation was discussed by Mathieu* in 1868 in
connexion with the problem of vibrations of an elliptic membrane in
the following manner :

* Journal de Math. (2), xiii. (1868), p. 137.

19*1] MA.THIEU FUXCTIOXS 405

Suppose that the membrane, which is in the plane XOY when it is

in equilibrium, is vibrating with frequency p. Then, if we write

V= u (cc, y) cos pt + e),

the equation becomes

d' u d u p- -, + -, + li = 0. cw dy c-

Let the foci of the elliptic membrane be (+ h, 0, 0), and introduce
new

real variables*, i] defined by the complex equation

cc + iy = h cosh ( + irj),

so that x = h cosh cos t, y = h sinh | sin r].

The curves, on which or r] is constant, are evidently ellipses or
hyper- bolas confocal with the boundary; if we take and — tt < ?? \$
tt, to each point (w, y, 0) of the plane corresponds one and only onef
value of (, ??).

The differential equation for u transforms into;]:

af-2 + a + - (cosh- f - cos- 7?) u = 0.

If we assume a solution of this equation of the form

u = F )G v), where the factors are functions of only and of t) only
respectively, we see that

1 d'J(g) Ay )\ ( 1 d'GM i jf

Since the left-hand side contains but not ? while the right-hand side
contains ri but not, F ) and G (tj) must be such that each side is a
constant, A, say, since | and tj are independent variables.

We thus arrive at the equations

 ) + ('i!£%osl.f- )l-(f) = 0,

By a slight change of independent variable in the former equation, we
see that both of these equations are linear differential equations, of
the second order, of the form

-T-; -f (a -1- 1 65 cos 2z) u = 0,

* The iutroduction of these variables is due to Lame, who called the
thermometric parameter. They are more usually known as confocal
coordinates. See Lame, Sur les fonctions inverses des transcendantes,
1'"'' Le(?on.

+ This may be seen most easily by considering the ellipses obtained by
giving f various positive values. If the ellipse be drawn through a
definite point (t, v) ot the plane, r? is the eccentric angle of that
point on the ellipse.

t A proof of this result, due to Lame, is given in numerous text-books
; see p. 401, footnote.

406 THE TRANSCENDENTAL FUNCTIONS [CHAP. XIX

where a and q are constants*. It is obvious that every point (infinity
ex- cepted) is a regular point of this equation.

This is the equation which is known as Jfathieu's equation and, in
certain circumstances (§ 19'2), particular solutions of it are called
Mathieu functions.

1911. The form of the solution of Mathieu s equation.

In the physical problems which suggested Mathieu's equation, the
constant a is not given a priori, and we have to consider how it is to
be determined. It is obvious from physical considerations in the
problem of the membrane that u (x, y) is a one-valued function of j
osition, and is consequently unaltered by increasing 77 by 27r; and
the condition f G (r) + lif) = G rj) is sufficient to determine a set
of values of a, in terms of q. And it will appear later (§§ 19'4,
19*41) that, when a has not one of these values, the equation

G 'n + 2'rr)=G (v) is no longer true.

When a is thus determined, q (and thence p) is determined by the fact
that F ) = on the boundary; and so the periods of the free vibrations
of the membrane are obtained.

Other problems of Mathematical Phy.sics which involve Mathieu
functions in their solution are (i) Tidal waves in a cylindrical
vessel with an elliptic boundary, (ii) Certain forms of steady vortex
motion in an elliptic cylinder, (iii) The decay of magnetic force in a
metal cylinder . The equation also occurs in a problem of Rigid
Dynamics which is of general interest g.

19'12. Hill's equation.

A differential equation, similar to Mathieu's but of a more general
nature, arises in G. W. HiU's]] method of determining the motion of
the Lunar Perigee, and in Adams' determination of the motion of the
Lunar Node. Hill's equation is

 ' + k + 2 i cos2 ) = 0.

The theory of Hill's equation is very similar to that of Mathieu's (in
spite of the increase in generality due to the presence of the
infinite series), so the two equations will, to some extent, be
considered together.

* Their actual values are a = A - h-p-l 2c-), q = h-2)-l(S2c-); the
factor 16 is inserted to avoid powers of 2 in the solution.

t An elementary analogue of this result is that a solution of -j- +aii
= has period 2v if,

and only if, a is the square of an integer.

* K. C. Maclaurin, Trans. Camb. Phil. Soc. xvii. p. 41.

§ A. W. Young, Proc. Edinburgh Math. Soc. xxxii. p. 81.

II Acta Math. viii. (1S86). Hill's memoir was originally published in
1877 at Cambridge, U.S.A.

H Monthly Notices R.A.S. xxxviii. p. 43.

19"l]-19"2l] MATHIEU FUNCTIONS 407

In the astronomical applications o> i> ••• are known constants, so the
problem of choosing them in such a way that the solution may be
periodic does not arise. The solution of Hill's equation in the Lunar
Theory is, in fact, not periodic.

19'2. Periodic solutions of Mathieus equation.

We have seen that in physical (as distinguished from astronomical)
problems the constant a in Mathieu's equation has to be chosen to be
such a function of q that the equation possesses a periodic solution.

Let this solution be G (z); then G z), in addition to being periodic,
is an integral function of z. Three possibilities arise as to the
nature of G z) : (i) G (z) may be an even function of z, (ii) G z) may
be an odd function of z, (iii) G (z) may be neither even nor odd.

In case (iii), G (z) + G - z)]

is an even periodic solution and

  G z)-G -z)]

is an odd periodic solution of Mathieu's equation, these two solutions
forming a fundamental system. It is therefore sufficient to confine
our attention to periodic solutions of Mathieu's equation which are
either even or odd. These solutions, and these only, will be called 31
athieu functions.

It will be observed that, suice the roots of the indicial equation at
z = are and 1, two even (or two odd) periodic solutions of Mathieu's
equation cannot form a fundamental system. But, so far, there seems to
be no reason why Mathieu's equation, for special values of a and q,
should not have one even and one odd periodic solution; for com-
paratively small values of 1 5' ' it can be seen [§ 19*3 example 2,
(ii) and (iii)] that Mathieu's equation has two periodic solutions
only in the trivial case in which q = 0; but for larger values o \ q\
there may be pairs of periodic solutions, though no such pairs have,
as yet, been discovered.

19"21. An integral equation satisfied hy even Mathieu functions*.

It will now be shewn that, if G r ) is any even Mathieu function, then
G(r)) satisfies the homogeneous integral equation

G(v)=-\ [" e'''°' '''' G(e)dO,

where k = \/(32 ). This result is suggested by the solution of
Laplace's equation given in § 1 83.

* This integral equation and the expansions of § 19"3 were published
by Wbittaker, Proc. Int. Congress of Math. 1912. The integral equation
was known to him as early as 1904; see Trans. Camb. Phil. Soc. xxi.
(1912), p. 193.

408 THE TRANSCENDENTAL FUNCTIONS [CHAP. XIX

For, if A' + i/ = h cosh, i +iri) and if F ) and G (v) are solutions
of the differential equations

 d - ( + f cosh I) F( ) = 0, ( '\ M + (A + nvJf- cos- 7;) G v) = 0,

then, by § 191, F ) G rj) e" is a particular solution of Laplace's
equation. If this solution is a special case of the general solution

/(// cosh cos 77 cos + h sinh sin 7; sin 6 + iz, 6) dO,

given in § 18"3, it is natural to expect that*

f(v,e) F 0)e' '<f>(e), where <f> (6) is a function of 6 to be
determined. Thus

F )G (v) e""' =1 F(0)(f) (6) exp mh cosh cos v cos 6

. —IT

+ mh sinh sin rjsmO + miz] dd.

Since and tj are independent, we may put = 0; and we are thus led to
consider the possibility of Mathieu's equation possessing a solution
of the form

J —77

19-22. Proof that the even Mathieu functions satisfy the integral
equation.

It is readily verified (§ 5-31) that, if (f> (6) be analytic in the
range (- tt, tt) and if 6 (7;) be defined by the equation

G (77) = I " e'"'' cosncose ( ) 0

then G (77) is an even periodic integral function of 7; and

— j- + (A + m-h- cos- 7;) G (7;) drf-

= [" ?u-/(2 (sin- 77 cos- + cos 77) - mh cos 77 cos + g'"'' cos >,
cose ( q

J - TV

= - n H/i sin e cos 77</) ( ) + </)' 6)] e' ''cos cos6

 < " (6') + ( + ni'h'' cos' ) (/) ( ) e"'''cos.,cosfl /

on integrating by parts.

* The constant F (0) is inserted to simplify the algebra.

19 "22, 19 '3] MATHIEU FUNCTIONS 409

But if(f) 6) be a ])eriodic function ivith period 27r) such that (f)"
(6) + A+ m h- cos 6) </> 6) = 0,

both the integral and the integrated part vanish; that is to say, G
(rj), defined by the integral, is a periodic sohition of Mathieu's
equation.

Consequently G (tj) is an even periodic solution of Mathieu's equation
if <f) (0) is a periodic solution of Mathieu's equation formed with
the same con- stants; and therefore (ft (6) is a constant multiple of
G (6); let it be \ G(6).

[In the case when the Mathieu equation has two periodic sokitions, if
this case exist, we have (p d) = XG (d) + Gi 6) where 6-'i 6) is an
odd periodic function; but

 mh cosv cos eg g

r

vanishes, so the subsequent work is unaffected.]

If we take a and q as the parameters of the Mathieu equation instead
of A and mh, it is obvious that mh = \/ S2q) = k.

We have thus proved that, if G 'r]) be an even periodic solution of
Mathieu's equation, then

G r]) = \ r e''-' ' °' G(0)de,

which is the result stated in § 19'21.

From § 11 "23, it is known that this integral equation has a solution
only when X has one of the ' characteristic values.' It will be shewn
in § 19-3 that for such values of \, the integral equation affords a
simple means of con- structing the even Mathieu functions.

Example 1. Shew that the odd Mathieu functions satisfy the integral
equation

G i]) = \ j sin (/• sin r] sin 6) G (d) dd.

Example 2. Shew that both the even and the odd Mathieu functions
satisfy the integral equation

G,j) = x[" e ''''''' G d)de.

Example 3. Shew that when the eccentricity of the fundamental ellipse
tends to zero, the confluent form of the integral equation for the
even Mathieu functions is

J,,(.r) = /" e'-*' cose cos

ZTTi" j -

d6.

19"3. The construction of Mathieu functions.

We shall now make use of the integral equation of § 19*21 to construct
Mathieu functions; the canonical form of Mathieu's equation will be
taken as

-T 4- (o + 1 6g cos z) u = 0.

410 THE TRANSCENDENTAL FUNCTIONS [CHAP. XIX

In the special case when q is zero, the periodic sohitions are
obtained by taking a= n-, where n is any integer; the solutions are
then

1, cos ', cos 22, ...,

sin z, sin z,

The Mathieu functions, which reduce to these when q- 0, will be called
ceo z, q), cei (z, q), ce. z, q), . . ., se?! z, q), seo (z, q), ....
To make the functions precise, we take the coefficients of cos nz and
sin nz in the respective Fourier series for ce z, q) and sen z, q) to
be unity. The functions cen z, q), sen i, Q) will be called Mathie it
functions of oixler n. Let us now construct ce (z, q).

Since ceo(3', 0)=1, we see that A,-*.(27r)~ as (y --* 0. Accordingly
we suppose that, for general values of q, the characteristic value of
X which gives rise to ce z, q) can be expanded in the form

(27r ) -i = 1 + ttig + Oioq- + . . ., and that ce (z, q) = l+ q i z)
+ q- o ( ) + . . .,

where Oj, Oo, ... are numerical constants and x z), 13.2 (z), ... are
periodic functions of z which are independent of q and which contain
no constant term.

On substituting in the integral equation, we find that

(1 + oc,q+a,q- + ...) l+q/3, (z) -i-q', z) + ...

1 T" = — / (1 + \/(32g) . cos cos + 16r/cos- 'cos- + ...

Equating coefficients of successive powers of q in this result and
making use of the fact that i(z), /SjC ),,. .. contain no constant
term, we find in succession

Oj = 4, /3i (z) = 4 cos 2z,

oTo = 14, /Sa (z) = 2 cos 4,

and we thus obtain the following expansion :

/ -77 29 \ / 160 N

cco (z, q) = l + Uq - 28q + "- q' - ...jcos2z + i2q-- ~ q + ...j cos
4iz

+ U'f-' 5' + • ••) cos 6z + ( — r/ - . . . j cos 8

the terms not written down being (q ) as -* 0.

210 99 The value of a is -S2q- + 224>q' l f/+0(r/); it will be
observed

that the coefficient of cos 2z in the series for C6?o(2, q) is
—ai(8q).

19 'SI] MATHIEU FUNCTIONS 411

The Mathieu functions of higher order may be obtained in a similar
manner from the same integral equation and from the integral equation
of § 19"22 example 1. The consideration of the convergence of the
series thus obtained is postponed to § 19"61.

Example 1. Obtain the following expansions*:

(i) ce, (., s, \ 1 +, J\ -. ~ \ y i ~ + (,/ • .) CO, 2..,

oc I" gV ' r+l qr + l

(n) cei(2, o) = cos5+ 2 \ \ — r ~ /, i m / "ttv. r=i i(?'+l)!?*! (r +
1)! (r+1)!

(ni) se, z, q) = sm .+ 2 ( j-yy + (.+ 1) ! (.+ 1)

(iv) C<?2 (, q) = I - 2(2 + — g3 + () ( 5-) I . COS 22

where, in each case, the constant implied in the symbol depends on r
but not on z.

(Whittaker.)

Example 2. Shew that the values of a associated with (i) ceo(5, </),
(ii) cei z, q), (iii) sfij (2, 5'), (iv) ce2 z, q) are respectively :

210 2Q (i) -32(?2 + 224(?4 \ \, 6 + ((? ),

(ii) l-8q- + -\ qi + 0 q%

(iii) l + 8j-8j2\ 823- <?* + 0(j5

(iv) 4 + j2- g* + 0(? ). (Mathieu.)

Example 3. Shew that, if n be an integer,

19*31. T e integral formulae for the Mathieu functions.

Since all the Mathieu functions satisfy a homogeneous integral
equation with a symmetrical nucleus (§ 19*22 example 3), it follows (§
11'61) that

cem z, q) cen (z, q)dz = (m n),

.' - jr

sem (z, q) sen (z, q)dz = m i= n)

T

cem z, q) sen (z, q)dz = 0.

T

* The leading terms of these series, as given in example 4 at the end
of the chapter (p. 427), were obtained by Mathieu.

412 THE TRAXSCENDENTAL FUNCTIONS [CHAP. XIX

Example 1. Obtain expansions of the form :

(ii) cos (/• sin z sin ) = 2 B ce (z, q) ce 6, q),

?l=0

(iii) sin (/ sin z sin 6)= 2 C se (2, q) se 6, q), where i=, 32q).

Example 2. Obtain the expansion

)! = — 3C

as a confluent form of expansions (ii) and (iii) of example 1.

19"4. The nature of tJie solution of Mathieu s general equation;
Floquet's theory.

We shall now discuss the nature of the solution of Mathieu's equation
when the parameter a is no longer restricted so as to give rise to
periodic solutions; this is the case which is of importance in
astronomical problems, as distinguished from other ph -sical
applications of the theory.

The method is applicable to any linear equation with periodic
coefficients which are one-valued functions of the independent
variable; the nature of the general solution of particular equations
of this type has long been per- ceived by astronomers, by inference
from the circumstances in which the equations arise. These inferences
have been confirmed by the following analytical investigation which
was published in 1883 by Floquet*.

Let g z), h (z) be a fundamental system of solutions of Mathieu's
equation (or, indeed, of any linear equation in which the coefficients
have period 27r); then, if F z) be any other integral of such an
equation, we must have

F z)=Ag(z)+Bh(z), where A and B are definite constants.

Since g z+ 27r), h (z -f 27r) are obviously solutions of the
equationf, they can be expressed in terms of the continuations of g
(z) and Ji (z) by equations of the type

g(z + 27r) = a g (z) + a,h (z), h z Itt) =,g z) + 0,h (z), where ttj,
a.>, /S, /?.\ > are definite constants; and then

F z + 27r) = (Aa + B,) g z) + Aoi., + B/3,) h (z).

* Ann. de VEcole norm. sitj). (2), xii. (1883), p. 47. Floquet's
analysis is a natural sequel to Picard's theory of differential
equations with doubly-periodic coefficients (§ 20-1), and to the
theory of the fundamental equation due to Fuchs and Hamburger.

t These solutions may not be identical with (j(z), h(z) respectively,
as the solution of an equation with periodic coefficients is not
necessarily periodic. To take a simple case, H = e sin z

is a solution of -r — (1 + cot ) 1/ = 0.

dz '

19-4, 19'41] MATHIEU FUNCTIONS 418

Consequently F z + 27r)= kF(z), where k is a constant*, if A and B are
chosen so that

A a, + BI3, = hA, A a, + B/3o = kB.

These equations will have a solution, other than A = B = 0, if, and
only if,

oc,-k, A =0;

ofo, /5o — A.-

and i k be taken to be either root of this equation, the function F(2)
can be constructed so as to be a solution of the differential equation
such that

F(z+27r) = kF(z).

Defining fi by the equation k= e-'" and writing ( z) for e~' F z), we
see that

(f>(z + 27r) = €-''''+-''> F z+-2tt)=( (z).

Hence the differential equation has a particular solution of the form
e' (f) z), where (f)(z) is a periodic function with period 27r.

We have seen that in physical problems, the jjarameters involved in
the differential equation have to be so chosen that k = l is a root of
the quadratic, and a solution is periodic. In general, however, in
astronomical problems, in which the parameters are given, A- 1 and
there is no periodic solution.

In the particular case of Mathieu's general equation or Hill's
equation, a fundamental system of solutions f is then e' -(f)(z), e~>
—z), since the equation is unaltered by writing — r for; so that the
complete solution of Mathieu's general equation is then

u = Cie'*-(f) z) + Coe~i ( (— z),

where Ci, c, are arbitrary constants, and /i is a definite function of
a and q.

Example. Shew that the roots of the equation

a,-k,,3i =0 ao, )io — k are independent of the particular pair of
solutions, g z) and h (z), chosen.

19"41. Hill's method of solution.

Now that the general functional character of the solution of equations
with periodic coefficients has been found by Floquet's theory, it
might be expected that the determination of an explicit expression for
the solutions of Mathieu's and Hill's equations would be a
comparatively easy matter; this however is not the case. For example,
in the particular case of Mathieu's general equation, a solution has
to be obtained in the form

y = e (f) (z),

* The symbol k is used in this particular sense only in this section.
It must not be confused with the constant A; of § 19-21, which was
associated with the parameter q of Mathieu's equation. t The ratio of
these solutions is not even periodic; still less is it a constant.

414 THE TRANSCENDENTAL FUNCTIONS [CHAP. XIX

where <f) (z) is periodic and /i is a function of the parameters a and
q. The crux of the problem is to determine /j.; when this is done,
the determination of (f) (2) presents comparatively little difficulty.

The first successful method of attacking the problem was published by
Hill in the memoir cited in § 19'12; since the method for Hill's
equation is no more difficult than for the special case of Mathieu's
general equation, we shall discuss the case of Hill's equation, viz.

where J (z) is an even function of z with period tt. Two cases are of
interest, the analysis being the same in each :

(I) The astronomical case when z is real and, for real values of z, J
(z) can be expanded in the form

J(z) = 00 + 2 1 cos 2z + 202 cos 4>z + 26 cos 6 + . . .; the
coefficients 6n are known constants and S 6n converges absolutely.

n=0

(II) The case when is a complex variable and J (2) is analytic in a
strip of the plane (containing the real axis), whose sides are
parallel to the

real axis. The expansion of J(z) in the Fourier series 6 + 2 "Z 6n cos
2nz

H = l

is then valid (§ 9"11) throughout the interior of the strip, and, as
before,

00

2 On converges absolutely. =o

Defining \ to be equal to 6n, we assume

00

71= -OC •

as a solution of Hill's equation.

[In case (II) this is the solution analytic in the strip (§> 10-2,
19'4); in case (I) it will have to be shewn ultimately (see the note
at the end of § 19*42) that the values of 6

which will be determined are such as to make 2 n-bn absolutely
convergent, in order to

n= —

justify the processes which we shall now carry out.] On substitution
in the equation, we find

M=— 00 \ n=-x / j=— 00 /

Multiplying out the absolutely convergent series and equating
coefficients of powers of e ' to zero (§§ 9"6-9"632), we obtain the
system of equations

(,i + 2niyb + i e X\,, = (n = ..., -2,-1,0,1,2, ...).

19-42]

MATHIEU FUNCTIONS

415

If we eliminate the coefficients bn determinantally (after dividing
the typical equation by 6o — 4 n" to secure convergence) we obtain*
Hill's deter- minantal equation :

( >+4)• -

io

- >

-00

-0, 42- 0

-0,

- 4 -0,

4— 0

4- -00

42- 0 "'

-e.

0> + 2)2-, 2- -00

-0,

 2' -Bo

-02

2- -00

- 3

'" 22-,,

2- -00 -

-Oo

-6,

W-do

02 -do

-0, 0- -00

-do

 •• 02 -. 0

0- - 6 0

02 -do -

-03

-6-2 22- 0

-0

22 - o

(z>-2)2- 22- 0

00

-01

"• 2 -60

22- do -

-0,

- 3

42-,,

-02

i--0o

- 1 42- 0

 ±

-4)2 -do 42 -do -

=0.

We write A ifx) for the determinant, so the equation determining yu.
is

A (i = 0.

19"42. The evaluation of Hill's determinant.

We shall now obtain an extremely simple expression for Hill's deter-
minant, namely

A iix) = A (0) - sin- ( tti/x) cosec- (|-7r V o)-

Adopting the notation of § 2*8, we write

A(i =[,,,J, (t'/i — 2m)- — Oq

where,, j =

4?7i- - dr,

A m—n

4m2 — 0Q

(m n).

The determinant [,,ii] is only conditionally convergent, since the
product of the principal diagonal elements does not converge
absolutely (§§ 2"81, 2'7). We can, however, obtain an absolutely
convergent determinant, Aj (i/x), by dividing the linear equations of
§ 19"41 by 6q— (i/x— 2n)- instead of dividing by (, — 4/1-. We write
this determinant Ai(2ju,) in the form [5, \ ], where

  m,ra — -••) - j/i, n — /.

-Or.

(m n).

 2in — i/x)-— Hf)

The absolute convergence of S 6,1 secures the convergence of the
deter-

minant [-S,, ], except when /x has such a value that the denominator
of one of the expressions B n vanishes.

* Since the coefl\&cients 6,j are not all zero, we may obtain the
infinite determinant as the eliminant of the system of linear
equations by multiplying these equations by suitably chosen cofactors
and adding up.

416 THE TRANSCENDENTAL FUNCTIONS [CHAP. XIX

From the definition of an infinite determinant (§ 2'8) it follows that

sin TT (?> - V o) sin tt (i> + \/Oo) and so A (i ) = - A, (i ) —
smM*7rV ) '

Now, if the determinant A, i/j,) be written out in full, it is easy to
see (i) that Ai (ifi) is an even periodic function of /x with period
2tV(ii) that Aj / ) is an analytic function (cf. §§ 2'81, 8"34, 5'3)
of yu, (except at its obvious simple poles), which tends to unity as
the real part of /j, tends to ± oo .

If now we choose the constant K so that the function D (/a), defined
by the equation

D (ijl) = Ai (i"/u.) - K [cot ir ifi, + \/0(,) - cot i TT (ifj, - V o)
,

has no pole at the point ij, = i 6, then, since D /j,) is an even
periodic function of yu., it follows that D (/x) has no pole at any of
the points

2ni ± i V o) where n is any integer.

The function D (/j.) is therefore a periodic function of /n (with
period 2 ) which has no poles, and which is obviously bounded as R
(//.) + x . The conditions postulated in Liouville's theorem (§ 5'63)
are satisfied, and so D (fi) is a constant; making / - + go, we see
that this constant is unity.

Therefore

Ai (ifi) = 1+K cot Itt ifi + \/d,) - cot -h TT ifi - V(9o), and so

sinl7r(t>- V o)sin 7r(i>+\/ o), o7- wi ia\ A rfi) = sinM -V o) " - '
'"' " " -

To determine K, put /a =; then

A(0) = l + 2/i:cot(i7rv/ o)- Hence, on subtraction,

A(, = MO)-| li

which is the result stated.

The roots of Hill's determinantal equation are therefore the roots of
the equation

sin -nifi) = A (0) . sin ( tt V o)-

When fj, has thus been determined, the coefficients bn can be
determined in terms of b, and cofactors of A ifi); and the solution
of Hill's differential equation is complete.

19 '5, 19 '51] MATHIEU FUNCTIOXS 417

[In case (I) of § 19*41, the convergence of 2 | 6 | follows from the
rearrangement theorem of § 2-82; for 2 2 1 6 | is equal to | 6o | 2 |
C j, o I - i o, o I where C, n is the cofactor of B,

ni= — x>

in Ai (ifi.)', and 2 | C i,o I is the determinant obtained by
replacing the elements of the row through the origin by numbers whose
moduli are bounded.]

It was shewn by Hill that, for the purposes of his astronomical
problem, a remarkably good approximation to the value of fj. could be
obtained by considering only the three central rows and columns of his
determinant.

19'5. The Lindemann-Stieltjes' theory of Mathieus general equation.

Up to the present, Mathieu's equation has been treated as a linear
differential equation with periodic coefficients. Some extremely
interesting properties of the equation have been obtained by
Lindemann* by the sub- stitution =cos, Avhich transforms the equation
into an equation with rational coefficients, namely

4 (1 - O, + 2 (1 - 20 + (a -I6q + 32 0 = 0.

This equation, though it somewhat resembles the hypergeometric
equation, is of higher type than the equations dealt with in Chapters
xiv and xvi, inasmuch as it has two regular singularities at and 1 and
an irregular singularity at x; whereas the three singularities of the
hypergeometric equation are all regular, while tlie equation for TFj.\
(3) has one irregular singularity and only one regular singularity.

We shall now give a short account of Linderaann's analysis, with some
modifications due to Stieltjesf.

19"51. Lindemann' s form of Floquet's theorem.

Since Mathieu's equation (in Lindemann's form) has singularities at =
and = 1, the exponents at each being 0, \, there exist solutions of
the form

W=0 M=0

2 o = i an (1 - y\ u = (1 - 0* i n (1 - KT;

>i = M =

the first two series converge when ] j < 1, the last two when 1 1 — |
< 1.

When the -plane is cut along the real axis from 1 to + x and from to —
00, the four functions defined by these series are one-valued in the
cut plane; and so relations of the form

Vw = ay 00 + Voi, Vn = 73/00 + i/oi will exist throughout the cut
plane.

Now suppose that describes a closed circuit round the origin, so that
the circuit crosses the cut from — oo to; the analytic continuation
of 3/10 is

* Math. Ann. xxii. (1883), p. 117.

t Astr. Nach. cix. (1884), cols. 145-152, 261-266. The analysis is
very similar to that employed by Hermite in his lectures at the Ecole
Polytechnique in 1872-1873 [Oeuvres, iii. (Paris, 1912), pp. 118-122]
in connexion with Lame's equation. See § 23"7.

W. M. A. 27

418 THE TRAXSCENDEXTAL FLECTIONS [CHAP. XIX

oj/oo — /3yoi (since l/ is unaffected by the description of the
circuit, but ?/oi changes sign) and the continuation of j/u is 7 00 -
j/oi; "c? so Ay - + By - will he unaffected by the description of the
circuit if

A ay,o + ySyoi)' + B yy + SyoO' s A ay - /3?/oi)' + B (73/00 - S oi)-,

i.e. if Aa + ByS = 0.

Also Ay f-h Byii- obviously has not a branch-point at f=l, and so, if
Aal3 + By8 — 0, this function has no branch-points at or 1, and, as it
has no other possible singularities in the finite part of the plane,
it must be an integral function of .

The two expressions

A y,o + iB -yn, -j/io - iB 'l/u are consequently two solutions of
Mathieu's equation whose product is an integral function of .

[This amounts to the fact (§ 19"4) that the product of ef" ( > (z) and
e~' ~ (— z) is a periodic integral function of z.']

1952. The determination of the integral function associated with
Mathieu's general equation.

The integral function F(z) = Ay o" + By, just introduced, can be
deter- mined without difficulty; for, if jo and y are any solutions
of

 :+p(n|+Q(r) =o,

their squares (and consequently any linear combination of their
squares) satisfy the equation*

 ! + 3P (D + [P' (0 + 4Q (0 + 2 [P (or ]

in the case under consideration, this result reduces to

+ (a-l-l6q + S2q ) J - 16qF (f) = 0.

X

Let the Maclaurin series for F ! ) be 1 c,i "; on substitution, we
easily obtain the recurrence formula for the coefficients c, namely

where

(n + I) (n - ly - a + 16q] \ \ n (n -h l)(2n + l)

""" ieq(2n + l) ' ''•" S2q 2n-1) "

* Appell, Comptes Rendus, xci. (1880), pp. 211-214; cf. example 10,
p. 298 supra.

19-52, 19-53]

MATHIEU FUNCTIONS

419

At first sight, it appears from the recurrence formula that Co and Ci
can be chosen arbitrarily, and the remaining coefficients C2, C3, ...
calculated in terms of them; but the third order equation has a
singularity at t= 1) nd the series thus obtained would have only unit
radius of convergence. It is necessary to choose the value of the
ratio Ci/Cq so that the series may con- verge for all values of .

The recurrence formula, when written in the form

suggests the consideration of the infinite continued firaction

Uu+'

V,

W2

n+i 1" W j o + • •

lim 1*, +

nn+i + ...+

The continued fraction on the right can be -sNTitten* w,i/r (n, n +
m)IK (n + 1, n + m),

where K (n, n + m) =

1

— u

-1

- Uni->, 1

The limit of this, as ??i - x, is a convergent determinant of von
Koch's type (by the example of § 2'82); and since

Vr+i

llrUr+l

as n - 00,

it is easily seen that K (n, x ) 1 as 7i -* x .

Cn Un K n, X )

Therefore, if

Cn+1 K n + 1, co)' then Cn satisfies the recurrence formula and, since
Cn i/Cn - as ?i - x, the resulting series for F ( ) is an integral
function. From the recurrence formula it is obvious that all the
coefficients c are finite, since they are finite when n is
sufficiently large. The construction of the integral function F ( )
has therefore been effected.

19"53. The solution of Mathieus equation in terms of F( ). If Wi and
Wo be two particular solutions of

g+P(r)|+(3(f)"=o,

thenf

W Wi — W1IV2

r=cexp|-j P(r)fzr >

* Sylvester, Phil. Mag. (4), v. (1S53), p. 446 [Math. Papers, i. p.
609].

t Abel, Journal fiir Math. ii. (1827), p. 22. Primes denote
differentiations with regard to f.

97 9

420 THE TRANSCENDENTAL FDNCTIONS [CHAP. XIX

where is a definite constant. Taking iv and w to be those two
solutions of Mathieu's general equation whose product is -P( ), we
have w w. C w/ w,' r( )

W, Wo f (l- t)*i (0' '"'' '2 iO' the latter following at once from the
equation tv iu. Fi ).

Solving these equations for iv lic, and tuJ/wo, and then integrating,
we at once get

where 71, y.. are constants of integration; obviously no real
generality is lost by taking Cq = 71 = 72 = 1-

From the former result we have, for small values of | |,

while, in the notation of § 1 "51, we have aJao = — a+ Sq.

Hence C = I69 — a — c .

This equation determines C in terms of a, q and Cj, the value of Ci
being

K(l, cc) uoK(0, x) .

Example 1. If the solutions of Mathieu's equation be e' ' (p ±z),
where </> (s) is periodic, shew that

Example 2. Shew that the zeros oi F C) are all simple, unless (7=0.

(Stieltjes.)

[If F () could have a re jeated zero, v and ivo would then have an
essential singularity.]

19"6. A second method of constructing the Mathieu function.

So far, it has been assumed that all the various series of § 19'3
involved in the expressions for cey(2, q) and sey(z, q) are
convergent. It will noiv be sheton that ce z, q) and scy z, q) are
integral functions of z and that the coefficients in their expansions
as Fourier series are power series in q which converge absolutely when
\ q\ is sufficiently small*.

To obtain this result for the functions ce2f z, q), we shall shew how
to determine a particular integral of the equation

- + (a + \ Qq cos 2z) u = y\ r a, q) cos Nz

* The essential part of this theorem is the proof of the convergence
of the series which occur in the coefficients; it is already known §§
10'2, 10-21) that solutions of Mathieu's equation are integral
functions of z, and (in the case of periodic solutions) the existence
of the Fourier expansion follows from § 9*11.

19-6] MATHIEU FCNCTIOXS 421

in the form of a Fourier series converging over the whole 2 -plane,
where yjr (a, q) is a function of the parameters a and q. The equation
-v/r (a, q) = then determines a relation between a and q which gives
rise to a Mathieu function. The reader who is acquainted with the
method of Frobenius* as applied to the solution of linear differential
equations in power series will recognise the resemblance of the
following analysis to his work.

Write a = iY + 8p, where JSf is zero or a positive or negative
integer.

Mathieu's equation becomes

 \ +:N'Hi = -S (p + 2q cos 2z) 11.

If jj and q are neglected, a solution of this equation is u — cos Nz=
Uo )> say.

To obtain a closer approximation, write —8(p + 2q cos 2z) ITq (z) as a
sum of cosines, i.e. in the form

- 8 cos X-2)z+p cos Nz + q cos (N + 2) z] = Fj z), say.

Then, instead of solving -r-- + X'-u = V z), suppress the terms f in V
z)

which involve cos Nz; i.e. consider the function W z) wherej

If, ( )=F,( ) + 8; cos iY . A particular integral of

,+NHl=W, z)

18

11 = 9.

iro iT) ' ' ( - 2) + roTiT) '°' ' + 1 = ' ' ' ' •

Now express —S(p + 2q cos 2z) L\ (z) as a sum of cosines; calling
this sum Vo (z), choose a. to be such a function of p and q that V (z)
+ a . cos Nz contains no term in cos Nz; and let V., z) + a., cos Nz
= W. z).

cP u Solve the equation -r-j + N- u = Wo z),

and continue the process. Three sets of functions Um z), Vm z), Wm z)
are thus obtained, such that U,n z) and W,n z) contain no term in cos
Nz when m 0, and

W z) = F, z) + a, cos Nz, F, ( ) = - 8 p + 2q cos 2z) U,n-i (z),

 J + NL ( z) = W, z), where, is a function of p and q hut not of z.

* Journal fiir Math, lxxvi. (1873), pp. 214-224.

, d-u -, t The reason for this suppression is that the particular
integral of + A'- = cosA

contains non-periodic terms.

+ Unless N = \, in which case \ \ \ \ {z)-1\ \ {z) + 9 i) + q) co%z.

422 THE TRANSCENDENTAL FUNCTIONS [CHAP. XIX

It follows that

\ az ) j=o /=i

n—i / n \

= - 8 (;9 + 25 COS 2ir) S [7',rt\ i ( ) + S a, ) cos Nz.

Therefore, if U z)= S f/', ( ' be a uniformly convergent series of
analytic

m=0

functions throughout a two-dimensional region in the •-plane, we have

(§ 5-3)

d?U(z

—7-2 + (' + 9. cos 'Iz) U (z) = ylr (a, q) cos Nz,

oc

Avhere fr a,q)= 0 .

It is obvious that, if a be so chosen that yjf (a, q) = 0, then U z)
reduces to cey z).

A similar process can obviously be carried out for the functions 5e y
z, q) by making use of sines of multiples of z.

19*61. The convergence of the series defining Mathieu functions.

We shall now examine the expansion of § 19"6 more closely, with a view
to investigating the convergence of the series involved.

When n" 1, we may obviously write

/! n

U-a\ Z)= 2 */3,,.cos(iy-2r)£-|- 2 a rCOfi N->r'2.r)z, r=l r=l

the asterisk denoting that the first summation ceases at the greatest
value of r for which r N.

 (12 1

-jpi+ \ n+i (-) = an + 1 cos Nz -8 p + 2q cos 22) £/" (2),

it follows on equating coefficients of cos (iV + 2r) z on each side of
the equation t that

0.1 + 1 = !? ("h,1+3,i),

/•(>- + .y)a +,,, = 2 /?a,, + j(a, r-i + a,r + i) (''=1,2, ...),

These formulae hold universally with the following conventions % :

(i) Vo = 3 .o = ( = 1,2,...); a,. =, = (;•> ),

(ii) j, . j = iv-i li6i' - i* 'en and r=|i, (i") n.H-V+D ' n.H V-i)
en .V is odd and r=h N- ) . t When A'=0 or 1 these equations must be
modified by the suppression of all the coefficients * The conventions
(ii) and (iii) are due to the fact that cos2; = cos (-2), cos 22 = cos
(- 22).

19-61]

MATHIEU FUNCTIONS

423

The reader will easily obtain the following special formulae :

(I) a = 8p, (iV= l); ai = 8ip + q), (iV=l),

(III) a,y and n,r homogeneous polynomials of degree n in p and q.

we have >/ (a, j) = 8jt? + 8y (Ji+5,) (iV- D,

rir + ]V)A,=2ipAr+q(Ar-, + A, i)] (A),

r(r-.V) B, = 2 pB,+q B,\, + Br i) (B),

where Ao=B = 1 and B,. is subject to conventions due to (ii) and (iii)
above. Now write w,.= -q r r + ]V)-2p -\ '/= -q r (r-y)-2p -\ The
result of eliminating Ai, Ao, ... Aj.-, A + i, ... from the set of
equations (A) is

where A,, is the infinite determinant of von Koch's type (§ 2'82)

A,.= 1, il'r+u 0,0,....

, Wr+3, 1, W +3, ...

The determinant converges absolutely (§ 2"82 example) if no
denominator vanishes; and Ar-*-l as r- -cc (cf. § 19"52). If p and q
be given such values that Ao fcO, 2p r r- N), where r = l, 2, 3, ...,
the series

2 — yiVx%o.2...Wj.Ar (r cos(iV+2r) z

r=\

represents an integral function of z.

In like manner B,.D(,= -Y Wi tc ...w ' J),., where D,. is the finite
determinant

1, w'r + i,, ... J,

w'r + 2, 1, w'r + 2, ••• !

the last row being 0, 0, ... 0, 2w\ \, 1 or 0, 0, ... 0, <''i(jvr-i))
l + '''i(,v\ i) according as N is even or odd.

The series 2 Un (z) is therefore

?t=0

COS V + Ao" 2 (-)'"?<?i?i'2...?<'rArCOS(iV+2r)2 r = l

+ D- 1 2 ( - )'• iv; 10,' . . . w,' I), cos (iY- 2;-) z,

these series converging uniformly in any bounded domain of values of
z, so that term-by- term differentiations are permissible.

Further, the condition yj/ (, g') = is equivalent to

If we multiply by

pAo o-q w- Ai Do + Wi'Di Aq) = 0.

424 THE TRANSCENDENTAL FUNCTIONS [CHAP. XIX

the expression on the left becomes an integral function of both p and
q, (a, q), h-a\; the terms of (a, q\ which are of lowest degrees.in p
and q, are respectively p and

 ow expand — -. . ..o, o c — — 5 — - - ap

in ascending powers of q (cf. § 7"31), the contour being a small
cii'cle in the p-plane, with centre at the origin, and | q I being so
small that (iV + Sjo, q) has only one zero inside the contour. Then it
follows, just as in § 7"31, that, for sufficiently small values of \
q\, we may expand p as a power series in q commencing* with a term in
q; and if | q be sufficiently small D and Aq will not vanish, since
both are equal to 1 when =0.

On substituting for p in terms of q throughout the series for U (z),
we see that the series involved in cex (s, q) are absolutely
convergent when | | is sufficiently small.

The series involved in se (2, q) may obviously be investigated in a
similar manner.

19'7 The method of change of parameter .

The methods of Hill and of Lindemann-Stieltjes are effective in
determining, but only after elaborate analysis. Such analysis is
inevitable, as is by no means a simple function of q; this may be
seen by giving q an assigned real value and making a vary from — C30
to + 00; then /x alternates between real and complex values, the
changes taking place when, with the Hill-]\ Iathieu notation, A (0)
sin- [hir s, a) passes through the values and 1; the complicated
nature of this condition is due to the fact that A (0) is an elaborate
expression involving both a and q.

It is, however, possible to express fj. and a in terms of q and of a
new parameter tr, and

the results are very well adapted for purposes of numerical
computation when | 5' | is small J.

The introduction of the parameter a- is suggested by the series for
cei z, q) and sey z, q)

given in § 19"3 example 1; a consideration of the.se series leads us
to investigate the

potentialities of a solution of Mathieu's general equation in the form
y=e' ' 0(s), where

  if) = sin z-(r) + a-i cos (82 - o-) + 63 sin (82 - cr) + 05 cos hz -
cr) + 65 sin (52 - tr) + . . ., the parameter o- being rendered
definite by the fact that no term in cos z — a) is to appear in (j)
z); the special functions 5 1(2, q), cei z, q) are the cases of this
solution in which o- is or \ ir.

On substituting this expression in Mathieu's equation, the reader will
have no difficulty in obtaining the following approximations, valid
for § small viilues of q and real values of cr :

  =4 ' sin 2o-- 12 -" sin 2o-- 12j*sin 4o- + (2" ),

a = 1+ 8y cos 2(7 + ( - 16 + 8 cos 4o-) 2 \ i cos 2o- + ( f -- 88 cos
4o-) q + O q% a3=Sq sin 2a- + '3q sin 4:a + - sin 2a- + 9 sin 6a) q +
q% h =q + q cos 2o- + ( - J f + 5 cos 4(r) j + ( \ IJ. cos 2o- + 7 cos
60-) </* + q% a- = Y? sin 2o-4-|f ? sin 4o- + q% h=W + j q cos 2(7 + (
- Vr + f f cos Aa) q +0 q->), 7 = f'ifs 9* sin 2a + (q ), 67 = q + (/*
cos 2<t+0 iq% a, = 0 q% h, = l,,q + 0 q% the constants involved in the
various functions 0 q ) depending on a-.

* If A = l this result has to be modified, since there is an
additional term q on the right and the term q jiN - 1) does not
appear. .

t Wbittaker, Proc. Edinburgh Math. Soc. xxxn. (1914), pp. 75-80.

X They have been applied to Hill's problem by luce, Monthly Notices of
the R. A. S. lxxv. (1915), pp. 436-448.

§ The parameters q and a are to be regarded as fundamental in this
analysis, instead of a and q as hitherto.

197, 19 '8] MATHIEU FUNCTIONS 425

The domains of values of q and o- for which these series converge have
not yet been determined*.

If the sokition thus obtained be called A z, a, q), then A (z, cr, q)
and A z, — o", q) form a fundamental system of solutions of Mathieu's
general equation if /x= 0.

Example 1. Shew that, if o- = / x 0'5 and = 0-01, then

a = M24,841,4..., /x = ?x 0-046,993,5 ...; shew also that, if (r = i
and 2' = 0'01, then

rt = l-,321, 169,3..., / = ix 0-145,027,6.... Example 2. Obtain the
equations

/x = 4 sin 2a- — 4 ja3, rt — 1 + 83' cos 2(r — /it- — 8363,
expressing n and a in finite terms as functions of q, a, a and 63.
Example 3. Obtain the recurrence formulae -4:n n+l) +
8qcos2a--8qb3±8qi 2n+l) as-sin2(r) z.2n + i + 8q z2 \ i+Z2 3) = 0,

where 22jh-i denotes bon+i + ict-in + i oi" -in + i - i 2)i + i>
according as the upper or lower sign is taken.

19"8. The asymptotic solution of MathiexCs equation.

If in Mathieu's equation

d' v. / 1,., \

-5— + a + - A,- cos 22 I M =

dz- \ 2 J

we write k sin 2 =, we get

where i/ = + P.

This equation has an irregular singularity at infinity. From its
resemblance to Bessel's equation, we are led to write u — e' |~- v,
and substitute

V=l+ a,/ ) + a,!e) + ... in the resulting equation for v; we then
find that

ai = - i (i - 3P + F), a., = - Hi - + •') (f - - H F) + IF, the
general coefficient being given by the recurrence formula

2i(r+l)a, + i = J-J/2 + F + /-(r+l) + (2?--l)zFa \ l-(/-2-2/- + |)Fa,\
2. The two series

e'U~-(l+ + jl+-], e-''r'(l- +

|,p,...), .-.,-.,. . \

are formal solutions of Mathieu's equation, reducing to the well-known
asymptotic solutions of Bessel's equation (§ 17-5) when -- 0. The
complete formulae which connect them with the solutions e ' (f)(±z)
have not yet been published, though some steps towards obtaining them
have been made by Dougall, F7'oc. Edinburgh Math. Soc. xxxiv. (1916),
pp. 176-196.

* It seems highly probable that, if | g | is sufficiently small, the
series converge for all real values of a, and also for complex values
of cr for which |I((r) | is sufficiently small. It may be noticed
that, when q is real, real and purely imaginary values of cr
correspond respectively to real and purely imaginary values of fi.

426 THE TRANSCEXDEXTAL FUNXTIONS [CHAP. XIX

KEFEREXCES.

E. L. Mathiec, Journal de Math. (2), xiii. (1868), pp. 137-203.

G. W. Hill, Acta Mathematica, viii. (1886), pp. 1-36.

G. Floquet, Ann. de VEcole norm. sup. (2), xii. (1883), pp. 47-88.

C. L. F. LiNDEMANN, Mcith. Ann. xxii. (1883), pp. 117-123.

T. J. Stieltjes, Astr. Naeh. cix. (1884), cols. 145-152, 261-266.

A. LiXDSTEDT, Astr. Nach. cm. (1882), cols. 211-220, 257-268; Civ.
(1883), cols. 145-150; cv. (1883), cols. 97-112.

H. Bruns, Astr. Xach. cvi. (1883), cols. 193-204; cvii. (1884), cols.
129-132.

R. C. Maclaurin, Trans. Camb. Phil. Soc. xvii. (1899), pp. 41-108.

K. AiCHi, Proc. Tokyo Math, and Phys. Soc. (2), iv. (1908), pp.
266-278.

E. T. Whittaker, Proc. International Congress of Mathematicians,
Cambridge, 1912, I. pp. 366-371.

E. T. Whittaker, Proc. Edinburgh Math. Soc. xxxii. (1914), pp. 75-80.

G. N, Watson, Proc. Edinburgh Math. Soc. xxxiii. (1915), pp. 25-30.

A. W. Young, Proc. Edinburgh Math. Soc. xxxii. (1914), pp. 81-90.

E. Lindsay Inge, Proc. Edinburgh Math. Soc. xxxiii. (1915), pp. 2-15.

J. Dougall, Proc. Edinburgh Math. Soc. xxxiv. (1916), pp. 176-196.

Miscellaneous Examples.

1. Shew that, if k= l Z \

2wce(, s, q) = cco 0,q) j cos k sin z sin 6) ccq (0, q) d6. J —It

2. Shew that the even Mathieu functions satisfy the integral equation

G (2)=X j Jo [ik (cos z + cos 6) G 6) d6.

3. Shew that the equation

(a22 + c) +2a5 + (X%2+ 0 =

(where o, c, X, m are constants) is satisfied by

u = \ \ v s)ds tAken round an appropriate contour, provided that v (s)
satisfies

 as + c) + 2as - X cs - + m)v 3) = 0,

which is the same as the equation for u.

Derive the integral equations satisfied by the Mathieu functions as
particular cases of this result.

MATHIEU FUNCTIONS 427

4. Shew that, if powers of q above the fourth are neglected, then

ce?! (s, q) = cos 2 + J cos 32 + q (J cos hz — cos 3s)

+ ( (i 8 cos 7s — f cos 5s + J cos 3s)

+ ?* (rl < ~ I's 0* ''' + H cos 5s + cos 3s),

sei (s, j) = sin s + g sin 3s + §'2 (i sin 5s + sin 3s)

+ <f ( jij sin 7s + f sin 5s + A sin 3s)

+ q (yIcj sin 9s + Jj sin 7s + sin 5s - - sin 3s),

C(?2 (s, g-) = cos 2s + g- (cos 4s — 2) + g cog g

+ ? (-/s cos 8s + If cos 4s + -* )

+ ?* (sTO cos 10s + f|§ cos 6s).

(Mathieu.)

5. Shew that

663(0, 2') = cos 32 + 2'( — coss+l cos5s)

+ j2 (cos s + J(5 COS 7s) + j3 ( \ I COS s + - COS 5s + Jq cos 9s) +
q*),

and that, in the case of this function

a = 9 + 4q -8q +0 q ).

(Mathieu.)

6. Shew that, if 1/ (s) be a Mathieu function, then a second sokition
of the corresponding differential equation is

Shew that a second solution * of the equation for ce (s, q) is zceQ
(z, q) — 4:qsiB 2s- Sq- sin 4s- ....

7. If ?/ (2) be a solution of Mathieu's general equation, shew that

 y(s + 2 )+3/(s-2:r) /j/(2) is constant.

8. Express the Mathieu functions as series of Bessel functions in
which the coefficients are multiples of the coefficients in the
Fourier series for the Mathieu functions.

[Substitute the Fourier series under the integral sign in the integral
equations of § 19-22.]

9. Shew that the confluent form of the equations for ce (s, q) and se
(2, q), when the eccentricity of the fundamental ellipse tends to
zero, is, in each case, the equation satisfied by J,, ii- eoH z).

10. Obtain the parabolic cylinder functions of Chapter xvi as
confluent forms of the Mathieu functions, by making the eccentricity
of the fundamental ellipse tend to unity.

11. Shew that ce (s, q) can be expanded in series of the form

2 J,cos2'"3 or 2 5,,,cos2' + i2,

 n=0 m=0

according as % is even or odd; and that these series converge when j
coss [ < 1.

* This solution is called in (z, q); the second sohitions of the
equations satisfied by Mathieu functions have been investigated by
Ince, Proc. Edinburgh Math. Soc. xxxiii. (1915), pp. 2-15. See also §
19-2.

428 THE TRANSCENDENTAL FUNCTIONS [cHAP. XIX

12. With the notation of example 11, shew that, if

ce z, q) = X I e*cos3oo60 e,, ((9, q) dd, then A is given by one or
other of the series

provided that these series converge.

13. Shew that the differential equation satisfied by the product of
any two solutions of Bessel's equation for functions of order n is

S S-2n) S+21l)u + -I. S + l) u = 0,

where 3 denotes z -j- . dz

Shew that one solution of this equation is an integral function of 2;
and thence, by the methods of 5; \$ 19'5-19'53, obtain the Bessel
functions, discussing particularly the case in which a is an integer.

14. Shew that an approximate solution of the equation

- -ir(A+k-sm\ i-z)u=Q dz

is \ i = C (cosech s) - sin k cosh z + e),

where C and e are constants of integration; it is to be assumed that
k is large, A is not very large and z is not small.

