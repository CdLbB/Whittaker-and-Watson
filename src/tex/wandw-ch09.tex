%
% 160
%
\chapter{Fourier Series and Trigonometrical Series}
\Section{9}{1}{Definition of Fourier series}\footnote{Throughout
  this chapter (except in \hardsubsectionref{9}{1}{1}) it is supposed that all
  the numbers involved are real.}.
Series of the type
\begin{align*}
  \half a_{0}
  + (a_{1} + \cos x + b_{1} \sin x)
  + (a_{2} + \cos 2x + b_{2} \sin 2x)
  + \cdots
  \\
  &
  \hfill
  \half a_{0}
  +
  \sum_{n=1}^{\infty} ( a_{n} \cos n x + b_{n} \sin n x),
\end{align*}
where $a_{n}$, $b_{n}$ are independent of $x$, are of great importance in many
investigations. They are called \emph{trigonometrical series}.

If there is a function $f(t)$ such that
$\int_{-\pi}^{\pi} f(t) \dmeasure t$ exists as a Riemann
integral or as an improper integral which converges absolutely, and such that
$$
\pi a_{n} = \int_{-\pi}^{\pi} f(t) \cos nt \dmeasure t,
\quad
\pi b_{n} = \int_{-\pi}^{\pi} f(t) \sin nt \dmeasure t,
$$
then the trigonometrical series is called a \emph{Fourier series}.

%\begin{smalltext}
Trigonometrical series first appeared in analysis in connexion with
the investigations of Daniel Bernoulli on vibrating
strings\index{Strings, vibrations of}\index{Vibrations of!strings};
d'Alembert had previously solved the equation of
motion
$ \ddot{y} = a^{2} \frac{\dd^{2} y}{\dd x^{2}}$
in the form
$y = \half \thebrace{f(x+at) + f(x-at)}$, where $y=f(x)$ is
the initial shape of the string starting from rest;
and Bernoulli shewed that a formal solution is
$$
y
=
\sum_{n=1}^{\infty}
b_{n}
\sin \frac{n \pi x}{l}
\cos \frac{n \pi a t}{l},
$$
the fixed ends of the string being $(0,0)$ and $(l,0)$; and he asserted
that this was the most general solution of the problem. This appeared
to d'Alembert and Euler to be impossible, since such a series, having
period $2l$, could not possibly represent such a function
as\footnote{This function gives a simple form to the initial shape of the string.}
$c x (l-x)$ when $t = 0$.
A controversy arose between these mathematicians, of which
an account is given in Hobson's \emph{Functions of a Real Variable}.

Fourier, in his \emph{TODO Theorie de la Chaleur}, investigated a number of
trigonometrical series and shewed that, in a large number of
particular cases, a Fourier series \emph{actually converged to the sum $f(x)$}.
Poisson attempted a general proof of this theorem. TODO Journal de VEcole
poly technique, xil. (1823), pp. 404-509. Two proofs were given by
Cauchy, TODO Men. de VAcad. R. des Sci. vi. (1823, publi-hed 1826), pp.
603-612 Oeuvres, (1), n. pp. 12-19) and Exercices de Math. 11. (1827),
pp. 341-376 (Oeuvres, (2), vil. pp. 393-430); these proofs, which are
based on the theory of contour integration, are concerned with rather
particular classes of functions and one is invalid. The second proof
has been investigated by Harnack, TODO Math. Ann. xxxii. (1888), pp.
175-202.

%\end{smalltext}
%
% 161
%

In 1829, Dirichlet gave the first rigorous proof\footnote{TODO Journal J iir Math. iv. (1829), pp. 157-169.}
that, for a general class of functions, the Fourier series, defined as above, does
converge to the sum $f(x)$. A modification of this proof was given later
by Bonnet\footnote{TODO Meiuoires des Savants etramjers of the Belgian Academy, xxiii.
(1848-1850). Bonnet employs the second mean value theorem directly,
while Dirichlet's original proof makes use of arguments precisely
similar to those by which that theorem is proved. See \hardsubsectionref{9}{4}{3}.}.

The result of Dirichlet\index{Statement of Fourier's theorem, \Dirichlet's}
is that\footnote{The conditions postulated for $f(t)$
  are known as \emph{Dirichlet's conditions}; as will be seen in TODO §§ 9-2, 9 42, they are unnecessarily
  stringent.}
if $f(t)$ is defined and bounded in the
range $(-\pi, \pi)$ and if $f(t)$ has only a finite number of maxima and
minima and a finite number of discontinuities in this range and,
further, if $f(t)$ is defined by the equation
$$
f(t + 2 \pi) = f(t)
$$
outside the range $(-\pi, \pi)$, then, provided that
$$
\pi a_{n}
=
\int_{-\pi}^{\pi} f(t) \cos nt \dmeasure t,
\quad
\pi b_{n}
=
\int_{-\pi}^{\pi} f(t) \sin nt \dmeasure t,
$$
the series $TODO$ converges to the sum /( +
0)+/( - 0) .

Later, Riemann and Cantor developed the theory of trigonometrical
series generally, while still more recently Hurwitz, \Fejer\ and others
have investigated properties of Fourier series when the series does
not necessarily converge. Thus \Fejer has proved the remarkable
theorem that a Fourier series (even if not convergent) is 'summable
$(C1)$' at all points at which $f(x \pm 0)$ exist, and its sum $(C1)$ is
$\half \thebrace{ f(x+0) + f(x-0) }$,
provided that $\int_{-\pi}^{\pi} f(t) \dmeasure t$ is an absolutely convergent integral.
One of the investigations of the convergence of Fourier series which we shall give later
(\hardsectionref{9}{4}{2}) is based on this result.

For a fuller account of investigations subsequent to Riemann, the
reader is referred to Hobson's \emph{Functions of a Real Variable}, and to
TODO de la Vallee Poussin's Cours dWnalyse Infinitesimale.

\Subsection{9}{1}{1}{Nature of the region within which a
  trigonometrical series converges.}

Consider the series
$$
\half a_{0}
+
\sum_{n=1}^{\infty}
\theparen{
  a_{n} \cos nz
  +
  b_{n} \sin nz
},
$$
where $z$ may be complex. If we write
$e^{iz} = \zeta$, the series becomes
$$
\half a_{0}
+
\sum_{n=1}^{\infty}
\thebrace{
  \half (a_{n} - i b_{n}) \zeta^{n}
  +
  \half (a_{n} + i b_{n}) \zeta^{-n}
}
$$
This Laurent series will converge, if it converges at all, in a region
in which $a \leq \absval{\zeta} \leq b$, where $a,b$ are positive
constants.

But, if $z = x + iy$, $\absval{\zeta} = e^{-y}$, and so we get, as the
region of convergence of the trigonometrical series, the strip in the
$z$ plane defined by the inequality
$$
\log a \leq -y \leq \log b.
$$

The case which is of the greatest importance in practice is that in which
$a = b = 1$, and the strip consists of a single line, namely the real axis.

TODO Example 1
Let
$$
f(z)
=
\sin z
- \half \sin 2z
+ \frac{1}{3} \sin 3z
- \frac{1}{4} \sin 4z
+ \ldots,
$$
where $z = x + iy$.
%
% 161
%

Writing this in the form
$$
f(z)
=
-
\half
i
\theparen{
  e^{iz}
  - \half e^{2iz}
  + \frac{1}{3} e^{3iz}
  -
  \cdots
}
+
\half
i
\theparen{
  e^{-iz}
  - \half e^{-2iz}
  + \frac{1}{3} e^{-3iz}
  -
  \cdots
}
$$
we notice that the first series converges\footnote{The series
  \emph{do converge} if $y=0$, see \hardsubsectionref{2}{3}{1} example TODO 2}
only if
$y \geq 0$, and the second only if $y \leq 0$.

Writing $x$ in place of $z$ ($x$ being real), we see that by Abel's
theorem (\hardsubsectionref{3}{7}{1}),
\begin{align*}
  f(x)
  &=
  \lim_{r \rightarrow 1} \theparen{
    r \sin x
    - \half r^{2} \sin 2x
    + \frac{1}{3} r^{3} \sin 3x
    - \cdots
  }
  \\
  &=
  \lim_{r \rightarrow 1} \thebrace{
    - \half \theparen{
      r e^{ix}
      - \half r^{2} e^{2ix}
      + \frac{1}{3} r^{3} e^{3ix}
      - \cdots
    }
    +
    \half i \theparen{
      r e^{-ix}
      - \half r^{2} e^{-2ix}
      + \frac{1}{3} r^{3} e^{-3ix}
      - \cdots
    }
  }
\end{align*}

This is the limit of one of the values of
$$
- \half i \log (1 + r e^{ix})
+ \half i \log (1 + r e^{-ix}),
$$
and as $r \rightarrow 1$ (if $-\pi < x < \pi$), this tends to
$\half x + k\pi$, where $k$ is some integer.

Now $\sum_{n=1}^{\infty} \frac{(-)^{n-1} \sin nx}{n}$ converges uniformly
(\hardsubsectionref{3}{3}{5} example TODO 1) and is therefore continuous in
the range $-\pi + \delta \leq x \leq \pi - \delta$, where
$\delta$ is any positive constant.

Since $\half x$ is continuous, $k$ has the same value wherever $x$ lies in the
range; and putting $x=0$, we see that $k=0$.

\emph{Therefore, when $-\pi < x < \pi$,
  $$
  f(x) = \half x.
  $$
}

But, when $\pi < x < 3\pi$,
$$
f(x)
=
f(x - 2\pi)
=
\half (x - 2\pi)
=
\half x - \pi,
$$
and generally, if $(2n - 1) \pi < x < (2n + 1) \pi$,
$$
f(x) = \half x - n \pi.
$$

We have thus arrived at an example in which $f(x)$ is
not represented by a single analytical expression.

It must be observed that this phenomenon can only occur when the
strip in which the Fourier series converges is a single line.
For if the strip is not of zero breadth, the associated Laurent
series converges in an annulus of non-zero breadth and represents an
analytic function of $\zeta$ in that annulus; and, since
$\zeta$ is an analytic function of $z$, the Fourier series
represents an analytic function of $z$; such a series is given by
$$
r \sin x
- \half r^{2} \sin 2x
+ \frac{1}{3} r^{3} \sin 3x
- \cdots,
$$
where $0 < r < 1$; its sum is
$\arctan \frac{r \sin x}{1 + r \cos x}$, the $\arctan$ always
representing an angle between $\pm \half \pi$.

Example TODO
When $-\pi \leq x \leq \pi$,
$$
\sum_{n=1}^{\infty}
\frac{(-)^{n-1} \cos nx}{n^{2}}
=
\frac{1}{12} \pi^{2}
-
\frac{1}{4} x^{2}.
$$

The series converges only when $x$ is real; by
\hardsubsectionref{3}{3}{4} the convergence is then
absolute and uniform.

Since
$$
\half x
=
\sin x
- \half \sin 2x
+ \frac{1}{3} \sin 3x
- \cdots
\quad
(-\pi + \delta \leq x \pi - \delta,
\delta > 0),
$$
and this series converges uniformly, we may integrate
term-by-term from $0$ to $x$ (\hardsectionref{4}{7}),
and consequently
$$
\frac{1}{4} x^{2}
=
\sum_{n=1}^{\infty}
\frac{(-)^{n-1} (1 - \cos nx)}{n^{2}}
\quad
(-\pi + \delta \leq x \leq \pi - \delta).
$$
%
% 163
%

That is to say, when $-\pi + \delta \leq x \leq \pi - \delta$,
$$
C - \frac{1}{4} x^{2}
=
\sum_{n=1}^{\infty} \frac{(-)^{n-1} \cos nx}{n^{2}},
$$
where $C$ is a constant, at present undetermined.

But since the series on the right converges uniformly throughout the
range $-\pi \leq x \leq \pi$, its sum is a continuous function of $x$ in this
extended range; and so, proceeding to the limit when
$x \rightarrow \pm \pi$, we see
that the last equation is still true when $x = \pm \pi$.

To determine $C$, integrate each side of the equation \hardsectionref{4}{7} between
the limits $-\pi, \pi$; and we get
$$
2 \pi C - \frac{1}{6} \pi^{3} = 0.
$$

Consequently
$$
\frac{1}{12} \pi^{2} - \frac{1}{4} x^{2}
=
\sum_{n=1}^{\infty}
\frac{ (-)^{n-1} \cos nx }{ n^{2} }
\quad
(-\pi \leq x \leq \pi).
$$

Example TODO.
By writing $\pi - 2x$ for $x$ in example TODO2, shew that
$$
\sum_{n=1}^{\infty} \frac{\sin^{2} nx}{n^{2}}
=
\begin{cases}
  \half x (\pi - x)                         & 0 \leq x \leq \pi,   \\
  \half \thebrace{ \pi \absval{x} - x^{2}} & -\pi \leq x \leq \pi.
\end{cases}
$$

\Subsection{9}{1}{2}{Values of the coefficients in terms of the sum of a
trigonometrical series.}
Let the trigonometrical series
$
\half c_{0}
+
\sum_{n=1}^{\infty} (c_{n} \cos nx + d_{n} \sin nx)
$
be uniformly convergent in the range $(-\pi, \pi)$ and let its sum be $f(x)$.
Using the obvious results
\begin{align*}
  \int_{-\pi}^{\pi} \cos mx \cos nx \dmeasure x
  =&
  \begin{cases}
    0 & m \neq n \\
    \pi & m = n \neq 0,
  \end{cases}
  \\
  \int_{-\pi}^{\pi} \sin mx \sin nx \dmeasure x
  =&
  \begin{cases}
    0 & m \neq n \\
    \pi & m = n \neq 0,
  \end{cases}
  \\
  \int_{-\pi}^{\pi} \dmeasure x
  =& 2\pi,
\end{align*}
we find, on multiplying the equation
$
\half c_{0}
+
\sum_{n=1}^{\infty}
(c_{n} \cos nx + d_{n} \sin nx)
= f(x)
$
by\footnote{Multiplying by these factors does not destroy the uniformity of the
  convergence.}
$\cos nx$ or by $\sin nx$ and integrating
term-by-term\footnote{These were given by TODO Euler (with limits $0$ and $2\pi$),
  Nova Acta Acad. Petrop. xi. (1793).}
(\hardsectionref{4}{7}),
$$
\pi c_{n} = \int_{-\pi}^{\pi} f(x) \cos nx \dmeasure x,
\quad
\pi d_{n} = \int_{-\pi}^{\pi} f(x) \sin nx \dmeasure x.
$$

TODO Corollary. A trigonometrical series uniformly convergent in the range
$(-\pi, \pi)$ is a Fourier series.

TODO Note. Lebesgue has given a proof (TODO Series trigonometriques, p. 124) of a
theorem communicated to him by Fatou that the trigonometrical series
$\sum_{n=2}^{\infty} \sin nx / \log n$, which converges for all real values of $x$
(\hardsubsectionref{2}{3}{1} example TODO), is \emph{not} a Fourier series.

\Section{9}{2}{On Dirichlet's conditions and Fourier's theorem.
\index{Statement of Fourier's theorem, \Dirichlet's}}
A theorem, of the type described in \hardsectionref{9}{1}, concerning the
expansibility of a function of a real variable into a trigonometrical
series is usually described
%
% 164
%
as \emph{Fourier's theorem}. On account of the length and difficulty of a
formal proof of the theorem (even when the function to be expanded is
subjected to unnecessarily stringent conditions), we defer the proof
until TODO §§ \hardsubsectionref{9}{4}{2}, \hardsubsectionref{9}{4}{3}.
It is, however, convenient to state here certain
\emph{sufficient} conditions under which a function can be expanded into a
trigonometrical series.

\emph{Let $f(t)$ he defined arbitrarily when $-\pi \leq t \leq \pi$
  and defined\footnote{This definition frequently results in $f(t)$ not being
    expressible by a single analytical expression for all real values of $t$.
    Cf.\hardsubsectionref{9}{1}{1} example TODO:1.}
  for all other real values of $t$ by means of the equation
  $$
  f(t + 2\pi) = f(t),
  $$
  so that $f(t)$ is a periodic function with period $2\pi$.
}
\emph{
  Let $f(t)$ be such that
  $\int_{-\pi}^{\pi} f(t) \dmeasure t$ exists; and if this is an improper
  integral, let it be absolutely convergent.
}

\emph{
  Let $a_{n}, b_{n}$ be defined by the
  equations\footnote{The numbers $a_{n}, b_{n}$ are called
    \emph{the Fourier constants\index{Fourier constants}} of
    $f(t)$, and the symbols $a_{n}, b_{n}$ will be used in this sense throughout
    §§ TODO \hardsectionref{9}{2}--\hardsectionref{9}{5}.
    It may be shewn that the convergence and absolute convergence of the
    integrals defining the Fourier constants are consequences of the
    convergence and absolute convergence of
    $\int_{-\pi}^{\pi} f(t) \dmeasure t$.
    Cf. §§ TODO \hardsubsectionref{2}{3}{2}, \hardsectionref{4}{5}.}
  $$
  \pi a_{n} = \int_{-\pi}^{\pi} f(t) \cos nt \dmeasure t,
  \quad
  \pi b_{n} = \int_{-\pi}^{\pi} f(t) \sin nt \dmeasure t
  \quad
  (n=0,1,2,\ldots).
  $$
}

\emph{
  Then, if $x$ be an interior point of any interval $(a, b)$ in which
  $f(t)$ has limited total fluctuation, the series
  $$
  \half a_{0}
  +
  \sum_{n=1}^{\infty} (a_{n} \cos nx + b_{n} \sin nx)
  $$
  is convergent, and its sum\footnote{The limits $f(x \pm 0)$ exist,
    by \hardsubsectionref{3}{6}{4} example TODO:3.}
  is $\half \thebrace{ f(x+0) + f(x-0) }$.
  If $f(t)$ is continuous at $t=x$, this sum reduces to $f(x)$.
}

This theorem will be assumed in
TODO \hardsubsectionref{9}{2}{1}--\hardsubsectionref{9}{3}{2};
these sections deal with theorems concerning Fourier series which
are of some importance
in practical applications. It should be stated here that every
function which Applied Mathematicians need to expand into Fourier
series satisfies the conditions just imposed on $f(t)$, so that the
analysis given later in this chapter establishes the validity of all
the expansions into Fourier series which are required in physical
investigations.

The reader will observe that in the theorem just stated,
$f(t)$ is subject to less stringent conditions than those contemplated by
Dirichlet, and this decrease of stringency is of
considerable practical importance. Thus, so simple a series as
$\sum_{n=1}^{\infty} (-)^{n-1} (\cos nx) / n$
is the
expansion of the function\footnote{Cf. example TODO:6 at the end of the chapter (p. TODO:190).}
$\log \absval{2 \cos \half x}$; and this function
does not satisfy Dirichlet's condition of boundedness at $\pm \pi$.

It is convenient to describe the series
$\half a_{0} + \sum_{n=1}^{\infty} (a_{n} \cos nx + b_{n} \sin nx)$
as \emph{the Fourier series associated with $f(t)$}. This description must,
however, be
%
% 165
%
taken as implying nothing concerning the convergence of the series in
question.

\Subsection{9}{2}{1}{The representation of a function by Fourier series for ranges
other than $(-\pi,\pi)$.}

Consider a function $f(x)$ with an (absolutely) convergent integral,
and with limited total fluctuation in the range $a \leq x \leq b$.

Write
$x = \half (a + b) - \half (a-b) \pi^{-1} x',
\quad
f(x) = F(x')$.

Then it is known (\hardsectionref{9}{2}) that
$$
\half [F(x'+0) + F(x'-0)]
=
\half a_{0} + \sum_{n=1}^{\infty} (a_{n} \cos nx' + b_{n} \sin nx'),
$$
and so
\begin{align*}
  & \half \thebrace{ f(x+0) + f(x-0)}
  \\
  &
  \hfill
  =
  \half a_{0}
  +
  \sum_{n=1}^{\infty}
  \thebrace{
    a_{n} \cos \frac{n \pi (2x-a-b)}{b-a}
    +
    b_{n} \sin \frac{n \pi (2x-a-b)}{b-a}
  },
\end{align*}
where by an obvious transformation
\begin{align*}
  \half (b-a) a_{n} =& \int_{a}^{b} f(x) \cos \frac{n \pi (2x-a-b)}{b-a} \dmeasure x,
  \\
  \half (b-a) b_{n} =& \int_{a}^{b} f(x) \sin \frac{n \pi (2x-a-b)}{b-a} \dmeasure x
  .
\end{align*}
\Subsection{9}{2}{2}{The cosine series and the sine series.}
Let $f(x)$ be defined in the range $(0,l)$ and let it have an
(absolutely) convergent integral and also let it have limited
total fluctuation in that range.
\emph{Define} $f(x)$ in the range
$\wandwtypo{(0,-l)}{(-l,0)}$
by the equation
$$
f(-x) = f(x).\index{Even functions}
$$

Then
$$
\half \thebrace{ f(x+0) + f(x-0) }
=
\half a_{0}
+
\sum_{n=1}^{\infty} \thebrace{
  a_{n} \cos \frac{n \pi x}{l}
  +
  b_{n} \sin \frac{n \pi x}{l}
},
$$
where, by \hardsubsectionref{9}{2}{1},
\begin{align*}
  l a_{n}
  =&
  \int_{-l}^{l} f(t) \cos \frac{n \pi t}{l} \dmeasure t
  =
  2 \int_{0}^{l} f(t) \cos \frac{n \pi t}{l} \dmeasure t,
  \\
  l b_{n}
  =&
  \int_{-l}^{l} f(t) \sin \frac{n \pi t}{l} \dmeasure t
  = 0,
\end{align*}
so that when $-l \leq x \leq l$,
$$
\half \thebrace{ f(x+0) + f(x-0) }
=
\half a_{0} + \sum_{n=1}^{\infty} a_{n} \cos \frac{n \pi x}{l};
$$
this is called the \emph{cosine series}.

If, however, we define $f(x)$ in the range $(0,-l)$ by the equation
$$
f(-x) = -f(\wandwtypo{-}{}x),\index{Odd functions}
$$
%
% 166
%
we get, when $-l \leq x \leq l$,
$$
\half \thebrace{ f(x+0) + f(x-0) }
=
\sum_{n=1}^{\infty} b_{n} \sin \frac{n \pi x}{l},
$$
where
$$
l b_{n}
=
2 \int_{0}^{l} f(t) \sin \frac{n \pi t}{l} \dmeasure t;
$$
this is called the \emph{sine series}.

Thus the series
$$
\half a_{0}
+
\sum_{n=1}^{\infty} a_{n} \cos \frac{n \pi x}{l},
\quad
\sum_{n=1}^{\infty} b_{n} \sin \frac{n \pi x}{l},
$$
where
$$
\half l a_{n}
=
\int_{0}^{l} f(t) \cos \frac{n \pi t}{l} \dmeasure t,
\quad
\half l b_{n}
=
\int_{0}^{l} f(t) \sin \frac{n \pi t}{l} \dmeasure t,
$$
\emph{have the same sum when $0 \leq x \leq l$;}
but their sums are numerically
equal and opposite in sign when $0 \geq x \geq -l$.

%\begin{smalltext}
The cosine series was given by Clairaut, TODO Hist, de I'Acad. R. des Sci.
1754 [published, 1759], in a memoir dated July 9, 1757; the sine
series was obtained between 1762 and 1765 by Lagrange, Oeuvres, l. p.
553.
%\end{smalltext}

TODO Example 1. Expand $\half (\pi - x) \sin x$ in a cosine series in the range
$0 \leq x \leq \pi$.
[We have, by the formula just obtained,
$$
\half (\pi - x) \sin x
=
\half a_{0}
+ \sum_{n=1}^{\infty} a_{n} \cos nx,
$$
where
$$
\half \pi a_{n}
=
\int_{0}^{\pi} \half (\pi - x) \sin x \cos nx \dmeasure x.
$$

But, integrating by parts, if $n \neq 1$,
\begin{align*}
  &
  \int_{0}^{\pi} 2 (\pi - x) \sin x \cos nx \dmeasure x
  \\
  & \quad
  \int_{0}^{\pi} (\pi - x) \thebrace{
    \sin (n+1) x - \sin (n-1) x
  } \dmeasure x
  \\
  & \quad
  \thebracket{
    (x - \pi)
    \thebrace{
      \frac{ \cos (n+1) x }{n+1}
      -
      \frac{ \cos (n-1) x }{n-1}
    }
  }_{0}^{\pi}
  -
  \int_{0}^{\pi}
  \thebrace{
    \frac{ \cos (n+1) x }{n+1}
    -
    \frac{ \cos (n-1) x }{n-1}
  }
  \dmeasure x
  \\
  & \quad
  \pi
  \theparen{
    \frac{1}{n+1}
    -
    \frac{1}{n-1}
  }
  =
  \frac{-2\pi}{(n+1)(n-1)}
\end{align*}
Whereas if $n=1$, we get
$\int_{0}^{\pi} 2 (\pi - x) \sin x \cos x \dmeasure x = \half \pi$.

Therefore the required series is
$$
\half
+ \frac{1}{4} \cos x
- \frac{1}{1 \cdot 3} \cos 2x
- \frac{1}{2 \cdot 4} \cos 3x
- \frac{1}{3 \cdot 5} \cos 4x
- \cdots.
$$

It will be observed that it is only for values of $x$ between
$0$ and $\pi$ that the sum of this series is proved to be
$\half (\pi - x) \sin x$; thus for
instance when $x$ has a value between $0$ and $-\pi$,
the sum of the series is not
$\half (\pi - x) \sin x$, but $-\half (\pi + x) \sin x$; when $x$ has a value
between $\pi$ and $2 \pi$, the sum of the series happens to be again
$\half (\pi - x) \sin x$, but this is a mere coincidence arising from the special
function considered, and does not follow from the general theorem.]

TODO Example 2. Expand $\frac{1}{8} \pi x (\pi - x)$ in a sine series,
valid when $0 \leq x \leq \pi$.

[The series is $\sin x + \frac{\sin 3x}{3^{3}} + \frac{\sin 5x}{5^{3}} + \cdots.$]
%
% 167
%

TOOD Example 3. Shew that, when $0 \leq x \leq \pi$,
$$
\frac{1}{96}
\pi
(\pi - 2x)
(\pi^{2} + 2 \pi x - 2 x^{2})
=
\cos x
+ \frac{\cos 3x}{3^{4}}
+ \frac{\cos 5x}{5^{4}}
+ \cdots.
$$

[Denoting the left-hand side by $f(x)$, we have, on integrating by
parts and observing that $f'(0) = f'(\pi) = 0$,
\begin{align*}
  \int_{0}^{\pi} f(x) \cos nx \dmeasure x
  =&
  \frac{1}{n} \thebracket{f(x) \sin nx}_{0}^{\pi}
  -
  \frac{1}{n} \int_{0}^{\pi} f'(x) \sin nx \dmeasure x
  \\
  =&
  \frac{1}{n^{2}} \thebracket{f'(x) \cos nx}_{0}^{\pi}
  -
  \frac{1}{n^{2}} \int_{0}^{\pi} f''(x) \cos nx \dmeasure x
  \\
  =&
  -\frac{1}{n^{3}} \thebracket{f''(x) \sin nx}_{0}^{\pi}
  +
  \frac{1}{n^{3}} \int_{0}^{\pi} f'''(x) \sin nx \dmeasure x
  =&
  -\frac{1}{n^{4}} \thebracket{f'''(x) \cos nx}_{0}^{\pi}
  =
  \frac{\pi}{4 n^{4}} (1 - \cos n \pi).]
\end{align*}
TODO Example 4. Shew that for values of $x$ between $0$ and $\pi$,
$e^{s x}$ can be expanded in the cosine series
$$
\frac{2 s}{\pi}
\theparen{e^{s \pi} - 1}
\theparen{
  \frac{1}{2 s^{2}}
  + \frac{\cos 2x}{s^{2} + 4}
  + \frac{\cos 4x}{s^{2} + 16}
  + \cdots
}
-
\frac{2 s}{\pi}
\theparen{e^{s \pi} - 1}
\theparen{
  \frac{\cos x}{s^{2} + 1}
  + \frac{\cos 3x}{s^{2} + 9}
  + \cdots
},
$$
and draw graphs of the function $e^{s x}$ and of the sum of the series.

TODO Example 5. Shew that for values of $x$ between $0$ and $\pi$,
the function $\frac{1}{8} \pi (\pi - 2x)$ can
be expanded in the cosine series
$$
\cos x
+ \frac{\cos 3x}{3^{2}}
+ \frac{\cos 5x}{5^{2}}
+ \cdots,
$$
and draw graphs of the function $\frac{1}{8} \pi (\pi - 2x)$ and of the sum of the
series.

\Section{9}{3}{The nature of the coefficients in a Fourier series.}\footnote{TODO The analysis of this section and of \hardsubsectionref{9}{3}{1} is contained in Stokes'
great memoir, Camb. Phil. Tratis. VIII. (1849), pp. 538-583 [Math.
Papers, i. pp. 236-313].}

Suppose that (as in the numerical examples which have been discussed)
the interval $(-\pi, \pi)$ can bo divided into a finite number of ranges
$(-\pi, k_{1}), (k_{1}, k_{2}), \ldots, (k_{n}, \pi)$
such that throughout each range $f(x)$
and all its differential coefficients are continuous with limited
total fluctuation and that they have limits on the right and on the
left (\hardsectionref{3}{2}) at the end points of these ranges.

Then
$$
\pi a_{m}
=
\int_{-\pi}^{k_{1}} f(t) \cos mt \dmeasure t
+ \int_{k_{1}}^{k_{2}} f(t) \cos mt \dmeasure t
+ \cdots
+ \int_{k_{n}}^{\pi} f(t) \cos mt \dmeasure t.
$$

Integrating by parts we get
\begin{align*}
  &
  \pi a_{m}
  =
  \thebracket{
    m^{-1} f(t) \sin mt
  }_{-\pi}^{k_{1}}
  +
  \thebracket{
    m^{-1} f(t) \sin mt
  }_{k_{1}}^{k_{2}}
  +
  \cdots
  +
  \thebracket{
    m^{-1} f(t) \sin mt
  }_{k_{n}}^{\pi}
  \\
  & \hfill
  - m^{-1} \int_{-\pi}^{k_{1}} f'(t) \sin mt \dmeasure t
  - m^{-1} \int_{k_{1}}^{k_{2}} f'(t) \sin mt \dmeasure t
  \cdots
  - m^{-1} \int_{k_{n}}^{\pi} f'(t) \sin mt \dmeasure t,
\end{align*}
so that
$$
a_{m} = \frac{A_{m}}{m} - \frac{b_{m}'}{m},
$$
%
% 168
%
where
$$
\pi A_{m}
=
\sum_{r=1}^{n}
\sin m k_{r}
\thebrace{
  f(k_{r} - 0) - f(k_{r} + 0),
}
$$
and $b_{m}'$ is a Fourier constant of $f'(x)$.

Similarly
$$
b_{m} = \frac{B_{m}}{m} + \frac{a_{m}'}{m},
$$
where
$$
\pi B_{m}
=
-
\sum_{r=1}^{n}
\cos m k_{r}
\thebrace{
  f(k_{r} - 0)
  -
  f(k_{r} + 0)
}
-
\cos m \pi
\thebrace{
  f(\pi - 0)
  -
  f(-\pi + 0)
}
,
$$
and $a_{m}'$ is a Fourier constant of $f'(x)$.

Similarly, we get
$$
a_{m}' = \frac{A_{m}'}{m} - \frac{b_{m}''}{m},
\quad
b_{m}' = \frac{B_{m}'}{m} - \frac{a_{m}''}{m},
$$
where $a_{m}'', b_{m}''$ are the Fourier constants of
$f''(x)$ and
\begin{align*}
  \pi A_{m}'
  =&
  \sum_{r=1}^{n} \sin m k_{r} \thebrace{
    f'(k_{r}-0) - f'(k_{r}+0)
  },
  \\
  \pi B_{m}'
  =&
  - \sum_{r=1}^{n} \cos m k_{r} \thebrace{
    f'(k_{r}-0) - f'(k_{r}+0)
  }
  \\
  \hfill
  - \cos m \pi \thebrace{
    f'(\pi - 0) - f'(-\pi + 0)
  }.
\end{align*}

Therefore
$$
a_{m} =
\frac{A_{m}}{m}
- \frac{B_{m}'}{m^{2}}
- \frac{a_{m}''}{m^{2}},
\quad
b_{m} =
\frac{B_{m}}{m}
+ \frac{A_{m}'}{m^{2}}
- \frac{b_{m}''}{m^{2}},
$$

Now as $m \rightarrow \infty$, we see that
$$
A_{m}' = \bigo(1),
\quad
B_{m}' = \bigo(1),
$$
and, since the integrands involved in $a_{m}''$ and $b_{m}''$
are bounded, it is evident that
$$
a_{m}'' = \bigo(1),
\quad
b_{m}'' = \bigo(1).
$$

Hence if $A_{m}=0, B_{m}=0$, the Fourier series for $f(x)$ converges
absolutely and uniformly, by \hardsubsectionref{3}{3}{4}.

The necessary and sufficient conditions that
$A_{m} = B_{m} = 0$ for all values of $m$ are that
$$
f(k_{r} - 0) = f(k_{r} + 0),
\quad
f(\pi - 0) = f(-\pi + 0)
$$
that is to say that\footnote{Of course $f(x)$ is also subject to the conditions stated at the
beginning of the section.} $f(x)$ should be continuous for \emph{all} values of $x$.

\Subsection{9}{3}{1}{Differentiation of Fourier series.}
The result of differentiating
$$
\half a_{0}
+ \sum_{m=1}^{\infty} (a_{m} \cos mx + b_{m} \sin mx)
$$
term by term is
$$
\sum_{m=1}^{\infty} \thebrace{
  m b_{m} \cos mx
  -
  m a_{m} \sin mx
}.
$$
%
% 169
%

With the notation of \hardsectionref{9}{3}, this is the same as
$$
\half a_{0}'
+
\sum_{m=1}^{\infty} ( a_{m}' \cos mx + b_{m}' \sin mx),
$$
provided that $A_{m} = B_{m} = 0$ and
$\int_{-\pi}^{\pi} f'(x) \dmeasure x = 0$;
these conditions are satisfied if $f(x)$ is continuous for all values of
$x$.

Consequently sufficient conditions for the legitimacy of
differentiating a Fourier series term by term are that $f(x)$ should be
continuous for \emph{all} values of $x$ and $f'(x)$ should have only a finite
number of points of discontinuity in the range $(-\pi, \pi)$, both
functions having limited total fluctuation throughout the range.

\Subsection{9}{3}{2}{Determination of points of discontinuity.}

The expressions for $a_{m}$ and $b_{m}$ which have been found in
\hardsectionref{9}{3} can
frequently be applied in practical examples to determine the points
at which the sum of a given Fourier series may be discontinuous. Thus,
let it be required to determine the places at which the sum of the
series
$$
\sin x
+ \frac{1}{3} \sin 3x
+ \frac{1}{5} \sin 5x
+ \cdots
$$
is discontinuous.

\emph{Assuming} that the series is a Fourier series and not \emph{any}
trigonometrical series and observing that
$a_{m} = 0, b_{m} = (2m)^{-1}(1 - \cos m \pi)$, we get on considering the
formula found in \hardsectionref{9}{3},
$$
A_{m} = 0,
\quad
B_{m} = \half - \half \cos m \pi,
\quad
a_{m}' = b_{m}' = 0.
$$

Hence if $k_{1}, k_{2},\ldots$ are the places at which the analytic
character of the sum is broken, we have
$$
0
=
\pi A_{m}
=
\thebracket{
  \sin m k_{1} \thebrace{
    f(k_{1} - 0) - f(k_{1} + 0)
  }
  +
  \sin m k_{2} \thebrace{
    f(k_{2} - 0) - f(k_{2} + 0)
  }
  +
  \cdots
}.
$$
Since this is true for all values of $m$, the numbers
$k_{1}, k_{2}, \ldots$ must
be multiples of $\pi$; but
there is only one even multiple of $\pi$ in the range
$-\pi < x \leq \pi$, namely zero.
So $k_{1} = 0$,
and $k_{2}, k_{3}, \ldots$ do not exist.
Substituting $k_{1} = 0$ in the equation
$B_{m} = \half - \half \cos m \pi$, we have
$$
\pi (\half - \half \cos m \pi)
=
- \thebracket{
  \cos m \pi \thebrace{
    f(\pi - 0) - f(-\pi + 0)
  }
  + f(-0)
  - f(+0)
}.
$$

Since this is true for all values of $m$, we have
$$
\half \pi = f(+0) - f(-0),
\quad
\half \pi = f(\pi - 0) - f(-\pi + 0).
$$

This shews that, if the
series is a Fourier series, $f(x)$ has discontinuities at the points
$n \pi$ ($n$ any integer), and since $a_{m}' = b_{m}' = 0$, we should
expect\footnote{In point of fact
  $$
  f(x)
  =
  \begin{cases}
    -\frac{1}{4} \pi & -\pi < x < 0;\\
    \frac{1}{4} \pi & 0 < x < \pi.
  \end{cases}
  $$
} $f(x)$
to be constant in the open range $(-\pi, 0)$ and to be another constant
in the open range $(0, \pi)$.

\Section{9}{4}{\Fejer's theorem.}

We now begin the discussion of the theory of Fourier series by proving
the following theorem, due to \Fejer,\footnote{TODO:Math. Ann. lviii. (1904), pp. 51-69.}
concerning the summability of
the Fourier series associated with an arbitrary function, $f(t)$:

\emph{Let $f(t)$ be a function of the real variable $t$, defined arbitrarily
  when $-\pi \leq t < \pi$, and defined by the equation
  $$
  f(t + 2\pi) = f(t)
  $$
%
% 170
%
  for all other real values of $t$; and let
  $\int_{-\pi}^{\pi} f(t) \dmeasure t$
  exist and (if it is an improper integral)
  let it he absolutely convergent.
}

\emph{Then the Fourier series associated with the function
  $f(t)$ is summable\footnote{See \hardsubsectionref{8}{4}{3}.} ($C1$)
  at all points $x$ at which the two limits $f(x \pm 0)$ exist.}

And its sum ($C1$) is
$$
\half \thebrace{
  f(x + 0) + f(x-0)
}.
$$

Let $a_{n}, b_{n}, (n=0,1,2,\ldots)$ denote the Fourier constants
(\hardsectionref{9}{2}) of
$f(t)$ and let
$$
\half a_{0} = A_{0},
\hfill
a_{n} \cos n x + b_{n} \sin n x = A_{n}(x),
\hfill
\sum_{n=0}^{m} A_{n}(x) = S_{m}(x).
$$

Then we have to prove that
$$
\lim_{m \rightarrow \infty}
\frac{1}{m} \thebrace{
  A_{0}
  + S_{1}(x) + S_{2}(x) + \cdots + S_{m-1}(x)
}
=
\half \thebrace{
  f(x+0) + f(x-0)
},
$$
provided that the limits on the right exist.

If we substitute for the Fourier constants their values in the form of
integrals (\hardsectionref{9}{2}), it is easy to verify
that\footnote{It is obvious that, if we write $\lambda$ for $e^{i(x-t)}$ in the second line,
  then
  \begin{align*}
    m + &
    (m-1) (\lambda + \lambda^{-1})
    + (m-2) (\lambda^{2} + \lambda^{-2})
    + \cdots
    + (\lambda^{m-1} + \lambda^{1 - m})
    \\
    =&
    (1 - \lambda)^{-1} \thebrace{
      \lambda^{1-m}
      + \lambda^{2-m}
      + \cdots
      + \lambda^{-1}
      + 1
      - \lambda
      - \lambda^{2}
      - \cdots
      - \lambda^{m}
    } \\
    =&
    (1 - \lambda)^{-2} \thebrace{
      \lambda^{1 - m}
      - 2 \lambda
      + \lambda^{m+1}
    }
    =
    (\lambda^{\half m} - \lambda^{-\half m})^{2}
    /
    (\lambda^{\half} - \lambda^{-\half})^{2}.
  \end{align*}
  m + (m - 1) (X + X-i) + (hi - 2) (X2 + X-2) + . . . + (X' -i + Xi-' )

  = (l-X)-i \ i-™ + X2-' +...+X-i + l-X--X'-i-...-X' = (1 - X)-2 xi- v -
  2X + X' +i = (X '" - X~ '"f /(X - X" )
}
\begin{align*}
  A_{0} + \sum_{n=1}^{m-1} S_{n}(x)
  =&
  m A_{0}
  + (m - 1) A_{1}(x)
  + (m - 2) A_{2}(x)
  + \cdots
  + A_{m-1}(x)
  \\
  =&
  \frac{1}{\pi}
  \int_{-\pi}^{\pi} \thebrace{
    \half m
    + (m-1) \cos (x-t)
    + (m-2) \cos 2(x-t)
    + \cdots
    + \cos (m-1)(x-t)
  }
  f(t) \dmeasure t
  \\
  =&
  \frac{1}{2\pi}
  \int_{-\pi}^{\pi}
  \frac{\sin^{2} \half m (x-t)}{\sin^{2} \half (x-t)}
  f(t) \dmeasure t
  \\
  =&
  \frac{1}{2\pi}
  \int_{-\pi+x}^{\pi+x}
  \frac{\sin^{2} \half m (x-t)}{\sin^{2} \half (x-t)}
  f(t) \dmeasure t,
\end{align*}
the last step following from the periodicity
of the integrand.

If now we bisect the path of integration and write $x \mp 2\theta$ in place of
$t$ in the two parts of the path, we get
$$
A_{0}
+ \sum_{n=1}^{m-1} S_{n}(x)
=
\frac{1}{\pi}
\int_{0}^{\half \pi}
\frac{\sin^{2} m \theta}{\sin^{2} \theta}
f(x + 2\theta) \dmeasure \theta
+
\frac{1}{\pi}
\int_{0}^{\half \pi}
\frac{\sin^{2} m \theta}{\sin^{2} \theta}
f(x - 2\theta) \dmeasure \theta
$$

\emph{Consequently it is sufficient to prove that, as
  $m \rightarrow \infty$, then}
$$
\frac{1}{m}
\int_{0}^{\half\pi}
\frac{\sin^{2} m \theta}{\sin^{2} \theta}
f(x + 2\theta)
\dmeasure \theta
\rightarrow
\half \pi f(x + 0),
\quad
\frac{1}{m}
\int_{0}^{\half\pi}
\frac{\sin^{2} m \theta}{\sin^{2} \theta}
f(x - 2\theta)
\dmeasure \theta
\rightarrow
\half \pi f(x - 0),
$$
%
% 171
%

Now, if we integrate the equation
$$
\half
\frac{\sin^{2} m \theta}{\sin^{2} \theta}
=
\half m + (m-1) \cos 2\theta + \cdots + \cos 2(m-1)\theta,
$$
we find that
$$
\int_{0}^{\half \pi}
\frac{\sin^{2} m \theta}{\sin^{2} \theta}
\dmeasure \theta
=
\half \pi m,
$$
and so we have to prove that
$$
\frac{1}{m}
\int_{0}^{\half \pi}
\frac{\sin^{2} m \theta}{\sin^{2} \theta}
\phi(\theta)
\dmeasure \theta
\rightarrow
0
\quad
\textrm{as} %TODO: clean up; do more properly
\quad
m \rightarrow \infty
$$
where $\phi(\theta)$ stands in turn for each of the two functions
$$
f(x + 2\theta) - f(x + 0),
\quad
f(x - 2\theta) - f(x - 0).
$$
Now, given an arbitrary positive number $\eps$, we can choose $\delta$ so
that\footnote{On the assumption that $f(x \pm 0)$ exist.}
$$
\absval{\phi(\theta)} < \eps
$$
whenever $0 < \theta \leq \half \delta$. This choice of $\delta$ is obviously independent of $m$.

Then
\begin{align*}
  \absval{
    \frac{1}{m}
    \int_{0}^{\half \pi}
    \frac{\sin^{2} m\theta}{\sin^{2} \theta}
    \phi(\theta)
    \dmeasure \theta
  }
  &
  \leq
  \frac{1}{m}
  \int_{0}^{\half \delta}
  \frac{\sin^{2} m\theta}{\sin^{2} \theta}
  \absval{ \phi(\theta) }
  \dmeasure \theta
  +
  \frac{1}{m}
  \int_{\half\delta}^{\half\pi}
  \frac{\sin^{2} m\theta}{\sin^{2} \theta}
  \absval{ \phi(\theta) }
  \dmeasure \theta
  \\
  &
  <
  \frac{\eps}{m}
  \int_{0}^{\half \delta}
  \frac{\sin^{2} m\theta}{\sin^{2} \theta}
  \dmeasure \theta
  +
  \frac{1}{m \sin^{2} \half \delta}
  \int_{\half \delta}^{\half \pi}
  \absval{\phi(\theta)}
  \dmeasure \theta
  \\
  &
  \leq
  \frac{\eps}{m}
  \int_{0}^{\half \pi}
  \frac{\sin^{2} m\theta}{\sin^{2} \theta}
  \dmeasure \theta
  +
  \frac{1}{m \sin^{2} \half\delta}
  \int_{0}^{\half \pi}
  \absval{\phi(\theta)}
  \dmeasure \theta
  \\
  &
  =
  \half \pi \eps
  +
  \frac{1}{m \sin^{2} \half\delta}
  \int_{0}^{\half \pi}
  \absval{\phi(\theta)}
  \dmeasure \theta
\end{align*}

Now the convergence of
$\int_{-\pi}^{\pi} \absval{f(t)} \dmeasure t$
entails the convergence of
$$
\int_{0}^{\half \pi}
\absval{\phi(\theta)}
\dmeasure \theta,
$$
and so, given $\eps$ (and therefore $\delta$), we can make
$$
\half \pi \eps
\sin^{2} \half \delta
>
\int_{0}^{\half\pi}
\absval{ \phi(\theta) }
\dmeasure \theta,
$$
by taking $m$ sufficiently large.

Hence, by taking $m$ sufficiently large, we can make
$$
\absval{
  \frac{1}{m}
  \int_{0}^{\half \pi}
  \frac{\sin^{2} m \theta}{\sin^{2} \theta}
  \phi(\theta)
  \dmeasure \theta
}
<
\pi \eps,
$$
where $\eps$ is an arbitrary positive number; that is to say, from the
definition of a limit,
$$
\lim_{m \rightarrow \infty}
\frac{1}{m}
\int_{0}^{\half\pi}
\frac{\sin^{2} m\theta}{\sin^{2} \theta}
\phi(\theta)
\dmeasure \theta
=
0,
$$
and so \Fejer's theorem is established.
%
% 172
%

%TODO:Corollary
Corollary 1. Let $U$ and $L$ be the upper and lower bounds of $f(t)$ in any
interval $(a, b)$ whose length does not exceed $2\pi$, and let
$$
\int_{-\pi}^{\pi} \absval{f(t)} \dmeasure t = \pi A.
$$
Then, if $a + \eta \leq x \leq b - \eta$, where $\eta$ is any positive number, we have
\begin{align*}
  U
  -
  \frac{1}{m}
  \thebrace{
    A_{0}
    +
    \sum_{n=1}^{m-1} S_{n}(x)
  }
  &
  =
  \frac{1}{2 m \pi}
  \thebrace{
    \int_{-\pi + x}^{x - \eta}
    + \int_{x - \eta}^{x + \eta}
    + \int_{x + \eta}^{\pi + x}
  }
  \frac{\sin^{2} \half m (x-t)}{\sin^{2} \half (x-t)}
  \thebrace{
    U - f(t)
  }
  \dmeasure t
  \\
  &
  \geq
  \frac{1}{2 m \pi}
  \thebrace{
    \int_{-\pi + x}^{x - \eta}
    + \int_{x + \eta}^{\pi + x}
  }
  \frac{\sin^{2} \half m (x-t)}{\sin^{2} \half (x-t)}
  \thebrace{
    U - f(t)
  }
  \dmeasure t
  \\
  &
  \geq
  -\frac{1}{2 m \pi}
  \thebrace{
    \int_{-\pi + x}^{x - \eta}
    + \int_{x + \eta}^{\pi + x}
  }
  \frac{\absval{U} + \absval{f(t)}}{\sin^{2} \half \eta}
  \dmeasure t
\end{align*}
so that
$$
\frac{1}{m}
\thebrace{
  A_{0}
  + \sum_{n=1}^{m-1} S_{n}(x)
}
\leq
U +
\thebrace{
  \absval{U}
  + \half A
}
/
\thebrace{
  m \sin^{2} \half \eta
}.
$$
Similarly
$$
\frac{1}{m}
\thebrace{
  A_{0}
  + \sum_{n=1}^{m-1} S_{n}(x)
}
\geq
L -
\thebrace{
  \absval{L}
  + \half A
}
/
\thebrace{
  m \sin^{2} \half \eta
}.
$$
Corollary 2. Let $f(t)$ be continuous in the interval $a \leq t \leq b$. Since
continuity implies uniformity of continuity (\hardsubsectionref{3}{6}{1}), the choice of
$\delta$ corresponding to any value of $x$ in $(a, b)$ is independent of $x$, and the
upper bound of $\absval{f(x \pm 0)}$, i.e. of $\absval{f(x)}$, is also independent of $x$,
so that
\begin{align*}
  \int_{0}^{\half \pi} \absval{\phi(\theta)} \dmeasure \theta
  =&
  \int_{0}^{\half \pi}
  \absval{
    f(x \pm 2\theta) - f(x \pm 0)
  }
  \dmeasure \theta
  \\
  \leq
  &
  \half \int_{-\pi}^{\pi} \absval{f(t)} \dmeasure t
  + \half \pi \absval{f(x \pm 0)},
\end{align*}
and the upper bound of the last expression is independent of $x$.

Hence the choice of $m$, which makes
$$
\absval{
  \frac{1}{m}
  \int_{0}^{\half \pi}
  \frac{\sin^{2} m \theta}{\sin^{2} \theta}
  \phi(\theta)
  \dmeasure \theta
}
<
\pi \eps,
$$
is independent of $x$, \emph{and consequently
  $\frac{1}{m}\thebrace{
    A_{0} + \sum_{n=1}^{m-1} S_{n}(x)
  }$ tends to the limit $f(x)$, as $m \rightarrow \infty$,
  uniformly throughout the interval $a \leq x \leq b$}.
\Subsection{9}{4}{1}{The Riemann-Lehesgue lemmas.}
In order to be able to apply Hardy's theorem (\hardsectionref{8}{5}) to deduce the
convergence of Fourier series from \Fejer's theorem, we need the two
following lemmas :

%TODO
(I) Let $\int_{a}^{b} \psi(\theta) \dmeasure \theta$ exist and (if it is an improper integral) let it be
absolutely convergent. Then, as $\lambda \rightarrow \infty$,
$$
\int_{a}^{b} \psi(\theta) \sin (\lamba \theta) \dmeasure \theta
\quad
\textrm{is}
\quad
\littleo(1).
$$
%TODO
(II) If, further, $\psi(\theta)$ has limited total fluctuation in the range
$(a,b)$ then, as $\lambda \rightarrow \infty$,
$$
\int_{a}^{b} \psi(\theta) \sin (\lambda\theta) \dmeasure \theta
\quad
\textrm{is}
\quad
\bigo(1 / \lambda).
$$

%
% 173
%

Of these results (I) %TODO:fixref
was stated by W. R. Hamilton\footnote{TODOTrans. Dublin Acad. xix. (1843), p. 267.} and by
Riemann\footnote{TODO:Ges. Math. IVerke, p. 241. For Lebesgue's investigation see his
Series trigonometriques (1906), Ch. III.} in
the case of bounded fiuictions.
The truth of (II) %TODO:fixref
seems to have been
well known before its importance was realised; it is a generalisation
of a result established by Dirksen\footnote{TODO:Journal fUr Math. iv. (1829), p. 172.}
and Stokes (see \hardsectionref{9}{3}) in the
case of functions with a continuous differential coefficient.

The reader should observe that the analysis of this section remains
valid when the sines are replaced throughout by cosines.

%TODO:fixref
(I) It is convenient\footnote{For this proof we are indebted to Mr Hardy;
  it seems to be neater than the proofs given by other writers,
  %TODO
  e.g. de la Vallee Poussin,
  Cours cV Analyse Infinitesiniale, ii. (1912), pp. 140-141.}
to establish this lemma first in the case in
which $\psi(\theta)$ is bounded in the range $(a, b)$. In this case, let $K$ be
the upper bound of $\absval{\psi(\theta)}$, and let $\eps$ be an arbitrary positive
number. Divide the range $(a, b)$ into $n$ parts by the points
$x_{1}, x_{2}, \ldots x_{n-1}$, and form the sums $S_{n}, s_{n}$ associated with the function
$\psi(\theta)$ after the manner of \hardsectionref{4}{1}. Take $n$ so large that
$S_{n} - s_{n} < \eps$; this is
possible since $\psi(\theta)$ is integrable.

In the interval $(x_{r-1}, x_{r})$ write
$$
\psi(\theta) = \psi_{r}(x_{r-1}) + \omega_{r}(\theta),
$$
so that
$$
\absval{ \omega_{r}(\theta) } \leq U_{r} - L_{r},
$$
where $U_{r}$ and $L_{r}$ are the upper and lower bounds of
$\psi(\theta)$ in the interval
$(x_{r-1}, x_{r})$.

It is then clear that

By taking X sufficiently large (n remaining fixed after e has been
chosen), the last expression may be made less than 2e, so that
%TODO:alignment
\begin{align*}
  &
  \absval{
    \int_{a}^{b}
    \psi(\theta) \sin(\lambda \theta) \dmeasure \theta
  }
  \\
  & \quad
  =
  \absval{
    \sum_{r=1}^{n}
    \psi_{r}(x_{r-1})
    \int_{x_{r-1}}^{x_{r}}
    \sin (\lambda \theta) \dmeasure \theta
    +
    \sum_{r=1}^{n}
    \int_{x_{r-1}}^{x_{r}}
    \omega_{r}(\theta) \sin(\lambda \theta) \dmeasure \theta
  }
  \\
  & \quad
  \leq
  \sum_{r=1}^{n}
  \absval{
    \psi_{r}(x_{r-1})
  }
  \cdot
  \absval{
    \int_{x_{r-1}}^{x_{r}}
    \sin (\lambda \theta) \dmeasure \theta
  }
  +
  \sum_{r=1}^{n}
  \int_{x_{r-1}}^{x_{r}}
  \absval{ \omega_{r} (\theta)} \dmeasure \theta
  \\
  & \quad
  \leq n K \cdot (2 / \lambda)
  +
  (S_{n} - s_{n})
  \\
  & \quad
  < (2nK / \lambda)
  +
  \eps.
\end{align*}

By taking $\lambda$ sufficiently large ($n$ remaining fixed after $\eps$ has been
chosen), the last expression may be less than $2\eps$, so that
$$
\lim_{\lambda \rightarrow \infty}
\int_{a}^{b} \psi(\theta) \sin(\lambda \theta) \dmeasure \theta = 0,
$$
and this is the result stated.

When $\psi(\theta)$ is unbounded, if it has an absolutely convergent integral,
by \hardsectionref{4}{5}, we may enclose the points at which it is unbounded in a
finite\footnote{The \emph{finiteness} of the number of intervals is assumed in the
definition of an improper integral,\hardsectionref{4}{5}.} number
%
% 174
%
of intervals $\delta_{1}, \delta_{2}, \ldots, \delta_{p}$ such that
$$
\sum_{n=1}^{p}
\int_{\delta_{r}}
\absval{ \psi(\theta) } \dmeasure \theta
<
\eps.
$$
If $K$ denote the upper bound of $\absval{\psi(\theta)}$ for values of
$\theta$ outside these intervals, and if
$\gamma_{1}, \gamma_{2}, \ldots, \gamma_{p+1}$ denote the portions of the interval
$(a, b)$ which do not belong to $\delta_{1}, \delta_{2}, \ldots, \delta_{p}$
we may prove as before that
\begin{align*}
  \absval{\int_{a}^{b} \psi(\theta) \sin (\lambda \theta) \dmeasure \theta}
  & =
  \absval{
    \sum_{r=1}^{p+1}
    \int_{\gamma_{r}} \psi(\theta) \sin(\lambda \theta) \dmeasure \theta
    +
    \sum_{r=1}^{p}
    \int_{\delta_{r}} \psi(\theta) \sin(\lambda \theta) \dmeasure \theta
  }
  \\
  & \leq
  \absval{
    \sum_{r=1}^{p+1}
    \int_{\gamma_{r}} \psi(\theta) \sin(\lambda \theta) \dmeasure \theta
  }
  +
  \sum_{r=1}^{p}
  \int_{\delta_{r}}
  \absval{\psi(\theta) \sin(\lambda \theta)}
  \dmeasure \theta
  \\
  & <
  (2nK / \lambda) + 2 \eps.
\end{align*}

Now the choice of $\eps$ fixes $n$ and $K$, so that the last expression may be
made less than $3\eps$ by taking $\lambda$ sufficiently large. That is to say
that, even if $\psi(\theta)$ be unbounded,
$$
\lim_{\lambda \rightarrow \infty}
\int_{a}^{b} \psi(\theta) \sin(\lambda \theta) \dmeasure \theta
=
0,
$$
provided that $\psi(\theta)$ has an (improper) integral which is absolutely
convergent.

The first lemma is therefore completely proved.

%TODO
(II) When $\psi(\theta)$ has limited total fluctuation in the range $(a, b)$,
by %TODO: un-hardcode example number
\hardsubsectionref{3}{6}{4} example 2, we may write
$$
\psi(\theta) = \chi_{1}(\theta) - \chi_{2}(\theta),
$$
where $\chi_{1}(\theta), \chi_{2}(\theta)$ are positive increasing bounded functions.

Then, by the second mean-value theorem (\hardsubsectionref{4}{1}{4}) a number
$\xi$ exists such that $a \leq \xi \leq b$ and
\begin{align*}
  \absval{
    \int_{a}^{b} \chi_{1}(\theta) \sin(\lambda \theta) \dmeasure \theta
  }
  &=
  \absval{
    \chi_{1}(b) \int_{\xi}^{b} \sin(\lambda \theta) \dmeasure \theta
  }
  \\
  & \leq
  2 \chi_{1}(b) / \lambda.
\end{align*}

If we treat $\chi_{2}(\theta)$ in similar manner, it follows that
\begin{align*}
  \absval{
    \int_{a}^{b} \psi (\theta) \sin(\lambda \theta) \dmeasure \theta
  }
  &
  \leq
  \absval{
    \int_{a}^{b} \chi_{1}(\theta) \sin(\lambda\theta) \dmeasure \theta
  }
  +
  \absval{
    \int_{a}^{b} \chi_{2}(\theta) \sin(\lambda\theta) \dmeasure \theta
  }
  \\
  &
  \leq
  2 \thebrace{ \chi_{1}(b) + \chi_{2}(b) } / \lambda
  \\
  = &
  \bigO(1 / \lambda),
\end{align*}
and the second lemma is established.

Corollary. If $f(t)$ lie such that $\int_{-\pi}^{\pi} f(t)$ %TODO:fix typo in book (missing 'dt')?
exists and is an absolutely convergent integral,
the Fourier constants $a_{n}, b_{n}$ of $f(t)$ are $\littleo{l}$ as
$n \rightarrow \infty$;
and if, further, $f(t)$ has limited total fluctuation in the range
$(-\pi, \pi)$, the Fourier constants are $\bigo{1/n}$.

[Of course these results are not sufficient to ensure the convergence
of the Fourier series associated with $f(t)$; for a series, in which the
terms are of the order of magnitude of the terms in the harmonic
series \hardsectionref{2}{3}), is not necessarily convergent.]

%
% 175
%

\Subsection{9}{4}{2}{The proof of Fourier's theorem.}

We shall now prove the theorem enunciated in \hardsectionref{9}{2}, namely:

%TODO:emphasize paragraphs
Let $f(t)$ be a function defined arbitrarily when $-\pi \leq t < \pi$, and defined by the
equation $f(t + 2\pi) = f(t)$ for all other real values of $t$; and let
$\int_{-\pi}^{\pi} f(t) \dmeasure t$
exist and (if it is an improper integral) let it be absolutely
convergent. Let $a_{n}, b_{n}$ be defined by the equations
$$
\pi a_{n} = \int_{-\pi}^{\pi} f(t) \cos nt \dmeasure t,
\quad
\pi b_{n} = \int_{-\pi}^{\pi} f(t) \sin nt \dmeasure t.
$$
Then, if $x$ be an interior point of any interval $(a, b)$ within which
$f(t)$ has limited total fluctuation, the series
$$
\half a_{0}
+
\sum_{n=1}^{\infty} (
a_{n} \cos nx + b_{n} \sin nx
)
$$
is convergent and its sum is $half \thebrace{f(x+0) + f(x-0)}$.

It is convenient to give two proofs, one applicable to functions for
which it is permissible to take the interval $(a, b)$ to be the interval
$(-\pi+x, \pi + x)$, the other applicable to functions for which it is
not permissible.

%TODO:autonumbering
(I) When the interval $(a, b)$ may be taken to be $(-\pi + x, \pi + x)$,
it follows from \hardsubsectionref{9}{4}{1} (II) %TODO:ref
that $a_{n} \cos nx + b_{n} \sin nx$ is $\bigo{1/n}$ as
$n \rightarrow \infty$. Now by \Fejer's theorem (\hardsectionref{9}{4})
the series under consideration
is summable (Cl) %TODO
and its sum (C'l) %TODO
is\footnote{The limits $f(x \pm 0)$ exist, by \hardsubsectionref{3}{6}{4} example 3.%TODO
}
$\half \thebrace{f(x+0) + f(x-0)}$. Therefore.,
by Hardy's convergence theorem \hardsectionref{8}{5}), the series under consideration
is CONVERGENT %TODO:emph?
and its sum (by \hardsubsectionref{8}{4}{3}) is
$\half \thebrace{f(x+0) + f(x-0)}$.

(II) %TODO
Even if it is not permissible to take the interval $(a, b)$ to be
the whole interval $(-\pi + x, \pi + x)$, it is possible, by
hypothesis, to choose a positive number $\delta$, less than $\pi$,
such that $f(t)$
has limited total fluctuation in the interval $(x-\delta, x+\delta)$.
We now define an auxiliary function $g(t)$, which is equal to $f(t)$ when
$x - \delta \leq t \leq x + \delta$,
and which is equal to zero throughout the rest of the interval
$(-\pi + x, \pi + x)$; and $g(t + 2\pi)$ is to be equal to $g(t)$ for all real
values of $t$.

Then $g(t)$ satisfies the conditions postulated for the functions under
consideration in (I),%TODO:ref
namely that it has an integral which is
absolutely convergent and it has limited total fluctuation in the
interval $(-\pi + x, \pi + x)$; and so, if
$a_{n}^{(1)}, b_{n}^{(1)}$ denote the Fourier
constants of $g(t)$, the arguments used in (I) %TODO:addref
prove that the Fourier
series associated with $g(t)$, namely
$$
\half a_{0}^{(1)}
+
\sum_{n=1}^{\infty}
\theparen{
  a_{n}^{(1)} \cos nx
  +
  b_{n}^{(1)} \sin nx
},
$$
is convergent and has the sum
$\half \thebrace{g(x+0) + g(x-0)}$, and this is
equal to
$$
\half \thebrace{
  f(x+0) + f(x-0)
}.
$$
%
% 176
%

Now let $S_{m}(x)$ and $S_{m}^{(1)} (x)$ denote the sums of the first $m + 1$
terms of the Fourier series associated with $f(t)$ and $g(t)$ respectively. Then
it is easily seen that
\begin{align*}
  S_{m}(x)
  =&
  \frac{1}{\pi}
  \int_{-\pi}^{\pi} \thebrace{
    \half
    + \cos (x-t)
    + \cos 2(x-t)
    + \cdots
    + \cos m(x-t)
  }
  f(t) \dmeasure t
  \\
  =&
  \frac{1}{2\pi}
  \int_{-\pi}^{\pi}
  \frac{\sin (m+\half) (x-t) }{\sin \half (x-t)}
  f(t) \dmeasure t
  \\
  =&
  \frac{1}{2\pi}
  \int_{-\pi+x}^{\pi+x}
  \frac{\sin (m+\half) (x-t) }{\sin \half (x-t)}
  f(t) \dmeasure t
  \\
  =&
  \frac{1}{\pi}
  \int_{0}^{\half \pi}
  \frac{\sin (2m+1)\theta}{\sin \theta}
  f(x + 2\theta) \dmeasure \theta
  +
  \frac{1}{\pi}
  \int_{0}^{\half \pi}
  \frac{\sin (2m+1)\theta}{\sin \theta}
  f(x - 2\theta) \dmeasure \theta,
\end{align*}
by steps analogous to those given in \hardsectionref{9}{4}.

In like manner
$$
S_{m}^{(1)}(x)
=
\frac{1}{\pi}
\int_{0}^{\half \pi}
\frac{\sin (2m+1)\theta}{\sin \theta}
g(x + 2\theta) \dmeasure \theta
+
\frac{1}{\pi}
\int_{0}^{\half \pi}
\frac{\sin (2m+1)\theta}{\sin \theta}
g(x - 2\theta) \dmeasure \theta,
$$
and so, using the definition of $g(t)$, we have
\begin{align*}
  S_{m}(x) - S_{m}^{(1)}(x)
  =&
  \frac{1}{\pi}
  \int_{\half\delta}^{\half \pi} \sin (2m+1)\theta
  \frac{f(x+2\theta)}{\sin \theta}
  \dmeasure \theta
  \\
  &\quad
  \frac{1}{\pi}
  \int_{\half\delta}^{\half \pi} \sin (2m+1)\theta
  \frac{f(x-2\theta)}{\sin \theta}
  \dmeasure \theta.
\end{align*}

Since $\cosec$ is a continuous function in the range $(\half \delta, \half \pi)$, it follows
that $f(x \pm 2\theta) \cosec \theta$ are integrable functions with absolutely
convergent integrals; and so, by the Riemann-Lebesgue lemma of
\hardsubsectionref{9}{4}{1} (I), %TODO:add reference to lemma (the `(I)' here)
\emph{both the integrals on the right in the last equation tend to zero
as $m \rightarrow \infty$}.

That is to say
$$
\lim_{m \rightarrow \infty}
\thebrace{
  S_{m}(x) - S_{m}^{(1)}(x)
}
= 0.
$$

Hence, since
$$
\lim_{m \rightarrow \infty}
S_{m}^{(1)}(x)
=
\half \thebrace{
  f(x+0) + f(x-0)
},
$$
it follows also that
$$
\lim_{m \rightarrow \infty} S_{m}(x)
=
\half \thebrace{
  f(x+0) + f(x-0)
}.
$$

\emph{We have therefore proved that the Fourier series associated with
  $f(t)$, namely
  $
  \half a_{0}
  + \sum \theparen{
    a_{n} \cos nx
    +
    b_{n} \sin nx
  }
  $
  is convergent and its sum is}
$$
\half \thebrace{
  f(x+0) + f(x-0)
}
$$
\Subsection{9}{4}{3}{The Dirichlet-Bonnet proof of Fourier's theorem.}
It is of some interest to prove directly the theorem of \hardsubsectionref{9}{4}{2},
without making use of the theory of summability; accordingly we now
give a proof which is on the same general lines as the proofs due to
Dirichlet and Bonnet.
%
% 177
%

As usual we denote the sum of the first $m + 1$ terms of the Fourier
series by $S_{m}(x)$, and then, by the analysis of \hardsubsectionref{9}{4}{2}, we have
$$
S_{m}(x)
=
\frac{1}{\pi}
\int_{0}^{\half \pi}
\frac{\sin (2m+1)\theta}{\sin \theta}
\f(x + 2\theta)
\dmeasure \theta
+
\frac{1}{\pi}
\int_{0}^{\half \pi}
\frac{\sin (2m+1)\theta}{\sin \theta}
\f(x - 2\theta)
\dmeasure \theta.
$$

Again, on integrating the equation
$$
\frac{\sin (2m+1)\theta}{\sin \theta}
=
1
+ 2 \cos 2\theta
+ 2 \cos 4\theta
+ \cdots
+ 2 \cos 2m\theta,
$$
we have
$$
\int_{0}^{\half\pi}
\frac{\sin (2m+1)\theta}{\sin \theta}
\dmeasure \theta
=
\half\pi,
$$
so that
\begin{align*}
  TODO
\end{align*}

In order to prove that
$$
\lim_{m \rightarrow \infty}
S_{m}(x)
=
\half \thebrace{
  f(x+0) + f(x-0)
},
$$
it is therefore sufficient to prove that
$$
\lim_{m \rightarrow \infty}
\int_{0}^{\half\pi}
\frac{\sin (2m+1)\theta}{\sin \theta}
\phi(\theta) \dmeasure \theta
=
0,
$$
where $\phi(\theta)$ stands in turn for each of the functions
$$
f(x+2\theta) - f(x+0),
\quad
f(x-2\theta) - f(x-0).
$$
Now, by \hardsubsectionref{3}{6}{4} example 4, %TODO:replace ref
$\theta \phi(\theta) \cosec \theta$ is a function with limited
total fluctuation in an interval of which $\theta=0$ is an
end-point;\footnote{The other end-point is $\theta = \half (b-x)$
  or $\theta = \half (x-a)$, according as $\phi(\theta)$
  represents one or other of the two functions.} and so
we may write
$$
\theta
\phi (\theta)
\cosec \theta
=
\chi_{1}(\theta)
-
\chi_{2}(\theta),
$$
where $\chi_{1}(\theta), \chi_{2}(\theta)$ are bounded positive
increasing functions of $\theta$ such that
$$
\chi_{1}(+0) = \chi_{2}(+0) = 0.
$$

Hence, given an arbitrary positive number $\eps$, we can choose a positive
number $\delta$ such that
$$
0 \leq \chi_{1}(\theta) < \eps,
\quad
0 \leq \chi_{2}(\theta) < \eps
$$
whenever $0 \leq \theta \leq \half \delta$.

We now obtain inequalities satisfied by the three integrals on the
right of the obvious equation
\begin{align*}
  &
 \int_{0}^{\half\pi}
 \frac{\sin (2m+1)\theta}{\sin \theta}
 \phi(\theta) \dmeasure \theta
 =
 \int_{\half \delta}^{\half \pi}
 \sin (2m+1)\theta
 \frac{\phi(\theta)}{\sin \theta}
 \dmeasure \theta
 \\
 &
 \quad
 \int_{0}^{\half \delta}
 \frac{\sin (2m+1)\theta}{\theta}
 \chi_{1}(\theta) \dmeasure \theta
 -
 \int_{0}^{\half \delta}
 \frac{\sin (2m+1)\theta}{\theta}
 \chi_{2}(\theta) \dmeasure \theta
\end{align*}
%
% 178
%

The modulus of the first integral can be made less than $\eps$ by taking $m$
sufficiently large; this follows from
\hardsubsectionref{9}{4}{1} (i) %TODO:subref
since $\phi(\theta) \cosec \theta$
has an integral which converges absolutely in the interval
$(\half \delta, \half \pi)$.

Next, from the second mean-value theorem, it follows that there is a
number $\xi$ between $0$ and $\delta$ such that
\begin{align*}
TODO
\end{align*}

Since
$\int^{\infty} \frac{\sin t}{t} \dmeasure t$
is convergent, it follows that
$\absval{\int_{\beta}^{\infty} \frac{\sin u}{u} \dmeasure u}$
has an upper bound\footnote{The reader will find it interesting to prove that
  $\int_{0}^{\infty} \frac{\sin u}{u} \dmeasure u = \half\pi$.}
$B$ which is independent of $\beta$, and it is then clear that
$$
\absval{
  \int_{0}^{\half\delta}
  \frac{\sin (2m+1)\theta}{\theta}
  \chi_{1}(\theta)
  \dmeasure \theta
}
\leq
2 B \chi_{1}(\half\delta)
<
2 B \eps.
$$

On treating the third integral in a similar manner, we see that we can
make
$$
\absval{
  \int_{0}^{\half\delta}
  \frac{\sin (2m+1)\theta}{\sin \theta}
  \phi(\theta)
  \dmeasure \theta
}
<
(4B+1) \eps
$$
by taking $m$ sufficiently large; \emph{and so we have proved that}
$$
\lim_{m \rightarrow \infty}
\int_{0}^{\half\pi}
\frac{\sin (2m+1)\theta}{\sin \theta}
\phi(\theta)
\dmeasure \theta
=
0.
$$
But it has been seen that this is a sufficient condition for the limit
of $S_{m}(x)$ to be $\half \thebrace{f(x+0) + f(x-0)}$; and we have therefore
established the con- vergence of a Fourier series in the circumstances
enunciated in \hardsubsectionref{9}{4}{2}.

%\begin{smallfontnote}%TODO
Note. The reader should observe that in either proof of the
convergence of a Fourier \emph{series} the second mean-value theorem is
required; but to prove the summability of the series, the \emph{first}
mean-value theorem is adequate. It should also be observed that, while
restrictions are laid upon $f(t)$ throughout the range $(-\pi, \pi)$ in
establishing the \emph{summability} at any point $x$, the only additional
restriction necessary to ensure \emph{convergence} is a restriction on the
behaviour of the function in the \emph{immediate neighbourhood} of the point
$x$. The fact that the convergence depends only on the behaviour of the
function in the immediate neighbourhood of $x$ (provided that the
function has an integral which is absolutely convergent) was noticed
by Riemann and has been emphasised by Lebesgue,
%TODO:ref
Series Trigonometriques p: 60.
%\end{smallfontnote}%TODO

It is obvious that the condition\footnote{Due to Jordan, Comptes Rendus, xcii. (1881), p. 228.}
that $x$ should be an interior point
of an interval in which $f(t)$ has limited total fluctuation is merely a
\emph{sufficient} condition for the convergence of the Fourier series; and
it may be replaced by any condition which makes
$$
\lim_{m \rightarrow \infty}
\int_{0}^{\half\pi}
\frac{\sin (2m+1)\theta}{\sin \theta}
\phi(\theta)
\dmeasure \theta
=
0.
$$
%
% 179
%

Jordan's condition is, however, a natural modification of the
Dirichlet condition that the function $f(t)$ should have only a finite
number of maxima and minima, and it does not increase the difficulty
of the proof.

Another condition with the same efifect is due to Dini,
% TODO:ref
Sopra le Serie di Foiirier (Pisa, 1880),
namely that, if
$$
\Phi(\theta)
=
\thebrace{
  f(x+2\theta) + f(x-2\theta) - f(x+0) - f(x-0)
}
\theta,
$$
then
$\int_{0}^{a} \Phi(\theta) \dmeasure \theta$ should converge absolutely for some positive value of
$a$.

[If the condition is satisfied, given $\eps$ we can find $\delta$ so that
$$
\int_{0}^{\half\delta}
\absval{
  \Phi(\theta)
}
\dmeasure \theta
<
\eps,
$$
and then
$$
\absval{
  \int_{0}^{\half\delta}
  \sin (2m+1)\theta
  \frac{\theta}{\sin \theta}
  \Phi(\theta)
  \dmeasure \theta
}
<
\half \pi \eps
$$
the proof that
$
\absval{
  \int_{\half\delta}^{\half \pi}
  \frac{\sin (2m+1)\theta}{\sin \theta}
  \phi(\theta)
  \dmeasure \theta
}
<
\eps
$
for sufficiently large values of $m$
follows from the Riemann-Lebesgue lemma.]

A more stringent condition than Dini's is due to Lipschitz,
Journal fuer Math. LXiil. (1864), p. 296, %TODO:ref
namely $\absval{\phi(\theta) < C \theta^{k}}$, where $C$ and
$k$ are positive and independent of $\theta$.

For other conditions due to Lebesgue and to
de la Vallee Poussin, %TODO:proper accents
see the latter's
Cours d' Analyse In/lnitesimale, ii. (1912), pp. 149-150. %TODO:ref
It should be noticed that Jordan's condition differs in character from
Dini's condition; the latter is a condition that the series may
converge \emph{at a point}, the former that the series may converge
\emph{throughout an interval}.

\Subsection{9}{4}{4}{The uniformity of the convergence of Fourier series.}
Let $f(t)$ satisfy the conditions enunciated in \hardsubsectionref{9}{4}{2},
and further let it be continuous
(in addition to having limited total fluctuation) in
an interval $(a, b)$. \emph{Then the Fourier series associated with $f(t)$
  converges uniformly to the sum $f(x)$ at all points $x$ for which
  $a + \delta \leq x \leq b - \delta$, where $\delta$ is any positive number.}

Let $h(t)$ be an auxiliary function defined to be equal to $f(t)$ when
$a \leq t \leq b$ and equal to zero for other values of $t$ in the range
$(-\pi, \pi)$, and
let $\alpha_{n}, \beta_{n}$ denote the Fourier constants of $h(t)$.
Also let $S_{m}^{2}(x)$ denote the sum of the first $m + 1$ terms of the
Fourier series associated with $h(t)$.

Then, by \hardsectionref{9}{4} corollary 2,%TODO:ref
it follows that
$\half \alpha_{0} + \sum_{n=1}^{\infty} \theparen{\alpha_{n} \cos nx + \beta_{n} \sin nx}$
is \emph{uniformly} summable throughout the interval $(a+\delta, b-\delta)$;
and since
$$
\absval{
  \alpha_{n} \cos nx + \beta_{n} \sin nx
}
\leq
\sqrt{
  \alpha_{n}^{2} + \beta_{n}^{2}
},
$$
which is independent of $x$ and which, by \hardsubsectionref{9}{4}{1} (ii), %TODO:ref
is $\bigO(1/n)$, it
follows from \hardsectionref{8}{5} corollary that
$$
\half \alpha_{0} + \sum_{n=1}^{\infty} \theparen{\alpha_{n} \cos nx + \beta_{n} \sin nx}
$$
converges uniformly to the sum $h(x)$, which is equal to $f(x)$.

Now, as in \hardsubsectionref{9}{4}{2},
$$
S_{m}(x)
-
S_{m}^{(2)}(x)
=
\frac{1}{\pi}
\int_{\half (b-x)}^{\half \pi}
\frac{\sin (2m+1)\theta}{\sin \theta}
f(x + 2\theta)
\dmeasure \theta
+
\frac{1}{\pi}
\int_{\half (x-a)}^{\half \pi}
\frac{\sin (2m+1)\theta}{\sin \theta}
f(x - 2\theta)
\dmeasure \theta.
$$
%
% 180
%

As in \hardsubsectionref{9}{4}{1} we choose an arbitrary positive number
$\eps$ and then enclose the points at which $f(t)$ is unbounded in a set of intervals
$\delta_{1}, \delta_{2}, \ldots, \delta_{p}$, such
that
$\sum_{r=1}^{p} \int_{\delta_{r}} \absval{f(t)} \dmeasure t < \eps$.

If $K$ be the upper bound of $\absval{f(t)}$ \ outside these intervals, we then
have, as in \hardsubsectionref{9}{4}{1},
$$
\absval{
  S_{m}(x) - S_{m}^{(2)}(x)
}
<
\theparen{
  \frac{2nK}{2m + 1} + 2\eps
}
\cosec \delta,
$$
where the choice of $n$ depends only on $a$ and $b$ and the form of the
function $f(t)$. Hence, by a choice of $m$ \emph{independent} of $x$ we can make
$$
\absval{
  S_{m}(x) - S_{m}^{(2)}(x)
}
$$
arbitrarily small; so that $\absval{S_{m}(x) - S_{m}^{(2)}(x)}$
tends uniformly to zero. Since
$S_{m}^{(2)}(x) \rightarrow f(x)$
uniformly, it is then obvious that
$S_{m}(x) \rightarrow f(x)$
uniformly; and this is the result to be proved.

%\begin{notefont}???
NOTE. It must be observed that no general statement can be made about
uniformity or absoluteness of convergence of Fourier series. Thus the
series of \hardsubsectionref{9}{1}{1} example 1 %TODO:ref
converges uniformly except near $x = (2n+1)\pi$
but converges absolutely only when $x = n\pi$, whereas the series of
\hardsubsectionref{9}{1}{1} example 2 %TODO:ref
converges uniformly and absolutely for all real values
of $x$.
%\begin{notefont}???
\begin{wandwexample}
 If $\phi(\theta)$ satisfies suitable conditions in the range $(0, \pi)$,
shew that
\begin{align*}
  \lim_{m \rightarrow \infty}
  \int_{0}^{\pi}
  \frac{\sin (2m+1)\theta}{\sin \theta}
  \phi(\theta) \dmeasure \theta
  =&
  \lim_{m \rightarrow \infty}
  \int_{0}^{\half \pi}
  \frac{\sin (2m+1)\theta}{\sin \theta}
  \phi(\theta) \dmeasure \theta
  \\
  & \quad
  +
  \lim_{m \rightarrow \infty}
  \int_{0}^{\half \pi}
  \frac{\sin (2m+1)\theta}{\sin \theta}
  \phi(\pi - \theta) \dmeasure \theta
  \\
  &
  = \half \pi \thebrace{
    \phi(+0) + \phi(\pi - 0)
  }.
\end{align*}
\end{wandwexample}
\begin{wandwexample}
 Prove that, if $a > 0$,
 $$
 \lim_{m \rightarrow \infty}
 \int_{0}^{\infty}
 \frac{\sin (2n+1)\theta}{\sin \theta}
 e^{-a\theta}
 \dmeasure \theta
 =
 \half \pi \coth \half a \pi.
 $$
 \addexamplecitation{Math. Trip. 1894.}
 [Shew that
 \begin{align*}
   \int_{0}^{\infty}
   \frac{\sin (2n+1)\theta}{\sin \theta}
   e^{-a\theta} \dmeasure \theta
   =&
   \lim_{m \rightarrow \infty}
   \int_{0}^{m \pi}
   \frac{\sin (2n+1)\theta}{\sin \theta}
   e^{-a\theta} \dmeasure \theta
   \\
   =&
   \lim_{m \rightarrow \infty}
   \int_{0}^{\pi}
   \frac{\sin (2n+1)\theta}{\sin \theta}
   \thebrace{
     e^{-a \theta}
     + e^{-a(\theta + \pi)}
     + \cdots
     + e^{-a(\theta + m \pi)}
   } \dmeasure \theta
   \\
   =&
   \int_{0}^{\pi}
   \frac{\sin (2n+1)\theta}{\sin \theta}
   \frac{e^{-a \theta} \dmeasure \theta}{1 - e^{-a\pi}},
 \end{align*}
 and use example l.] %TODO:ref
\end{wandwexample}
\begin{wandwexample}
  Discuss the uniformity of the convergence of Fourier series
  by means of the Dirichlet-Bonnct integrals, without making use of the
  theory of summability.
\end{wandwexample}
%
\Section{9}{5}{The TODO-TODO* theorem concerning Fourier constants.}
\emph{Let $f(x)$ be bounded in the interval $(-\pi, \pi)$ and let
$\int_{-\pi}^{\pi} f(x) \dmeasure x$ exist, so
%TODO:move into section header
* Math. Ann. lvii. (1903), p. 429. Liapounoff discovered the theorem
in 1896 and published it in the Proceedings of the Math. Soc. of the
Univ. of Kharkov. See Comptes Rendw, cxxvi. (1898), p. 1024.
%
% 181
%
that the Fourier constants $a_{n}, b_{n}$ of $f(x)$ exist. Then the series
$$
\half a_{0}^{2}
+
\sum_{n=1}^{\infty} \theparen{
  a_{n}^{2} + b_{n}^{2}
}
$$
is convergent and its sum is\footnote{This integral exists by
  \hardsubsectionref{4}{1}{2} example 1. %TODO:ref
  A proof of the theorem has been given by de la Valine Poussin, %TODO:accent(s)
  in which the sole restrictions on $f(x)$ are that the (improper) integrals of
  $f(x)$ and $\thebrace{f(x)}^{2}$ exist in the interval $(-\pi, \pi)$. See his
  Cours d' Analyse Infinitesimale, ii. (1912), pp. 165-166.}
$$
\frac{1}{\pi}
\int_{-\pi}^{\pi}
\thebrace{
  f(x)
}^{2}
\dmeasure x.
$$}.

It will first be shewn that, with the notation of \hardsectionref{9}{4},
$$
\lim_{m \rightarrow \infty}
\int_{-\pi}^{\pi}
\thebrace{
  f(x)
  -
  \frac{1}{m}
  \sum_{n=0}^{m-1}
  S_{n}(x)
}^{2}
\dmeasure x
=
0.
$$

Divide the interval $(-\pi, \pi)$ into $4r$ parts, each of length $\delta$;
let the upper and lower bounds of $f(x)$ in the interval
$\thebrace{(2p-1)\delta - \pi, (2p+3)\delta - \pi}$
be $U_{p}, L_{p}$, and let the upper bound of $\absval{f(x)}$ in the
interval $(-\pi, \pi)$ be $K$. Then, by
\hardsectionref{9}{4} corollary 1, %TODO:ref
\begin{align*}
  \absval{
    f(x)
    -
    \frac{1}{m}
    \sum_{m=0}^{m-1} S_{n}(x)
  }
  <&
  U_{p} - L_{p} + 2K / \thebrace{m \sin^{2} \half\delta}
  \\
  <&
  2K \thebracket{
    1 + 1/\thebrace{m \sin^{2} \half\delta}
  }
\end{align*}
when $x$ lies between $2p\delta$ and $(2p+2)\delta$.

Consequently, by the first mean-value theorem,
$$
\int_{-\pi}^{\pi}
\thebrace{
  f(x)
  -
  \frac{1}{m}
  \sum_{n=0}^{m-1} S_{n}(x)
}^{2}
\dmeasure x
<
2K
\thebrace{
  1
  +
  \frac{1}{m \sin^{2} \half\delta}
}
\thebrace{
  2\delta
  \sum_{p=0}^{2r-1}
  (U_{p}-L_{p})
  +
  \frac{4Kr}{m \sin^{2} \half \delta}
}
$$

Since $f(x)$ satisfies the Rieraann condition of integrability
(\hardsubsectionref{4}{1}{2}), it follows that both
$4\delta \sum_{p=0}^{r-1} (U_{2p}-L_{2p})$
and
$4\delta \sum_{p=0}^{r-1} (U_{2p+1}-L_{2p+1})$ can be made arbitrarily
small by giving $r$ a sufficiently large value.
When $r$ (and therefore also $\delta$) has been given
such a value, we may choose $m_{1}$ so large that
$r / \thebrace{m_{1} \sin^{2} \half\delta}$ is
arbitrarily small. That is to say, we can make the expression on the
right of the last iiiecpiality arbitrarily small by giving $m$ any value
greater than a determinate value $m_{1}$. Hence the expression on the left
of the inequality tends to zero as $m \rightarrow \infty$.

But evidently
\begin{align*}
  &
  \int_{-\pi}^{\pi}
  \thebrace{
    f(x)
    -
    \frac{1}{m}
    \sum_{n=0}^{m-1} S_{n}(x)
  }^{2} \dmeasure x
  \\
  &\quad
  \int_{-\pi}^{\pi}
  \thebrace{
    f(x)
    -
    \sum_{n=0}^{m-1}
    \frac{m-n}{m}
    A_{n}(x)
  }^{2} \dmeasure x
  \\
  &\quad
  \int_{-\pi}^{\pi}
  \thebrace{
    f(x)
    -
    \sum_{n=0}^{m-1}
    A_{n}(x)
    +
    \sum_{n=0}^{m-1}
    \frac{n}{m}
    A_{n}(x)
  }^{2} \dmeasure x
  \\
  &\quad
  \int_{-\pi}^{\pi}
  \thebrace{
    f(x)
    -
    \sum_{n=0}^{m-1}
    A_{n}(x)
  }^{2} \dmeasure x
  +
  \int_{-\pi}^{\pi}
  \thebrace{
    \sum_{n=0}^{m-1}
    \frac{n}{m}
    A_{n}(x)
  }^{2} \dmeasure x
  \\
  &\quad\quad
  +
  2
  \int_{-\pi}^{\pi}
  \thebrace{
    f(x)
    -
    \sum_{n=0}^{m-1}
    A_{n}(x)
  }
  \cdot
  \thebrace{
    \sum_{n=0}^{m-1}
    \frac{n}{m}
    A_{n}(x)
  } \dmeasure x
  \\
  &\quad
    \int_{-\pi}^{\pi}
    \thebrace{
      f(x)
      -
      \sum_{n=0}^{m-1}
      A_{n}(x)
    }^{2} \dmeasure x
    +
    \frac{\pi}{m^{2}}
    \sum_{n=0}^{m-1}
    n^{2} (a_{n}^{2} + b_{n}^{2}),
\end{align*}
%
% 182
%
since
$$
\int_{-\pi}^{\pi}
f(x) A_{r}(x) \dmeasure x
=
\int_{-\pi}^{\pi}
\thebrace{
  \sum_{n=0}^{m-1}
  A_{n}(x)
}
A_{r}(x)
\dmeasure x
$$
when $r = 0,1,2,\ldots,m-1$.

Since the original integral tends to zero and since it has been proved
equal to the sura of two positive expressions, it follows that each of
these expressions tends to zero; that is to say
$$
\int_{-\pi}^{\pi}
\thebrace{
  f(x)
  -
  \sum_{n=0}^{m-1}
  A_{n}(x)
}^{2} \dmeasure x
\rightarrow
0.
$$

Now the expression on the left is equal to
\begin{align*}
  \int_{-\pi}^{\pi}
  \thebrace{
    f(x)
  }^{2} \dmeasure x
  -&
  2
  \int_{-\pi}^{\pi}
  \thebrace{
    f(x)
    -
    \sum_{n=0}^{m-1}
    A_{n}(x)
  }
  \cdot
  \thebrace{
    \sum_{n=0}^{m-1}
    A_{n}(x)
  }
  \dmeasure x
  \\
  &\quad\quad
  \int_{-\pi}^{\pi}
  \thebrace{
    \sum_{n=0}^{m-1}
    A_{n}(x)
  }^{2}
  \dmeasure x
  \\
  =&
  \int_{-\pi}^{\pi}
  \thebrace{
    f(x)
  }^{2}
  \dmeasure x
  -
  \int_{-\pi}^{\pi}
  \thebrace{
    \sum_{n=0}^{m-1}
    A_{n}(x)
  }^{2}
  \dmeasure x
  \\
  =&
  \int_{-\pi}^{\pi}
  \thebrace{
    f(x)
  }^{2}
  \dmeasure x
  -
  \pi
  \thebrace{
    \half a_{0}^{2}
    +
    \sum_{n=1}^{m-1}
    (a_{n}^{2} + b_{n}^{2})
  }
  ,
\end{align*}
so that, as $m \rightarrow \infty$,
$$
\int_{-\pi}^{\pi}
\thebrace{
  f(x)
}^{2}
\dmeasure x
-
\pi
\thebrace{
  \half a_{0}^{2}
  +
  \sum_{n=1}^{m-1}
  (a_{n}^{2} + b_{n}^{2})
}
\rightarrow
0.
$$
This is the theorem stated.

% TODO:formatting
Corollary. Parseval's theorem\footnote{Mem. par divers senium, i. (1805), pp. 639-648.%TODO:ref
  Parseval, of course, assumed the permissibility of integrating the trigonometrical series
  term-by-term.}. If $f(x), F(x)$ both satisfy the
conditions laid on $f(x)$ at the beginning of this section, and if
$A_{n}, B_{n}$ be the Fourier constants of $F(x)$, it follows by subtracting the pair
of equations which may be combined in the one form
$$
\int_{-\pi}^{\pi}
\thebrace{
  f(x) \pm F(x)
}^{2}
\dmeasure x
=
\pi
\thebrace{
  \half (a_{0} \pm A_{0})^{2}
  +
  \sum_{n=1}^{\infty}
  ((a_{n} \pm A_{n})^{2}
  +
  (b_{n} \pm B_{n})^{2})
}
$$
that
$$
\int_{-\pi}^{\pi}
f(x)F(x)
\dmeasure x
=
\pi
\thebrace{
  \half a_{0}A_{0}
  +
  \sum_{n=1}^{\infty}
  (a_{n}A_{n} + b_{n}A_{n})
}
$$
\Section{9}{6}{Riemann's theory of trigonometrical series.}
The theory of Dirichlet concerning Fourier series is devoted to series
which represent given functions. Important advances in the theory were
made by Riemann, who considered properties of functions defined by a
series of the type\footnote{Throughout \hardsectionref{9}{6}--\hardsubsubsectionref{9}{6}{3}{2} %TODO:cite multiple
  the letters $a_{n}, b_{n}$ do not necessarily denote Fourier constants.}
$\half a_{0} + \sum_{n=1}^{\infty} (a_{n} \cos nx + b_{n} \sin nx)$
where it is assumed that
$\lim_{n \rightarrow \infty} (a_{n} \cos nx + b_{n} \sin nx) = 0$.
We shall give the propositions leading
up to Riemann's theorem\footnote{The proof given is due to G. Cantor, Journal fiir
  Math, lxxii. (1870), pp. 130-142.} %TODO:ref
that if two trigonometrical series converge and
are equal
%
% 183
%
at all points of the range $(-\pi, \pi)$ with the possible exception of a
finite number of points, corresponding coefficients in the two series
are equal.

\Subsection{9}{6}{1}{Riemann's associated function.}
Let the sum of the series
$\half a_{0} + \sum_{n=1}^{\infty} (a_{n} \cos nx + b_{n} \sin nx)
=
A_{0} + \sum_{n=1}^{\infty} A_{n}(x)$,
at any point $x$ where it converges, be denoted by $f(x)$.

Let
$$
F(x) = \half A_{0} x^{2} - \sum_{n=1}^{\infty} n^{-2} A_{n}(x).
$$

\emph{Then, if the series defining $f(x)$ converges at all points of any
finite interval, the series defining $F(x)$ converges for all real
values of $x$.}

To obtain this result we need the following Lemma due to Cantor :

Cantor's lemma\footnote{Riemann appears to have regarded this result as obvious.
  The proof here given is a modification of Cantor's proof,
  Math. Ann. iv. (1871), pp. 139-143, and Journal fiir Math, lxxii. (1870), pp. 130-138.}.%TODO:ref
\emph{If $\lim_{n \rightarrow \infty} A_{n}(x) = 0$ for all values of $x$ such that $a \leq x \leq b$,
then $a_{n} \rightarrow 0, b_{n} \rightarrow 0$.
}

For take two points $x, x+\delta$ of the interval. Then, given $\eps$, we can
find $n_{0}$ such that\footnote{The value of $n_{0}$ depends on $x$ and on $\delta$.},
when $n > n_{0}$
$$
\absval{
  a_{n} \cos nx + b_{n} \sin nx
}
<
\eps,
\quad
\absval{
  a_{n} \cos n(x+\delta) + b_{n} \sin n(x+\delta)
}
<
\eps.
$$

Therefore
$$
\absval{
  \cos n\delta
  (a_{n} \cos nx + b_{n} \sin nx)
  +
  \sin n\delta
  (-a_{n} \sin nx + b_{n} \cos nx)
}
<
\eps.
$$

Since
$$
\absval{
  \cos n\delta
  (a_{n} \cos nx + b_{n} \sin nx)
}
<
\eps,
$$
it follows that
$$
\absval{
  \sin n\delta
  (-a_{n} \sin nx + b_{n} \cos nx)
}
<
2\eps,
$$
and it is obvious that
$$
\absval{
  \sin n\delta
  (a_{n} \sin nx + b_{n} \cos nx)
}
<
2\eps.
$$

Therefore, squaring and adding,
$$
\sqrt{a_{n}^{2} + b_{n}^{2}}
\absval{ \sin n\delta }
<
2\eps\sqrt{2}
$$

Now suppose that $a_{n}, b_{n}$ have not the unique limit $0$; it will be shewn
that this hypothesis involves a contradiction. For, by this
hypothesis, \emph{some} positive number $\eps_{0}$ exists such that there is an
unending increasing sequence $n_{1}, n_{2}, \ldots$ of values of $n$, for which
$$
\sqrt{a_{n}^{2} + b_{n}^{2}} > 4\eps_{0}.
$$

Now let the range of values of $\delta$ be called the interval $I_{1}$ of length
$L_{1}$ on the real axis.

Take $n_{1}'$ the smallest of the integers $n_{r}$ such that
$n_{1}' L_{1} > 2\pi$; then
$\sin n_{1}'y$ goes through all its phases in the interval $I$; call
$I_{2}$
that sub-interval\footnote{If there is more than one such sub-interval,
  take that which lies on the left.}
of $I_{1}$ in which $\sin n_{1}'y > 1/\sqrt{2}$; its length is
$\pi / (2n_{1}') = L_{2}$. Next take $n_{2}'$ the smallest of the integers
$n_{r} (> n_{1}')$
such that $n_{2}' L_{2} > 2\pi$, so that
$\sin n_{2}'y$ goes through all its phases in
the interval $I_{2}$; call $I_{3}$ that sub-interval%TODO:repeat previous footnote
of $I_{2}$ in which $\sin n_{2}'y > 1/\sqrt{2}$;
its length is $\pi / (2 n_{2}') = L_{3}$. We thus get a sequence of
decreasing intervals $I_{1}, I_{2},\ldots$ each contained in all the previous
ones. It is obvious from the definition of an irrational number that
there is a certain point $a$ which is not outside any of these
intervals, and $\sin na \geq 1/\sqrt{2}$ when
$n=n_{1}', n_{2}', \ldots (n_{r+1}' > n_{r}')$.
For these values of $n$,
$\sqrt{a_{n}^{2} + b_{n}^{2}} \sin na > 2\eps_{0} \sqrt{2}$.
But it has been shewn that
corresponding
%
% 184
%
to given numbers $a$ and $\eps$ we can find $n_{0}$ such that when
$n > n_{0}$,
$\sqrt{a_{n}^{2} + b_{n}^{2}} \sin na < 2\eps_{0} \sqrt{2}$;
since some values of $n_{r}'$ are greater than $n_{0}$, the
required contradiction has been obtained, because we may take
$\eps < \eps_{0}$; therefore
$a_{n} \rightarrow 0, b_{n} \rightarrow 0$.

Assuming that the series defining $f(x)$ converges at all points of a
certain interval of the real axis, we have just seen that
$a_{n} \rightarrow 0, b_{n} \rightarrow 0$. Then, for all real values of
$x$,
$\absval{a_{n} \cos nx + b_{n} \sin nx} \leq \sqrt{a_{n}^{2} + b_{n}^{2}} \rightarrow 0$,
and so, by \hardsubsectionref{3}{3}{4}, the
series
$\half A_{0} x^{2} - \sum_{n=1}^{\infty} n^{-2} A_{n}(x) = F(x)$
converges absolutely and uniformly for
all real values of $x$; therefore, (\hardsubsectionref{3}{3}{2}), $F(x)$ is continuous for all
real values of $x$.

\Subsection{9}{6}{2}{Properties of Riemann's associated function;
  Riemann's first lemma.}
It is now possible to prove Riemann's first lemma \emph{that if
  $$
  G(x, \alpha)
  =
  \frac{F(x + 2\alpha) + F(x - 2\alpha) - 2F(x)}{4 \alpha^{2}},
  $$
  then $\lim_{\alpha \rightarrow 0} G(x,\alpha) = f(x)$, provided that
  $\sum_{n=0}^{\infty} A_{n}(x)$ converges for the value
  of $x$ under consideration.}

Since the series defining $F(x), F(x \pm 2\alpha)$ converge absolutely, we may
rearrange them; and, observing that
\begin{align*}
  \cos n(x + 2\alpha) + \cos n(x - 2\alpha) - 2 \cos nx =& -4 \sin^{2} n\alpha \cos nx,\\
  \sin n(x + 2\alpha) + \sin n(x - 2\alpha) - 2 \sin nx =& -4 \sin^{2} n\alpha \sin nx,
\end{align*}
it is evident that
$$
G(x, \alpha)
=
A_{0}
+
\sum_{n=1}^{\infty}
\theparen{
  \frac{\sin n\alpha}{n\alpha}
}^{2}
A_{n}(x).
$$

It will now be shewn that this series converges uniformly with regard
to $\alpha$ for all values of $\alpha$, provided that
$\sum_{n=1}^{\infty} A_{n}(x)$ converges. The result
required is then an immediate consequence of \hardsubsectionref{3}{3}{2}:
for, if
$f_{n}(\alpha) = \theparen{
  \frac{\sin n\alpha}{n\alpha}
}^{2}, (\alpha \neq 0)$
and $f_{n}(0) = 1$, then $f_{n}(\alpha)$ is continuous for all values of $\alpha$, and so
$G(x,\alpha)$ is a continuous function of $\alpha$, and therefore, by \hardsectionref{3}{2},
$G(x, 0) = \lim_{\alpha \rightarrow \infty} G (x, \alpha)$.

To prove that the series defining $G(x, \alpha)$ converges uniformly, we
employ the test given in \hardsubsectionref{3}{3}{5} example 2. %TODO:ref
The expression corresponding to $\omega_{n}(x)$ is
$f_{n}(\alpha)$, and it is obvious that
$\absval{f_{n}(\alpha)} \leq 1$; it is therefore sufficient to shew
that
$\sum_{n=1}^{\infty} \absval{f_{n+1}(\alpha) - f_{n}(\alpha)} < K$,
where $K$ is independent of $\alpha$.

%\begin{smallfont}
In fact\footnote{Since $x^{-1} \sin x$ decreases as $x$ increases from $0$ to $\pi$.},
if $s$ be the integer such that
$a \absval{\alpha} \leq \pi < (s+1) \absval{\alpha}$,
when $\alpha \neq 0$ we have
$$
\sum_{n=1}^{s-1}
\absval{
  f_{n+1}(\alpha) - f_{n}(\alpha)
}
=
\sum_{n=1}^{s-1}
(
f_{n}(\alpha)
??? %TODO
f_{n+1}(\alpha)
)
=
\frac{ \sin^{2} \alpha }{\alpha^{2}}
-
\frac{ \sin^{2} s\alpha }{s^{2} \alpha^{2}}.
$$
%
% 185
%

Also
\begin{align*}
  \sum_{n=s+1}^{\infty}
  \absval{
    f_{n+1}(\alpha) - f_{n}(\alpha)
  }
  =&
  \sum_{n=s+1}^{\infty}
  \absval{
    \thebrace{
      \frac{\sin^{2} n\alpha}{\alpha^{2}}
      \theparen{
        \frac{1}{n^{2}}
        -
        \frac{1}{(n+1)^{2}}
      }
    }
    +
    \frac{\sin^{2} n\alpha - \sin^{2} (n+1)\alpha }{(n+1)^{2} \alpha^{2}}
  }
  \\
  \leq &
  \sum_{n=s+1}^{\infty}
  \frac{1}{\alpha^{2}}
  \theparen{
    \frac{1}{n^{2}}
    -
    \frac{1}{ (n+1)^{2} }
  }
  +
  \sum_{n=s}^{\infty}
  \frac{\absval{
      \sin^{2} n\alpha - \sin^{2} (n+1)\alpha
    }}{ (n+1)^{2} \alpha^{2} }
  \\
  \leq &
  \frac{1}{ (s^{2} + 1)^{2} \alpha^{2} }
  +
  \sum_{n=s+1}^{\infty}
  \frac{\absval{
      \sin \alpha \sin (2n+1)\alpha
    }}{ (n^{2} + 1)^{2} \alpha^{2} }
  \\
  \leq &
  \frac{1}{ (s^{2} + 1)^{2} \alpha^{2} }
  +
  \frac{\absval{\sin \alpha}}{\alpha^{2}}
  \sum_{n=s+1}^{\infty}
  \frac{1}{(n+1)^{2}}
  \\
  \leq &
  \frac{1}{\pi^{2}}
  +
  \frac{\absval{\sin \alpha}}{\alpha^{2}}
  \int_{s}^{\infty}
  \frac{\dmeasure x}{(x+1)^{2}}
  \\
  \leq &
  \frac{1}{\pi^{2}}
  +
  \frac{1}{(s+1) \alpha}.
\end{align*}

Therefore
\begin{align*}
  \sum_{n=1}^{\infty}
  \absval{
    f_{n+1}(\alpha) - f_{n}(\alpha)
  }
  \leq &
  \frac{\sin^{2} \alpha}{\alpha^{2}}
  -
  \frac{\sin^{2} s\alpha}{s^{2}\alpha^{2}}
  +
  \theparen{
    \frac{\sin^{2} s\alpha}{s^{2}\alpha^{2}}
    +
    \frac{\sin^{2} (s+1)\alpha}{(s+1)^{2}\alpha^{2}}
  }
  + \frac{1}{\pi^{2}} + \frac{1}{\pi}.
  \\
  \leq &
  1 + \frac{1}{\pi} + \frac{2}{\pi^{2}}.
\end{align*}

Since this expression is independent of $\alpha$, the result required has
been obtained\footnote{This inequality is obviously true when $\alpha = 0$.}.
%\end{smallfont}

Hence, if $\sum_{n=0}^{\infty} A_{n}(x)$ converges, the series defining $G(x,\alpha)$
converges uniformly with respect to $\alpha$ for all values of $\alpha$, and, as stated above,
$$
\lim_{\alpha \rightarrow \infty} G(x, \alpha)
= G(x,0)
= A_{0} + \sum_{n=1}^{\infty} A_{n}(x)
= f(x).
$$

\begin{wandwexample}
  If
  $H(x, \alpha, \beta) = \frac{F(x+\alpha+\beta) ? %TODO
    F(x+\alpha-\beta) ? %TODO
    F(x-\alpha+\beta) +
    F(x-\alpha-\beta)
  }{4 \alpha \beta}$
  shew that $H(x,\alpha,\beta) \rightarrow f(x)$ when $f(x)$ converges if
  $\alpha,\beta \rightarrow 0$ in such a
  way that $\alpha/\beta$ and $\beta/\alpha$ remain finite.
  \addexamplecitation{Riemann.}
\end{wandwexample}
\Subsubsection{9}{6}{2}{1}{Riemann's second lemma.}
\emph{With the notation of
\hardsectionref{9}{6}--\hardsubsectionref{9}{6}{2}, %TODO:refmultiple
if
$a_{n}, b_{n} \rightarrow 0$, then
$\lim_{\alpha \rightarrow 0} \frac{F(x+2\alpha)+F(x-2\alpha)-2 F(x)}{4\alpha}$ for all values of $x$.
}

For
$\frac{1}{4} \alpha^{-1} \thebrace{ F(x + 2\alpha) + F(x - 2\alpha) - 2 F(x)
  =
  A_{0} \alpha + \sum_{n=1}^{\infty} \frac{\sin^{2} n\alpha}{n^{2}\alpha} A_{n}(x)
}$; but by \hardsubsectionref{9}{1}{1} example 3, %TODO:example ref
if $\alpha > 0$,
$\sum_{n=1}^{\infty} \frac{\sin^{2} n\alpha}{n^{2}\alpha} = \half (\pi - \alpha)$; and so,
since
\begin{align*}
  &
  A_{0}(x) \alpha
  +
  \sum_{n=1}^{\infty}
  \frac{\sin^{2} n\alpha}{n^{2} \alpha} A_{n}(x)
  \\ &
  A_{0}(x) \alpha
  +
  \half (\pi - \alpha) A_{1}(x)
  +
  \sum_{n=1}^{\infty}
  \thebrace{
    \half (\pi - \alpha)
    -
    \sum_{m=1}^{n}
    \frac{\sin^{2} m\alpha}{m^{2} \alpha}
  } \thebrace {
    A_{n+1}(x) - A_{n}(x)
  },
\end{align*}
it follows from \hardsubsectionref{3}{3}{5} example 2, %TODO:ref example
that this series converges uniformly
with regard to $\alpha$ for all values of $\alpha$ greater than, or equal to,
zero\footnote{If we define $g_{n}(\alpha)$ by the equations
  $g_{n}(\alpha) = \half (\pi - \alpha) - \sum_{m=1}^{n} \frac{\sin^{2} m\alpha}{m^{2} \alpha}, (\alpha \neq 0)$,
  and $g_{n}(0) = \half \pi$, then $g_{n}(\alpha)$ is continuous when
  $\alpha \geq 0$, and $g_{n+1}(\alpha} \leq g_{n}(\alpha)$.
}
%
% 186
%

But
\begin{align*}
  &
  \lim_{\alpha \rightarrow +0} \alpha^{-1}
  \thebrace{
    F(x + 2\alpha) + F(x - 2\alpha) - 2 F(x)
  }
  \\
  &
  \lim_{\alpha \rightarrow +0}
  \thebracket{
    A_{0}(x) \alpha
    + \half (\pi - \alpha) A_{1}(x)
    + \sum_{n=1}^{\infty} g_{n}(\alpha)
    \thebrace{
      A_{n+1}(x) - A_{n}(x)
    }
  },
\end{align*}
and this limit is the value of the function when $\alpha = 0$, by
\hardsubsectionref{3}{3}{2};
and this value is zero since $\lim_{n \rightarrow \infty} A_{n}(x)$.
By symmetry we see that
$\lim_{\alpha \rightarrow +0} = \lim_{\alpha \rightarrow -0}$.

\Subsection{9}{6}{3}{Riemann's theorem\footnote{
    The proof we give is due to Gr. Cantor, Journal fit r Math, lxxii.
    (1870), pp. 139-142. + Quoted by G. Cantor, Journal filr Math, i.xxii.
    (1870).}
on trigonometrical series.}
\emph{Two trigonometrical series which converge and are equal at all points
  of the range $(-\pi, \pi)$, with the possible exception of a finite
  number of points, must have corresponding coefficients equal.}
%\begin{smallfont}
An immediate deduction from this theorem is that a function of the
type considered in \hardsubsectionref{9}{4}{2} cannot be represented
by any trigonometrical series in the range $(-\pi, \pi)$ other than
the Fourier series. This fact was first noticed by Du Bois Reymond.

We observe that it is certainly possible to have other expansions of
(say) the form
$$
\alpha_{0}
+
\sum_{m=1}^{\infty} \theparen{
  \alpha_{m} \cos \half mx
  +
  \beta_{m} \sin \half mx
},
$$
which represent $f(x)$ between $-\pi$ and $\pi$; for write
$x = 2\xi$, and
consider a function $\phi(\xi)$, which is such that
$\phi(\xi) = f(2\xi)$ when $-\half\pi < \xi < \half\pi$, and
$\phi(\xi) = g(\xi)$ when $-\pi < \xi < -\half \pi$, and when
$\half \pi < \xi < \pi$,
where $g(\xi)$ is any function satisfying the conditions of
\hardsubsectionref{9}{4}{3}. Then if we expand
$\phi(\xi)$ in a Fourier series of the form
$$
\alpha_{0}
+
\sum_{m=1}^{\infty} \theparen{
  \alpha_{m} \cos m \xi
  +
  \beta_{m} \sin m \xi
},
$$
this expansion represents $f(x)$ when $-\pi < x < \pi$; and clearly by
choosing the function $g(\xi)$ in different ways an unlimited number of
such expansions can be obtained.

The question now at issue is, whether other series proceeding in
sines and cosines of \emph{integral} multiples of $x$ exist, which differ from
Fourier's expansion and yet represent $f(x)$ between $-\pi$ and $\pi$.
%\end{smallfont}

If possible, let there be two trigonometrical series satisfying the
given conditions, and let their difference be the trigonometrical
series
$$
A_{0}
+
\sum_{n=1}^{\infty} A_{n}(x)
=
f(x).
$$

Then $f(x) = 0$ at all points of the range $(-\pi, \pi)$ w th a finite number
of exceptions; let $\xi_{1}, \xi_{2}$ be a consecutive pair of these exceptional
points, and let $F(x)$ be Riemann's associated function. We proceed to establish a
lemma concerning the value of $F(x)$ when $\xi_{1} < x < \xi_{2}$.

\Subsubsection{9}{6}{3}{1}{Schwartz' lemma.}\footnote{Quoted by TODO} \emph{In the range $\xi_{1} < x < \xi_{2}$,
  $F(x)$ is a linear function of $x$, if $f(x) = 0$ in this range.}

For if $\theta = 1$ or if $\theta = -1$
$$
\phi(x)
=
\theta \thebracket{
  F(x)
  - F(\xi_{1})
  - \frac{x - \xi_{1}}{\xi_{2} - \xi_{1}} \thebrace{
    F(\xi_{2}) - F(\xi_{1})
  }
}
-
\half h^{2} (x - \xi_{1})(\xi_{2} - x)
$$
is a continuous function of $x$ in the range
$\xi_{1} \leq x \leq \xi_{2}$ and $\phi(\xi_{1}) = \phi(\xi_{2}) = 0$.
%
% 187
%

If the first term of $\phi(x)$ is not zero throughout the
range\footnote{If it is zero throughout the range, $F(x)$
  is a linear function of $x$.}
there will be some point $x=c$ at which it is not zero. Choose the sign of $\theta$
so that the first term is positive at $c$, and then choose $h$ so small
that $\phi(c)$ is still positive.

Since $\phi(x)$ is continuous it attains its upper bound (\hardsubsectionref{3}{6}{2}), and
this upper bound is positive since $\phi(c) > 0$. Let $\phi(x)$ attain its
upper bound at $c_{1}$, so that $c_{1} \neq \xi_{1}, c_{1} \neq \xi_{2}$.

Then, by Riemann's first
lemma,
$$
\lim_{\alpha \rightarrow 0}
\frac{\phi(c_{1} + \alpha) + \phi(c_{1} - \alpha) - 2 \phi(c_{1})}{\alpha^{2}}
=
h^{2}.
$$

But $\phi(c_{1} + \alpha) \leq \phi(c_{1}), \phi(c_{1} - \alpha) \leq \phi(c_{1})$, so this limit must be
negative or zero.

Hence, by supposing that the first term of $\phi(x)$ is not everywhere zero
in the range $(\xi_{1}, \xi_{2})$ we have arrived at a contradiction. Therefore it
is zero; and consequently $F(x)$ is a linear function of $x$ in the range
$\xi_{1} < x < \xi_{2}$
The lemma is therefore proved.

\Subsubsection{9}{6}{3}{2}{Proof of Riemann's Theorem.}
We see that, in the circumstances under consideration, the curve $y = F(x)$
represents a series of segments of straight lines, the beginning
and end of each line corresponding to an exceptional point; and as $F(x)$,
being uniformly convergent, is a continuous function of $x$, these
lines must be connected.

But, by Riemann's second lemma, even if $\xi$ be an exceptional point,
$$
\lim_{\alpha \rightarrow 0}
\frac{F(\xi + \alpha) + F(\xi - \alpha) - 2F(\xi)}{\alpha}
=
0.
$$

Now the fraction involved in this limit is the difference of the
slopes of the two segments which meet at that point whose abscissa is
$\xi$; therefore the two segments are continuous in direction, so the
equation $y = F(x)$ represents a single line. If then we write
$F(x) = cx + c'$, it follows that $c$ and $c'$ have the same values for
\emph{all} values of $x$. Thus
$$
\half A_{0} x^{2} - cx - c'
=
\sum_{n=1}^{\infty} n^{-2} A_{n}(x),
$$
the right-hand side of this equation being periodic, with period $2\pi$.

The left-hand side of this equation must therefore be periodic, with
period $2\pi$. Hence
$$
A_{0} = 0,
\quad
c = 0,
$$
and
$$
-c' = \sum_{n=1}^{\infty} n^{-2} A_{n}(x).
$$

Now the right-hand side of this equation converges uniformly, so we
can multiply by $\cos nx$ or by $\sin nx$ and integrate.

This process gives
\begin{align*}
  \pi n^{-2} \alpha_{n}
  =&
  - c' \int_{-\pi}^{\pi} \cos nx \dmeasure x = 0,
  \\
  \pi n^{-2} b_{n}
  =&
  - c' \int_{-\pi}^{\pi} \sin nx \dmeasure x = 0.
\end{align*}
%
% 188
%
Therefore all the coefficients vanish, and therefore the two
trigonometrical series whose difference is
$A_{0} + \sum_{n=1}^{\infty} A_{n}(x)$ have
corresponding coefficients equal.
This is the result stated in \hardsubsectionref{9}{6}{3}.

\Section{9}{7}{Fourier's representation of a function by an integral.}\footnote{TODO:La Theorie Analytique de la Chaleur, Cb. ix.}
It follows from \hardsubsectionref{9}{4}{3} that, if $f(x)$ be continuous except at a
finite number of discontinuities and if it have limited total
fluctuation in the range $(-\infty, \infty)$, then, if $x$ be any
\emph{internal} point of the range $(-\alpha, \beta)$,
$$
\lim_{m \rightarrow \infty}
\int_{-\alpha}^{\beta}
\frac{\sin (2m + 1)(t - x)}{t - x}
f(t) \dmeasure t
=
\lim_{\theta \rightarrow 0}
\half \pi \theta^{-1} \sin \theta
\thebrace{
  f(x + 2\theta) + f(x - 2 \theta)
}.
$$

Now let $\lambda$ be any real number, and choose the integer $m$ so that
$\lambda = 2m + 1 + 2\eta$ where $0 \leq \eta < 1$.

Then
\begin{align*}
  &
  \int_{-\alpha}^{\beta}
  \thebrace{
    \sin \lambda(t - x) - \sin (2m+1)(t-x)
  }
  (t - x)^{-1} f(t) \dmeasure t
  \\
  &\quad
  \int_{-\alpha}^{\beta}
  2 \thebrace{
    \cos (2m + 1 + \eta)(t-x)
  } \cdot \thebrace{
    \sin \eta (t-x)
  }
  (t - x)^{-1} f(t) \dmeasure t
  \\
  &\quad
  \rightarrow 0,
\end{align*}
as $m \rightarrow \infty$ by
\hardsubsectionref{9}{4}{1} (ii), %TODO:ref
since $(t - x)^{-1} f(t) \sin \eta(t - x)$ has
limited total fluctuation.

Consequently, from the proof of the Riemann-Lebesgue lemma of \hardsubsectionref{9}{4}{1},
it is obvious that if $\int_{0}^{\infty} \absval{f(t)} \dmeasure t$ and
$\int_{-\infty}^{0} \absval{f(t)} \dmeasure t$ converge, then\footnote{$\int_{-\infty}^{\infty}$ means the double limit
  $\lim_{\phi\rightarrow\infty, \sigma\rightarrow\infty} \int_{-\rho}^{\sigma}$. If this limit exists, it is of course
  equal to $\lim_{\rho\rightarrow\infty}\int_{-\rho}^{\rho}$.}
$$
\lim_{\lambda\rightarrow\infty}
\int_{-\infty}^{\infty}
\frac{\sin \lambda(t-x)}{(t-x)}
f(t) \dmeasure t
=
\half
\pi
\thebrace{
  f(x+0) + f(x-0)
},
$$
and so
$$
\lim_{\lambda\rightarrow\infty}
\int_{-\infty}^{\infty}
\thebrace{
  \int_{0}^{\lambda}
  \cos u(t-x) \dmeasure u
}
f(t) \dmeasure t
=
\half
\pi
\thebrace{
  f(x+0) + f(x-0)
},
$$

To obtain Fourier's result, we must reverse the order of integration
in this repeated integral.

For any given value of $\lambda$ and any arbitrary value of $\eps$, there exists a
number $\beta$ such that
$$
\int_{\beta}^{\infty}
\absval{f(t)}
\dmeasure t
<
\half \eps / \lambda;
$$
%
% 189
%
writing $\cos u(t-x) \cdot f(t) = \phi(t,u)$, we
have\footnote{The equation
  $\int_{0}^{\beta} \int_{0}^{\lambda} = \int_{0}^{\lambda} \int_{0}^{\beta}$
  is easily justified by \hardsectionref{4}{3}, by considering the ranges within
  which $f(x)$ is continuous.}
\begin{align*}
  \absval{
    \int_{0}^{\infty} \thebrace{
      \int_{0}^{\lambda} \phi(t,u) \dmeasure u
    } \dmeasure t
    -
    \int_{0}^{\lambda} \thebrace{
      \int_{0}^{\infty} \phi(t,u) \dmeasure t
    } \dmeasure u
  }
  \\
  \quad &
  =
  \int_{0}^{\beta}
  \thebrace{
    \int_{0}^{\lambda} \phi(t,u) \dmeasure u
  } \dmeasure t
  +
  TODO
\end{align*}

Since this is true for all values of e, no matter how small, we infer
that /oo rk r\ foD /- Qo r\ r\ r- x

=; similarly = .

J a J J J I) J i) Jo J . )

Hence | ir [f x + 0) +/(./; - 0) = lim | cos u (t - oc)f t) dtdu

=

A- . JO. -00

00;" 00

COS u t - oe)f t) dtdu. .' -

This result is known as Fourier s integral tlieorem. Example. Verify
Fourier's integral theorem dinx-tlj (i) fur the function

(ii) for the function defined by the equations

/(.r)=l, -\ < x< ); fix) = 0, (|x|>l). (Rayleigh.j

REFERENCES. G. F. B. Riemann, Ges. Math. Werke, pp. 213-250. E. W.
HoBSON, Functions of a Real Variable (1907), Ch. vii. H. Lebesgde,
Lemons sur les Series Trigonometriques. (Paris, 1906.) C. J. DE LA
VallJ e Poussix, Cours d: Analyse Infinitesimale, ir. (Louvain and
Paris,

1912), Ch. IV.

H. Burkhardt, Ena/clopddie der Math. WUg. ii. 1 (7). (Leipzig, 1914.)

G. A. Carse and G. Shearer, A course in Fourier's analysis and
periodogram analysis

(Edinburgh Math. Tracts, No. 4, 1915).

t For a proof of the theorem when f(x] is subject to less stringent
restrictions, see Hobson, Functions of a Real Variable, %% 492-493.
The reader should observe that, although

Um I I exists, the repeated integral I \ I sin m (( -x) duj- /(<) dt
does not,

A-*oo y -00 7 7 -00 U J

%
% 190
%

Miscellaneous Examples.

1. Obtain the expansions

  ' l-2rcos2 + r2

(6) - log ( 1 - 2r cos z + r ) = - r cos J - '' cos 2 - - r cos 3 - .
. .,

,,, ?'sinz . 1 . 1 . \,

(c) arc tan, = rsins + -?' sin 22 + ?: /' sin 3s + ...,

   r cos 2 2 3

,, 2rsin2 . I,  1 c

(a) arc tan 3- = r sin z- -r sin 3 + - r° sm 02 + . . .,

1 - r'' 3 5

and shew that, when | r | < 1, thej- are convergent for all values of
z in certain strips parallel to the real axis in the -plane.

2. Expand x and x in Fourier sine series valid when - \pi < .r < \pi;
and hence find the value of the sum of the series

sin X - sin 2.r + -3 sin 3*' - - sin 4.r + . . ., for all values of
a.-. \addexamplecitation{Jesus, 1902.}

3. Shew that the function of x I'epresented by 2 ~i sin sin /2a, is
constant

n = \

(0 < X < 2a) and zero (2a < x < \pi), and draw a graph of the function.

\addexamplecitation{Pembroke, 1907.}

4. Find the cosine series representing /(.r) where

/ (x) = sin X + cos X (0 < . ' \pi ),

/(.r) = sin.r-cos.r (in- .r < \pi). \addexamplecitation{Peterhouse, 1906.}

5. Shew that

sin 3n- sinSjTA' sin Ttt., r -,

sin TTX + - - + - - + - - - + ... = Jtt [x],

where [x] denotes 4-1 or - 1 according as the integer next inferior to
x is even or uneven, and is zero if x is an integer. \addexamplecitation{Trinity, 1895.}

6. Shew that the expansions

log and

1

2 cos - X 2

log 2 .sin - X

= cos X-- cos 2x + - cos 3x

- COS X - - COS 2x - - cos 3x

are valid for all real A'alues of x, except multiples of \pi.

7. Obtain the expansion

"(-)"* cos m.r, o M / 1 \ 1 /  T  \

2 ~ - -yT-. = (cos X + cos 2x) log ( 2 cos -x\ +-,:?; (sin zx + sin x)
- cos x,

and find the range of values of . ' for which it is applicable.
\addexamplecitation{Trinity, 1898.}

8. Prove that, if < a- < 27r, then

sin X 2 sin 2x Z sin 3.:*; \ \pi sinh a (\pi - x) a' + Y' " a +2- "a T +
' ' " " 2 ' sinh aTr '

\addexamplecitation{Trinity, 1895.}

FOURIER SERIES

%
% 191
%

9. Shew that between the values -\pi and +n o( x the following
expansions hold

2 . / sin 2 sin 2x 3 sin Zx

sin in.v = - sin ran -. r - -, - + -= - - ...

IT \ V-m'' -l -m 32-7 2

2 . ( \ m cos X m cos 2.r m cos 3 ' cos mx = - sin ??l7r;;- + - - -r
+ - 5 -

TT \ 2m P - 7rt- 22-77l2 32-J/l2

e-mx g-mx 2 / 1 ??IC0S.¥ ?/l COS 2.r 7?2C0S3.1- \

10. Let X be a real variable V)etween and 1, and let n be an odd
number 3. Shew that

(-1)* = - + - 2 -tan - cos2OTn-:r, if is not a multiple of -, where s
is the gi-eatest integer contained in nx; but

 1 2 1 WITT

0= - + - 2 - tan - cos 2m7rx, n 7r, =i i n

if jp is an integer multiple of 1/n. \addexamplecitation{Berger.}

11. Shew that the sum of the series

X

 5 + 47r ~ 2 m~ sin mn cos 2m7rx

m = 1

is 1 when 0<x<l, and when fj<A'<l, and is -1 when l<.r<|.
\addexamplecitation{Trinity, 1901.}

12. If

shew that, when - 1 <x <\,

a" V (x)

ae"'

e" - 1 =o n

cos 2irx+

s47r.r cosB\pij: \ \ -i 2" ""*' " i-

22"

32

2/i !

sin 47ra; sin 6nx "2-" + 1 "*" "32

. -, 8IU 'iTTX Sin OTTX-, .,, -,. .,, .

sm27ra;+-j; +-; +. .. = (-)"+! \ \,, V,,, x).

22" 2 + l

2 + 1 :

\addexamplecitation{Math. Trip. 1896.}

13. If 7?i is an integer, shew that, for all real values of .v,

cos2' a;=2

1.3.5...(2m-

2.4.6 ...2m

1) fl, r,i (2 i + l

m m- )

cos 2x + - - ~ cos 4x

(ot4-1)(7 + 2)

m m-l) m - 2) * (m + l)(?\pi + 2)(w+3)

cos 6.r + ...V,

  \ ., 4 2.4.6...(27ft-2) (1 2w-l . f2m-l)(27n-3),,

r.ns2m-l\;p|\ \ - \ - \ \ \ \ J -!;; + ., COS 2x+-,,,, -;
C0S4a-+.

1.3.5 ...(2 i-l) (2 2/u + l

(2to + 1)(2 h-3)

/

14. A point moves in a straight line with a velocity which is
initially u, and which receives constant increments, each equal to u,
at equal intervals r. Prove that the velocity at any time t after the
beginning of the motion is

u ut V- 1 . 2miit 77 + - + - 2 - sin,

and that the di-stance traversed is

lit

( + ) + T7>-o32 2 -2 cos

2mTTt

\addexamplecitation{Trinity, 1894.}

%
% 192
%

15. If

X 2 = 1

f x)= 2 sin (6?i - 3) j; - 2 2 -sm 2n-l)x

ji=i '2n - 1 71=1 zn - i

3 v'3 f . sin 5.r sin 7.r sin ll r ]

shew that / ( + 0) =/ (\pi - 0) = - \pi,

and /G +0)-/(i7r-0)=-i7r, /(;:|7r+0)-/(37r-0)=*7r.

Observing that the last series is

6 sin (2ft- 1 ) TTsin 2n-l)x

draw the graph of /(* ) \addexamplecitation{Math. Trip. 1893.}

16. Shew that, when <.r < n,

 / N 2 /3/ 1 .1 1 Ti, \

f(,v) = '- I cos .r-- cos 0. '+- cos /.<--- cos 11. r,'-f... ' ' 3 \
/ 11 /

= sin 2x' + - sin 4.f + - sin 8.v + r sin lO.r + . . . 2 4 5

where / ) = i (0< < 7r),

f x) = (l7r<.r<g7r),

f x)=-\ v %7T<X<1t).

Find the sum of each series when .r = 0, \ ir, qtt, n, and for all
other values of x.

\addexamplecitation{Trinity, 1908.}

17. Prove that the locus represented by

2 2 - '' " '*'' ny =

n=l "

is two systems of lines at right angles, dividing the coordinate plane
into squares of area \pi'\ \addexamplecitation{Math. Trip. 1895.}

18. Shew that the equation

" ( - )" ~ sin ny cos nx \

n=i n ~

represents the lines i/= ±mir, (wi = 0, 1, 2, ...) together with a set
of arcs of ellipses whose

semi-axes are \pi and nlslZ, the arcs being placed in squares of area
27r" . Draw a diagram

of the locus. \addexamplecitation{Trinity, 1903.}

19. Shew that, if tlie point x,y, z) lies inside the octahedron
bounded by the planes

±x±y±z = Tr, then

",,, sin 7ia;sin 7jwsin wj 1

 j-Y- - =2-y-

\addexamplecitation{Math. Trip. 1904.}

20. Circles of radius a are diawn having their centres at the
alternate angular jjoints of a regular hexagon of side a. Sliew that
the equation of the trefoil formed by the outer arcs of the circles
can be put in the form

-nr- = -; + - cos 3(9- --= cos 6 + -- - cos9 - ..., 6s'3a 22.4 5.7 8.
10

the initial line being taken to pass through the centre of one of the
circles.

\addexamplecitation{Pembroke, 1902.}

2m . = 1 H sm

%
% 193
%

21. Draw the graph represented by

rr J 1 ( - )" COS nm6\ m [2 a=i 1 - nm)- J ' where m is an integer.
\addexamplecitation{Jesus, 1908.}

22. With each vertex of a regular hexagon of side 2a as centre the arc
of a circle of radius 2a lying within the hexagon is drawn. Shew that
the equation of the figure formed by the six arcs is

f-=6-3,.i + 2Si< -)-' j- f co.6 <>,

4a a=i 6n-l) bn+l)

the prime vector bisecting a petal. \addexamplecitation{Trinity, 1905.}

23. Shew that, if c> 0,

lim f

e ' cot A" sin 2ii + l)x . dx=- n tanh -c\pi.

\addexamplecitation{Trinity, 1894.}

24. Shew that

sin(2 + l) dx 1

lim r

IT O

oth 1.

\addexamplecitation{King's, 1901.}

sin X l+x' 2

25. Shew that, when - 1 < .r < 1 and a is real,

/ '°sin(2n + l) sin(H-.r)(9, 1 sinha.r

hm I -.;; 5 -r, d6= --TT .-, .

  .xjo sm a + 6- 2 smh a

\addexamplecitation{Math. Trip. 1905.}

26. Assuming the possibility of expanding f(x) in a uniformly
convergent series of

the form A -ainkx, where is a root of the equation /-cosa/t-Hft sin a
- = and the

  . . . *

summation is extended to all positive roots of this equation,
determine the constants Jj..

\addexamplecitation{Math. Trip. 1898.}

1 ""

27. If f(x) = -a + 2 a cosnx + b,t>im nx)

2,(=1

is a Fourier series, shew that, iif x) satisfies certain general
conditions,

a = - P i f (t) cos nt ta.n - t --, 6 = - / f t)smnttn,n-t - .

7T J ' 'It ""yO 'It

\addexamplecitation{Beau.}

W. M. A. 13
