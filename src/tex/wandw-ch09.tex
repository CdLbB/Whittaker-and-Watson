%
% 160
%
\chapter{Fourier Series and Trigonometrical Series}
\Section{9}{1}{Definition of Fourier series}\footnote{Throughout
  this chapter (except in\hardsubsectionref{9}{1}{1}) it is supposed that all
  the numbers involved are real.}.
Series of the type
\begin{align*}
  \half a_{0}
  + (a_{1} + \cos x + b_{1} \sin x)
  + (a_{2} + \cos 2x + b_{2} \sin 2x)
  + \cdots
  \\
  &
  \hfill
  \half a_{0}
  +
  \sum_{n=1}^{\infty} ( a_{n} \cos n x + b_{n} \sin n x),
\end{align*}
where $a_{n}$, $b_{n}$ are independent of $x$, are of great importance in many
investigations. They are called \emph{trigonometrical series}.

If there is a function $f(t)$ such that
$\int_{-\pi}^{\pi} f(t) \dmeasure t$ exists as a Riemann
integral or as an improper integral which converges absolutely, and such that
$$
\pi a_{n} = \int_{-\pi}^{\pi} f(t) \cos nt \dmeasure t,
\quad
\pi b_{n} = \int_{-\pi}^{\pi} f(t) \sin nt \dmeasure t,
$$
then the trigonometrical series is called a \emph{Fourier series}.

%\begin{smalltext}
Trigonometrical series first appeared in analysis in connexion with
the investigations of Daniel Bernoulli on vibrating
strings\index{Strings, vibrations of}\index{Vibrations of!strings};
d'Alembert had previously solved the equation of
motion
$ \ddot{y} = a^{2} \frac{\dd^{2} y}{\dd x^{2}}$
in the form
$y = \half \thebrace{f(x+at) + f(x-at)}$, where $y=f(x)$ is
the initial shape of the string starting from rest;
and Bernoulli shewed that a formal solution is
$$
y
=
\sum_{n=1}^{\infty}
b_{n}
\sin \frac{n \pi x}{l}
\cos \frac{n \pi a t}{l},
$$
the fixed ends of the string being $(0,0)$ and $(l,0)$; and he asserted
that this was the most general solution of the problem. This appeared
to d'Alembert and Euler to be impossible, since such a series, having
period $2l$, could not possibly represent such a function
as\footnote{This function gives a simple form to the initial shape of the string.}
$c x (l-x)$ when $t = 0$.
A controversy arose between these mathematicians, of which
an account is given in Hobson's \emph{Functions of a Real Variable}.

Fourier, in his \emph{TODO Theorie de la Chaleur}, investigated a number of
trigonometrical series and shewed that, in a large number of
particular cases, a Fourier series \emph{actually converged to the sum $f(x)$}.
Poisson attempted a general proof of this theorem. TODO Journal de VEcole
poly technique, xil. (1823), pp. 404-509. Two proofs were given by
Cauchy, TODO Men. de VAcad. R. des Sci. vi. (1823, publi-hed 1826), pp.
603-612 Oeuvres, (1), n. pp. 12-19) and Exercices de Math. 11. (1827),
pp. 341-376 (Oeuvres, (2), vil. pp. 393-430); these proofs, which are
based on the theory of contour integration, are concerned with rather
particular classes of functions and one is invalid. The second proof
has been investigated by Harnack, TODO Math. Ann. xxxii. (1888), pp.
175-202.

%\end{smalltext}
%
% 161
%

In 1829, Dirichlet gave the first rigorous proof\footnote{TODO Journal J iir Math. iv. (1829), pp. 157-169.}
that, for a general class of functions, the Fourier series, defined as above, does
converge to the sum $f(x)$. A modification of this proof was given later
by Bonnet\footnote{TODO Meiuoires des Savants etramjers of the Belgian Academy, xxiii.
(1848-1850). Bonnet employs the second mean value theorem directly,
while Dirichlet's original proof makes use of arguments precisely
similar to those by which that theorem is proved. See\hardsubsectionref{9}{4}{3}.}.

The result of Dirichlet\index{Statement of Fourier's theorem, \Dirichlet's}
is that\footnote{The conditions postulated for $f(t)$
  are known as \emph{Dirichlet's conditions}; as will be seen in TODO §§ 9-2, 9 42, they are unnecessarily
  stringent.}
if $f(t)$ is defined and bounded in the
range $(-\pi, \pi)$ and if $f(t)$ has only a finite number of maxima and
minima and a finite number of discontinuities in this range and,
further, if $f(t)$ is defined by the equation
$$
f(t + 2 \pi) = f(t)
$$
outside the range $(-\pi, \pi)$, then, provided that
$$
\pi a_{n}
=
\int_{-\pi}^{\pi} f(t) \cos nt \dmeasure t,
\quad
\pi b_{n}
=
\int_{-\pi}^{\pi} f(t) \sin nt \dmeasure t,
$$
the series $TODO$ converges to the sum /( +
0)+/( - 0) .

Later, Riemann and Cantor developed the theory of trigonometrical
series generally, while still more recently Hurwitz, \Fejer\ and others
have investigated properties of Fourier series when the series does
not necessarily converge. Thus \Fejer has proved the remarkable
theorem that a Fourier series (even if not convergent) is 'summable
$(C1)$' at all points at which $f(x \pm 0)$ exist, and its sum $(C1)$ is
$\half \thebrace{ f(x+0) + f(x-0) }$,
provided that $\int_{-\pi}^{\pi} f(t) \dmeasure t$ is an absolutely convergent integral.
One of the investigations of the convergence of Fourier series which we shall give later
(\hardsectionref{9}{4}{2}) is based on this result.

For a fuller account of investigations subsequent to Riemann, the
reader is referred to Hobson's \emph{Functions of a Real Variable}, and to
TODO de la Vallee Poussin's Cours dWnalyse Infinitesimale.

\Subsection{9}{1}{1}{Nature of the region within which a
  trigonometrical series converges.}

Consider the series
$$
\half a_{0}
+
\sum_{n=1}^{\infty}
\theparen{
  a_{n} \cos nz
  +
  b_{n} \sin nz
},
$$
where $z$ may be complex. If we write
$e^{iz} = \zeta$, the series becomes
$$
\half a_{0}
+
\sum_{n=1}^{\infty}
\thebrace{
  \half (a_{n} - i b_{n}) \zeta^{n}
  +
  \half (a_{n} + i b_{n}) \zeta^{-n}
}
$$
This Laurent series will converge, if it converges at all, in a region
in which $a \leq \absval{\zeta} \leq b$, where $a,b$ are positive
constants.

But, if $z = x + iy$, $\absval{\zeta} = e^{-y}$, and so we get, as the
region of convergence of the trigonometrical series, the strip in the
$z$ plane defined by the inequality
$$
\log a \leq -y \leq \log b.
$$

The case which is of the greatest importance in practice is that in which
$a = b = 1$, and the strip consists of a single line, namely the real axis.

TODO Example 1
Let
$$
f(z)
=
\sin z
- \half \sin 2z
+ \frac{1}{3} \sin 3z
- \frac{1}{4} \sin 4z
+ \ldots,
$$
where $z = x + iy$.
%
% 161
%

Writing this in the form
$$
f(z)
=
-
\half
i
\theparen{
  e^{iz}
  - \half e^{2iz}
  + \frac{1}{3} e^{3iz}
  -
  \cdots
}
+
\half
i
\theparen{
  e^{-iz}
  - \half e^{-2iz}
  + \frac{1}{3} e^{-3iz}
  -
  \cdots
}
$$
we notice that the first series converges\footnote{The series
  \emph{do converge} if $y=0$, see \hardsubsectionref{2}{3}{1} example TODO 2}
only if
$y \geq 0$, and the second only if $y \leq 0$.

Writing $x$ in place of $z$ ($x$ being real), we see that by Abel's
theorem (\hardsubsectionref{3}{7}{1}),
\begin{align*}
  f(x)
  &=
  \lim_{r \rightarrow 1} \theparen{
    r \sin x
    - \half r^{2} \sin 2x
    + \frac{1}{3} r^{3} \sin 3x
    - \cdots
  }
  \\
  &=
  \lim_{r \rightarrow 1} \thebrace{
    - \half \theparen{
      r e^{ix}
      - \half r^{2} e^{2ix}
      + \frac{1}{3} r^{3} e^{3ix}
      - \cdots
    }
    +
    \half i \theparen{
      r e^{-ix}
      - \half r^{2} e^{-2ix}
      + \frac{1}{3} r^{3} e^{-3ix}
      - \cdots
    }
  }
\end{align*}

This is the limit of one of the values of
$$
- \half i \log (1 + r e^{ix})
+ \half i \log (1 + r e^{-ix}),
$$
and as $r \rightarrow 1$ (if $-\pi < x < \pi$), this tends to
$\half x + k\pi$, where $k$ is some integer.

Now $\sum_{n=1}^{\infty} \frac{(-)^{n-1} \sin nx}{n}$ converges uniformly
(\hardsubsectionref{3}{3}{5} example TODO 1) and is therefore continuous in
the range $-\pi + \delta \leq x \leq \pi - \delta$, where
$\delta$ is any positive constant.

Since $\half x$ is continuous, $k$ has the same value wherever $x$ lies in the
range; and putting $x=0$, we see that $k=0$.

\emph{Therefore, when $-\pi < x < \pi$,
  $$
  f(x) = \half x.
  $$
}

But, when $\pi < x < 3\pi$,
$$
f(x)
=
f(x - 2\pi)
=
\half (x - 2\pi)
=
\half x - \pi,
$$
and generally, if $(2n - 1) \pi < x < (2n + 1) \pi$,
$$
f(x) = \half x - n \pi.
$$

We have thus arrived at an example in which $f(x)$ is
not represented by a single analytical expression.

It must be observed that this phenomenon can only occur when the
strip in which the Fourier series converges is a single line.
For if the strip is not of zero breadth, the associated Laurent
series converges in an annulus of non-zero breadth and represents an
analytic function of $\zeta$ in that annulus; and, since
$\zeta$ is an analytic function of $z$, the Fourier series
represents an analytic function of $z$; such a series is given by
$$
r \sin x
- \half r^{2} \sin 2x
+ \frac{1}{3} r^{3} \sin 3x
- \cdots,
$$
where $0 < r < 1$; its sum is
$\arctan \frac{r \sin x}{1 + r \cos x}$, the $\arctan$ always
representing an angle between $\pm \half \pi$.

Example TODO
When $-\pi \leq x \leq \pi$,
$$
\sum_{n=1}^{\infty}
\frac{(-)^{n-1} \cos nx}{n^{2}}
=
\frac{1}{12} \pi^{2}
-
\frac{1}{4} x^{2}.
$$

The series converges only when $x$ is real; by
\hardsubsectionref{3}{3}{4} the convergence is then
absolute and uniform.

Since
$$
\half x
=
\sin x
- \half \sin 2x
+ \frac{1}{3} \sin 3x
- \cdots
\quad
(-\pi + \delta \leq x \pi - \delta,
\delta > 0),
$$
and this series converges uniformly, we may integrate
term-by-term from $0$ to $x$ (\hardsectionref{4}{7}),
and consequently
$$
\frac{1}{4} x^{2}
=
\sum_{n=1}^{\infty}
\frac{(-)^{n-1} (1 - \cos nx)}{n^{2}}
\quad
(-\pi + \delta \leq x \leq \pi - \delta).
$$
%
% 163
%

That is to say, when $-\pi + \delta \leq x \leq \pi - \delta$,
$$
C - \frac{1}{4} x^{2}
=
\sum_{n=1}^{\infty} \frac{(-)^{n-1} \cos nx}{n^{2}},
$$
where $C$ is a constant, at present undetermined.

But since the series on the right converges uniformly throughout the
range $-\pi \leq x \leq \pi$, its sum is a continuous function of $x$ in this
extended range; and so, proceeding to the limit when
$x \rightarrow \pm \pi$, we see
that the last equation is still true when $x = \pm \pi$.

To determine $C$, integrate each side of the equation \hardsectionref{4}{7} between
the limits $-\pi, \pi$; and we get
$$
2 \pi C - \frac{1}{6} \pi^{3} = 0.
$$

Consequently
$$
\frac{1}{12} \pi^{2} - \frac{1}{4} x^{2}
=
\sum_{n=1}^{\infty}
\frac{ (-)^{n-1} \cos nx }{ n^{2} }
\quad
(-\pi \leq x \leq \pi).
$$

Example TODO.
By writing $\pi - 2x$ for $x$ in example TODO2, shew that
$$
\sum_{n=1}^{\infty} \frac{\sin^{2} nx}{n^{2}}
=
\begin{cases}
  \half x (\pi - x)                         & 0 \leq x \leq \pi,   \\
  \half \thebrace{ \pi \absval{x} - x^{2}} & -\pi \leq x \leq \pi.
\end{cases}
$$

\Subsection{9}{1}{2}{Values of the coefficients in terms of the sum of a
trigonometrical series.}
Let the trigonometrical series
$
\half c_{0}
+
\sum_{n=1}^{\infty} (c_{n} \cos nx + d_{n} \sin nx)
$
be uniformly convergent in the range $(-\pi, \pi)$ and let its sum be $f(x)$.
Using the obvious results
\begin{align*}
  \int_{-\pi}^{\pi} \cos mx \cos nx \dmeasure x
  =&
  \begin{cases}
    0 & m \neq n \\
    \pi & m = n \neq 0,
  \end{cases}
  \\
  \int_{-\pi}^{\pi} \sin mx \sin nx \dmeasure x
  =&
  \begin{cases}
    0 & m \neq n \\
    \pi & m = n \neq 0,
  \end{cases}
  \\
  \int_{-\pi}^{\pi} \dmeasure x
  =& 2\pi,
\end{align*}
we find, on multiplying the equation
$
\half c_{0}
+
\sum_{n=1}^{\infty}
(c_{n} \cos nx + d_{n} \sin nx)
= f(x)
$
by\footnote{Multiplying by these factors does not destroy the uniformity of the
  convergence.}
$\cos nx$ or by $\sin nx$ and integrating
term-by-term\footnote{These were given by TODO Euler (with limits $0$ and $2\pi$),
  Nova Acta Acad. Petrop. xi. (1793).}
(\hardsectionref{4}{7}),
$$
\pi c_{n} = \int_{-\pi}^{\pi} f(x) \cos nx \dmeasure x,
\quad
\pi d_{n} = \int_{-\pi}^{\pi} f(x) \sin nx \dmeasure x.
$$

TODO Corollary. A trigonometrical series uniformly convergent in the range
$(-\pi, \pi)$ is a Fourier series.

TODO Note. Lebesgue has given a proof (TODO Series trigonometriques, p. 124) of a
theorem communicated to him by Fatou that the trigonometrical series
$\sum_{n=2}^{\infty} \sin nx / \log n$, which converges for all real values of $x$
(\hardsubsectionref{2}{3}{1} example TODO), is \emph{not} a Fourier series.

\Section{9}{2}{On Dirichlet's conditions and Fourier's theorem.
\index{Statement of Fourier's theorem, \Dirichlet's}}
A theorem, of the type described in \hardsectionref{9}{1}, concerning the
expansibility of a function of a real variable into a trigonometrical
series is usually described
%
% 164
%
as \emph{Fourier's theorem}. On account of the length and difficulty of a
formal proof of the theorem (even when the function to be expanded is
subjected to unnecessarily stringent conditions), we defer the proof
until TODO §§ \hardsubsectionref{9}{4}{2}, \hardsubsectionref{9}{4}{3}.
It is, however, convenient to state here certain
\emph{sufficient} conditions under which a function can be expanded into a
trigonometrical series.

\emph{Let $f(t)$ he defined arbitrarily when $-\pi \leq t \leq \pi$
  and defined\footnote{This definition frequently results in $f(t)$ not being
    expressible by a single analytical expression for all real values of $t$.
    Cf.\hardsubsectionref{9}{1}{1} example TODO:1.}
  for all other real values of $t$ by means of the equation
  $$
  f(t + 2\pi) = f(t),
  $$
  so that $f(t)$ is a periodic function with period $2\pi$.
}
\emph{
  Let $f(t)$ be such that
  $\int_{-\pi}^{\pi} f(t) \dmeasure t$ exists; and if this is an improper
  integral, let it be absolutely convergent.
}

\emph{
  Let $a_{n}, b_{n}$ be defined by the
  equations\footnote{The numbers $a_{n}, b_{n}$ are called
    \emph{the Fourier constants\index{Fourier constants}} of
    $f(t)$, and the symbols $a_{n}, b_{n}$ will be used in this sense throughout
    §§ TODO \hardsectionref{9}{2}--\hardsectionref{9}{5}.
    It may be shewn that the convergence and absolute convergence of the
    integrals defining the Fourier constants are consequences of the
    convergence and absolute convergence of
    $\int_{-\pi}^{\pi} f(t) \dmeasure t$.
    Cf. §§ TODO \hardsubsectionref{2}{3}{2}, \hardsectionref{4}{5}.}
  $$
  \pi a_{n} = \int_{-\pi}^{\pi} f(t) \cos nt \dmeasure t,
  \quad
  \pi b_{n} = \int_{-\pi}^{\pi} f(t) \sin nt \dmeasure t
  \quad
  (n=0,1,2,\ldots).
  $$
}

\emph{
  Then, if $x$ be an interior point of any interval $(a, b)$ in which
  $f(t)$ has limited total fluctuation, the series
  $$
  \half a_{0}
  +
  \sum_{n=1}^{\infty} (a_{n} \cos nx + b_{n} \sin nx)
  $$
  is convergent, and its sum\footnote{The limits $f(x \pm 0)$ exist,
    by \hardsubsectionref{3}{6}{4} example TODO:3.}
  is $\half \thebrace{ f(x+0) + f(x-0) }$.
  If $f(t)$ is continuous at $t=x$, this sum reduces to $f(x)$.
}

This theorem will be assumed in
TODO\hardsubsectionref{9}{2}{1}--\hardsubsectionref{9}{3}{2};
these sections deal with theorems concerning Fourier series which
are of some importance
in practical applications. It should be stated here that every
function which Applied Mathematicians need to expand into Fourier
series satisfies the conditions just imposed on $f(t)$, so that the
analysis given later in this chapter establishes the validity of all
the expansions into Fourier series which are required in physical
investigations.

The reader will observe that in the theorem just stated,
$f(t)$ is subject to less stringent conditions than those contemplated by
Dirichlet, and this decrease of stringency is of
considerable practical importance. Thus, so simple a series as
$\sum_{n=1}^{\infty} (-)^{n-1} (\cos nx) / n$
is the
expansion of the function\footnote{Cf. example TODO:6 at the end of the chapter (p. TODO:190).}
$\log \absval{2 \cos \half x}$; and this function
does not satisfy Dirichlet's condition of boundedness at $\pm \pi$.

It is convenient to describe the series
$\half a_{0} + \sum_{n=1}^{\infty} (a_{n} \cos nx + b_{n} \sin nx)$
as \emph{the Fourier series associated with $f(t)$}. This description must,
however, be
%
% 165
%
taken as implying nothing concerning the convergence of the series in
question.

\Subsection{9}{2}{1}{The representation of a function by Fourier series for ranges
other than $(-\pi,\pi)$.}

Consider a function $f(x)$ with an (absolutely) convergent integral,
and with limited total fluctuation in the range $a \leq x \leq b$.

Write
$x = \half (a + b) - \half (a-b) \pi^{-1} x',
\quad
f(x) = F(x')$.

Then it is known (\hardsectionref{9}{2}) that
$$
\half [F(x'+0) + F(x'-0)]
=
\half a_{0} + \sum_{n=1}^{\infty} (a_{n} \cos nx' + b_{n} \sin nx'),
$$
and so
\begin{align*}
  & \half \thebrace{ f(x+0) + f(x-0)}
  \\
  &
  \hfill
  =
  \half a_{0}
  +
  \sum_{n=1}^{\infty}
  \thebrace{
    a_{n} \cos \frac{n \pi (2x-a-b)}{b-a}
    +
    b_{n} \sin \frac{n \pi (2x-a-b)}{b-a}
  },
\end{align*}
where by an obvious transformation
\begin{align*}
  \half (b-a) a_{n} =& \int_{a}^{b} f(x) \cos \frac{n \pi (2x-a-b)}{b-a} \dmeasure x,
  \\
  \half (b-a) b_{n} =& \int_{a}^{b} f(x) \sin \frac{n \pi (2x-a-b)}{b-a} \dmeasure x
  .
\end{align*}
\Subsection{9}{2}{2}{The cosine series and the sine series.}
Let $f(x)$ be defined in the range $(0,l)$ and let it have an
(absolutely) convergent integral and also let it have limited
total fluctuation in that range.
\emph{Define} $f(x)$ in the range
$\wandwtypo{(0,-l)}{(-l,0)}$
by the equation
$$
f(-x) = f(x).\index{Even functions}
$$

Then
$$
\half \thebrace{ f(x+0) + f(x-0) }
=
\half a_{0}
+
\sum_{n=1}^{\infty} \thebrace{
  a_{n} \cos \frac{n \pi x}{l}
  +
  b_{n} \sin \frac{n \pi x}{l}
},
$$
where, by \hardsubsectionref{9}{2}{1},
\begin{align*}
  l a_{n}
  =&
  \int_{-l}^{l} f(t) \cos \frac{n \pi t}{l} \dmeasure t
  =
  2 \int_{0}^{l} f(t) \cos \frac{n \pi t}{l} \dmeasure t,
  \\
  l b_{n}
  =&
  \int_{-l}^{l} f(t) \sin \frac{n \pi t}{l} \dmeasure t
  = 0,
\end{align*}
so that when $-l \leq x \leq l$,
$$
\half \thebrace{ f(x+0) + f(x-0) }
=
\half a_{0} + \sum_{n=1}^{\infty} a_{n} \cos \frac{n \pi x}{l};
$$
this is called the \emph{cosine series}.

If, however, we define $f(x)$ in the range $(0,-l)$ by the equation
$$
f(-x) = -f(\wandwtypo{-}{}x),\index{Odd functions}
$$
%
% 166
%
we get, when $-l \leq x \leq l$,
$$
\half \thebrace{ f(x+0) + f(x-0) }
=
\sum_{n=1}^{\infty} b_{n} \sin \frac{n \pi x}{l},
$$
where
$$
l b_{n}
=
2 \int_{0}^{l} f(t) \sin \frac{n \pi t}{l} \dmeasure t;
$$
this is called the \emph{sine series}.

Thus the series
$$
\half a_{0}
+
\sum_{n=1}^{\infty} a_{n} \cos \frac{n \pi x}{l},
\quad
\sum_{n=1}^{\infty} b_{n} \sin \frac{n \pi x}{l},
$$
where
$$
\half l a_{n}
=
\int_{0}^{l} f(t) \cos \frac{n \pi t}{l} \dmeasure t,
\quad
\half l b_{n}
=
\int_{0}^{l} f(t) \sin \frac{n \pi t}{l} \dmeasure t,
$$
\emph{have the same sum when $0 \leq x \leq l$;}
but their sums are numerically
equal and opposite in sign when $0 \geq x \geq -l$.

%\begin{smalltext}
The cosine series was given by Clairaut, TODO Hist, de I'Acad. R. des Sci.
1754 [published, 1759], in a memoir dated July 9, 1757; the sine
series was obtained between 1762 and 1765 by Lagrange, Oeuvres, l. p.
553.
%\end{smalltext}

TODO Example 1. Expand $\half (\pi - x) \sin x$ in a cosine series in the range
$0 \leq x \leq \pi$.
[We have, by the formula just obtained,
$$
\half (\pi - x) \sin x
=
\half a_{0}
+ \sum_{n=1}^{\infty} a_{n} \cos nx,
$$
where
$$
\half \pi a_{n}
=
\int_{0}^{\pi} \half (\pi - x) \sin x \cos nx \dmeasure x.
$$

But, integrating by parts, if $n \neq 1$,
\begin{align*}
  &
  \int_{0}^{\pi} 2 (\pi - x) \sin x \cos nx \dmeasure x
  \\
  & \quad
  \int_{0}^{\pi} (\pi - x) \thebrace{
    \sin (n+1) x - \sin (n-1) x
  } \dmeasure x
  \\
  & \quad
  \thebracket{
    (x - \pi)
    \thebrace{
      \frac{ \cos (n+1) x }{n+1}
      -
      \frac{ \cos (n-1) x }{n-1}
    }
  }_{0}^{\pi}
  -
  \int_{0}^{\pi}
  \thebrace{
    \frac{ \cos (n+1) x }{n+1}
    -
    \frac{ \cos (n-1) x }{n-1}
  }
  \dmeasure x
  \\
  & \quad
  \pi
  \theparen{
    \frac{1}{n+1}
    -
    \frac{1}{n-1}
  }
  =
  \frac{-2\pi}{(n+1)(n-1)}
\end{align*}
Whereas if $n=1$, we get
$\int_{0}^{\pi} 2 (\pi - x) \sin x \cos x \dmeasure x = \half \pi$.

Therefore the required series is
$$
\half
+ \frac{1}{4} \cos x
- \frac{1}{1 \cdot 3} \cos 2x
- \frac{1}{2 \cdot 4} \cos 3x
- \frac{1}{3 \cdot 5} \cos 4x
- \cdots.
$$

It will be observed that it is only for values of $x$ between
$0$ and $\pi$ that the sum of this series is proved to be
$\half (\pi - x) \sin x$; thus for
instance when $x$ has a value between $0$ and $-\pi$,
the sum of the series is not
$\half (\pi - x) \sin x$, but $-\half (\pi + x) \sin x$; when $x$ has a value
between $\pi$ and $2 \pi$, the sum of the series happens to be again
$\half (\pi - x) \sin x$, but this is a mere coincidence arising from the special
function considered, and does not follow from the general theorem.]

TODO Example 2. Expand $\frac{1}{8} \pi x (\pi - x)$ in a sine series,
valid when $0 \leq x \leq \pi$.

[The series is $\sin x + \frac{\sin 3x}{3^{3}} + \frac{\sin 5x}{5^{3}} + \cdots.$]
%
% 167
%

TOOD Example 3. Shew that, when $0 \leq x \leq \pi$,
$$
\frac{1}{96}
\pi
(\pi - 2x)
(\pi^{2} + 2 \pi x - 2 x^{2})
=
\cos x
+ \frac{\cos 3x}{3^{4}}
+ \frac{\cos 5x}{5^{4}}
+ \cdots.
$$

[Denoting the left-hand side by $f(x)$, we have, on integrating by
parts and observing that $f'(0) = f'(\pi) = 0$,
\begin{align*}
  \int_{0}^{\pi} f(x) \cos nx \dmeasure x
  =&
  \frac{1}{n} \thebracket{f(x) \sin nx}_{0}^{\pi}
  -
  \frac{1}{n} \int_{0}^{\pi} f'(x) \sin nx \dmeasure x
  \\
  =&
  \frac{1}{n^{2}} \thebracket{f'(x) \cos nx}_{0}^{\pi}
  -
  \frac{1}{n^{2}} \int_{0}^{\pi} f''(x) \cos nx \dmeasure x
  \\
  =&
  -\frac{1}{n^{3}} \thebracket{f''(x) \sin nx}_{0}^{\pi}
  +
  \frac{1}{n^{3}} \int_{0}^{\pi} f'''(x) \sin nx \dmeasure x
  =&
  -\frac{1}{n^{4}} \thebracket{f'''(x) \cos nx}_{0}^{\pi}
  =
  \frac{\pi}{4 n^{4}} (1 - \cos n \pi).]
\end{align*}
TODO Example 4. Shew that for values of $x$ between $0$ and $\pi$,
$e^{s x}$ can be expanded in the cosine series
$$
\frac{2 s}{\pi}
\theparen{e^{s \pi} - 1}
\theparen{
  \frac{1}{2 s^{2}}
  + \frac{\cos 2x}{s^{2} + 4}
  + \frac{\cos 4x}{s^{2} + 16}
  + \cdots
}
-
\frac{2 s}{\pi}
\theparen{e^{s \pi} - 1}
\theparen{
  \frac{\cos x}{s^{2} + 1}
  + \frac{\cos 3x}{s^{2} + 9}
  + \cdots
},
$$
and draw graphs of the function $e^{s x}$ and of the sum of the series.

TODO Example 5. Shew that for values of $x$ between $0$ and $\pi$,
the function $\frac{1}{8} \pi (\pi - 2x)$ can
be expanded in the cosine series
$$
\cos x
+ \frac{\cos 3x}{3^{2}}
+ \frac{\cos 5x}{5^{2}}
+ \cdots,
$$
and draw graphs of the function $\frac{1}{8} \pi (\pi - 2x)$ and of the sum of the
series.

\Section{9}{3}{The nature of the coefficients in a Fourier series.}\footnote{TODO The analysis of this section and of\hardsubsectionref{9}{3}{1} is contained in Stokes'
great memoir, Camb. Phil. Tratis. VIII. (1849), pp. 538-583 [Math.
Papers, i. pp. 236-313].}

Suppose that (as in the numerical examples which have been discussed)
the interval $(-\pi, \pi)$ can bo divided into a finite number of ranges
$(-\pi, k_{1}), (k_{1}, k_{2}), \ldots, (k_{n}, \pi)$
such that throughout each range $f(x)$
and all its differential coefficients are continuous with limited
total fluctuation and that they have limits on the right and on the
left (\hardsectionref{3}{2}) at the end points of these ranges.

Then
$$
\pi a_{m}
=
\int_{-\pi}^{k_{1}} f(t) \cos mt \dmeasure t
+ \int_{k_{1}}^{k_{2}} f(t) \cos mt \dmeasure t
+ \cdots
+ \int_{k_{n}}^{\pi} f(t) \cos mt \dmeasure t.
$$

Integrating by parts we get
\begin{align*}
  &
  \pi a_{m}
  =
  \thebracket{
    m^{-1} f(t) \sin mt
  }_{-\pi}^{k_{1}}
  +
  \thebracket{
    m^{-1} f(t) \sin mt
  }_{k_{1}}^{k_{2}}
  +
  \cdots
  +
  \thebracket{
    m^{-1} f(t) \sin mt
  }_{k_{n}}^{\pi}
  \\
  & \hfill
  - m^{-1} \int_{-\pi}^{k_{1}} f'(t) \sin mt \dmeasure t
  - m^{-1} \int_{k_{1}}^{k_{2}} f'(t) \sin mt \dmeasure t
  \cdots
  - m^{-1} \int_{k_{n}}^{\pi} f'(t) \sin mt \dmeasure t,
\end{align*}
so that
$$
a_{m} = \frac{A_{m}}{m} - \frac{b_{m}'}{m},
$$
%
% 168
%
where
$$
\pi A_{m}
=
\sum_{r=1}^{n}
\sin m k_{r}
\thebrace{
  f(k_{r} - 0) - f(k_{r} + 0),
}
$$
and $b_{m}'$ is a Fourier constant of $f'(x)$.

Similarly
$$
b_{m} = \frac{B_{m}}{m} + \frac{a_{m}'}{m},
$$
where
$$
\pi B_{m}
=
-
\sum_{r=1}^{n}
\cos m k_{r}
\thebrace{
  f(k_{r} - 0)
  -
  f(k_{r} + 0)
}
-
\cos m \pi
\thebrace{
  f(\pi - 0)
  -
  f(-\pi + 0)
}
,
$$
and $a_{m}'$ is a Fourier constant of $f'(x)$.

Similarly, we get
$$
a_{m}' = \frac{A_{m}'}{m} - \frac{b_{m}''}{m},
\quad
b_{m}' = \frac{B_{m}'}{m} - \frac{a_{m}''}{m},
$$
where $a_{m}'', b_{m}''$ are the Fourier constants of
$f''(x)$ and
\begin{align*}
  \pi A_{m}'
  =&
  \sum_{r=1}^{n} \sin m k_{r} \thebrace{
    f'(k_{r}-0) - f'(k_{r}+0)
  },
  \\
  \pi B_{m}'
  =&
  - \sum_{r=1}^{n} \cos m k_{r} \thebrace{
    f'(k_{r}-0) - f'(k_{r}+0)
  }
  \\
  \hfill
  - \cos m \pi \thebrace{
    f'(\pi - 0) - f'(-\pi + 0)
  }.
\end{align*}

Therefore
$$
a_{m} =
\frac{A_{m}}{m}
- \frac{B_{m}'}{m^{2}}
- \frac{a_{m}''}{m^{2}},
\quad
b_{m} =
\frac{B_{m}}{m}
+ \frac{A_{m}'}{m^{2}}
- \frac{b_{m}''}{m^{2}},
$$

Now as $m \rightarrow \infty$, we see that
$$
A_{m}' = \bigo(1),
\quad
B_{m}' = \bigo(1),
$$
and, since the integrands involved in $a_{m}''$ and $b_{m}''$
are bounded, it is evident that
$$
a_{m}'' = \bigo(1),
\quad
b_{m}'' = \bigo(1).
$$

Hence if $A_{m}=0, B_{m}=0$, the Fourier series for $f(x)$ converges
absolutely and uniformly, by \hardsubsectionref{3}{3}{4}.

The necessary and sufficient conditions that
$A_{m} = B_{m} = 0$ for all values of $m$ are that
$$
f(k_{r} - 0) = f(k_{r} + 0),
\quad
f(\pi - 0) = f(-\pi + 0)
$$
that is to say that\footnote{Of course $f(x)$ is also subject to the conditions stated at the
beginning of the section.} $f(x)$ should be continuous for \emph{all} values of $x$.

\Subsection{9}{3}{1}{Differentiation of Fourier series.}
The result of differentiating
$$
\half a_{0}
+ \sum_{m=1}^{\infty} (a_{m} \cos mx + b_{m} \sin mx)
$$
term by term is
$$
\sum_{m=1}^{\infty} \thebrace{
  m b_{m} \cos mx
  -
  m a_{m} \sin mx
}.
$$
%
% 169
%

With the notation of \hardsectionref{9}{3}, this is the same as
$$
\half a_{0}'
+
\sum_{m=1}^{\infty} ( a_{m}' \cos mx + b_{m}' \sin mx),
$$
provided that $A_{m} = B_{m} = 0$ and
$\int_{-\pi}^{\pi} f'(x) \dmeasure x = 0$;
these conditions are satisfied if $f(x)$ is continuous for all values of
$x$.

Consequently sufficient conditions for the legitimacy of
differentiating a Fourier series term by term are that $f(x)$ should be
continuous for \emph{all} values of $x$ and $f'(x)$ should have only a finite
number of points of discontinuity in the range $(-\pi, \pi)$, both
functions having limited total fluctuation throughout the range.

\Subsection{9}{3}{2}{Determination of points of discontinuity.}

The expressions for a, and b,n which have been found in\hardsectionref{9}{3} can
frequently be applied in practical examples to determine the [loints
at which the sum of a given Fourier series may be discontinuous. Thus,
let it be required to determine the places at which the sum of the
series

sin u'+ i sin 3.'; + 4 sin ")./ + . . . is discontinuous.

  Assmning that the series is a Fourier series and not <i/ij/
trigonometrical series and

observing that 0 =0, b = 2m)- ' l-coa nin), we get on considering the
formula found in

! 9-3,

,,, = 0, B = -h cos nirr, a, ' = 6,' = 0.

Hence if y(-,, k., ... are the places at which the analytic
character of the sum is broken, we have

= 7r.4, = [sin ml'i f(i- - 0) -/(/(, + 0) +sin mk.> / h, - 0) -f(L;,
+ 0) + ...].

Since this is true for all values of ?, the numbers ki, h, ... must
be multiples of tt; but

there is only one even multiple of it in the range - 7r<.r 7r, namely
zero. So i = 0,

and X ) '3, ... do not exist. Substituting TODO in the equation 5m =
i-i cos mir, we have

7r( - COSJ 7r)=-[cosm7r /(7r-0)-/(-7r+0) +/(-0)-/( + 0)].

Since this is true for all values of ?n, we have

i7r=/( + 0)-/(-0), |7r=/(7r-0)-/(,r + 0). This shews that, if the
series is a Fourier series, /(.r) has discontinuities at the points
niT n any integer), and since a,' = 6m' = 0, we should expect* /( )
to be constant in the open range ( - tt, 0) and to be another constant
in the open range (0, tt).

\Section{9}{4}{\Fejer's theorem.}

We now begin the discussion of the theory of Fourier series by proving
the following theorem, due to \Fejer, concerning the summability of
the Fourier series associated with an arbitrary function, f t):

Let f(t) he a function of the real variable t, defined arbitrarily
luhen - TT t< TT, and defined by the equation

f(t + 27r)=f(t)

* lu point of fact f(x)=- Tr (-7r<:x<0);

f(x) = ir (0<x<?r).

t Math. Ann. lviii. (1904), pp. 51-69.

%
% 170
%

for all other real values oft; and let I f t)dt exist and if it is an
improper

J -n

integral) let it he ahsolutely convergent.

Then the Fourier ser'ies associated with the function f (f) is
summable* (CI) at all points x at which the two limits f(x ± 0) exist.

And its sum (CI) is

i /(*' + 0)+/( -0) .

Let an, bn, (n = 0, 1, 2, ...) denote the Fourier constants \hardsectionref{9}{2}) of
/( ) and let

 ao = Ao, anCosnx + bnSinnx = An x), S An(x) = S ni )-

n=0

Then we have to prove that

lira - A, + S, x) + S, x) + ... + S, \, x)] = f x + 0)+f x -0)],

in-*-x W

provided that the limits on the right exist.

If we substitute for the Fourier constants their values in the form of
integrals \hardsectionref{9}{2}), it is easy to verify thatf

j - 1 0 + 2 >S (x) = mAo + (m -l)A (x) + (m -2) A., x) + ... + A,n-i
x)

n=\

1 /""

= - [ m + m - 1) cos x-t) + (m - 2) cos 2 x-t)+ ...

 "j -:r

+ cos (m - l) x - 1)] fit) dt - i\ r sin iM // -x 7, ~27rj\ . sin 1(
-0 -

-27rj\ .+, sin 1( -0 - ' the last step following from the periodicity
of the integrand.

If now we bisect the path of integration and write a; + 2 in place of
t in the two parts of the path, we get

A + 1 Sn (x) = - - - r f x + 2d)de + - -- r,' j oc- 26) dO.

Consequently it is sufficient to prove that, as m- cc, then 1 r sin'
wi,. \ .,, -/ x 1 f- 'sin-m ..,,, ., ..

* See\hardsubsectionref{8}{4}{3}.

t It is obvious that, if we write X for e ( -') in the second line,
tlien

m + (m - 1) (X + X-i) + (hi - 2) (X2 + X-2) + . . . + (X' -i + Xi-' )

= (l-X)-i \ i-™ + X2-' +...+X-i + l-X--X'-i-...-X' = (1 - X)-2 xi- v -
2X + X' +i = (X '" - X~ '"f /(X - X" )

%
% 171
%

! the equation m + iu- l)cos2 + ... + cos 2 m - 1)6,

Now, if we integrate the equation 1 sin ni0

2 sin- 6* we find that

' " sin m

da - TTin,

J sin

and so we have to prove that

1 r ' sin m .,.

- - /i 4>(6)(i6->0 as m-¥x,

mJo siTi'd

where (f)(0) stands in turn for each of the two functions

fix + 26) -fix + 0), fix - 26) -fix - 0).

Now, given an arbitrary positive number e, we can choose S so that*

( ( ) I < e

whenever 0< 6 S. This choice of B is obviously independent of m.

Then

1 f sin- m6 ., 1 fi' sin m

1 fi"" sin- m6,,,, 1 f sm-mO,, 1 f " sm" ni6,,, ., - - n
4>id)de % - --- rQ- 4><6) d6+ '.---4>i6)\ d6

mJo sin-6 ' mj sin-6 //i'js sin f -r \ /

ri*sin-?,, 1 [i", - ..

-r-r dc)+ -.-- -. \ 4> 6) d6 'o sin- msmHSJw'

e ri*sin-?,, 1 /"i"

m.'o sin-c msm ojji

6 fi'sin / - - 1 ri"

 - r7 f +  ...J \ i>i6)\ d6 )i\ sm-6 711 sm- 6 Jo ' '

= i7re+ - ijl" (f>(6)d6.

Now the convergence of I /(O f? entails the convergence of

J -1T

<f> 6) d6,

. and so, given e (and therefore h), we can make

 irem sin' lh> \'\ < i6) dO,

Jo

by taking m sufficiently large.

Hence, by taking m sufficiently large, we can make

I 1 fi' sin-??i . //ix j/i - 4>i6)d6 <7re,

\ in J Q sm- 6

where e is an arbitrary positive number; that is to say, from the
definition of a limit,

m oo m ' sm- e and so \Fejer's theorem is established.

* On the assumption that/(x±0) exist.

%
% 172
%

Corollary 1. Let V and L be the upper and lower bounds o f(t) in any
interval a, b) whose length does not exceed £77, and let

P \ f t)\ dt = nA.

Then, if a +r) .v b-rj, where;; is any positive number, we have

 -.- r Mx-'" = 2T - j-... + j.-, j..J sinH(i-0 - (

  1 iP- /- -|- sinH> ( -0 r2rw7rij\ +.,. ./a;+J sin (x - - -

i 1 0+" 'i' 'S' (.x-)| U- \ U + U l m sin2 i;, .

so that

Similarly

    o+' i ' " ] >L- \ L'H ]l r>i sinH .

Corollary 2. Let/( ) be continuous in the interval a t b. Since
continuity implies uniformity of continuity \hardsubsectionref{3}{6}{1}), the choice of S
corresponding to any value of x in (a, b) is independent of x, and the
upper bound of |/(.r±0) |, i.e. of \ f(x) |, is also independent of X,
so that

/ '" j ( ( ) i d6= ['" I f x±U)-f x±Q)\ de Jo Jo'

and the upper bound of the last expression is independent of x.

Hence the choice of ??z, which makes
TODO
is independent of x, and consequently - \ A( + 2 S.n x)\ tends to the
limit f x), as

in \ n=\ )

m- -a:, uniformly throughout the interval a x%b.

\Subsection{9}{4}{1}{The Riemann-Lehesgue lemmas.}

In order to be able to apply Hardy's theorem \hardsectionref{8}{5}) to deduce the
con- vergence of Fourier series from \Fejer's theorem, we need the two
following lemmas :

(I) Let ( -x 6) dd exist and if it is an improper integral) let it he

  a

absolutely convergent. Then, as X- x>,

yl/ e)sm \ e)de is 0(1).

(II) If, further, i/r (6) Juts limited total fluctuation in the range
a, h) then, as X-><X),
$$
TODO
$$

%
% 173
%

Of these results (I) was stated by W. R. Hamilton* and by Riemannt in
the case of bounded fiuictions. The truth of (II) seems to have been
well known before its importance was realised; it is a generalisation
of a result established by Dirksenf and Stokes (see § 9 '3) in the
case of functions with a continuous differential coefficient.

The reader should obser\'e that the analysis of this section remains
valid when the sines are replaced throughout by cosines.

(I) It is convenient to establish this lemma first in the case in
which yjr (0) is bounded in the range (a, b). In this case, let K be
the upper bound of r\ fr d)\, and let e be an arbitrary positive
number. Divide the range (a, 6) into n parts by the points x,, x.2,
... a'n-i, and form the sums Sn, Sn associated with the function i/r
(0) after the manner of § 41. Take n so large that n - Sn<e; this is
possible since -yjr [6) is integrable.

In the interval (a-v\ i, a,) write

y r d) = ylr,(xr-,) + co,(e),

SO that ( 0r(6)\ <z Ur - L,-,

where Ur and Lr are the upper and lower bounds of - 6) in the interval
yXf-i, Xf).

It is then clear that

6

[ yjr (6) sin (\ d)de

J a

= 1 t i/r,. (av-i) I r=l

n

sm \ d)(W+ :i

(o, \$) sin (kO) dO

sm \ 0)dd

+ S

r=l . Xr-i

O)r(O)\ d0

< 2nK/X) + €.

By taking X sufficiently large (n remaining fixed after e has been
chosen), the last expression may be made less than 2e, so that

lim ylr(0)sm(xe)dd = O,

K-*-ac J a

and this is the result stated.

When yfr 6) is unbounded, if it has an absolutely convergent integral,
by\hardsectionref{4}{5}, we may enclose the points at which it is unbounded in a
finite |j number

* Trans. Dublin Acad. xix. (1843), p. 267.

t Ges. Math. IVerke, p. 241. For Lebesgue's investigation see his
Series trigonometriques (1906), Ch. III.

X Journal fUr Math. iv. (1829), p. 172.

§ For this proof we are indebted to Mr Hardy; it seems to be neater
than the proofs given by other writers, e.g. de la Vallee Poussin,
Cours cV Analyse Infinitesiniale, ii. (1912), pp. 140-141.

li The finiteness of the number of intervals is assumed in the
definition of an improper integral,\hardsectionref{4}{5}.

%
% 174
%

of intervals Sj, So, ... Bp such that

d6<€.

If K denote the upper bound of \ ylr(6)\ for values of outside these

intervals, and if 71, 70, ... 7 +1 denote the portions of the interval
(a, b) which

do not belong to 8, Bo, ... 8p we may prove as before that

rb I ?3+i r p r I

ylr e)sm \ e)de =\ t ylr(e)sm \ e)d0+ t ylr e)sm \ e)de\

J a ' \ r=lJyr r=lJ S,- '

 \ \ X I f e)sin Xe)de + S f \ ylr(d)sm \ d)ide

I r=l. V,. r = lJ Sr

< 2nKj ) + 26.

Now the choice of e fixes n and K, so that the last expression may be
made less than 3e by taking X. sufficiently large. That is to say
that, even if yjr (6) be unbounded,

lim [ f e)sm(ke)dd = 0,

A.- .oo J a

provided that yjr 6 has an (improper) integral which is absolutely
convergent.

The first lemma is therefore completely proved.

(II) When yjr 6) has limited total fluctuation in the range (a, b), by\hardsubsectionref{3}{6}{4} example 2, we may write

 W = %:( )-%2( ),

where Xi ( )> %2 ( ) are positive increasing bounded functions.

Then, by the second mean-value theorem \hardsubsectionref{4}{1}{4}) a number exists such
that a- b and

f Yi (6) sin (XO) dd\ = \ xAb)l' sin (Xd) dO,

 2xAb)IX. If we treat X2. ( ) in similar manner, it follows that

X, (0) sin (Xd) dd

+

X', 0) \ n xe)dd-

yjr (d) sin (XO) d6

 2te(6) +;,(6) /X = 0(1/X), and the second lemma is established.

Corollary. If f(t) lie such that I f(t) exists and is an absolutely
convergent

integral, the Fourier constants a, bn oi f(t) are o(l) as n- -oo;
and if, further, /(<) has limited total fluctuation in the range (-tt,
it), the Fourier constants are 0 l/n).

[Of course these results are not sufl\&cient to ensure the convergence
of the Fourier series associated with/(i); for a series, in which the
terms are of the order of magnitude of the terms in the harmonic
series \hardsectionref{2}{3}), is not necessarily convergent.]

%
% 175
%

\Subsection{9}{4}{2}{The proof of Fourier's theorem.}

We shall now prove the theorem enunciated in\hardsectionref{9}{2}, namely : Let f(t)
be a function defined arbitrarily luhen -ir tKir, and defined by the
equation f t + 27r) =f t) for all other real values oft; and let 1
fit) dt exist and (if it is an improper integral) let it be absolutely
convergent. Let an, bn be defined by the equations

7ran= I f(t) cos ntdt, 7rb,i= j f(t)sinntdt.

J -n J -TT

Then, if OS be an interior point of any interval (a, b) luithin tuhich
f(t) has limited total fluctuation, the series

00

 ao+ 2 (tt,j cos n + 6,1 sin ?ia;)

is convergent and its sum is [f x + 0) +f x - 0) .

It is convenient to give two proofs, one applicable to functions for
which it is permissible to take the interval (o, b) to be the interval
(- tt + a-, it - x), the other applicable to functions for which it is
not permissible.

(I) When the interval (a, b) may be taken to be (- tt + a?, tt + x),
it follows from\hardsubsectionref{9}{4}{1} (II) that a cos nx + bn sin nx is 0(1/??) as n-
> oo . Now by \Fejer's theorem \hardsectionref{9}{4}) the series under consideration
is summable (Cl) and its sum (C'l) is* /( + 0) +f x - 0)|. Therefore.,
by Hardy's convergence theorem \hardsectionref{8}{5}), the series under consideration
is convergent and its sum (by\hardsubsectionref{8}{4}{3}) isif/(*' + 0)+/(.r-0) .

(II) Even if it is not permissible to take the interval (a, b) to be
the whole interval (- tt + o;, tt + a;), it is possible, by
hypothesis, to choose a positive number S, less than tt, such that/( )
has limited total fluctuation in the interval x - h, x - h). We now
define an auxiliary function g (t), which is equal to f(t) when x-B%t
x + 8, and which is equal to zero throughout the rest of the interval
-ir + x, tt + x); and g(t + 27r) is .to be equal to g (t) for all real
values of t.

Then g (t) satisfies the conditions postulated for the functions under
consideration in (I), namely that it has an integral which is
absolutely convergent and it has limited total fluctuation in the
interval (-tt + x, tt + x); and so, if a, "*, 6 "' denote the Fourier
constants of git), the arguments used in (I) prove that the Fourier
series associated with g (t), namely

  cto"' + S ( '" COS nx + 6 * ' sin nx),

n = l

is convergent and has the sum [g (x + 0) + g (x - 0), and this is
equal to

1 /( + 0) +/(.. -0) .

* The limits /(.r±0) exist, by\hardsubsectionref{3}{6}{4} example 3.

%
% 176
%

Now let Sm ) and Sm (x) denote the sums of the first m + 1 terms of
the Fourier series associated with f(t) and c/ (t) respectively. Then
it is easily seen that

Sm ( ) = - I 2 + cos x-t) + cos'2 a;-t) + ... + cos vi (x - t) f(t) dt

~27rJ\ / m x-t) - " \ 1\ /-+- sm(m + )(x-t)

IT Jo sm * IT Jo sm -

by steps analogous to those given in § d'4>. In like manner

e (1)/ \ 1 / " sin(2 t + l) 1 |"i' sin(2m + 1), om /a

'Sfm'''( ) = - -3- g x+2e)d6 + - - W-5 g x-2e)dd,

TT J sm 17 TT j Sin t/

and so, using the definition of g t), we have

S>n (x) - S, > (x) =- r sin (2m + 1)6' M dd TTJjs smt/

+ - \'\ m 2m+ ) d- ''~l '> dd.

TT J ks smu

Since cosec is a continuous function in the range 8, tt), it follows
that f x ± 20) cosec 6 are integrable functions with absolutely
convergent integrals; and so, by the Riemann-Lebesgue lemma of\hardsubsectionref{9}{4}{1}
(I), both the integrals on the right in the last equation tend to zero
as m- >oo .

That is to say lim >S x) - S,J' (x)] = 0.

Hence, since lim S (x) = f(x + 0) +f(x - 0)|,

it follows also that

lim>Sf ( ) =l /( + 0)+/(. '-0)|.

We have therefore proved that the Fourier series associated luith f
t), namely a + 2 (a,iCos nx + 6 sin nx), is convergent and its sum is

i /( - + 0)+/( -0) .

\Subsection{9}{4}{3}{The Dirichlet-Bonnet proof of Fourier's theorem.}
It is of some interest to prove directly the theorem of \hardsubsectionref{9}{4}{2},
without making use of the theory of summability; accordingly we now
give a proof which is on the same general lines as the proofs due to
Dirichlet and Bonnet.

%
% 177
%

As usual we denote the sum of the first m + 1 terms of the Fourier
series by Sm (jc), and then, by the analysis of\hardsubsectionref{9}{4}{2}, we have

., ., 1 ri''sin(2Hi + l)6 . 1 r ' sin(2m + l)6' . om zj

TT j Sin f TT J sm (7

Again, on integrating the equation

sin (2m + ) 6

sin 6

= 1 + 2 cos 2 + 2 cos 4 + . . . + 2 cos 2 i 6,

we have 1 - -x - -dd = l'n-,

\ sin

5''siu(2m + l)

J so that

1 ri' sin(2m + l)

>S; ( )-H/(- + 0)+/( --0) =i/ '

  1 1*' sin(2m + l) J . \ 2 \, . \ y

In order to prove that

lim <S', (.r) = \ /(./ + 0) f x - 0)),

it is therefore sufficient to prove that

where ( ) stands in turn for each of the functions

f x + 20) -fix + 0), f x - 26) -fix - 0).

Now, by\hardsubsectionref{3}{6}{4} example 4, 6 <t) (6) cosec 6 is a func on with limited
total fluctuation in an interval of which = is an end-point*; and so
we may write

  < ( ) cosec = Xi( )-X.( ), where Xi C )- %2 (6) are bounded positive
increasing functions of 6 such that

X (+0) = xA+0) = 0.

Hence, given an arbitrary positive number e, we can choose a positive
number 5 such that

whenever % h8.

We now obtain inequalities satisfied by the three integrals on the
right of the obvious equation

r ' sin(2m + l), ..,. f  / . i x zj < ( ) jzj

 - - -, - -6(6) (16=1 sm(2 i + l)6 .-. a6

Jo sm6 .' ks sin r>

( 'sm 2m + l)6,,, psin(2m + l),,,. + 1 - '-xA dd-j - g X2( )d6.

* The other end-point is d - (b-x) or d - (x -a), according as (p (0)
represents one or other of the two functions.

W. M. A. 12

%
% 178
%

The modulus of the first integral can be made less than e by taking m
sufficiently large; this follows from\hardsubsectionref{9}{4}{1} (i) since </> (6) cosec
6 has an integral which converges absolutely in the interval hS, Itt).

Next, from the second mean-value theorem, it follows that there is a
number between and S such that

i sin(2 i + l)6'

d

i sin(2m + l)

X, (0) dd = N:i (i S) . ' dd

  / (' + ) sinw,

Since I

sin, .,-,,.,,,,, P sin u

dt is convergent, it follows that I du has an upper

f

Jo

lim |-*'' in(2m + l) p

... ' sm f/ r V /

bound* B which is independent of /3, and it is then clear that

I [i sin (2m + 1)6

N - - '-xi(0)d£ 2Bxi(i8)<2B6.

On treating the third integral in a similar manner, we see that we can

make

  ' sin(2??i + l),

. j - . .( ) c <(4B 4- 1).

by taking m sufficiently large; and so ive have proved that

sin (2m + 1)

... .00-0 sin

But it has been seen that this is a sufficient condition for the limit
of Sm (oc) to be i f(oo + 0) + f(x - 0)]; and we have therefore
established the con- vergence of a Fourier series in the circumstances
enunciated in\hardsubsectionref{9}{4}{2}.

NoTK. The reader should observe that in either proof of the
convergence of a Fourier series the second mean-value theorem is
required; but to prove the summability of the series, the first mean
-value theorem is adequate. It should also be observed that, while
restrictions are laid upon / () throughout the range ( - tt, tt) in
establishing the summability/ at any point .r, the only additional
restriction necessary to ensure convergence is a re- striction on the
behaviour of the function in the immediate neighbourhood of the point
x. The fact that the convergence depends only on the behaviour of the
function in the immediate neighbourhood of x (provided that the
function has an integral which is absolutely convergent) was noticed
by Riemann and has been emphasised by Lebesgue, Series
Trigonometriques p: 60.

It is obvious that the condition t that x should be an interior point
of an interval in which f(t) has limited total fluctuation is merely a
sufficient condition for the con- vergence of the Fourier series; and
it may be replaced by any condition which makes

,. /'i' sin(2TO-f 1)(9, ...,-,,

lim I -;r- -(b(d)dd=0.

m xjo sm(9

* The reader will find it interesting to prove that B= | du = hTr.

J u

t Due to Jordan, Comptes Rendus, xcii. (1881), p. 228.

%
% 179
%

Jordan's condition is, however, a natural modification of the
Dirichlet condition that the function /( ) should have only a finite
number of maxima and minima, and it does not increase the difficulty
of the proof.

Another condition with the same efifect is due to Dini, Sopra le Serie
di Foiirier (Pisa, 1880), namely that, if

*( ) = /(:p + 2 )+/(.r-2 )-/(ar + 0)-/(a.'-0) /,

then / * \&)dQ should converge absolutely for some positive value of
a.

[If the condition is satisfied, given e we can find S so that

/

and then

<\ fTi;

Jo sni

the proof that I - - (f)(6) dBl < t for sufficiently large values of m
follows

from the Riemann-Lebesgue lemma.]

A more stringent condition than Dini's is due to Lipschitz, Journal
f>ir Math. LXiil. (1864), p. 296, namely \ (f) B)\ < 06", where C and
k are positive and independent of 6.

For other conditions due to Lebesgue and to de la Vallee Poussin, see
the latter's Cours d' Analyse In/lnitesimale, ii. (1912), pp. 149-150.
It should be noticed that Jordan's condition differs in character from
Dini's condition; the latter is a condition that the series may
converge at a point, the former that the .series may converge
throughout an interval.

\Subsection{9}{4}{4}{The uniformity of the convergence of Fourier series.}
Let f(t) satisfy the conditions enunciated in ! 9'42, and further let
it be continuous (in addition to having limited total fluctuation) in
an interval (a, b). Then the Fourier series associated with f (t)
convei'ges uniformly to the sum f(x) at all points .v for which a + b
x b - b., where 8 is any positive number.

Let h t) be an auxiliary function defined to be equal to/( ) when a t
b and equal to zero for other values of t in the range (-tt, tt), and
let a, /3,i denote the Fourier constants oi hit). Also let S i.v)
denote the sum of the first m + 1 terms of the Fourier

  ' TO '

series associated with h t).

Then, by\hardsectionref{9}{4} corollary 2, it follows that \ af - 2 (a cos; .i' +
/3,i sin j7) is xmiformly summable throughout the interval (o +S, b-h)
; and since

i a cos 1lX- n sin nx \ \$ an- n'r-,

which is independent of x and which, by\hardsubsectionref{9}{4}{1} (ii), is 1/n), it
follows from\hardsectionref{8}{5} corollary that

 ao+ 2 (an cos nx + n sin n.v) n-l

converges uniformly to the sum h x), which is equal to/(.i'). Now, as
in\hardsubsectionref{9}{4}{2},

   J hib-x) smO ' ' 7rj ( \ a) sm ' " '

12-2

%
% 180
%

As in\hardsubsectionref{9}{4}{1} we choose an arbitrary positive number e and then enclose
the points at

p f which f(t) is unbounded in a set of intervals Sj, 80, ... 8,, such
that 2 I \ f(i) \ dt <e.

r = l J Sr

If K be the upper bound of >f t) \ outside these intervtils, we then
have, as in\hardsubsectionref{9}{4}{1},

i S, ) - S[ l ) I < ( 1 + 2e) cosec 8,

where the choice of n depends only on a and b and the form of the
function /(<). Hence, by a choice of m mdependent of x we can make

arbitrarily small; so that S x) - S ' x) tends uniformly to zero.
Since Sj x)- f(x) uniformly, it is then obvious that S x)-*~f x)
uniformly; and this is the result to be proved.

Note. It must be observed that no general statement can be made about
uniformity or absoluteness of convergence of Fourier series. Thus the
series of § 9* 11 example 1 converges uniformly except near .r = (2 +
l) n but converges absolutely only when x=nir, whereas the series of §
9'11 example 2 converges uniformly and absolutely for all real values
of x.

Example 1. If 6) satisfies suitable conditions in the range (0, tt),
shew that

+ lim [''''2 L( : <i.in-e)d0

Example 2. Prove that, if > 0,

' .;'" Siu(2 +1) a 7A . .

hm / -a e- ae de = \ Tr coth hair.

\addexamplecitation{Math. Trip. 1894.} [Shew that

  . e~ d6= hm .], e-a dd

J SI" d,n - J sm

w-*-* J sm 6

  sin(27i+l) e-a\&dd

sin 6 1 - e-af' and use example l.J

Example 3. Discuss the uniformity of the convergence of Fourier series
by means of the Dirichlet-Bonnct integrals, without making use of the
theory of summability.

\Section{9}{5}{The TODO-TODO* theorem concerning Fourier constants.}
Let f dc) he hounded in the interval (- tt, tt) and let I f(x) dx
exist, so

 -n

* Math. Ann. lvii. (1903), p. 429. Liapounoff discovered the theorem
in 1896 and published it in the Proceedings of the Math. Soc. of the
Univ. of Kharkov. See Comptes Rendw, cxxvi. (1898), p. 1024.

%
% 181
%

that the Fourier constants, n off x) exist. Then the series

 a + i an + h, )

n = l 1 t"

is convergent and its sum is* - I fix)Ydx.

It will first be she m that, with the notation of S 9"4,

lim \ f x)-- 1 Sn(x)\ dx = 0.

Divide the interval ( - tt, tt) into 4r parts, each of length 8; let
the upper and lower bounds of /(.v) in the interval 2p~l) 8-it, (2p +
'S)8 - 7r be Up, Lp, and let the upper bound of \ f x)\ in the
interval ( - tt, tt) be K. Then, by § 94 corollary 1,

I 1 "t-i

\ f x)-- 2 S x) I "> n=0

< -Zp + 2 / OTsin2 8

< 2A'[l + l/ wsin2 S ],

when X lies between 2pb and (2/? + 2) S.

Consequently, by the first mean-value theorem,

fir 1 w-l 12 C 1 1 f 2r-l 4/f;. \

\ f )-- 2 >% x)\ dx<2K\ \ + r! T-.N2a 2 Up-Lp) ~ \

Since /(j;) satisfies the Rieraann condition of integrability (§
4'12), it follows that both

r-l r-l

48 2 U2p-L2p) and 48 2 ((7.,p+i~ 2iJ + i) can be made arbitrarily
small by giving ? a

sufficiently large value. When r (and therefore also 8) has been given
such a value, we may choose mj so large that ?/ ) ! sin 8 is
arbitrarily small. That is to say, we can make the expression on the
right of the last iiiecpiality arbitrarily small by giving m any value
greater than a determinate value nti. Hence the expression on the left
of the inequality tends to zero as wi-s-jo.

But evidently

/( )-- % Sn(x)\ dx

= r \ f x)-t'' A,, x)Xdx

m - 1 ni - 1 )j Vj

/( )- S 4 ( )+ S - ( ) dx

\ f(x)- X An x)\ dx+ 1 -An(x)\ dx

1 JJ = j .' -TT [n = ofn

/it ( m-l ) (m-l )

+ 2 f(x) - S An (x)\ At A,, x)\ dx

J -n ( n=0 J n=--0 )

= \ f(x) - 1 An x)\ dx + t n (an' + K

/ I *. - A \ ill, -t \ A

H =

* This integral exists by\hardsubsectionref{4}{1}{2} example 1. A proof of the theorem has
been given by de la Valine Poussin, in which the sole restrictions on
/(.r) are that the (improper) integrals of f(x) and f(x) ' exist in
the interval -ir, tt). See his Cours d' Analyse Infinitesimale, ii.
(1912), pp. 165-166.

%
% 182
%

,'n /"T hn-l "I

since I f w)Ar(x)dx=\ ] S An (i 'n A, (x) dx

J -n J -n (. = J

when r = 0, 1, 2, ... m-1.

Since the original integral tends to zero and since it has been proved
equal to the sura of two positive expressions, it follows that each of
these expressions tends to zero; that is to say

\ f x)- 2 An x)\ dx-*0.

Now the expression on the left is equal to

r f x)Ydx-2r \ f(x)-'''SA,,, A.\'''i'A,, x)\ dx

rn- im-l "la

- 2 ( ) c?

J -TT in=0 J

/'tt r TT I'm- 1 ") 2

 = \ f(x)Ydx- \ 2,i(*')t d

J -n ' .' -TT [h = )

rir ( W-1 )

= /( )P dx-irl- tto" + 2 (a,r + h,i) \,

J -n (."' n = \ )

so that, as m- >oo,

This is the theorem stated.

Corollary. ParsevaVs theorem*. l f(x), F(x) both satisfy the
conditions laid on/(.r) at the beginning of this section, and if, B
be the Fourier constants of F x), it follows by subtracting the pair
of equations which may be combined in the one form

f" f x)±F x) dx = n\ \ a,±A,f+ I a,,±A,,f + K±B,,) \

J -TT L =1 J

that I / x) F(x) rfx = TT j ao J + (a + 6 5,0 \ .

\Section{9}{6}{Riemann's theory of trigonometrical series.}

The theory of Dirichlet concerning Fourier series is devoted to series
which represent given functions. Important advances in the theory were
made by Riemarin, who considered properties of functions defined by a
series

of the typef o+ S (a,i cos wa' + 6 sin /ia-), where it is assumed that

lim (In cos nx + 6 sin nx) = 0. We shall give the propositions leading
up to

Riemann's theorem; : that if two trigonometrical series converge and
are equal

* Mem. par divers senium, i. (1805), pp. 639-648. Parseval, of course,
assumed the permissi- bility of integrating the trigonometrical series
term-by-term.

t Throughout §§ 9-6-9-6i52 the letters, 6 do not necessarily denote
Fourier constants. X The proof given is due to G. Cantor, Journal fiir
Math, lxxii. (1870), pp. 130-142.

%
% 183
%

at all points of the range (- tt, tt) with the possible exception of a
finite number of points, corresponding coefficients in the two series
are equal.

\Subsection{9}{6}{1}{Riemann's associated function.}

Let the sum of the series rrOa + 2 (a,i cos nx + hn sin nx) = A - S An
x),

  n-\ n=l

at any point x where it converges, be denoted by /(a;).

Let F(x) = \ a,x'- i n- An x).

Then, if the series defining f(x) converges at all points of any
finite interval, the series defining F(x) converges for all real
values of x.

To obtain this result we need the following Lemma due to Cantor :

Cantor's lemma*. If lim J x) = for all values of x such that a x b,
then o -0,

 .- 0.

For take two points x, x-i-8 of the interval. Then, given e, we can
find ?io such thatt,

when n> n

j a cos nx + b sin nx < e, \ a cos n x + 8) + 6 sin n x + S) | < e.

Therefore

I cos nb (a cos ax + 6 sin nx) + sin n8 ( - sin nx + 6 cos nx) \ < e.
Since 1 cos n8 (a cos ?; + b sin nx) \ < 6,

it follows that | .sin n8 ( - a sin nx + bn cos nx) i < 2f,

and it is obvious that ] sin 7i8 (a cos nx + bn sin nx) j < 2c.
Therefore, squaring and adding,

( H6 2)*! siii 8! <2ev'2.

Now suppose that,, b have not the unique limit; it will be shewn
that this hypothesis involves a contradiction. For, by this
hypothesis, sotne positive number e exists such that there is an
unending increasing sequence /ii, n.,, ... of values of n, for which

Now let the range of values of S be called the interval /j of length L
on the real axis.

Take n the smallest of the integers ?v s ch that Hi Zi > 27r; then
sin /i/// goes through all its phases in the interval I; call I.,
that sub-interval; of /j in which .sin j'y > 1 V2; its length is
7r/(2H/) = Zo. Next take iio the smallest of the integers nr >)ii)
such that n 'Zo >27r, so that sin n 'i/ goes through all its phases in
the interval I2; call I3 that .sub- interval + of L in which sin ?j2'y
> l/\ \ 2; its length is 7r/(2n2') = - 3. We thus get a sequence of
decreasing intervals Zj, /.,, ... each contained in all the previous
ones. It is obvious from the definition of an irrational number that
there is a certain point a which is not outside any of these
intervals, and sin a l/ /2 when w= 2', ... ( .'r + i> ' /)- For these
values of n, a,' + b,c) sin na > 2€q 1%. But it has been shewn that
corresponding

* Riemann appears to have regarded this result as obvious. The proof
here given is a modification of Cantor's proof, Math. Ann. iv. (1871),
pp. 139-143, and Journal fiir Math, lxxii. (1870), pp. 130-138.

t The value of /( depends on .r and on 5.

X If there is more than one such sub-interval, take that which lies on
the left.

%
% 184
%

to given numbers a and e we can find ?Jo such that when n > .o, a J +
6 ) (sin na) < 2e f2; since some values of / are greater than n, the
required contradiction has been obtained, because we may take e< e,,;
therefore Un- O, 5 -*-0.

Assuming that the series defining /(a:) converges at all points of a
certain interval of the real axis, we have just seen that a,i- >0, 6
,- >0. Then, for all real values of x, \ an cos nx + hn sin nx \ (a,/
+ 6,f)2- 0, and so, by\hardsubsectionref{3}{8}{4}<, the

00

series A x-- S n~-An x) = F(x) converges absolutely and uniformly for
all real values oi x; therefore, \hardsubsectionref{3}{3}{2}), F(x) is continuous for all
real values of c.

\Subsection{9}{6}{2}{Properties of Riemann's associated function;
  Riemann's first lemma.}

It is now possible to prove Riemann's first lemma that if F x- '2a) +
F x- 2a) - 2F x)

G x, a) =

4a

then limG(*', a.)=f(x), provided that S An x) converges for the value
of x

o-*-0 =

under consideration.

Since the series defining F x), F x ± 2a) converge absolutely, we may
rearrange them; and, observing that

cos n (x + 2a) + cos n (x - 2a) - 2 cos nx = - 4 sin na cos nx,

sin n x + 2a) + sin n (x - 2a) - 2 sin nx = - 4 sin na sin nx,

it is evident that

/-, / \ A V /sinna\'' .,,

G(x,a) = Ao+ 1 An x).

M=i \ na J

It will now be shewn that this series converges uniformly with regard
to

00

a for all values of a, provided that S An x) converges. The result
required

is then an immediate consequence of\hardsubsectionref{3}{3}{2} : for, if f (a) = ( j, (a
0),

and/ (0) = 1, then /" ( ) is continuous for all values of a, and so G
x, a) is a continuous function of a, and therefore, by\hardsectionref{3}{2}, G (x, 0)
= lim G (x, a).

To prove that the series defining G (x, a) converges uniformly, we
employ the test given in\hardsubsectionref{3}{3}{5} example 2. The expression
corresponding to \&> (x) is fn oi), a-nd it is obvious that |/n(a)j 1
; it is therefore sufficient to shew

that X fn+\ (a) - A( ) < i, where K is independent of a.

n = \ '

In fact*, if s be the integer such that s j a j  <(* + 1) i a, when
o=t=0 we have

s-l s-l si i2

SUi a Sin-. a

2 I / + 1 (a) -/. (a) ! = 2 (/ (a) -/ ., (a)) =

* Since x~ sin x decreases as .r increases from to w.

9-62,

9-62i;

TRIGONOMETRICAL SERIES

Also

2

=s+l

l/ +l

(a)-

-fn a)

, /sin2

na (\ 1 M sin2

n

'=?+! 1 a'

W ( + i)VJ

- 1/1

1

1 sin nn - * 1 sin a sin (271 + 1) a 1

-siii2(n+l)a 1

  s + lYa \

 (s2+l)2a "'"

1 sin a 1 1

1 1 sin a 1

TT- a-

f' dx

Is (-v + lY

1 1

-r- 1

 TT- (S+l)

a

%
% 185
%

Therefore
$$
TODO
$$

Since this expression is independent of a, the result required has
been obtained*.

00

Hence, if S A x) converges, the series defining G x, a) converges
uniformly with respect to a for all values of a, and, as stated above,
\ \ {mG x,a)=Gix,0) = A,+ S Ar, x)=f x).

a-J-O M = l

Example. If H (.., . ) = ' '( a m-Fi'c+a-t -F(.-a SHFi.-.-

that 5(.r, a, /3) -/( ) when /(a-) converges if n, /3- -0 in such a
way that a//3 and /a remain finite. \addexamplecitation{Riemann.}

\Subsubsection{9}{6}{2}{1}{Riemann's second lemma.}
With the notation of §§ 9'6-9'62, if

dn, t>ji- >0, /;en Inn ~ - = for all values of x.

a u 'la

For a->;/ '(- '+2 ) + i (;r-2a)-2/ (a;) =Aa+ S - An x); but

  = 1 '' '

  sin" ?jft by\hardsubsectionref{9}{1}{1} example 3, if a > 0, - - = h (tt - a) : and so,
since

/j=i n ot

.,,, sin wa ., .

M=i " a

  = 1 (. m = l W Of )

it follows from\hardsubsectionref{3}{3}{5} example 2, that this series converges uniformly
with regard to a for all values of a greater than, or equal to, zerof.

* This inequality is obviously true when a = 0. t If we define g (a)
by the equations ( j (a) = then (7,j (a) is continuous when a 0, and
g, \ .l (a) <; (a)

t If we define fif (a) by the equations n, (a) = (ir - a) - 2 - -,
(a + 0), and,9,((0) = i7r,

m=l '"' *

%
% 186
%

But lim la-i [F x - 2a) + F x - 2a) - 2F(.x)]

= lim

.4o (x)a + 7r-a.)A, (x) + 2 g (a) [An+i (x) - An ix)

w = l

and this limit is the value of the function when a = 0, by | 3"32;
and this value is zero since lim An(x) = 0. By symmetry we see that
lim = lim.

\Subsection{9}{6}{3}{Riemann's theorem* on trigonometrical series.}

Two trigonometrical series which converge and are equal at all j oints
of the range (- tt, tt), witli the possible exception of a finite
number of points, must have corresponding coefficients equal.

An immediate deduction from this theorem is that a function of the
type considered in\hardsubsectionref{9}{4}{2} cannot be represented bv auy trigonometrical
series in the range (- tt, tt) other than the Fourier series. This
fact was first noticed by Du Bois Reymond.

We observe that it is certainly possible to have other expansions of
(say) the form oq + 2 (a, cos i rax + /3, sin rax),

which represent /(a-) between - tt and tt; for write .r=2, and
consider a function c ( ), which is such that ( ( ) =/(2|) when -\ ir
< <\ tv, and (|) =g (|) when - tt < | < - tt, and when tt < < tt,
where g ( ) is any function satisfying the conditions of\hardsubsectionref{9}{4}{3}. Then
if we expand ( (|) in a Fourier series of the form

0,1 + 2 (a cos m + /3 i cos m ),

this expansion represents /(.r) when - Tr<x<ir; and clearly by
choosing the function g (|) in different ways an unlimited number of
such expansions can be obtained.

The question now at issue is, whether other series- proceeding in
sines and cosines of integral multiples of x exist, which differ from
Fourier's expansion and yet represent / (.r) between - tt and tt.

If possible, let there be two trigonometrical series satisfying the
given conditions, and let their difference be the trigonometrical
series

Ao+ t An x)=f(x). =1

Then f(x) = at all points of the range (- tt, tt) w th a finite number
of

exceptions; let i, o be a consecutive pair of these exceptional
points, and

let F (x) be Riemann's associated functiou. We proceed to establish a

lemma concerning the value of F(x) when j< x< fo-

\Subsubsection{9}{6}{3}{1}{Schwartz' lemma.}\footnote{TODO} In the range i<x< ., F(x) is a linear
function of x, if f (x) = in this range.

For if = 1 or if <9= -1

cl> ) = e F x)-Fi,)-l F i,)-F(i,) y,hHx-,),-x) is a continuous
function of x in the range i j;,, nd ( i) = ( ( 2) = 0.

* The proof we give is due to Gr. Cantor, Journal fit r Math, lxxii.
(1870), pp. 139-142. + Quoted by G. Cantor, Journal filr Math, i.xxii.
(1870).

%
% 187
%

If the first term of (f) (x) is not zero throughout the range* there
will be some point x=c at which it is not zero. Choose the sign of \$
so that the first term is positive at c, and then choose h so small
that (f) (c) is still positive.

Since cf) (x) is continuous it "attains its upper bound \hardsubsectionref{3}{6}{2}), and
this upper bound is positive since (f> (c) > 0. Let (x) attain its
upper bound at Cj, so that Ci + |i, Ci4= 2  Then, by Riemanifs first
lemma,

lini </'(gi + °) + < (gi- )-20(< i)\ .

But (c'l + a) (f) (c']), (f> (ci-a) - (p (0|), SO this limit must be
negative or zero.

Hence, by supposing that the first term of (x) is not everywhere zero
in the range ( j, 2)5 we have arrived at a contradiction. Therefore it
is zero; and consequently F(x) is a linear function of x in the range
1 < x < 2- li J lemma is therefore proved.

\Subsubsection{9}{6}{3}{2}{Proof of Riemann's Theorem.}
We see that, in the circumstances under consideration, the curve y = F
x) represents a series of segments of straight lines, the beginning
and end of each line corresponding to an exceptional point; and as F
x), being uniformly convergent, is a continuous function of x, these
lines must be connected.

But, by Riemann's second lemma, even if be an exceptional point,

a a a

Now the fraction involved in this limit is the difference of the
slopes of the two segments which meet at that point whose abscissa is
; therefore the two segments are continuous in direction, so the
equation y = F(x) represents a single line. If then we write F(x) =
cx- c, it follows that c and c' have the same values for all values of
. Thus

X

 A x" - ex - c' = S n~-A n (x), =i

the right-hand side of this equation being periodic, with period 27r.

The left-hand side of this equation must therefore be periodic, with
period

27r. Hence

 4,, = (), c = 0,

and - c' = - ir'-A,i(x).

>i = \

Now the right-hand side of this equation converges uniformly, so we
can multiply by cos nx or by sin nx and integrate.

This process gives

7rn~-a,i = - cj cos nxdx = 0,

7rn~ bn= - c j sin nxdx = 0. * If it is zero throughout the range, F
(x) is a linear function of z.

%
% 188
%
Therefore all the coefficients vanish, and therefore the two
trigonometrical series whose difference is J.o+ S An(x) have
corresponding coefficients equal. This is the result stated in\hardsubsectionref{9}{6}{3}.

\Section{9}{7}{Fourier s representation of a function by an integral.}*

It follows from § 9 "43 that, if / ai) be continuous except at a
finite number of discontinuities and if it have limited total
fluctuation in the range (- x, x ), then, if x be any internal point
of the range (- a, /3),

lim' r siii( + l)( - ) y( ) 11 iTT - sin 6 [f x + 26) +/( - W)].

Now let X be any real number, and choose the integer m so that \ = 27n
+ 1 + 277 where t; < 1.

Then sin X t- x) - sin 2m + 1 ) ( - x)] (t - x)-' f (t) dt

J -a.

= I 2 cos (2 i + 1 + t;) ( - x)] . sin rj t - x)] t - x)-'f t) dt

. -a

as ??i- >x by\hardsubsectionref{9}{4}{1} (ii), since (t - x)~' f (t) sin 7 t - x) has
limited total fluctuation.

Consequently, fi'om the proof of the Riemann-Lebesgue lemma of\hardsubsectionref{9}{4}{1},

it is obvious that if I \ f t)' dt and | 1/(0 i converge, thenf

Jo . -00

1" r f(t)di = h f( + )+f( - )]>

A-*-x J -X

 t-x)

and so

lim r \ I cos u (t - x) du\ f(t) dt = tt f x + 0) +f(x - 0) .

To obtain Fourier's result, we must reverse the order of integration
in this repeated integral.

For any given value of \ and any arbitrary value of e, there exists a
number such that

r\ f(t)idt< el J p

* La Theorie Analytique de la Chaleur, Cb. ix.

t r°° . . . ("

I means the double limit lim I . If this limit exists, it is of course
equal

J -X p -  X, cr -* X J -p

I'-

to lim

p x J -p

%
% 189
%

writing cos u(t- ir) .f(t) = (f> (t, u), we have*

\ r\ I (f>(t, a) dul dt-l \ I (f> t, u) dt] du !

I ' ( J 7 ( . ) '

= j \ t (f) t, a) duldt+ j I (f> t, u)dul dt

-[ \ j <f) t, u) dt\ du - j \ j (f> t,ii)dtldu\ = j \ l 4>(t, u) dul
dt-\ \ I (fi t, u) dt\ du

  P [J ) . fi )

< r \ i''\ < t>(t, u)\ duldt+i'' j' (t>(t, u)[dtdu

< 2\ I \ f t)\ dt<€.

Since this is true for all values of e, no matter how small, we infer
that /oo rk r\ foD /- Qo r\ r\ r- x

=; similarly = .

J a J J J I) J i) Jo J . )

Hence | ir [f x + 0) +/(./; - 0) = lim | cos u (t - oc)f t) dtdu

=

A- . JO. -00

00;" 00

COS u t - oe)f t) dtdu. .' -

This result is known as Fourier s integral tlieorem. Example. Verify
Fourier's integral theorem dinx-tlj (i) fur the function

(ii) for the function defined by the equations

/(.r)=l, -\ < x< ); fix) = 0, (|x|>l). (Rayleigh.j

REFERENCES. G. F. B. Riemann, Ges. Math. Werke, pp. 213-250. E. W.
HoBSON, Functions of a Real Variable (1907), Ch. vii. H. Lebesgde,
Lemons sur les Series Trigonometriques. (Paris, 1906.) C. J. DE LA
VallJ e Poussix, Cours d: Analyse Infinitesimale, ir. (Louvain and
Paris,

1912), Ch. IV.

H. Burkhardt, Ena/clopddie der Math. WUg. ii. 1 (7). (Leipzig, 1914.)

G. A. Carse and G. Shearer, A course in Fourier's analysis and
periodogram analysis

(Edinburgh Math. Tracts, No. 4, 1915).

* The equation I / = I I is easilj' justified by § 4-.3, by
considering the ranges within

which /(x) is continuous.

t For a proof of the theorem when f(x] is subject to less stringent
restrictions, see Hobson, Functions of a Real Variable, %% 492-493.
The reader should observe that, although

Um I I exists, the repeated integral I \ I sin m (( -x) duj- /(<) dt
does not,

A-*oo y -00 7 7 -00 U J

%
% 190
%

Miscellaneous Examples.

1. Obtain the expansions

  ' l-2rcos2 + r2

(6) - log ( 1 - 2r cos z + r ) = - r cos J - '' cos 2 - - r cos 3 - .
. .,

,,, ?'sinz . 1 . 1 . \,

(c) arc tan, = rsins + -?' sin 22 + ?: /' sin 3s + ...,

   r cos 2 2 3

,, 2rsin2 . I,  1 c

(a) arc tan 3- = r sin z- -r sin 3 + - r° sm 02 + . . .,

1 - r'' 3 5

and shew that, when | r | < 1, thej- are convergent for all values of
z in certain strips parallel to the real axis in the -plane.

2. Expand x and x in Fourier sine series valid when - tt < .r < tt;
and hence find the value of the sum of the series

sin X - sin 2.r + -3 sin 3*' - - sin 4.r + . . ., for all values of
a.-. \addexamplecitation{Jesus, 1902.}

3. Shew that the function of x I'epresented by 2 ~i sin sin /2a, is
constant

n = \

(0 < X < 2a) and zero (2a < x < tt), and draw a graph of the function.

\addexamplecitation{Pembroke, 1907.}

4. Find the cosine series representing /(.r) where

/ (x) = sin X + cos X (0 < . ' tt ),

/(.r) = sin.r-cos.r (in- .r < tt). \addexamplecitation{Peterhouse, 1906.}

5. Shew that

sin 3n- sinSjTA' sin Ttt., r -,

sin TTX + - - + - - + - - - + ... = Jtt [x],

where [x] denotes 4-1 or - 1 according as the integer next inferior to
x is even or uneven, and is zero if x is an integer. \addexamplecitation{Trinity, 1895.}

6. Shew that the expansions

log and

1

2 cos - X 2

log 2 .sin - X

= cos X-- cos 2x + - cos 3x

- COS X - - COS 2x - - cos 3x

are valid for all real A'alues of x, except multiples of tt.

7. Obtain the expansion

"(-)"* cos m.r, o M / 1 \ 1 /  T  \

2 ~ - -yT-. = (cos X + cos 2x) log ( 2 cos -x\ +-,:?; (sin zx + sin x)
- cos x,

and find the range of values of . ' for which it is applicable.
\addexamplecitation{Trinity, 1898.}

8. Prove that, if < a- < 27r, then

sin X 2 sin 2x Z sin 3.:*; \ tt sinh a (tt - x) a' + Y' " a +2- "a T +
' ' " " 2 ' sinh aTr '

\addexamplecitation{Trinity, 1895.}

FOURIER SERIES

%
% 191
%

9. Shew that between the values -tt and +n o( x the following
expansions hold

2 . / sin 2 sin 2x 3 sin Zx

sin in.v = - sin ran -. r - -, - + -= - - ...

IT \ V-m'' -l -m 32-7 2

2 . ( \ m cos X m cos 2.r m cos 3 ' cos mx = - sin ??l7r;;- + - - -r
+ - 5 -

TT \ 2m P - 7rt- 22-77l2 32-J/l2

e-mx g-mx 2 / 1 ??IC0S.¥ ?/l COS 2.r 7?2C0S3.1- \

10. Let X be a real variable V)etween and 1, and let n be an odd
number 3. Shew that

(-1)* = - + - 2 -tan - cos2OTn-:r, if is not a multiple of -, where s
is the gi-eatest integer contained in nx; but

 1 2 1 WITT

0= - + - 2 - tan - cos 2m7rx, n 7r, =i i n

if jp is an integer multiple of 1/n. \addexamplecitation{Berger.}

11. Shew that the sum of the series

X

 5 + 47r ~ 2 m~ sin mn cos 2m7rx

m = 1

is 1 when 0<x<l, and when fj<A'<l, and is -1 when l<.r<|.
\addexamplecitation{Trinity, 1901.}

12. If

shew that, when - 1 <x <\,

a" V (x)

ae"'

e" - 1 =o n

cos 2irx+

s47r.r cosBttj: \ \ -i 2" ""*' " i-

22"

32

2/i !

sin 47ra; sin 6nx "2-" + 1 "*" "32

. -, 8IU 'iTTX Sin OTTX-, .,, -,. .,, .

sm27ra;+-j; +-; +. .. = (-)"+! \ \,, V,,, x).

22" 2 + l

2 + 1 :

\addexamplecitation{Math. Trip. 1896.}

13. If 7?i is an integer, shew that, for all real values of .v,

cos2' a;=2

1.3.5...(2m-

2.4.6 ...2m

1) fl, r,i (2 i + l

m m- )

cos 2x + - - ~ cos 4x

(ot4-1)(7 + 2)

m m-l) m - 2) * (m + l)(?tt + 2)(w+3)

cos 6.r + ...V,

  \ ., 4 2.4.6...(27ft-2) (1 2w-l . f2m-l)(27n-3),,

r.ns2m-l\;p|\ \ - \ - \ \ \ \ J -!;; + ., COS 2x+-,,,, -;
C0S4a-+.

1.3.5 ...(2 i-l) (2 2/u + l

(2to + 1)(2 h-3)

/

14. A point moves in a straight line with a velocity which is
initially u, and which receives constant increments, each equal to u,
at equal intervals r. Prove that the velocity at any time t after the
beginning of the motion is

u ut V- 1 . 2miit 77 + - + - 2 - sin,

and that the di-stance traversed is

lit

( + ) + T7>-o32 2 -2 cos

2mTTt

\addexamplecitation{Trinity, 1894.}

%
% 192
%

15. If

X 2 = 1

f x)= 2 sin (6?i - 3) j; - 2 2 -sm 2n-l)x

ji=i '2n - 1 71=1 zn - i

3 v'3 f . sin 5.r sin 7.r sin ll r ]

shew that / ( + 0) =/ (tt - 0) = - tt,

and /G +0)-/(i7r-0)=-i7r, /(;:|7r+0)-/(37r-0)=*7r.

Observing that the last series is

6 sin (2ft- 1 ) TTsin 2n-l)x

draw the graph of /(* ) \addexamplecitation{Math. Trip. 1893.}

16. Shew that, when <.r < n,

 / N 2 /3/ 1 .1 1 Ti, \

f(,v) = '- I cos .r-- cos 0. '+- cos /.<--- cos 11. r,'-f... ' ' 3 \
/ 11 /

= sin 2x' + - sin 4.f + - sin 8.v + r sin lO.r + . . . 2 4 5

where / ) = i (0< < 7r),

f x) = (l7r<.r<g7r),

f x)=-\ v %7T<X<1t).

Find the sum of each series when .r = 0, \ ir, qtt, n, and for all
other values of x.

\addexamplecitation{Trinity, 1908.}

17. Prove that the locus represented by

2 2 - '' " '*'' ny =

n=l "

is two systems of lines at right angles, dividing the coordinate plane
into squares of area tt'\ \addexamplecitation{Math. Trip. 1895.}

18. Shew that the equation

" ( - )" ~ sin ny cos nx \

n=i n ~

represents the lines i/= ±mir, (wi = 0, 1, 2, ...) together with a set
of arcs of ellipses whose

semi-axes are tt and nlslZ, the arcs being placed in squares of area
27r" . Draw a diagram

of the locus. \addexamplecitation{Trinity, 1903.}

19. Shew that, if tlie point x,y, z) lies inside the octahedron
bounded by the planes

±x±y±z = Tr, then

",,, sin 7ia;sin 7jwsin wj 1

 j-Y- - =2-y-

\addexamplecitation{Math. Trip. 1904.}

20. Circles of radius a are diawn having their centres at the
alternate angular jjoints of a regular hexagon of side a. Sliew that
the equation of the trefoil formed by the outer arcs of the circles
can be put in the form

-nr- = -; + - cos 3(9- --= cos 6 + -- - cos9 - ..., 6s'3a 22.4 5.7 8.
10

the initial line being taken to pass through the centre of one of the
circles.

\addexamplecitation{Pembroke, 1902.}

2m . = 1 H sm

%
% 193
%

21. Draw the graph represented by

rr J 1 ( - )" COS nm6\ m [2 a=i 1 - nm)- J ' where m is an integer.
\addexamplecitation{Jesus, 1908.}

22. With each vertex of a regular hexagon of side 2a as centre the arc
of a circle of radius 2a lying within the hexagon is drawn. Shew that
the equation of the figure formed by the six arcs is

f-=6-3,.i + 2Si< -)-' j- f co.6 <>,

4a a=i 6n-l) bn+l)

the prime vector bisecting a petal. \addexamplecitation{Trinity, 1905.}

23. Shew that, if c> 0,

lim f

e ' cot A" sin 2ii + l)x . dx=- n tanh -ctt.

\addexamplecitation{Trinity, 1894.}

24. Shew that

sin(2 + l) dx 1

lim r

IT O

oth 1.

\addexamplecitation{King's, 1901.}

sin X l+x' 2

25. Shew that, when - 1 < .r < 1 and a is real,

/ '°sin(2n + l) sin(H-.r)(9, 1 sinha.r

hm I -.;; 5 -r, d6= --TT .-, .

  .xjo sm a + 6- 2 smh a

\addexamplecitation{Math. Trip. 1905.}

26. Assuming the possibility of expanding f(x) in a uniformly
convergent series of

the form A -ainkx, where is a root of the equation /-cosa/t-Hft sin a
- = and the

  . . . *

summation is extended to all positive roots of this equation,
determine the constants Jj..

\addexamplecitation{Math. Trip. 1898.}

1 ""

27. If f(x) = -a + 2 a cosnx + b,t>im nx)

2,(=1

is a Fourier series, shew that, iif x) satisfies certain general
conditions,

a = - P i f (t) cos nt ta.n - t --, 6 = - / f t)smnttn,n-t - .

7T J ' 'It ""yO 'It

\addexamplecitation{Beau.}

W. M. A. 13
