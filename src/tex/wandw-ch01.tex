\chapter{Complex Numbers} 

11. Rational numbers.

The idea of a set of numbers is derived in the first instance from the
consideration of the set of positive* integral numbers, or positive
integers; that is to say, the nnmbL-rs 1. 2, 8, 4, .... Positive
integers have many properties, which will be found in treatises on the
Theory of Integral Numbers; but at a very early stage in the
development of mathematics it was found that the operations of
Subtraction and Division could only be performed among them subject to
inconvenient restrictions; and consequently, in elementary
Arithmetic, classes of numbers are constructed such that the
operations of subtraction and division can always bB performed among
them.

To obtaiti a class of numbers among which the operation of subtraction
can be performed without restraint wc construct the class of integers,
which consists of the class of positive f integers (+1, +2, +3, ...)
and of the class of negative integers (-1, -2, -3, ...) and the number
0.

To obtain a class of numbers among which the operations both of sub-
traction and of division can be performed freely:):, we construct the
class of rational numbers. Symbols which denote members of this class
are |, 3,

0, -Y--

We have thus introduced three classes of numbers, (i) the signless
integers, (ii) the integers, (iii) the rational numbers.

It is not part of the scheme of this work to discuss the construction
of the class of integers or the logical foundations of the theory of
I'ational numbers .

The extension of tbe idea of number, which has just been described,
was not effected without some opposition from the more conservative
mathematicians. In the latter half of the eighteenth century, Maseras
(1731-1824) and Frend (1757-1841) published works on Algebra,
Trigonometry, etc., in which the use of negative numbers was
disallowed, although Descartes had used them imrestrictedly more than
a hundred years before.

* Strictly speakinpr, a more appropriate epithet would be, not
positive, but signless.

t In the strict sense.

:*: With the exception of division by tbe rational number 0.

§ Such a discussion, defining a rational number as an ordered
number-pair of iute ers in a similar manner to that in which a complex
number is defined in § 1'3 as an ordere i number-pair of real numbers,
will be found in Hobsou's Functions of a Real Variable, §§ 1-12.

1—2



4 THE PROCESSES OF ANALYSIS [CHAP. I

A rational number x may be represented to the eye in the following
manner :

If, on a straight line, we take an origin and a fixed segment OPi (Pi
being on the right of 0), we can measure from a length OFx such that
the ratio OPx/OPi is equal to x; the point P is taken on the right or
left of according as the number x is positive or negative. We may
regard either the point P or the displacement OP (which Avill be
written OPx) as repre- senting the number x.

All the rational numbers can thus be represented by points on the
line, but the converse is not true. For if we measure off on the line
a length OQ equal to the diagonal of a square of which OPi is one
side, it can be proved that Q does not correspond to any rational
number.

Points oil the line which do not represent rational numbers may be
said to represent irrational numbers; thus the jjoint Q is said to
rej resent the irrational number , /2 = l •414213.... But while such
an explanation of the existence of irrational numbers satisfied the
mathematicians of the eighteenth century and may still be sufficient
for those whose interest lies in the applications of mathematics
rather than in the logical upbuilding of the theory, yet from the
logical standpoint it is improper to introduce geometrical intuitions
to supply deficiencies in arithmetical arguments; aud it was shewn by
Dedekind in 1858 that the theory of irrational numbers can be
established on a purely arithmetical basis without any appeal to
geometry.

1"2. Dedekind' s* theory of irrational numbers.

The geometrical property of points on a line which suggested the
starting point of the arithmetical theory of irrationals was that, if
all points of a line are separated into two classes such that every
point of the first class is on the right of every point of the second
class, there exists one and only one point at which the line is thus
severed.

Following up this idea, Dedekind considered rules by which a
separationf or section of all rational numbers into two classes can be
made, these classes (which will be. called the Z-class and the
P-class, or the left class and the right class) being such that they
possess the following properties :

(i) At least one member of each class exists.

(ii) Every member of the X-class is less than every member of the
P-class.

It is obvious that such a section is made b ' an rational number x;
and X is either the greatest number of the Z-class or the least number
of the

* The theory, though elaborated in 1858, was not published before the
appearance of Dede- kind's tract, Stetigkeit und irrationale Zahlen,
Brunswick, 1872. Other theories are due to Weitrstrass [see von
Dantscher, Die Weierstrasa'sche Theorie der irrationulen Zahlen
(Leipzig, 1908)] and Cantor, Math. Ann. v. (1872), pp. 123-130.

t This procedure formed the basis of the treatment of irrational
numbers by the Greek mathematicians in the sixth aud fifth centuries
b.c. The advance made by Dedekind consisted in observing that a purely
arithmetical theory could be built ap on it. *



1'2] COMPLEX NUMBERS 5

i?-class. But sections can be made in which no rational number x plays
this part. Thus, since there is no rational number* Avhose square is
2, it is easy to see that we may form a section in which the i?-c]ass
consists of the positive rational numbers whose squares exceed 2, and
the Z-class consists of all other rational numbers.

Then this section is such that the i?-class has no least member and
the Z-class has no greatest member; for, if x be any positive
rational fraction,

and 2 are in order of magnitude; and therefore given any member x of
the Zr-class, we can always find a greater member of the Z-class, or
given any member x of the 7?-class, we can always find a smaller
member of the /i- class, such numbers being, for instance, y and \ j ,
where y' is the same function of x' as y of x.

If a section is made in which the i -chiss has a least member Jo, or
if the /> class has a greatest member J,, the section determines a
rational-real number, which it is convenient to denote by the samef
symbol Ao\ or A .

If a section is made, such that the i -class has no least member and
the Z-class has no greatest member, the section determines an
irrational-real number I.

If X, y are real numbers (defined by sections) we say that x is
greater than 3/ if the Z-class defining x contains at least two§
members of the i -class defining y.

Lot a, , ... be real numbers and let .4j, i?,. ... be any members of
the corresponding Z-classes while A., B„... are any members of the
corresponding i -classes. The classes of which A , Ao, ... are
respectively members will be denoted by the symbols A ), A. ), ....

Then the sum (written a + ) of two real numbers a and /3 is defined as
the real number (rational or irrational) which is determined by the
Z-class A + B,) and the E-class A. + B.,).

It is, of course, necessary to prove that these classes determine a
section of the rational numbers. It is evident that A i + 5, < A, +
B., and that at least one member of each of the classes Ai + B ),
A.,-\-B.,) exists. It remains to prove that there is, at most, one
rational

* For if piq be such a number, this fraction being in its lowest
terms, it may be seen that (2'if-i')/(p-9) is another such number, and
0<p-q<q, so that pjq is not in its lowest terms. The contradiction
implies that such a rational number does not exist.

+ This causes no confusion in practice.

X B. A, W. Russell defines the class of real numbers as actualhj being
the class of all L-classes; the class of real numbers whose i classes
have a greatest member corresponds to the class of rational numbers,
and though the rational-real number x which corresponds to a rational
number .r is conceptually distinct from it, no confusion arises from
denoting both by the same symbol.

§ If the classes had only one member in common, that member might be
the greatest member of the JL-class of .r and the least member of the
i -class of y.



6 THE PROCESSES OF ANALYSIS [CHAP. I

number which is greater than every Ai + B and less than every A., +
B.,; suppose, if possible, that tliere are two, .v and y (?/>.r). Let
cfj be a member of (.Ij) and let a-> be a member of (Ao); and let lY
be the integer next greater than (a2- i)/ i(y-. ) - Take the last of

the numbers ai-|- ((,,\ flj), (where m=0, 1, ... lY), which belongs to
(A ) and the first of

them which belongs to (Ao); let these two numbers be Cj, c,. Then

 2 -' i = Jr ao - i) < i (3/ - )- Choose c/i, do in a similar maimer
from the classes defining /3; then

C2 + cL - f 1 - c?i <y - A'. But C2 + d., i Ci+di x, and therefore C2
+ c?2-Ci-c/i > i/-.?-; we have therefore arrived at a contradiction by
supjjosing that two rational numbers a;, y exist belonging neither to
( Jj + B ) nor to ( 2 + 2)-

If every rational number belongs either to the class (J 1 + 1) or to
the class ( 2+ 2), then the classes (Jj + i), ( 2 + 2) define an
irrational number. If one rational number.)- exists belonging to
neither class, then the Z-class formed by x and (Jj+ j) and the
i2-class ( 2 + -52) define the rational-real number x. In either
ca.se, the number defined is called the sum a + /3.

The difference a-/3 of two real numbers is defined by the Z-class
(J1-Z2) and the Z'-class ( 2- 1).

The product of two positive real numbers a, /3 is defined by the 7
class A B-i) and the Z-class of all other rational numbers.

The reader will see without difficulty how to define the product of
negative real num- bers and the quotient of two real numbers; and
further, it may be shewn that real numbers may be combined in
accordance with the associative, distributive and commuta- tive laws.

The aggregate of rational-real and irrational-real numbers is called
the aggregate of real numbers; for brevity, rational-real numbers and
irrational- real numbers are called rational and irrational numbers
respectively.

1"3. Complex numbers.

We have seen that a real number may be visualised as a displacement
along a definite straight line. If, however, P and Q are any two
points in a plane, the displacement PQ needs two real numbers for its
specification; for instance, the differences of the coordinates of P
and Q referred to fixed rectangular axes. Ifthe coordinates of P be (
, 77) and those oi Q( + cc,7 +i/), the displacement PQ may be
described by the symbol [x, y]. We are thus led to consider the
association of real numbers in ordered* pairs. The natural definition
of the sum of two displacements [x, y\ [x, y'] is the displacement
which is the result of the successive applications of the two
displacements; it is therefore convenient to define the sum of two
number-pairs by the equation

[.c, y] + [x', y'] = [x + x\ y -f y'].

The order of the two terms distiuguishes the ordered number-pair [.r,
ij] from the ordered number-pair [? ,r].



rS] COMPLEX NUMBERS 7

The product of a number-pair and a real, number x is then naturally
defined by the equation

x X \ x, y\ = \ x'x, x'y\

We are at liberty to define the product of two number-pairs in any
convenient manner; but the only definition, which does not give rise
to results that are merely trivial, is that symbolised by the equation

[x, y] X [x, ij] = [xx' - i/t/', xy + xy\

It is then evident that

[x, 0] X [x, y ] = [xx, xy' = xx [x, y']

and [0, y] x [x, y] = [- yy', x'y] = y x [- y', x'].

The geometrical interpretation of these results is that the effect of
multiplying by the displacement [x, 0] is the same as that of
multiplying by the real number x; but the effect of multiplying a
displacement by [0, y] is to multiply it by a real number y and turn
it through a right angle.

It is convenient to denote the number-pair [x, y] by the compound
symbol x + iy; and a number-pair is now conveniently called (after
Gauss) a complex number; in the fundamental operations of Arithmetic,
the complex number x+ iO may be replaced by the real number x and,
defining i to mean + il, we have i- = [0, 1] x [0, 1] = [— 1, 0]; and
so r may be replaced by — 1.

The reader will easily convince himself that the definitions of
addition and multiplication of numbei--pairs have been so framed that
we may perform the ordinary operations of algebra with complex numbers
in exactly the same way as with real numburs, treating the symbol i as
a number and replacing the product it by — I wherever it occurs.

Thus he will verify that, if a, b, c are complex numbers, we have

a + b = b + a,

ab = b(i,

(a + b) +c = a +(b -\-c),

ab .c = a.bc,

a b -\- c)= ab + ac,

and if ab is zero, then either a or 6 is zero.

It is found that algebraical operations, direct or inverse, when
applied to complex numbers, do not suggest numbers of any fresh type;
the complex number will therefore for our purposes be taken as the
most general type of number.

The introduction of the complex number has led to many important
developments in mathematics. Functions which, when real variables only
are considered, appear as essentially distinct, are seen to be
connected when complex variables are introduced :



8 THE PROCESSES OF ANALYSIS [CHAP. I

thus the circular functions are found to be expressible in terms of
exponential functions of a complex argument, by the equations

TODO

Again, many of the most important theorems of modern analysis are not
true if the numbers concerned are restricted to be real; thus, the
theorem that every algebraic equation of degree n has % roots is true
in general only when regarded as a theorem concerning complex numbers,

Hamilton's quaternions furnish an example of a still fiu'ther
extension of the idea of number. A quaternion

iv+xi+1/j + zk

is formed from four real numbers w, .i; y, z, and four number- units
1, i, j, l\ in the same way that the ordinary complex number x-\-iy
might be regarded as being formed from two real numbers x, y, and two
number-units 1, i. Quaternions however do not obey the commutative law
of multiplication.

1'4. The modulus of a complex number.

Let X + iy be a complex number, x and y being real numbers. Then the
positive square root of x- 4- y- is called the modulus of x + iy), and
is

written

x + iy.

Let us consider the complex number which is the sum of two given
complex numbers, x + iy and u + iv. We have

TODO

The modulus of the sum of the two numbers is therefore

\ \ {X + Uf + ((/ + V) 2,

or [ x- + y-) + ur + V-) -+• 2 xu + ijv)] .

But

 \ x-\-iy\-\-\ u + iv\ ] -= [ x- + /y"') + (w- + v' yf

= ( 2 4- y ) + ( <2 +;2 + 2 x' + t/'f tC + V') '

= (x- + y-) + (u + tj2) + 2 (xu + yv)- -h (xv - yu)-] , and this
latter expression is greater than (or at least equal to) x- + y' ) +
(u" + V-) + 2 (xu + yv). We have therefore

TODO

i.e. the moduhis of tJie sum of two complex numbers cannot be greater
than the sum of their moduli; and it follows by induction that the
modulus of the sum of any number of complex numbers cannot be greater
than the sum of their moduli.



14, 1-5] COMPLEX NUMBERS 9

Let us consider next the complex number which is the product of two
given complex numbers, x + iy and u + iv; we have

 pG 4- iy) u + iv) = xu — yv) + i xv + yu), and therefore

TODO

The modulus of the product of two complex numbers (and hence, by in-
duction, of any number of complex numbers) is therefore equal to the
product of their moduli.

15. The Argand diagram.

We have seen that complex numbers may be represented in a geometrical
diagram by taking rectangular axes Ox, Oy in a plane. Then a point P
whose coordinates referred to these axes are x, y may be regarded as
representing the complex number x + iy. In this way, to every point of
the plane there corresponds some one complex number; and, conversely,
to every possible complex number there corresponds one, and only one,
point of the plane.

The complex number x + iy may be denoted by a single letter* z. The
point P is then called the representative point of the number z; we
shall also speak of the number z as being the ajfx of the point P.

If we denote (jf + /) by r and choose 6 so that r cos 6 = x, r sin 9
=y, then r and 6 are clearly the radius vector and vectorial angle of
the point P, referred to the origin and axis Ox.

The representation of complex numbers thus afforded is often called
the Argand diagram .

By the definition already given, it is evident that r is the modulus
of z. The angle 6 is called the argument, or amplitude, or phase, of
z.

We write 6 — arg z.

From geometrical considerations, it appears that (although the modulus
of a complex number is unique) the argument is not unique;; if be a
value of the argument, the other values of the argument are comprised
in the expression 2mr + d where a is any integer, not zero. The
principal value of arg 2 is that which satisfies the inequality TODO.

* It is convenient to call .r and y the real and imaginary parts of z
respectively. We fre- quently write x = R(~), y = I(z).

t .J. E. Argand published it in 1806; it had however previously been
used by Gauss, and by Caspar Wessel, who discussed it in a memoir
presented to the Danish Academy in 1797 and published by that Society
in 1798-9.

X See the Appendix, § A-521.



10 THE PROCESSES OF ANALYSIS [CHAP. I

If P, and P2 are the representative points corresponding to values z-
and Z.2 respectively of z, then the point which represents the value
z- + z. is clearly the terminus of a line drawn from P , equal and
parallel to that which joins the origin to P. 2.

To find the point which represents the complex number z Zo, where z
and Zo are two given complex numbers, we notice that if

z = 1 (cos 6 + i sin 6 ),

z.2, = ?-2 (cos f 2 + * sin o) then, by multiplication,

Z\ Z% = n 2 [cos ((9i + 2) + sin ( j + 6. ]. The point which
represents the number z z has therefore a radius vector measured by
the product of the radii vectores of Pj and P2, and a vectorial angle
equal to the sum of the vectorial angles of Pj and Pg.

REFERENCES.

Tke gical foundations of the theory of number. u . X. Whitehead axd B.
A. W. Russell, Principia Mathematica (1910-1913). 'B. A. W. Russell,
Introduction to Mathematical Philosophy (1919).

Onirrational numbers.

R. Dedekixd, Stetigkeit unci irrationale Zahlen. (Brunswick, 1872.)
''V. VON Dantscher, Vorlesungen ueber die Weierstrass'sehe Theorie der
irrationalen

Zahlen. (Leipzig, 1908.) E .W. HoBSON, Functions of a Real Variable
(1907), Ch. i. T.'*?. I'a. Bromwich, Theory of Infinite Series (1908),
Appendix i.

On omplex numbers.

H. Hankel, Theorie der complexen Zahlen-systeme. (Leipzig, 1867.) O.
Stolz, Yorlesungen iiber allgemeine Arithmetik, II. (Leipzig, 1886.)
G. H. Hardy, A course of Pure Mathematics (1914), Ch. in.

Miscellaneous Examples.

1. Shew that the representative points of the complex numbers 1 + 4 2
4-7?', 34-10 are coHinear.

2. Shew that a paralwla can be drawn to pass through the
representative points of the complex numbers

2 + V, 44-4t, 6 + 9 8+ 16 10 + 2.5i.

3. Determine the nth. roots of unity by aid of the Argand diagram;
and shew that the number of primitive roots (roots the powers of each
of which give all the roots) is the number of integers (including
unity) less than n and prime to it.

Prove that if j, 2? 3, ••• he the arguments of the primitive roots, 2
cosjo =0 when

p m \& positive integer less than -7 . , where a, b, c, ... k are the
difterent constituent

n ( — 'f n

primes of 71; and that, when p— , — -? , 2 cos p6= ,- , , where /x is
the number of

(toe , m iC ccoc , , , fc

the constituent primes. (Math. Trip. 1895.)

