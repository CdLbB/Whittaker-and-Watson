\chapter{The Zeta Function of Riemann}

13"1. Definition of the Zeta-f unction.

Let s = a + it where a and t are real*; then, if S > 0, the series

= 1 n=i n is a uniformly convergent series of analytic functions (§§
2"33, 3'34) in any domain in which <r 1 + 8; and consequently the
series is an analytic function of s in such a domam. The function is
called the Zeta-function; although it was known to Eulerf, its most
remarkable properties were not discovered before RiemannJ who
discussed it in his memoir on prime numbers; it has since proved to
be of fundamental importance, not onl - in the Theory of Prime
Numbers, but also in the higher theory of the Gamma-function and
allied functions.

1311. The generalised Zeta-function .

Many of the properties possessed by the Zeta-function are particular
cases of properties possessed by a more general function defined, when
cr' 1 + 8, by the equation

where a is a constant. For simplicity, we shall suppose || that < \$
1, and then we take arg(a + n) = 0. It is evident that (s, 1) = (s).

1312. The expression of s, a) as an infinite integral.

Since (a-F ??)-' T (s) = / af- e' '' ''''' do-., when arg.r = and o- >
(and a fortiori when a \ - S), we have, when o- 1 + 8, V s) s,a)= \ \
m :L | .-r*-' e" < + '* c a;

.v xjio 1-6- .'o l-e- "

* The letters o", t will be used in this sense throughout the chapter.

t Commentatioiies Acad. ScL Imp. Petropolitunae, ix. (1737), pp.
160-188.

+ Berliner MonaUherichte, 1659, pp. 671-680. Ges. Werke (1876), pp.
136-141.

§ The definition of this function appears to be due to Hurwitz,
Zeitschrift fiir Math, ttnd Phys. xxvii. (1882), pp. 86-101.

II When a has this range of values, the properties of the function
are, in general, much simpler than the corresponding properties for
other values of a. The results of\hardsubsectionref{13}{1}{4} are true for all values of a
(negative integer values excepted); and the results of §§ 13-12,
13-13, 13-2 are true when R (a) > 0.

%
% 266
%

Now, when ic' i), e l + x, and so the modulus of the second of these
integrals does not exceed

I "" A-'-2e-(- + ) rf = (N f aV- r (o- - 1),

-'0

which (when o- 1 + 8) tends to as JV x . Hence, when a l + 8 and arg x
= 0,

this formula corresponds in some respects to Euler's integral for the
Gamma- function.

13"13. The expression* of (s, a) as a contour integral. When a 1 + 8,
consider

(0+) / y-i g-a2

' 1 - e

the contour of integration being of Hankel's t3rpe \hardsubsectionref{12}{2}{2}) and not
containing the points + 2n7ri(n = l, 2, 3,,..) which are poles of the
integrand; it is supposed (as in\hardsubsectionref{12}{2}{2}) that, arg(- z)\ ir.

It is legitimate to modify the contour, precisely as in\hardsubsectionref{12}{2}{2}, whenf
o- 1 + B; and we get

'(0+) - A -ip-az r° 8-ip-az

Therefore

27n . ' 1 - e

Now this last integral is a one-valued analytic function of s for all
values of s. Hence the only possible singularities of s, a) are at the
singularities of r (1 - s), i.e. at the points 1, 2, 3, ..., and, with
the exception of these points, the integral affords a representation
of s, a) valid over the whole plane. The result obtained corresponds
to Hankel's integral for the Gamma- function. Also, we have seen that
s, a) is analytic when o- 1 -H 8, and so the only singularity of s, a)
is at the point .s = 1. Writing 6- = 1 in the

integral, we get

1 r(o+) e-,

ZTTl J 1 - e ' which is the residue at £; = of the integrand, and this
residue is 1.

Hence lim f \ = -l.

* il (1 -s)

* Given by Riemann for the ordinary Zeta-function.

+ If (7 1, the integral taken along any straight line up to the origin
does not converge.

%
% 267
%

Since T (1 - s) has a single pole at s = 1 with residue - 1, it
follows that the only singularity of s, a) is a simple pole with
residue + 1 at 5 = 1.

Example 1. Shew that, when R (s) > 0,

(1 \ 21 -*) /-(s) = i: \ 1 + 1 \ i . 1 2* 3' 4s ~

1 / * x ~

Exuraple 2. -Shew that, when R s)> 1,

Example 3. Shew that

where the contour does not include any of the points +rr ±877 ±57ri',
....

1314. Values of s, a) for special values of s.

In the special case when s is an integer (positive or negative),

is a one- valued function of z. We may consequently apply Cauchy's
theorem, so that - . -r: dz is the residue of the intesfrand at = 0,
that

is to say, it is the coefficient of z'" in V .

1 - e~

To obtain this coefficient we differentiate the expansion \hardsectionref{7}{2})

. e-" - 1 \ \ I (-)n< (a)2 e- -l n=\ n'.

term-by-term with regard to a, where (f)n iO denotes the Bernoullian
poly- nomial.

(This is obviously legitimate, by\hardsectionref{4}{7}, when | s j < tt, since - \,
can be expanded into a power series in z imiformly convergent with
respect to a.)

Then i ll-r Ml".

Therefore f s is zero or a negative integer (= - m), we have

  - m, a) = - </)', +o(a)/[(m -f- 1) ( i -f- 2)|. In the special case
when a = 1, if s = - m, then (5) is the coefficient

/ y, yji I Z

of z-~ in the expansion of - ' ' .

%
% 268
%

Hence, by\hardsectionref{7}{2},

 (-2m) = 0, l-2m) = (-y-BJ 2m) (m = 1, 2, S, ...),

r(0)=-|.

These equations give the value of (s) ivhen s is a negative integer or
zero.

13"15. TJie forniula* of Hurwitz for s, a) when cr< 0.

Consider - - -; - r-; - dz taken round a contour C consisting of

27n J c I - e~ °

a (large) circle of radius (2iV+l)7r, (iV an integer), starting at the
point

(2iV"+ l)7r and encircling the origin in the positive direction, arg
(- z) being

zero at z = -(2N+l) ir.

In the region between C and the contour 2NTr +7r; +), of which the
contour of § 13"lo is the limiting form, (- zy~' e~' (1 - e~ )~ is
analytic and one-valued except at the simple poles + 2Tri, + 47, ...,
± 2N7ri.

Hence

27riJc l-e~' 27ri j n+i) n l-e'- n=i

where i2, Rn are the residues of the integrand at 2n7ri, - 2 7
respectively. At the point at which - z = 2mre~ ', the residue is

(2n7r)'-le- '''('-l)e-2 "''',

and hence Rn + Rn = (27?7r) ~ 2 sin i ' " 2'n-an j .

Hence

1 /(0+) (\ )s-ig-a

27riJ(2iv'+i) l-e-

\ 2 sin STT cos 2'rran) 2 cos sir sin (27ra? ) " "(27rr,=i /? - " ' \
2 ) = n':i n'-'

+ H-  T -, dz.

2in J c I - e

Now, since < a 1, it is easy to see that we can find a number K
independent of N' such that | e~" (1 - e~ )~ \ < K when z is on C.

Hence

1 r ( -z\ \ ~' e~"' I 1 f""

-  1 -- . dz\ < K\, (2.Y + 1) -rrYe'"' I rf

27r

<ir (2iV-|-l)7r|' e-l*l as i\" X if cr < 0.

* Zeitschrift fill- Math, und Phys. xsvii. (1882), p. 95,

%
% 269
%

Making N y:, we obtain the result of Hurwitz that, if o- < 0,

 u. r(l-,v) ( /I cos(27ra ) /i \ 4 sin(27ra/i))

each of these series being convergent.

13151. Riemairii's relation between (s) and (l - s).

If we write a = 1 in the formula of Hurwitz given in\hardsubsectionref{13}{1}{5}, and
employ\hardsubsectionref{12}{1}{4}, we get the remarkable result, due bo Riemann, that

2 -* r (S) (S) cos (I STTJ = (1 \ s).

Since both sides of this equation are analytic functions of s, save
for isolated values of s at which they have poles, this equation,
proved when a < 0, persists (by\hardsectionref{5}{5}) for all values of s save those
isolated values.

Example 1. If m be a positive integer, shew that

C (2>n) = S- "*- 1 7r2"' BJ 2m) ! .

Example 2. Shew that r is)n~ C (s) is unaltered by replacing s by 1 -
s.

(Riemann.)

Example 3. Deduce from Riemann's relation that the zeros of f (5) at -
2, - 4, - 6, ... are zeros of the tirst order.

13"2. Hennites* fornuda for (s, a).

Let us apply Plana's theorem (example 7, p. 145) to the function (p
(z) =(a + z)~\ where arg ii + z) has its principal value.

Define the function q x, y) by the equation

(/ < '' y) = 9- - K + + wT' - ( + A' - iy)~']

= - Ha + A')- + if] ~ * sin \ s arc tan - - .

  X + a)

Since + arc tan - - does not exceed the smaller of tt and ', we X + a
' x+ a

have

\ q x,ij)\ \ (a + x)'+f-] *< 1 y-'; sinh JItt, 5 j j,

' q (X, y) : [ a + xf + f]- -''\ jsinh | | | . Using the first result
when y > a and the second when y < a it is

* Annali di Matemalica, (3),fv. (1901), pp. 57-72.

t If t>0, arc tan = :,<. I 5; and arc tan < / dt.

J 1-rt- J 1-i-t- Jo

%
% 270
%

evident that, if o- > 0, q (.v, y) (e-" - 1) dy is convergent when x
and

Jo

tends to as ic - X; also (a + x)~ dx converges if a > 1.

.0

Hence, if o- > 1, it is legitimate to make 'o - oo in the result
contained in the example cited; and we have

 (s,a) = la- +\ \ a+ocy dx + 'lj (a2 + 2)-i jsin fs arc tan 0j- J .

So

  s, a) = la-s + f +2j\ a + f)-y jsin (. arc tan )| - .

This is Hermite's formula*; using the results that, if y 0,

arc tan y/a y/a f y < dT j, arc tan y/a < 2 tt (y>: a7r],

we see that the integral involved in the formula converges for all
values of s. Further, the integral defines an analytic function of s
for all values of s.

To prove this, it is sufficient \hardsubsectionref{5}{3}{1}) to shew that the integral
obtained by differentiating under the sign of integration converges
uniformly; that is to say we have to prove that

/ - 1 log a- + \ y2) (a' +y-) - 5* sin i s arc tan -

dij

o ' y

- I (( +y ) * arc tan - cos f s arc tan - j

di/

~' y -I

converges uniformly with respect to s in any domain of values of s.
Now when s ! A, where A is any positive number, we have

1 (a2+y-) ~ * arc tan '- cos (s arc tan j < a- + i/' ')i cosh (|n-A);

since a-+f)

  J

,2 iA y±y\

converges, the second integral converges uniformly by\hardsubsubsectionref{4}{4}{3}{1} (I).

By dividing the path of integration of the first integral into two
parts (0, rra), Una, X ) and using the results

sin'lsarctau- I <sinh -, sin (s arc tan -) l<sinhJr7rA

V / 1 V /,

in the respective parts, we can simihxrly shew that the first integral
converges uniformly.

Consequently Hermite's formula is valid \hardsectionref{5}{5}) for all values of s,
and it is legitimate to differentiate under the sign of integration,
and the differentiated integral is a continuous function of s.

* The corresponding formula when = 1 had been previously giveu by
Jensen.

%
% 271
%

13'21. Deductions from Hermites formula. Writing s = in Hermite's
formula, we see that

Making s - 1, from the uniformity of convergence of the integral
involved in Hermite's formula we see that

li,, 1 (,, a)-- \ = lim + 1 + 2 r, -,, .

  i( ' *-lj,-*! s-\ 2a Jo a- + y-') e y-l)

Hence, by the example of\hardsubsectionref{12}{3}{2}, we have

hm|ri., )- - | = -j .

Further, differentiating* the formula for l s, a) and then making s -
0, we get

\ d, A y 1 \, a -*' log a a'-'

+ 2 - o log (a- + 2/-) . (a- + y-) ~ *' sin ( s arc tan -)

Jo i \ ci/

+ (a- + V-) ~ arc tan - cos ( s arc tan - ) [ - -

    a \ a)] e' y - 1

I i\, ' arc tan (Wa),

= ( a-,jlog - + 2J \ \ \ A rfy.

Hence, by § 1232,

These results had previously been obtained in a different manner by
Lerch -j*.

Corollary. lim k(s) 1 = 7> T (0) = -5 log(27r).

13"3. Euler's product for (s).

Let (T l+B; and let 2, 3, 5, .../>,... be the prime numbers in order.
Then, subtracting the series for 2~* (s) from the series for (s), we
get

 (.).(l-2-) = ~ +,+~ +, + ...,

* This was justified in § 13 -2.

t The formula for f (s, a) from which Lerch derived these results is
given in a memoir published by the Academy of Sciences of Prague. A
summary of his memoir is contained in the Jahrbuch iiber die
Fortschritte der Math. 1893-1894, p. 484.

%
% 272
%

all the terms of Sn~ for which n is a multiple of 2 being omitted;
then in like manner

all the terms for which n is a miiltiple of 2 or 3 being omitted; and
so on; so that

 (s) . (1 - 2-0(1 - 3-0  (1 -p-') = 1 + 2' -

the ' denoting that only those values of n (greater than p) which are
prime to 2, 3, ... j:? occur in the summation.

Now* i t'n-' I \$ I'n-'- \$ 2 n' - as ja co .

Therefore if a I + 8, the product (s) Yl (1 -p~ ) converges to 1,
tuhere

p the number p assumes the prime values 2, 3, 5, ... only.

But the product 11 (1 -p~ ) converges when a I + 8, for it consists of
p

some of the factors of the absolutely convergent product 11 (1 - n~ ).

Consequently we infer that (s) has no zeros at which a 1 + 8; for if

it had any such zeros, IT (1 -p~ ) would not converge at them. p

Therefore, ii a 1 + 8,

This is Euler's result.

13"31. Riemanns hypothesis concerning the zeros of (s).

It has just been proved that (s) has no zeros at which a >1.

From the formula (| 13*1 51)

y s) = 2 -i r(.9)|-isec ( STT Ul-s)

it is now apparent that the only zeros of (.5) for which a < are the
zeros

of r(s) - sec (2 '''') ) i-e- the points s = - 2, - -4, ....

Hence all the zeros of f (s) except those at - 2, - ]>, ... lie in
that strip of the domain of the complex variable s ivhich is defined
by - a 1.

It was conjectured by Riemann, but it has not yet been proved, that
all

the zeros of (.s) in this strip lie on the line o" = 2 ' " ' ile it
has quite recently

been proved by Hardy -f- that an infinity of zeros of (s) actually lie
on cr = : .

It is highly probable that Riemann's conjecture is correct, and the
proof of it would have far-reaching consequences in the theory of
Prime Numbers.

* The first term of S' starts with the prime next greater than p. t
Comptes Rendiif!, clviii. (1914), p. 1012; see p. 280.

%
% 273
%

13'4. Riemanns integral for (s). It is easy to see that, if cr > 0,

n' Hence, when a > 0,

   s) r ( ) TT - i = lim f 1 e- "''  x -'- dx.

X

Now, if OT x)= S e"""' '", since, by example 17 of Chapter vi (p.
124), 1 + 2ct x) = x~ ' 1 4- 2ts (I/*'), we have lim x -sr x) = 1;
and hence

 st(x)x ~ dx converges when a > .

Jo

Consequently, if a > 2, (s)r ('.7s')7r-'''"=lim ! (x) x -''' dx- i 1
e''''''' xi'-' dx'] .

V" / N o lJt) .0 M = .V+1 J

Now, as in\hardsubsectionref{13}{1}{2}, the modulus of the last integral does not exceed

Jo \ n = N+l j .'o l\ e-< V+l) x

.0

= 7r(iY+l)|-'|(iV H2;\ \ )7r l-i< rQ<r-l)

-*- as uV - - X, since a > 2. Hence, when cr > 2,

= ri-. + -i +x- -x;7(llx)\ x -'-' dx+ I t!r( )a:i -l(fa;

= - + 7 + f .rizTOr)a;-i'' + l(- )(;a;+[ nT(x)x '-Ux.

Consequently

r(5)r(L')7r-i*- - v -r,= I (x - -''> +x ')x-' (x)dx.

Now the integral on the right represents an analytic function of s for
all values of s, by\hardsubsectionref{5}{3}{2}, since on the path of integration

tn- x) < e-' * S e-"" e-"" (1 - e-"")-'.

;i=0

Consequently, by § o S, the above equation, proved when cr> 2,
persists for all values of 5.

w. M. A.  18

%
% 274
%

If now we put

s = l + it, ls(s- 1) (s) r ( s 7r- *- = 1(0, we have

  (t) = I - ff + j x-i (x) cos ( tlogx dx.

Since x~ -st x) log x\ cos U log x + - mr) dx

satisfies the test of\hardsubsectionref{4}{4}{4} corollary, we may differentiate any
number of times under the sign of integration, and then put = 0.
Hence, by Taylor's theorem, we have for all values* of

 (0= S a t''', =o

by considering the last integral ag i is obviously real. This result
is fundamental in Riemann's researches.

13*5. Inequalities satisfied by (s, a) when o-> 0.

We shall now investigate the behaviour of (s, a) as t - + oo, for
given values of cr.

When cr> 1, it is easy to see that, if N be any integer,

as, -)=U- + n)- - -,\ s)il aY- -LM' where

\ 1 J 1 \ 1 \ \ 1

/'"+! u-n,

fn

Now, when tr'> 0, l/ ( ) I ! I /

Jn (u + a)

" +] a-n

ai

fn+l J n (n

di

 n + af- ' = slin + a)-"-'. Therefore the series i" f (s) is a
uniformly convergent series of analytic functions

when cr >; so that 2 / (s) is an analytic function when <t> 0; and
consequently, bj'

;i=.V
\hardsectionref{5}{5}, the function ( (s, a) may be defined when (r>0 by the series

C (., a) = £ (a + )-.- (i\,)(; ).-. -j/. . Now let [t] be the
greatest integer in | < |; and take iV=[ ]. Then

|C )I 2 \ \ {a + nr \ + \ \ { l-sr'i[t] + ay- + 2 \ s\ \ {n + ar'' '

71=0 n=[t] [t] X

< 2 a + 7i)- + \ t mt] + ay ' + \ s\ 2 (n + a) ' \ n=0 i>=[t]

* In this particular piece of analysis it is convenieut to regard t as
a complex variable, defined by the equation s = + it; and then | (t)
is an integral function of t.

%
% 275
%

Using the Maclaurin-Cauchy sum formula \hardsubsectionref{4}{4}{3}), we get

r[t] r

Jo J[t]-l

Now when 8 a- 1 - S where S > 0, we have \ (s, a) \ <a- + l-a)- a +
[t]y-'' -a - + \ t [t] + af-'' + \ s\ < r-H[t]-l + a)-''. Hence f (s,
a) = 0 \ t p"* ), the constant implied in the symbol being independent
of s. But, when 1 - 8 cr l + S, we have

I C (, a) I = ( i i \'-n + / ( + A-)-'' dx

<0 \ t f-") + '-'+( + tf-"] I '' (a+x) - 1 dx,

since (a + x)~' a ~' a+x)-'>- when o- l, and (a+x)'" a+[t]) ~' (a +
x)- when o" 1, anoJ so

Cis, a) = 'Itr'' log \ t\ \ }. When 0- 1+8,

|C(, )| a~ + i (a + r'-* = 6'(l).

13'51. Inequalities satisfied by f (s, a) wlien cr 0.

We next obtain inequalities of a similar nature when <j h. In the case
of the function f (s) we use Riemaim's relation

C(s) = 2''7r -i r (1 -s) f (1 -s) sin (isTr). Now, when o- < 1 - 8, we
have, by § 1233,

r(l-s) = 0 e(*"*)'*' ( -*)-( -*' and ao

C (s) = [exp .V I | + ( -o--i01og|l-s|+iarctan /(l-o-) ]C(l-s).

Since arc tan i/(l - cr)= ±i7r4-0 ( ~ ), according as is positive or
negative, we see, from the results already obtained for f (s, a), that

i B) = 0 \ t\ \ -''\ i s).

In the case of the function (s, ), we have to use the formula of
Hurwitz \hardsubsectionref{13}{1}{5}) to obtain the generalisation of this result; we
have, when o- < 0,

i s,a): -i ±nY- V s)\ \ e ' ' Uiy- )- -''''" i-a s)\ where Ca (!-' )=
2

1 %'-

. - Hence (1 -e" ''"') f (l - ) = e2' + 2 e2 ' ' [/i'-i- (n- l) -i]

+ (S-1) i / T' /"" tt -2£;

since the series on the right is a uniformly convergent series of
analytic functions whenever o- l-S, this equation gives the
continuation of fa(l-*) over the range O o-: 1-S; so that, whenever cr
1 - S, we have

sin7raCa(l-s) 1 1+ 2 /i' ~' + (n-l)' -iH-|s-l I 2 /" n"-- dv..

1=2 re=iV+l ] n-X

18-2

276

THE TRANSCENDENTAL FUNCTIONS [cHAP. XIII

And obviously

Taking V=[ ], we obtain, as in\hardsectionref{13}{5},

Ca l-s)=0 \ tr) 8 a l-8) .

= 0 \ tf\ og\ t ) -8 (T<8).

C s)=0 ) a<-8).

Consequently, whether a is unity or not, we have the results

C s,a) = 0 \ t\ \ -'') (a 8)

= 0 \ t\ \ ) (8 0- 1-8)

= Oi\ t\ \ \ og\ t ) -8 a 8).

We may combine these results and those of\hardsectionref{13}{5}, into the single
formula

C(s,a) = 0(i<r" 'log| |), where*

r(o-)-i-(r, (o- O); 7-(o-) = A, (O a i); r(a) = l-(r, (*-\$<t 1);
r(cr) = 0, ( r l);

and the log | t \ may be suppressed except when - 8 o- S or when 1 - S
cr l + S.

13*6. The asymptotic expansion of log T z + a). From\hardsectionref{12}{1} example 3,
it follows that

\ aJ =i (.V a+nJ J T (z + a) Now, the principal values of the
logarithms being taken,

= 2

n=l

- az

  (-)"'"'

+ 2

(\ yn-i r

ji a + n)J 2 'ni (a - + w)"'J j,,' ! m a'' If I I < a, the double
series is absolutely convergent since

' ' - log 1 + -

 aA- n) ° V a+ n) a+ n

= 1 [\ /i(a + n)

converges.

Consequently

log

az

+

e-v r(a)

r ( + a) a =1 71 (a + n) ',,,=2 wi

1 1 772'

X / yn- 1

2 5 z'"' m,a).

Now consider;;; -; - -. tis, a) ds, the contour of integration being

ziri J c ssimrs

similar to that of\hardsubsectionref{12}{2}{2} enclosing the points 5 = 2, 3, 4, ... but
not the

points 1, 0, -1,-2, ...; the residue of the integrand at s = m(m 2) is

- z m, a); and since, as cr x (where s= a + it), s, a) = (1), the

integral converges if | 2 | < 1.

* It can be proved that t a) may be taken to be i (1 - a) when (t 1.
See Landau, Prim- zahlen, % 237.

%
% 277
%

Consequently

, e-y'r a) z az \ -rrz .. ..

I z + a) a =i n (a + m) Ztti q s sin its

Hence

- V a) V' a) 1 f -TTZ' .,,

  r ( + a) r (a) 27ri j c- 5 sin tts '

Now let D be a semicircle of (large) radius N with centre at s = f,
the semicircle lying on the right of the line <r = |. On this
semicircle (s, a)=0(l), !ir*| = |;<'e- ' '-g', and so the integrand
is* 5:; e-' i'f- ' s3l. Hence if | | < 1 and - tt + 8 arg tt - 8,
where S is positive, the integrand is \ zY e' *" ), and hence

t, s, a)ds 0

J j)S sm 7r5

as iV -* 00 . It follows at once that, if; arg z \ : 7r - 8 and j [ <
1,

, T(a) r'(a) i r + '' 'rrz',,,

log Y ~- -. =-z + r-. ~. (s, a) ds.

° 1 ( + a) 1 (a) 'Itti J s issimrs

But this integral defines an analytic function of for all values of 12
| if

j arg z] TT - 8.

Hence, by\hardsectionref{5}{5}, the above equation, proved when [ | < 1, persists for
all values of I I when ] arg zI' tt - 8.

Now consider I --. (s, a) ds, where n is a fixed integer and

-/ -n- ±iii sin TTS'

R is going to tend to infinity. By\hardsubsectionref{13}{5}{1}, the integrand is [z' e'
R"' ',"' where - n - - cr :\$ -; and hence if the upper signs be
taken, or if the lower signs be taken, the integral tends to zero as
i2 - x . Therefore, by Cauchy's theorem,

Via) T'ia) 1 f->'-h + i n

log \ \ = -z j -~- + -. - (s,a)ds+ X R,n,

where R is the residue of the integrand at s = - in. Now, on the new
path of integration

I s sin ITS \

where K is independent of z and t, and t (t) is the function defined
in\hardsubsectionref{13}{5}{1}.

* The constants implied in the symbol are independent of s and z
throughout.

%
% 278
%

Consequently, since j e~ - ',t]' - -i)dt converges, we have

when \ 2\ is large.

Now, when m is a positive integer, i?, = - - '- and so

- m

by\hardsubsectionref{13}{1}{4}, Rm = r '-,, where 6,/ (a) denotes the derivate of

m(m + l)(m + 2)

Bernoulli's polynomial.

Also Ro is the residue at s = of

and so i2 = f - - a j log + ' (0, a)

= (i - ) log + log r ( ) - log (27r), by\hardsubsectionref{13}{2}{1}.

And, using\hardsubsectionref{13}{2}{1}, R\, is the residue* at >Sf= of

\ ia\ s=\ ..,(i, >...),(i,s, g,+..,(l\ i:g,...).

XT 75 1 r' (a)

Hence R\ = - z\ oo'z+ z tt + z.

r(a)

Consequently, finally, if | arg z ir- S and | j is large, log r (z +
a) == (z + a -~ \ og z - z + l\ og 27r)

+ i (-)"'~ </>w+2( -), \ \ s

w=i w(77H- l)(m + 2) *

In the special case when a = 1, this reduces to the formula found
previously in\hardsubsectionref{12}{3}{3} for a more restricted range of values of arg 2.

The asymptotic expansion just obtained is valid when a is not
restricted by the inequality < a 1; but the investigation of it
involves the rather more elaborate methods which are necessary for
obtaining inequalities satisfied by (s, a) when a does not satisfy the
inequality 0<a%l. But if, in the formula just obtained, we write a=l
and then put z + a for z, it is easily seen that, when j arg ( + a) [
< tt - 8, we have

log r (2 + a + 1) = f + a + 2) log (z + a)-z-a + l ' + o l);

* Writings = 5+1.

%
% 279
%

subtracting log (z + a) from each side, we easily see that when both

I arg z + a) 7r- 8 and \ arg zI tt - 8, we have the asymptotic formula

logr( + a)=( + a- jlog - + . log(27r) + o(l),

where the expression which is o (1) tends to zero as \ Zi->X).

REFERENCES. G. F. B. RiEMANN, Ges. Werke, pp. 145-155. E. G. H.
Landau, Handbuch der Primzahlen. (Leipzig, 1909.) E. L. LiNDELOF, Le
Calcid des Residue, Ch. iv. (Paris, 1905.) E. W. Barnes, Messenger of
Mathematics, xxix. (1899), pp. 64-128. G. H. Hardy and J. E.
Littlewood, Acta Mathematical xli. (1917), pp. 119-196.

Miscellaneous Examples.

1. Shew that

(2 - 1 ) f (5) = - - + 2 / (i +y2)-* sin (s arc tan 2y)

(Jensen, D Intermediaire des Math. (1895), 'p. 346.)

2. Shew that

2 -i / * dv

C(s)= \ j-2 j (1+/)-** sin (5 arc tan y) - - .

(Jensen.)

3. Discuss the asymptotic expansion of \ ogG z + a), (Chapter xii
example 48) by aid of the generahsed Zeta-function. (Barnes.)

4. Shew that, if cr > 1,

p m=.l mp"

the summation extending over the prime numbers jd = 2, 3, 5,

(Dirichlet, Journal de Math. iv. (1839), p. 407.)

5. Shew that, if o-> 1,

where A (w) = when n is not a power of a prime, and A n) = \ ogp when
is a power of a prime p.

6. Prove that e~- -dx

log C (5) = 2 2

 \&

'(is) Jo

** r (is) 1

(Lerch, KraMw Rozprawy*, ll. See the Jahrbuch ilber die Fortschritte
der Math. 1893-1894, p. 482.

%
% 280
%

7. If 00

where | a; | < 1, and the real part of s is positive, shew that

and, if 5 < 1,

lim (1 - xy- (j) (s, A-) = r (1 - s).

(Appell, Comptes Rendus, lxxxvii.)

8. If X, a, and s be real, and < a < 1, and s > 1, and if

< (--' ' ).= ?,( : .' . .

shew that

and

(b (x, a, 1 - s) = tttVo

(Lerch, Acta Math, xi.)

9. By evaluating the residues at the poles on the left of the straight
line taken as contour, shew that, if k > 0, and | arg 3/ 1 < Att,

1 fk+cci

e-y= --.l rhi)y-''du,

and deduce that, if - > i,

9 - f ' ' (")  ('" )" " '') du = w x\

k - xi

and thence that, if a is an acute angle.

r 1 (0 = TT cos ia - W" 1 + 2 or (e -) . 1 t + t

(Hardy.)

10. By differentiating 2?i times under the integral sign in the last
result of example 9, and then making a - Jtt, deduce from example 17
on p. 124 that

r-*i- <*. (,)*= < -cos|

By taking n large, deduce that there is no number 0 such that | t) is
of fixed sign when t > to, and thence that f (s) has an infinity of
zeros on the line <r = .

(Hardy.)

[Hardy and Littlewood, P?'oc. London Math. Soc. xix. (1920), have
shewn that the number of zeros on the hue o- = i for which < < T' is
at least ( T) as - 00; if the

Kiemann hypothesis is true, the number is -- 7' log - " 7"+ (log 7;
see

Landau, Pnmzahlen, i. p. 370.]
