\chapter{The Fundamental Properties of Analytic Functions; 
Taylor's, Laurent's and Liouville's Theorems} 

51. Property of the elementary functions.

The reader will be already familiar with the term elementary function,
as used (in text-books on Algebra, Trigonometry, and the Differential
Calculus) to denote certain analytical expressions* depending on a
variable z, the symbols involved therein being those of elementary
algebra together with exponentials, logarithms and the trigonometrical
functions ; examples of such

expressions are

1 • §

Z-, e~, iogz, arcsm '-.

Such combinations of the elementary functions of analysis have in
common a remarkable property, which will now be investigated.

Take as an example the function e .

Write e'=f z).

Then, if 2 be a fixed point and if z' be any other point, we have

f z')-f z) \ e~ -e' \ , e' '- ' - 1 z —z z — z z — z

f, z' — z (z —z)-

and since the last series in brackets is uniformly convergent for all
values of it follows (§ 3*7) that, as z'—>z, the quotient

z — z tends to the limit e , uniformly for all values of arg z — z).
This shews that tlie limit of

f z:)-f z)

z — z

is in this case independent of the path by which the point z tends
towards coincidence witJt z.

It wall be found that this property is shared by many of the
well-known elementary functions ; namely, that iif(z) be one of these
functions and h. be

* The reader will observe that this is uot the sense in which the term
function is defined (§ 3"1) in this work. Thus e.g. .r - hj and | z \
are functions of z (—x + iy) in the sense of § 3-1, but are not
elementary functions of the type under consideration.



5 1-5 "12] FUNDAMENTAL PROPERTIES OF ANALYTIC FUNCTIONS 83 any complex
number, the limiting value of

exists and is independent of the mode in luhich h fends to zero.

The reader will, however, easily prove that, ii f(z)=x —iy, where z =
x- iy, then lim' - — — — IXJ- is not independent of the mode in which
A— >0.

5'11. Occasional failure of the jwoperty.

For each of the elementary functions, however, there will be certain
points z at which this property will cease to hold good. Thus it does
not hold for the function l/( — a) at the point z = a, since

 i Qh\ z — a+h z—a

does not exist when z = a. Similarly it does not hold for the
functions log z

and z at the point z = Q.

These exceptional points are called singular points or singulaHties of
the function f z) under consideration ; at other points f z) is said
to be analytic.

The property does not hold good at any point for the function \ z.

5'12. Cauchy's* definition of an analytic function of a complex
variable.

The property considered in § b\ \ will be taken as the basis of the
definition of an analytic function, which may be stated as follows.

Let a two-dimensional region in the -plane be given ; and let w be a
function of z defined uniquely at all points of the region. Let z, z-
hz be values of the variable z at two points, and u, u + Bu the
corresponding values

of u. Then, if, at any point z within the area, - tends to a limit
when 8x—>0,

By—*0, independently (where 8z = 8x + iBy), u is said to be a function
of z which is monogenic or analytic j" at the point. If the function
is analytic and one-valued at all points of the region, we say that
the function is analytic throughout the region.

We shall frequently use the word ' function ' alone to denote an
analytic function, as the functions studied in this work will be
almost exclusively analytic functions.

* See the memoir cited in § 5 "2.

t The words ' regular ' and ' holomorphic ' are sometimes used. A
distinctiou has been made by Borel between ' monogenic ' and '
analytic ' functions in the case of functions with an infinite number
of singularities. See § 5*51.

X See § 5-2 cor. 2, footnote.

6—2



84 THE PROCESSES OF ANALYSIS [CHAP. V

In the foregoing definition, the function u has been defined only
within a certain region in the -plane. As will be seen subsequently,
however, the function u can generally be defined for other values of 2
not included in this region ; and (as in the case of the elementary
functions already discussed) may have singularities, for which the
fundamental property no longer holds, at certain points outside the
limits of the region.

We shall now state the definition of analytic functionality in a more
arithmetical form.

Let f z) be analytic at z, and let e be an arbitrary positive number;
then we can find numbers I and h, h depending on e) such that

z —z I

whenever \ z' — z\ < h.

Vi f z) is analytic at all points of a region, I obviously depends on
z ; we consequently write 1 = f z).

Hence /(/) = f z) + - z) f z) + v z'- z\

where v is a function of z and z such that ! v < e when \ z —z\ < t>.

Example 1. Find the points at which the following functions are not
analytic :

z — \ (i) 2 . (ii) cosec2 (2 = /itt, w any integer). (iii) - — - — -
(3 = 2,3).

1

(iv) ez (j = 0). (V) z- ) zf (2 = 0,1).

Example 2. If z = x - iy, f z) = u ii\ where u, v, x, y are real and /
is an analytic function, shew that

8m "bv cu cv /T - s

 :r = , : =— ; . (Kiemann.)

ex oy oy ex

5*13. An application of the modified Heine-Borel theorem.

Let f(z) be analytic at all points of a continuum ; and on any point z
of

the boundar ' of the continuum let numbers f (z), S (S depending on z)
exist

such that

\ f(z')-f(z)- z'-z)Mz)<e,z'-z\

whenever \ z — z < 8 and z is a point of the continuum or its
boundary.

[We write /i iz) instead of /' z) as the differential coefficient
might not exist when c approaches z from outside the boundary so
that/j (2) is not necessarily a unique derivate.]

The above inequality is obviously satisfied for all points z of the
continuum as well as boundary points.

Applying the two-dimensional form of the theorem of § 3"6, we see that
the region formed by the continuum and its boundary can be divided
into a jinite number of parts (squares with sides parallel to the axes
and their



5" 13, 5-2] FUNDAMENTAL PROPERTIES OF ANALYTIC FUNCTIONS 85

interiors, or portions of such squares) such that inside or on the
boundary of any j art there is one point z such that the inequality

fiz') - f z,) - (/ - z,)f, z,) .<e\ z'-z, \ is satisfied by all points
z inside or on the boundary of that part.

5*2. CaUCHY'S theorem* on the integral OF A FUNCTION ROUND A CONTOUR.

A simple closed curve C in the plane of the variable z is often called
a contour ; ii A, B, D he points taken in order in the
counter-clockwise sense along the arc of the contour, and if f z) be a
one-valued continuousf function of z (not necessarily analytic) at all
points on the arc, then the integral

f f z)dz or f f z)dz

taken round the contour, starting from the point A and returning to A
again, is called the integral of f z) taken along the contour. Clearly
the value of the integral taken along the contour is unaltered if some
point in the contour other than A is taken as the starting-point.

We shall now prove a result due to Cauchy, which may be stated as
follows. If fiz) is a function of z, analytic at all points on and
inside a contour G, then

I f z)dz = 0.

For divide up the interior of C by lines parallel to the real and
imaginary

axes in the manner of § .5"13 ; then the interior of Cis divided into
a number

of regions whose boundaries are squares Cj, C , ... Cm and other
regions

whose boundaries D,, D,, ... Dy are portions of sides of squares and
parts

of G ; consider

Mr N r

X f(z)dz+ S f z)dz,

n = lJ(,C ) n=lJ(.D )

each of the paths of integration being taken counter-clockwise ; in
the complete sum each side of each square appears twice as a path of
integration, and the integrals along it are taken in opposite
directions and consequently cancel §; the only parts of the sum which
survive are the integrals oi f z)

* Memoire sur les integrates definies prises entre des limites
imarjinaires (1825). The proof here given is that due to Goursat,
Trans. American Math. Soc. i. (1900), p. 14.

t It is sufficient for f z) to be continuous when variations of z
along the arc only are considered.

:J: It is not necessary that f(z) should be analytic on C (it is
sufficient that it be continuous on and inside C), but if /( ) is not
analytic on C, the theorem is much harder to prove. This proof merely
assumes that /' [z) exists at all points on and inside C. Earlier
proofs made more extended assumptions; thus Cauchy's proof assumed the
continuity of f' z). Eiemann's proof made an equivalent assumption.
Goursat's first proof assumed that f z) was uniformly differentiable
throughout C.

§ See § 4G, example.



/

J (On)



86 THE PROCESSES OF ANALYSIS [CHAP. V

taken along a number of arcs which together make up G, each arc being

taken in the same sense as in I f\ z)dz ; these integrals therefore
just make

hc)'

up f z)dz.

Now consider 1 f z)dz. With the notation of § 5-12,

J (Cn)

f z) dz = [ f z,) + ( - z,)f' (.-,) + z- z,) v] dz

-' (C )

= /( i) - V ( 01 f dz +/' (z,) I zdz+l (z- z,) vdz.

\ f? = Mc = 0, f zdz

by the examples of § 4*6, since the end points of C coincide. Now let
In be the side of Cn and An the area of C . Then, using § 4-62,

TODO

But



l\ \ l n



= 0,



I



! (C )



In like manner



< e V2 . f \ dz\= eln \/2 . 4 ,, = 4e sj'2.

J Cn

I f z)dz\ \ i \ \ {z-z )vdz\

J (Dn) I J (Dn)



(Dn)

  4>e (An + In Xi)\ \ '2,

Avhere An is the area of the complete square of which Dn is part, In
is the side of this square and \ n is the length of the part of G
which lies inside this square. Hence, if \ be the whole length of G,
while I is the side of a square which encloses all the squares Gn and
Dn,

f(z)dz k S f z)dz + t \ f z)dz

|J(0 I n = \'J(Cn) I n = \ \ J Dn) |

( M N iV )

<4eV2 An+ S An' + l tXn\ (m = 1 m = 1 n=l )

< 4e V2 . I' + X).

Now e is arbitrarily small, and I, \ and 1 f z)dz are independent of
e.

•I iC)'

It therefore follows from this inequality that the only value which I
f(z) dz can have is zero ; and this is Cauchy's result.



5*2] FUNDAMENTAL PROPERTIES OF ANALYTIC FUNCTIONS 87

Corollary . If there are two jjaths z AZ and Zq,BZ from 2o to Z, and
if /(z) is a function of z analytic at all points on these curves and
throughout the domain enclosed by

these two paths, then / f z) dz has the same value whether the path of
integration is

Z(iAZ (jv ZqBZ. This follows from the fact that ZqAZBzq is a contour,
and so the integral taken round it (which is the difference of the
integrals along zqAZ and ZoBZ) is zero.

Thus, if /(2) be an analytic function of z, the value of / f z)dz is
to a certain extent

J AB

independent of the choice of the arc AB, and depends only on the
terminal points A and B. It must be borne in mind that (his is only
the case whenf z) is an analytic function in the sense of § 5*12.

Corollary 2. Suppose that two simple closed curves (7 and Cj are
given, such that Co completely encloses Cj, as e.g. would be the case
if Cq and Ci were confocal ellipses.

Suppose moreover that/ (2) is a function which is analytic* at all
points on Cq and Cj and throughout the ring-shaped region contained
between Cq and C . Then by drawing a network of intersecting lines in
this ring-shaped space, we can shew, exactly as in the theorem just
proved, that the integral



//(



dz



is zero, ivhere the integration is taken round the whole boundary of
the nng-shaped space; this boundary consisting of two curves Co and C
, the one described in the counter-clochvise direction aiid the other
described in the qlockwise direction.

Corollary 3. In general, if any connected region be given in the
3-plane, bounded by any number of simple closed curves Co, Ci, Co,
..., and if /(z) be any function of z which is analytic and one-valued
everywhere in this region, then



I



f z)dz

is zero, where the integral is taken round the whole boundary of the
region ; this boundary consisting of the curves Co, Ci, ..., each
described in such a sense that the region is kept either uhvays on the
right or always on the left of a person walking in the sense in
question round the boundary.

An extension of Cauchy's theorem I f z) dz = 0, to curves lying on a
cone whose vertex

is at the origin, has been made by Ravut (N'ouv. Annales de Math. (3)
xvi. (1897), pp. 365-7). Morera, Moid, del 1st. Lombardo, xxii.
(1889), p. 191, and Osgood, Bull.

Amer. Math. Soc. II. (1896), pp. 296-302, have shewn that the property
f z)dz =

may be taken as the property defining an analytic function, the other
properties being deducible from it. (See p. 110, example 16.)

Example. A ring-shaped region is bounded by the two circles | s | = 1
and | z | = 2 in the 2-plane. Verify that the value of I — , where the
integral is taken round the boundary of this region, is zero.

* The phrase 'analytic throughout a region' implies one-valuedness (§
5-12); that is to say that after z has described a closed path
surrounding Co, f z) has returned to its initial value. A function
such as log z considered in the region 1 | | 2 will be said to be '
analytic at all points of the region. '



  



88 THE PROCESSES OF ANALYSIS [CHAP. V

For the boundary consists of the circumference ]2| = 1, described in
the clockwise direction, together with the circumference |s| = 2,
described in the counter-clockwise direction. Thus, if for points on
the iirst circumference we write 2 = e , and for points on the second
circumference we write z = '2e<'t>, then 6 and cp are real, and the
integral becomes

jo e' jo 2e'*

5'21. 2'he value of an analytic function at a 'point, expressed as an
integral taken round a contour enclosing the point.

Let C he a contour within and on which /( ) is an analytic function of
z. Then, if a be any point within the contour,

z — a is a function of z, which is analytic at all points within the
contour G except the point z = a.

Now, given e, we can find 3 such that

\ fi z)-f a)- z-a)f' a)\ \ \ z-a\

whenever | — a | < S ; with the point a as centre describe a circle 7
of radius 7' < 8, r being so small that 7 lies wholly inside C

Then in the space between 7 and G f z)\ \ {z — a) is analytic, and so,
by § 5-2 corollary 2, we have

f z)dz\ [f z)dz



c z — a Jy z — a where I and I denote integrals taken
counter-clockwise along the curves

G and 7 respectively.

But, since 1 2 - a | < S on 7, we have

r f(z) dz\ f f a)+iz- a)f (a) + v(z-a) J y z — a J y z — a

where \ v\ < e; and so

r m y j f dz f ( ) dz -\ vdz.

J C Z-a \ lyZ -a - Jy Jy

Now, if z be on 7, we may write

z — a = re , where r is the radius of the circle 7, and consequently

"2t ire dO



jyZ — a Jo



= I I de = 2771, y j , re'" Jo



and dz= ire' dd = 0;



also, by § 4"62,



vdz



< e . 27rr.



i



5"21, 5 -22] FUNDAMENTAL PEOPERTIES OF ANALYTIC FUNCTIONS 89

r f(z)d2 \ 2 if( a) I = f €dz\ \ 2irre.

J c z — a \ J y 1



( r ( 5: 1 n.7: i /'

Thus

c z — a



But the left-hand side is independent of e, and so it must be zero,
since e is arbitrary ; that is to say



    2771 ] c z — a



This remarkable result expresses the value of a function f z\ (which
is analytic on and inside (7) at any point a within a contour C, in
terms of an integral which depends only on the value of/( ) at points
on the contour itself.

Corollary. If f z) is an analytic one-valued function of 2 in a
ring-shaped region bounded by two curves C and C", and a is a point in
the region, then



•' 2ni J c -(i "tti c' Z — a



where C is the outer of the ciu-ves and the integi-als are taken
counter-clockwise.

5"22. The derivates of an analytic function f z).

The function/' (2'), which is the limit of

f z + h)-f z) h

as h tends to zero, is called the derivate of / z). We shall now shew
that f' z) is itself an analytic function of z, and consequently
itself possesses a derivate.

For if be a contour surrounding the point a, and situated entirely
within the region in which f z) is analytic, we have

f(a+h)-f(a)

f (a) = hm -- ,

h o n

,, 0 2TTih [J c z-a-h (j z-a )

= Km -I- \ /( ) dz

h Q liri J c z-a) z- a~h)

Itti J c z - af h 27ri J c z - ay z-a- h)

Now, on C, f z) is continuous and therefore bounded, and so is (z —
a)~ ; while we can take | h less than the upper bound of U — a |.



\



90



Therefore



(z — a)- z — a — h) Then, if I be the length of C,

h f f z)dz



[chap. V

is bounded ; let its upper bound be K.



THE PROCESSES OF ANALYSIS



lim



<lim !A|(27r)- 7 : = 0,



I h Q lirij c z -af z-a- h)

and consequently f (a) = — . | , .

   Ziri J c z — a)- '

a formula which expresses the value of the derivate of a function at a
point as an integral taken along a contour enclosing the point.

From this formula we have, if the points a and a + h are inside C,

f(a + h)-f'(a) J\ r f z)dz 1 1

h 27ri J c h



(z — a— hy z — ay

c (z — a — hy z — ay

f(z) dz



= 9:Z;f f )dz.

iTTl J c



... +hAh,

ZTTi J c \ Z — ay

and it is easily seen that J./ is a bounded function of z when \ h\ <
\ z — a.

Therefore, as h tends to zero, A~' /'( - + A) — /' (a) tends to a
limit, namely

J\ /• f(z)dz 27ri J c z — ay '

Since /' (a) has a unique differential coefficient, it is an analytic
function of a; its derivate, which is represented by the expression
just given, is denoted by/" (a), and is called the second derivate of
/(a).

Similarly it can be shewn that /"(a) is an analytic function of a,
possessing a derivate equal to

2 f f(z)dz\ 27ri Jc z-ay'

this is denoted by f" (a), and is called the third derivate of /(a).
And in general an nth derivate/*"' (a) of /(ct) exists, expressible by
the integral

r. n r f(z)dz P 2'rTiJc z-aT+ '

and having itself a derivate of the form

(n + 1)! r f z)dz .



27ri Jciz- a)"+ ' the reader will see that this can be proved by
induction without difficulty.



 '



5 '23, 5 3] FUNDAMENTAL PROPERTIES OF ANALYTIC FUNCTIONS 91

A function which possesses a first derivate with respect to the
complex variable z at all points of a closed two-dimensional region in
the •-plane therefore possesses derivates of all orders at all points
inside the region.

5'23. Caiichys inequality for f " (a).

Let f(z) be analytic on and inside a circle G with centre a and radius
?\ Let M be the upper bound oif z) on the circle. Then, by § 4*62,

!/""( )i ;4/c; 'i'' '

M.n\

Example. l f z) is analytic, z = x- iij and V- = ; , + -2, .shew that

V2 log 1/(2) 1 = 0; andy2|/(e)|>0 unle.ss/(2) = or/' z) = Q. (Trinity,
1910.)

5"3. Analytic functions represented by uniformly convergent series.

 X)

Let X fn (z) be a series such that (i) it converges uniformly along a
=o

contour C, (ii) / (z) is analytic throughout C and its interior.

00 Then 2 fni ) converges, and the sum* of the series is an analytic

n =

function throughout C and its interior.

00

For let a be any point inside C; on C, let S fn z) = (z).

Then -. — dz = j— \ fn( )\

27ri J c z — a 27ri j c Im=o ] z -a

 .o \ 2mlc 2-a \ '

00

by* § 47. But this last series, by § 5-21, is S fiid)', the series
under

n =

consideration therefore converges at all points inside C; let its sum
inside G (as well as on C) be called (z). Then the function is
analytic if it ha,s a unique differential coefficient at all points
inside G.

But if a and a + h be inside G,

 (a + h)- (a) \ J f ( ) (

A 27rt J c z — a) z — a — h)'

and hence, as in § 5'22, lim W a- h) — a)] A~ ] exists and is equal to

7t-*0 * Since | 2 - a |~i is bounded when a is fixed and z is on C,
the uniformity of the convergence of S / z)l[z - a) follows from that
of 2 / [z).

n=0 n=0



92 THE PROCESSES OF ANALYSIS [CHAP. V

  — : 7 — dz ; and therefore <I> (z) is analytic inside G. Further, by
27rt J c z— a)

transforming the last integral in the same way as we transformed the
first

00 ao

one, we see that <!>' (a) = S / ' (a), so that 2 fn (a) may be '
differentiated

n=0 n=0

term by term.'

If a series of analytic functions converges only at points of a curve
which is not closed nothing can be inferred as to the convergence of
the derived series*.

'COS 7hJ7

Thus 2 ( - )" — 2 — converges uniformly for real values of x (§ 3*34).
But the derived

H = l 'i

  sin ij series 2 ( — )""* converges non-uniformly near A' = (2m+1)
tt, (m any integer) ; and

H = l 'ii

the derived series of this, viz. 2 ( - )"~ cos n.v, does not converge
at all.

(1=1

Corollary. By § 3-7, the sum of a power series is analytic inside its
circle of con- vergence.

531. Analytic functions represented hy integrals.

Let f t, z) satisfy the following conditions when t lies on a certain
path of integration (a, h) and z is any point of a region >S' :

(i) f and ~ are continuous functions of t.

   dz

(ii) / is an analytic function of z.

df * (in) The continuity of ~- qua function of z is uniform with
respect to

the variable t.

rb .

Then I f(t, z)dt is an analytic function of z. For, by § 4*2, it has
the

, !* f* dt\ t, z) . unique derivate - — — - dt.

5*32. Analytic functions represented hy infinite integrals.

From § 4*44 (II) corollary, it follows that I f (t, z) dt is an
analytic

J a

function of z at all points of a region >S' if (i) the integral
converges, (ii) f t, z) is an analytic function of z when t is on the
path of integration and z is on S,

(iii) - : ' is a continuous function of both variables, (iv) - — - dt

dz J a OZ

converges uniformly throughout 8.

For if these conditions are satisfied f t, z) dt has the unique
derivate

J a



J a



dz



* This might have been anticipated as the main theorem of this section
deals with uniformity of convergence over a two-dimensional region.



5 -3 1-5 -4] Taylor's, Laurent's and liouville's theorems 93

A case of very great importance is afforded by the integral I e~' /(0
dt,

Jo where /(t) is continuous and \ f t)\ < Ke' where K, r are
independent of t; it is obvious from the conditions stated that the
integral is an analytic function of z when R z) r, > r. [Condition
(iv) is satisfied, by § 4-431 (I),

r

since I fe"""''*' rf converges.]

Jo

5-4. Taylor's Theorem*.

Consider a function f z), which is analytic in the neighbourhood of a
point z = a. Let (7 be a circle with a as centre in the -plane, which
does not have any singular point of the function f z) on or inside it
; so that f z) is analytic at all points on and inside C. Let z = a +
h be any point inside the circle C. Then, by § 5*21, we have

    Itti J c z-a-h

 -rriJc Xz-a iz-af ' ' ' z - ay- ' z - a)'' ' z - a - h)]

f z) But when z is on C, the modulus of — — —j is continuous, and so,

z — a — h

by § 3*61 cor. (ii), will not exceed some finite number M. Therefore,
by § 4-62,



1 f f(z)dz.h''+



27riJc z-ay'+' z-a-h) " 27r [rJ '

where R is the radius of the circle C, so that 'IttR is the length of
the path of integration in the last integral, and R = \ z — a\ for
points z on the cir- cumference of C.

The right-hand side of the last inequality tends to zero as ?? — > oo
. We have therefore

/(a + /0=/(a) + / /'(a)-h| ,/"(a)+...-f |/-'(a) + ..., which we can
write

f(z)=f a) + (z - a)/' (a) + ~ ~ff" a) + ... + ri /< ) (a) + ....

This result is known as Taylors Theorem ; and the proof given is due
to Cauchy. It follows that the radius of convergence of a poiuer
series is always

* The formal expansion was first published by Dr Brook Taylor (1715)
in his Methodus Incrementonim.



94 THE PROCESSES OF ANALYSIS [CHAP. V

at least so large as only just to exclude from the interior of the
circle of con- vergence the nearest singularity of the function
represented by the series. And by § 5*3 corollary, it follows that the
radius of convergence is not larger than the number just specified.
Hence the radius of convergence is just such as to exclude from the
interior of the circle that singularity of the function which is
nearest to a.

At this stage we may introduce some terms which will be frequently
used.

If f a) = 0, the function f(z) is said to have a zero at the point z =
a. If at such a point f (a) is different from zero, the zero of f(a)
is said to be simple; if, however,/' a),f" a), .../"*"'' (a) are all
zero, so that the Taylor's expansion of f z) at z = a begins with a
term in (z — a)", then the function f z) is said to have a zero of the
nth. order at the point z = a.

Example 1. Find the function / (s), which is analytic throughout the
circle C and its interior, whose centre is at the origin and whose
radius is unity, and has the value

a — cos 6 . sin 6



a -2acos6 + l a -2acos0 + l

(where a> 1 and 6 is the vectorial angle) at points on the
circumference of C.

[We have

f z) dz



/( )(0) = .f - - •' ' 2mjc z"

n ! /"St

27nJo

\ n\ n e~"i9d6 \ n \ f dz f d"" 1 "1

~2ffjo a-e<9 ~2iriJcz''(a-z)~~\ \ ds a-zJ



e- 'O.idd. -7- — ~;r . , (puttuig z = e*e) a- -2a cos + 1 ' ° '



Therefore by Maclaurin's Theorem*,

)l=0 "

or/(2) = (a-2)~' for all points within the circle.

This example raises the interesting question, Will it still be
convenient to define f z) as (a-2)~ at points outside the circle ?
This will be discussed in § 5-51.]

Example 2. Prove that the arithmetic mean of all values of 2"" 2 cv,
for points z on

the circumference of the circle \ z\ = l, is a , if Sc? " is analytic
throughout the circle and its interior.

/ (") (0) [Let 2 v2''=/(2), so that a , = ;- . Then, writing z = e' ,
and calling C the circle

K=0 "

277 jo 2" ~ 2ni j c 2"* ~ n\ """ -'

z'i * The re8ult/ 2) =/(0) +2/' (0) + -/" (0) + ..., obtained By
putting a = in Taylor's Theorem,

is usually called Maclaurin's Theorem; it was discovered by Stirling
(1717) and published by Maclaurin (1742) in his Fluxions.


%
% 95
%

Example 3. Let $f(z) = z^{r}$; then $f(z+h)$ is an analytic function
of $h$ when $\absval{h} < \absval{z}$ for all values of $r$; and so
$(z + h)^{r} = z^{r} + rz^{r-1} h + \frac{ r (r-1) }{2} z^{r-2} h^{2}
+ \cdots, $ this series converging when $\absval{h} < \absval{z}$.
This is the binomial theorem.\index{Binomial theorem}

Example 4. Prove that if h is a positive constant, and (1 - 2zh- h?) ~
in expanded in the form

\ + hP z) + h''P.2 z) + h P z) + (A),

(where P (2) is easily seen to be a polynomial of degree n in z), then
this series converges so long as z is in the interior of an ellipse
whose foci are the points z = \ and 2= —1, and whose semi-major axis
is (A + A"').

Let the series be first regarded as a function of A. It is a power
series in A, and therefore converges so long as the point A lies
within a circle in the /; -plane. The centre of this circle is the
point A = 0, and its circumference will be such as to pass through
that

singularity of (1 - 2zh- h' )~ which is nearest to A = 0.

But 1 - 22A -1- A2 = A - 2 + (22 - 1 )5>. /i \ 2 \ ( 2 \ 1 )ij,

so the singularities of (1 - 22A-|-A-)~2 are the points h=z — z' - ) '
and h=z + z — ) . [These singularities are branch points (see § 5'7).]

Thus the series (A) converges so long as | A | is less than both
|2-(22-l) | and |2-f(22\ l) |.

Draw an ellipse in the 2-plane passing through the point 2 and having
its foci at +L Let a be its semi-major axis, and 6 the eccentric angle
of 2 on it.

Then 2 = a cos -f i (a - 1 ) sin 9,

which gives 2 ± (22 - 1 )i = a + (a2 \ i) (cos + 1 sin 6),

so i2±(22-l)i | = a + (a2\ i)4.

Thus the series (A) converges so long as A is less than the smaller of
the numbers a-|-(a2- 1) and a- a - 1)2, i.e. so long as A is less than
a-(a2\ i)5. But A = a — (a2- 1) when a = |(A-f-A~i).

Therefore the series (A) converges so long as 2 is within an ellipse
whose foci are 1 and — 1, and whose semi-major axis is h h + h~ ).

5'41. Forms of the remainder in Taylor's series.

Let f x) be a real function of a real variable ; and let it have
continuous differential coefficients of the first n orders when a x a
+ h.

If O i l, we have

fl (71-1 hm •) /i n —f\ n-\

It li. 7! (1 - ') "/'"" ( + *> = ifr /'"' ( + "'> - ''/' ( + "')•

Integrating this between the limits and 1, we have

n-l hm /•! hn /I \ f\ n—i

f(a + h)=f a)+ -,f (a)+ ] ' , f'Ha + th)dt.

ni = im: Jo n—L)l

Let Rn = J~iyi ] \ l - 0' - V'"* ( + ih) dt ;

and let j-j be a positive integer such that p n.



96 THE PROCESSES OF ANALYSIS [cHAP. V

Then Rn = r - f ' ( " y~" ' ( " O"" /'"* ( + th) dt. Let U, L be the
upper and lower bounds of (1 - )' -p/("' (a + th). Then

f ' X (1 - 0 "' dt<\ \ t)P-' . (1 - O' -P/"'' (a + th) dt<[ U(l- t)P-'
dt. Jo Jo Jo

Since (1 — t)' ~P / '" (a + th) is a continuous function it passes
through all

values between U and L, and hence we can find 6 such that 1, and

[ I- ) -y <' ' (a + th) dt = -1(1- ey-Pf " (a + Oh). Jo

Therefore R = .J \ y ;(1 - f- /*"' (a + 6h).

A" Writing p = n, we get Rn = — /"" (a + Oh), which is Lagrange s form
for

A" Me remainder ; and writing j9 = 1, we get Rn = - \ , y , (1 -
)''-'/''" (a + A),

which is Cauchys form for the remainder. Taking n = \ in this result,
we get

f a h)-f a) = hf a + 6h) \ i f x) is continuous when a x a + h; this
result is usually known as the First Mean Value Theorem (see also §
4-14).

Darboux gave in 1876 Journal de Math. (3) ii. p. 291) a form for the
remainder in Taylor's Series, which is applicable to complex variables
and resembles the above form given by Lagrange for the case of real
variables.

55. The Process of Continuation.

Near every point P, Zq, in the neighbourhood of which a function f z)
is analytic, we have seen that an expansion exists for the function as
a sei'ies of ascending positive integral powers of z — Zq), the
coefficients in which involve the successive derivates of the function
at z .

Now let A be the singularity of f z) which is nearest to P. Then the
circle within which this expansion is valid has P for centre and PA
for radius.

Suppose that we are merely given the values of a function at all
points of the circumference of a circle slightly smaller than the
circle of convergence and concentric with it together with the
condition that the function is to be analytic throughout the interior
of the larger circle. Then the preceding theorems enable us to find
its value at all points within the smaller circle and to determine the
coefficients in the Taylor series proceeding in powers of z — Zq. The
question arises, Is it possible to define the function at points
outside the circle in such a way that the function is analytic
throughout a larger domain than the interior of the circle ? ,



5*6] Taylor's, Laurent's and liouvtlle's theorems 97

In other words, given a potver series which converges and represents a
function only at poiiits within a circle, to define hy means of it the
values of the function at points outside the circle.

For this purpose choose any point Pi within the circle, not on the
line PA. We know the value of the function and all its derivates at
Pj, from the series, and so we can form the Taylor series (for the
same function) with Pi as origin, which will define a function
analytic throughout some circle of centre Pj. Now this circle will
extend as far as the singularity* which is nearest to Pi, which may or
may not be A ; but in either case, this- new circle will iisuall '!
lie partly outside the old circle of convergence, and for jjoints in
the region which is included in the new circle but not in the old
circle, the new series may he used to define the values of the
function, although the old series failed to do so.

Similarly we can take any other point Po, in the region for which the
values of the function are now known, and form the Taylor series with
P as origin, which will in general enable us to define the function at
other points, at which its values were not previously known ; and so
on.

This process is called continuation . By means of it, starting from a
representation of a function by any one power series we can find any
number of other power series, which between them define the value of
the function at all points of a domain, any point of which can be
reached from P without passing through a singularity of the function ;
and the aggregate § of all the power series thus obtained constitutes
the analytical expression of the function.

It is important to know whether continuation by two different paths
fBQ, PB'Q will give the same final power series ; it will be seen that
this is the case, if the function have no singularity inside the
closed curve PBQB'P, in the following way : Let P be any point on PBQ,
inside the circle C' with centre P; obtain the continuation of the
function with Pi as origin, and let it converge inside a circle Ci ;
let P be any point inside both circles and also inside the curve
PBQB'P; let S, Si, Si be the power series with P, Pi, Pi as origins ;
then|| *S'i = S'i' over a certain domain which will contain Pi, if Pi'
be taken sufficiently near Pi ; and hence Si will be the continuation
of Si ; for if Ti were the continuation of Si, we have Ti = Si over a
domain containing Pj, and so (§ 3"73) corresponding coefficients in i
and Ti are the same. By carrying out such a process a sufficient
number of times, we deform the path PBQ into the path PB'Q if no
singular point is inside PBQB'P. The reader will convince himself by
drawing a figure that the process can be carried out in a finite
number of steps.

* Of the function defined by the new sei'ies.

+ The word ' usually ' must be taken as referring to the cases which
are likely to come under the reader's notice while studying the less
advanced parts of the subject.

X French, prolongement ; German, Fortsetzung.

§ Such an aggregate of power series has been obtained for various
functions by M. J. M. Hill, by purely algebraical processes, Proc.
London Math. Soc. xxxv. (1903), pp. 388-416.

II Since each is equal to S.

W. M. A. 7



98 THE PROCESSES OF ANALYSIS [CHAP. V

Example. The series

1 , Z 22 3

a a a" a* represents the function

/('-) = —- a — z

only for points z within the circle | I = | a | .

But any number of other power series exist, of the type

1 z-h z-hf z-hf

a-b' a-hf" a-bf' a-bf '-' '

if b/a is not real and positive these converge at points inside a
circle which is partly inside and partly outside | s | = | a j ; these
series represent this same function at points outside this circle.

5-501. .On functions to which the continuation-process cannot be
applied.

It is not always possible to carry out the process of continuation.
Take as an example the function /(2) defined by the power series

which clearly converges in the interior of a circle whose radius is
unity and w hose centre is at the origin.

Now it is obvious that, as -1-0, /(2) +qc ; the point +1 is therefore
a singularity of/ (2).

But /(2)=22+/( 2)

and if z-- 0, f(z )- x and so /(s)- x, and hence the points for which
z- = l are singularities oi f z) ; the point 2= - 1 is therefore also
a singularity oif z). Similarly since

we see that if 2 is such that 2* = 1, then z is a singularity of/ (2)
; and, in general, any root of any of the equations

22=1, 2* = 1, 28 = 1, 2l =l, ...,

is a singularity of f z). But these points all lie on the circle | 2 |
= 1 ; and in any arc of this circle, however small, there are an
unlimited number of them. The attempt to carry out the process of
continuation will therefore be frustrated by the existence of this
imbroken front of singularities, beyond which it is impossible to
pass.

In such a case the function f z) cannot be continued at all to points
2 situated outside the circle i 2 | = 1 ; such a function is called a
lacuvary f miction, and the circle is said to be a limiting circle for
the function.

551. The identity of tivo functions. .,

The two series

1 + 2 + ' + 2' + . . .

and - 1 + ( - 2) - ( - 2y- + ( - 2) - (2 - 2 + ...

do not both converge for any value, of z, and are distinct expansions.

Nevertheless, we generally say that they represent the same function,
on the

strength of the fact that they can both be represented by the same
rational

1 expression .



/ > -" J DEPARTMENT OF

V MATMEMATIC 5501, 5"51] Taylor's, Laurent's and liouville's THEOREl|
k)l)B R A R y

This raises the question of the identity of two functions. When can
two different expansions be said to represent the same function ?

We might define a function (after Weierstrass), by means of the last
article, as consisting of one power series together with all the other
power series which can be derived from it by the process of
continuation. Two different analytical expressions will then define
the same function, if they represent power series derivable from each
other by continuation.

Since if a function is analytic (in the sense of Cauchy, § 5"12) at
and near a point it can be expanded into a Taylor's series, and since
a convergent power series has a unique differential coefficient (§
5'3), it follows that the definition of Weierstrass is really
equivalent to that of Cauchy.

It is important to observe that the limit of a combination of analytic
functions can represent different analytic functions in different
parts of the plane. This can be seen by considering the series



5(' + j)\!,('"i)(TT7.-r;i )



The sum of the first n + 1 terms of this series is

1 / 1\ 1



z \ zJ' I + z' '

The series therefore converges for all values of z (zero excepted) not
on the circle j 2 | = 1. But, as w — > oo , | 2:'* | — > or j £•" i -
oo according as | j is less or greater than unity; hence we see that
the sum to infinity of the series is

z when \ z\ < 1, and - when | j > 1. This series therefore represents
one

function at points in the interior of the circle | j = 1, and an
entirely different function at points outside the same circle. The
reader will see from § 5"3 that this result is connected with the
non-uniformity of the convergence of the series near | j = 1.

It has been shewn by Borel* that if a region C is taken and a set of
points S such that points of the set S are arbitrarily near every
point of 6', it may be possible to define a function which has a
unique differential coefficient (i.e. is monogenic) at all points of C
which do not belong to *S'; but the function is not analytic in C in
the sense of Weierstrass.



Such a function is



., , 00 w n exp(-expw*) f z)= 2 2 2 ; . . n=ip=oq=o z- p + qi)jn



* Proc. Math. Congress, Cambridge (1912), i. pp. 137-liJ8. Leqons sur
les fonctions mono- genes (1917). The functions are not monogenic
strictly in the sense of § 5'1 because, in the example quoted, in
working out f z + h) -f(z)]jli, it must be su Dposed that R z + h) and
I z + ]i) are not botli rational fractions.



100 THE PROCESSES OF ANALYSIS [CHAP. V

5-6. Laurent's Theorem.

A very important theorem was published in 1843 by Laurent* ; it
relates to expansions of functions to which Taylor's Theorem cannot be
applied.

Let G and C be two concentric circles of centre a, of which C is the
inner ; and let f z) be a function which is analytic i* at all points
on G and G' and throughout the annulus between G and C. Let a + A be
any point in this ring-shaped space. Then we have (§ 5"21 corollary)

• ZTTt j cz — a - h liTi j c z — a — h

where the integrals are supposed taken in the positive or
counter-clockwise direction round the circles.



This can be written

We find, as in the proof of Taylor's Theorem, that

f(z)dz.h- r f(z)dz(z-ar+

c(z-a)"+' z-a-h) Jc z-a-h)h +'

tend to zero as n— cc; and thus we have

/(a + h) = cio + a h + ajr + ... + -~ + j + ...,

where + a = - / K , and 6 = 5— . I z - ay'-'f(z) dz.

This result is Laurent's Theorem ; changing the notation, it can be
expressed in the following form: If f(z) be analytic on the concentric
circles G and G' of centre a, and throughout the annulus between them,
then at any point z of the annidus f z) can he expanded in the form

f z) = tto -h i z -a) + a.2 z-ay- ...+ J -f y: y • • • • ,

where a = . j a.nd b = . [ (t - aT' f t) dt.

An important case of Laurent's Theorem arises when there is only one
singularity within the inner circle G' , namely at the centre a. In
this case the circle G' can be taken as small as we please, and so
Laurent's expansion is valid for all points in the interior of the
circle C, except the centre a.

* Comptes Rendus, xvii. (1843), pp. 348-349. t See § 5'2 corollary 2,
footnote.

X We cannot write a- = fW (a)ln ! as in Taylor's Theorem since /(2) is
not necessarily analytic inside C.



5*6] Taylor's, Laurent's and liouville's theorems 101

Example 1. Prove that

e ' = J x)+zJ x) + z' J x) + ...-irz-J,, x)- ...

1 /"St

where Jn (•*) = s~ I ' ~ " ) •

 TT J

[For the function of z under consideration is analytic in any domain
which does not include the point z=Q ; and so by Laurent's Theorem,

g2V z) = a,, + axZ + aoz'-+... + - ->r + ...,

where n = s — I — n and G = i, — . I e z dz,

and where 6' and C" are any circles with the origin as centre. Taking
C to be the circle of

radius unity, and writing = e' , we have

1 Z"' "" 1 /' -'f

 -=-—, / e:' sin .e- ' irf = jr- / COS (n6 - X H\ n 0) dB, 27ri. Jo
27r y

I sin (/i - i'sin ) tZ vanishes, as may be seen by writing in-cf) for
9. ThiLS



suice



a = J (A'), and \& = ( -)" , since the function expanded is unaltered
if —z • l)e written for z, so that 6 = ( - )" Ai(.:i ), and the proof
is complete.]

Example 2. Shew that, in the annulus defined by|a|<|3|<i6|, the
function

r bz i



I'



\ \ {z-a) b-z) can be expanded in the form

*"+.?,* '(? + 6-.) c. * 1.3. ..(2 -1). 1.3... (2i + 2>i-l) /a\' " ' =
,!o 2 .ll l + n)l [b)

The function is one- valued and analytic in the annulus (see § 5'7),
for the branch-points 0, a neutralise each other, and so, by Laurent's
Theorem, if C denote the circle \ z\=r, where | o ' < /• < | 6 | , the
coefficient of 2" in the required expansion is

1 f dz ( bz 1

27riJ c '- \ \ {z-a) b-z) ' Putting z = re' , this becomes

1 [-' . ,,• 1.3... (2i-l) ?*<*•" 1.3... (2i-l)a' -''

the series being absolutely convergent and uniformly convergent with
regard to 6.

The only terms which give integrals different from zero are those for
which k = l + n. So the coefficient of z" is

(2 -1) 1 . 3 ...(2 + 2/i-l) a \ Sn



1 r 'T * 1

2ir J 1=0



2Kl\ 2* + ™.(/+m) ! ¥*"• b" '



Similarly it can be shewn that the coefficient of — is S a'\



102 THE PROCESSES OF ANALYSIS [cHAP. V

Example 3. Shew that

Z Z''

1 rsT

where = '" '' "> '''" cos ( - v) sin (9 - ?i(9 o? ,

ZTT / 1 /"-"

and i = 5- / e'" + ")'=°' cos (i'-?Osi" - '<9 < <9-

ZTry

5"61. T ie nature of the singularities of one-valued functions.

Consider first a function f z) which is analytic throughout a closed
region 8, except at a single point a inside the region.

Let it be possible to define a function z) such that (i) z) is analj
tic throughout S,

(ii) when a, /( ) = < ( ) + - 4-A . + ...+ "



z — a (z — a)- z — ay-

Then f(z) is said to have a 'pole of order n at a'; and the terms

h , -T + ... + -. are called the principal part of f(z) near a.

z — a. (z — a) (z — aY r r j \ /

By the definition of a singularity (§ 5"12) a pole is a singularity.
If n = 1,

the singularity is called a simple pole.

Any singularity of a one-valued function other than a pole is called
an essential singularity.

If the essential singularity, a, is isolated (i.e. if a region, of
which a is an interior point, can be found containing no singularities
other than a), then a Laurent expansion can be found, in ascending and
descending powers of a valid when dk>\ z — a h, where A depends on the
other singularities of the function, and 8 is arbitrarily small. Hence
the ' principal part ' of a function near an isolated essential
singularity consists of an infinite series.

It should be noted that a pole is, by definition, an isolated
singularity, so that all singularities which are not isolated (e.g.
the limiting point of a sequence of poles) are essential
singularities.

There does not exist, in general, an expansion of a function valid
near a non-isolated singularity in the way that Laurent's expansion is
valid near an isolated singularity.

Corollary. If f z) has a pole of order n at a, and z) = z — aYf z) z
a), i\ r a)= lim z-a)' f z), then y\ r z) is analytic at a.

Example 1. A function is not bounded near an isolated essential
singularity.

[Prove that if the function were bounded near z=a, the coefficients of
negative powers of 2 — a would all vanish.]



5-61, 5 -62] Taylor's, Laurent's and liouville's theorems 103

c z\

Example 2. Find the singularities of the function e ~ 7 e - 1 .

At 2 = 0, the numerator is analytic, and the denominator has a simple
zero. Hence the function has a simple pole at 2 = 0.

Similarly there is a simple pole at each of the points mria ( = + 1,
+2, +3, ...); the denominator is analytic and does not vanish for
other values of z.

A.t z = a, the numerator has an isolated singularity, so Laurent's
Theorem is applicable, and the coefficients in the Laurent expansion
may be obtained from the quotient

c c

z- a 2 I (z — aV



Ml+'--" + ...)-l



which gives an expansion involving all positive and negative powers of
z — a). So there is an essential singularity at 2 = a.

Example 3. Shew that the function defined by the series

I wg"- (l +)t- )"-l

 =1 (2 -l) 2 -(l+/i-l)

has simple poles at the points 2 = (1 + ~i)e- ''' / ( =0, 1, 2, ... n
— \; ?i = l, 2, 3, ...). y (Math. Trip. 1899.)

5'62. The 'point at infinity.'

The behaviour of a function /C ') as | ' — oo can be treated in a
similar way to its behaviour as z tends to a finite limit.

If we write z = -,, so that large values of z are represented by small

values of z' in the '-plane, there is a one-one correspondence between
z and z , provided that neither is zero ; and to make the
correspondence complete it is sometimes convenient to say that when z
is the origin, z is the * point at infinity.' But the reader must be
careful to observe that this is not a definite point, and any
proposition about it is really a proposition concerning the point / =
0.

Let/(2 ) = 4> z'). Then < z') is not defined at z = 0, but its
behaviour near z = is determined by its Taylor (or Laurent) expansion
in powers of z \ and we define < (0) as lim (/) if that limit exists.
For instance

the function (/) may have a zero of order m at the point 2' = ; in
this case the Taylor expansion of ( ') will be of the form

and so the expansion of f z) valid for sufficiently large values of |
.0 j will be of the form

In this case,/(0) is said to have a zero of order m at ' infinity.'



104 THE PROCESSES OF ANALYSIS [CHAP. V

Again, the function (f)(2') may have a pole of order m at the point z'
= 0; in this case

and so, for sufficiently large values of \ z\, f(z) can be expanded in
the form

N P

f z) = Az"' + Bz'''-' + Cz'"-- +...+Lz + M+- + - + ....

In this case,/(2 ) is said to have a pole of order m at ' infinity. '
Similarly f z) is said to have an essential singularity at infinity,
if z) has an essential singularity at the point / = 0. Thus the
function e' has an

essential singularity at infinity, since the function e~' or

1 \ 1\ 1

has an essential singularity at z = 0.

Example. Discuss the function represented by the series

2 — , :; 5—5, ( >1).

Z 1

The function represented by this series has singularities at 2=— and
2= - i

 n=\, 2, 3, ...), since at each of these points the denominator of one
of the terms in the series is zero. These singularities are on the
imaginary axis, and have 3 = as a limiting point ; so no Taylor or
Laurent expansion can be foi med for the function valid throughout any
region of which the origin is an interior point.

For values of z, other than these singularities, the series converges
absolutely, since the limit of the ratio of the (?i + l)th term to the
?ith is lim (?i+ l)~i a~ = 0. The function is

an even function of z (i.e. is unchanged if the sign of z be changed),
tends to zero as I 2 I - Qc , and is analytic on and outside a circle
C of radius greater than unity and centre at the origin. So, for
points outside this circle, it can be expanded in the form

h + j + b+

where, by Laurent's Theorem,

'" 27nJ c =o ! a-2 + 22

This double .series converges absolutely when | 2 | > 1, and if it be
rearranged in powers of 2 it converges uniformly.

Since the coefficient of 2 ~ Ms 2 — ; and the only term which
furnishes a non-

H=o n !

zero integral is the term in z~ , we have

(\ )fc-ia-2A- dz



b'k • 1 - .



Itti J c =o , fo n\ a2*



5-63, 5-64] Taylor's, Laurent's and liouville's theorems 105

Therefore, when | 2 | > 1, the function can be expanded in the form

ill

e"' e * e

The function has a zero of the second order at infinity, since the
expansion begins with a term in z~' .

5-63. Liouville's Theorem*.

Let f z) he analytic for all values of z and let \ f z)\ < K for all
values of z, where K is a constant (so that \ f z) is hounded as | r |
— > x ). Then f z) is a constant.

Let z, z' be any two points and let C be a contour such that z, z are
inside it. Then, by § 5'21,

take C to be a circle whose centre is z and whose radius is p 2 | / —
s | ; on

C write \ \ = z pe' \ since jf — /| 2P when is on C it follows from §
4-62 that

= 2\ z -z\ Kp-K Make p- cc keeping z and z' fixed; then it is obvious
that/(/) —f(z) = 0; that is to say, f(z) is constant.

As will 1)0 seen in the next article, and again frequently in the
latter half of this volume (Chapters xx, xxi and xxii), Liouville's
theorem furnishes short and convenient proofs of some of the most
important results in Analysis.

564. Functions with no essential singularities.

We shall now shew that the only one-valued functions luhich have no
singidarities, except poles, at any jjoint (including oo ) are
rational functions.

For let f(z) be such a function ; let its singularities in the finite
part of the plane be at the points Cj, c-j, ... Ck'. and let the
principal part (§ 5"61) of its expansion at the pole Cr be

+ 1 ::; + . . . + — 7 —



Z — Cr (Z — Crf '" (Z - CyJ

Let the principal part of its expansion at the pole at infinity be

a- z + a-iZ + ... + anZ \

if there is not a pole at infinity, then all the coefficients in this
expansion

will be zero.

* This theorem, which is really due to Cauchy, Comptes Rendus, xix.
(1844), pp. 1377, 1378, was given this name by Borchardt, Journal fvr
Math, lxxxviii. (1880), pp. 277-310, who heard it in Liouville's
lectures in 1847.



106 THE PROCESSES OF ANALYSIS [CHAP. V

Now the function

has clearly no singularities at the points c , c.j, ... Ck, or at
infinity; it is therefore analytic everywhere and is bounded as \ z -
>cc, and so, by Liou\ dlle's Theorem, is a constant; that is,

/(.) = + .. + a..= + . . . + ,,.. + I l i + ( . + • • + (i J •

where C is constant ; f(z) is therefore a rational function, and the
theorem is established.

It is evident from Liouville's theorem (combined with § 361 corollary
(ii)) that a function which is analytic everywhere (including oc ) is
merely a constant. Functions which are analytic everywhere except at
oc are of considerable importance; they are known as integral
functions*. Examples of such functions are e , sin z, e . From § 5-4
it is apparent that there is no finite radius of convergence of a
Taylor's series which represents an integral function ; and from the
result of this section it is evident that all integral functions
(except mere polynomials) have essential singularities at oo .

5"7. Many-valued functions.

In all the previous work, the functions under consideration have had a
unique value (or limit) corresponding to each value (other than
singularities) of .

But functions may be defined which have more than one value for each
value of z ; thus if 2- = r (cos 6 + i sin 6), the function 2- has the
two values

r* (cos 1(9 + t sin (9) , r jcos \ 0 + lir) + i sin \ (6 + 27r)| ;

and the function arc tan x (x real) has an unlimited number of values,
viz. Arc tan x + mr, where — tt < Arc tan x <- 'tt and n is any
integer ; further

examples of many- valued functions are log z, z , sin z~).

Either of the two functions which z represents is. however, analytic
except at = 0, and we can apply to them the theorems of this chapter ;
and the two functions are called ' branches of the many-valued
function z'-.' There will be certain points in general at which two or
more branches coincide or at which one branch has an infinite limit ;
these points are called ' branch-points.' Thus z- has a branch-point
at ; and, if we consider the change in z as z describes a circle
counter-clockwise round 0, we see that 6

* Vxench, fonction entilre; Germa,n, game Funktion.



57] TAYLORS, Laurent's and ltouvtlle's theorems 107

increases by 27r, r remains unchanged, and either branch of the
function passes over into the other branch. This will be found to be a
general characteristic of branch-points. It is not the purpose of this
book to give a full discussion of the properties of many-valued
functions, as we shall always have to consider particular branches of
functions in regions not containing branch- points, so that there will
be comparatively little difficulty in seeing whether or not Cauchy's
Theorem may be applied.

Thus we cannot apply Cauchy's Theorem to such a function as z when the
path of iutegi-ation is a circle surrounding tlie origin ; but it is
permissible to apply it to one of

the branches of z when the path of integration is like that shewn in §
6 '24, for through- out the contour and its interior the function has
a single definite value.

Example. Prove that if the different values of a , corresponding to a
given value of z, are represented on an Argand diagram, the
representative points will be the vertices of an equiangular polygon
inscribed in an equiangular spiral, the angle of the spiral being
independent of a.

(Math. Trip. 1899.)

The idea of the different braiickes of a function helps us to
understand such a paradox as the following.

Consider the function y = ' i

for which ~=x log x).

AVhen x is negative and real, is not real. But if x is negative and of
the form

P (where p and q are positive or negative integers), y is real.

Z'j -f- 1

If therefore we draw the real c\ irve

we have for negative values of ./ a set of conjugate points, one point
corresponding to each rational value of x with an odd denominator ;
and then we might think of proceeding to form the tangent as the limit
of the chord, just as if the curve were continuous ; and

thus —-, when derived from the inclination of the tangent to the axis
of x, would appear

to be real. The question thus arises, Why does the ordinary process of
differentiation

give a non-real value for — ? The explanation is, that these conjugate
points do not all

arise from the same branch of the function y = x' . We have in fact •

y — 5

where k is any integer. To each value of k corresponds one branch of
the function y. Now in order to get a real value of y when x is
negative, we have to choose a suitable value for k : and this value of
k varies as we go from one conjugate point to an adjacent one. So the
conjugate points do not represent values of y arising from the same
branch of the

function y=x , and consequently we cannot expect the value of — when
evaluated

for a definite branch to be given by the tangent of the inclination to
the axis of x of the line joining two arbitrarily close members of the
series of conjugate points.



108 THE PROCESSES OF ANALYSIS [CHAP. V

EEFEREXCES.

E. GouRSAT, Cours dJ Analyse, ii. (Paris, 1911), Chs. xiv and xvi.

J. Hadamard, La Serie de Taylor et son prolongement analytique
(Scientia, 1901).

E. LiNDELOF, Le Calmd des Residus (Paris, 1905).

C. J. DE LA Vallee Poussin, Covrs d' Analyse Infinitesimale, i. (Paris
and Louvain, 1914), Ch. X.

E. BoREL, Lecons stir les Fonctions Entieres (Paris, 1900).

G. N. Watson, Complex Integration and Cauchy's Theorem (Camb. Math.
Tracts, no. 15, 1914).

Miscellaneous Examples.

1. Obtain the expansion

/(.)=/( ) + 2 1 -/ (- + -3-3-r/ ( -2-j + -25751- ' \ \ r-] ' and
determine the circumstances and range of its vaUdity.

2. Obtain, under suitable circumstances, the expansion

+ .... (Corey, Ann. of Math. (2), i. (1900), p. 77.)

3. Shew that for the series

  1



)i=o -r~

the region of convergence consists of two distinct areas, namely
outside and inside a circle of radius unity, and that in each of these
the series represents one function and represents it completely.

(Weierstrass, Berliner Monatsherichte., 1880, p. 731 ; Ges. Werke, 11.
(1895), p. 227.)

4. Shew that the function

2 s"'

tends to infinity as 2- -exp i-niplm !) along the radius through the
point ; where m is any integer and p takes the values 0, 1, 2, ..'.
(ni I - 1).

Deduce that the function cannot be continued beyond the unit circle.

(Lerch, Sitz. BOhm. Acad., 1885-6, pp. 571-582.)

5. Shew that, if z-— 1 is not a positive real number, then

-". •.s:)"<'-")- /:'=-('-'- '-**-

(Jacobi and Scheibner.)



Taylor's, Laurent's and liouville's theorems 109

6. Shew that, if s - 1 is not a positive real number, then

,, , , m, m(m+ ) to (m + 1) ... (to + ?i — 1)

(1-2) — =l+-2 + — 2 "" +-+ -1 .

n : J

(Jacobi and Scheibner.)

7. Shew that, if z and 1—2 are not negative real numbers, then

  Jo m+l [ m + 3 (m+3) ... (m + 27i- 1) J

+ "') (m+l)(w + 3)...(m + 27i-l)Jo ' -

(Jacobi and Scheibner.)

8. If, in the expansion of (a + i2 + a22 )'" by the multinomial
theorem, the remainder after n terms be denoted by R (2), so that

(a4- i2 + a2S-)'" = o + - i2 + -'-l2-"" + --- + n-iS"'"' + (s), shew
that



/4(.) = ( + a.. + ..r j, — (a + .M+a, r " ' •

9. If (ao + i2 + a22 )""'"' /'(ao + i< + 2 -)'"o?<

y



(Scheibner.)



be expanded in ascending powers of 2 in the form

Jl2 + .4222+...,

shew that the remainder after n-\ terms is

(ao + ai2 + a22 )~'"" I (ao + a, + a2 )'" o .i-(2?'i + ? + 1) 2 -i
""'c C

(Scheibner*.)

10. Shew that the series

where X (2)= - 1 +2- f, + I",- - + (- )" , '

and where (2) is analytic near 2 = 0, is convergent near the point 2 =
; and shew that if the sum of the series be denoted by/(2), then/(2)
satisfies the differential equation

/'(--)=/( )-0(4

(Pincherle, Rend, dei Lincei (5), v. (1896), p. 27.)

11. Shew that the arithmetic mean of the squares of the moduli of all
the values of the series "2 0, on a circle |2| = r, situated within
its circle of convergence, is equal

 i=0

to the sum of the squares of the moduli of the separate terms.

(Gutzmer, Math. Ann. xxxii. (1888), pp. .596-600.)

* The results of examples 5, 6 and 7 are special cases of formulae
contained in Jacobi's dis- sertation (Berlin, 182.5) published in bis
Ges. Werke, in. (1884), pp. 1-44. Jacobi's formulae were generalised
by Scheibner, Lelpziger Berichte, xlv. (1893), pp. 432-443,



110 THE PROCESSES OF ANALYSIS [CHAP. V

12. Shew that the series

2 e-2(am) -m-l m = l

converges when | 2 | < 1 ; and that, when a > 0, the function which it
represents can also be represented when | s ! < 1 by the integral



/ay f" e-"" dx n-J Jo ex\ 2 x '



and that it has no singularities except at the point z=l.

(Lerch, Monatshefte fiir Math, und Phys. viii.) 13. Shew that the
series



2



\ \ ,, 2 r z\ zj \

 z + z )-V- 2 ( i\ 2 \ 2y'zi)(2v + iv'zif' iv-2v'z- i) 2v + 2v'z-'
xf]'



in which the summation extends over all integral values of v, v',
except the combination (i' = 0, v' = 0), converges absolutely for all
values of z except purely imaginary values ; and that its sum is + 1
or — 1, according as the real part of z is positive or negative.

(Weierstrass, Berliner Monatsherichte, 1880, p. 735.)

14. Shew that sin \ u ( + -)[ can be expanded in a series of the type

a , + aiZ + aoy-+...+- + j, + ...,

in which the coefficients, both of s" and of z~'\ are

1 / 2t

-— I sm (2m cos ) cos Ji o? . •2ir J ) •

n=i n-z' + a-

shew that/ (2) is finite and continuous for all real values of z, but
cannot be expanded as a Maclaurin's series in ascending powers of z ;
and explain this apparent anomaly.

[For other cases of failure of Maclaurin's theorem, see a posthumous
memoir by Cellerier, Bidl. des Set. Math. (2), xiv. (1890), pp.
145-599 ; Lerch, Journal fiir Math. cm. (1888), pp. 126-138;
Pringsheim, Math. Ann. XLii. (1893), pp. 153-184 ; and Du Bois
Reymond, Miinehener Sitzungsberichte, vi. (1876), p. 235.]

16. If f z) be a continuous one- valued function of z throughout a
two-dimensional region, and if

   f z)dz =



h



for all closed contours C lying inside the region, then f z) is an
analytic function of z throughout the interior of the region.



[Let a be any point of the region and let

F z)=\'f z)dz

J a



It follows from the data that F z) has the unique derivate f z). Hence
F z) is analytic (§ 5*1) and so (§ 5*22) its derivate /(z) is also
analytic. This important converse of Cauchy's theorem is due to
Morera, Rendiconti del R. 1st. Lomhardo Milano), xxil. (1889), p.
191.]

