\chapter{Ellipsoidal Harmonics and \Lame's Equation}

231. The definition of ellipsoidal harmonics.

It has been seen earlier in this work (§ 18-4) that solutions of
Laplace's equation, which are analytic near the origin and which are
appropriate for the discussion of physical problems connected with a
sphere, may be con- veniently expressed as linear combinations of
functions of the type

cos r' P (cos ), r' Pn'" (cos 6) . md),

where n and m are positive integers (zero included).

When Pn (cos 6) is resolved into a product of factors which are linear
in cos 6 (multiplied by cos 6 when n is odd), we see that, if cos 9 is
replaced by zjr, then the zonal harmonic r"P (cos 6) is expressible as
a product of factors which are linear in a, y and z, the whole being
multiplied by z when n is odd. The tesseral harmonics are similarly
resoluble into factors which are linear in a, y and z- multiplied by
one of the eight products 1, cc, y, z, yz, zx, xy, xyz.

The surfaces on which any given zonal or tesseral harmonic vanishes
are surfaces on which either 6 or (f) has some constant value, so that
they are circular cones or planes, the coordinate planes being
included in certain cases.

When we deal with physical problems connected with ellipsoids, the
structure of spheres, cones and planes associated with polar
coordinates is replaced by a structure of confocal quadrics. The
property of spherical harmonics which has just been explained suggests
the construction of a set of harmonics which shall vanish on certain
members of the confocal system.

Such harmonics are known as ellipsoidal harmonics; they were studied
by Lame* in the early part of the nineteenth century by means of
confocal coordinates. The expressions for ellipsoidal harmonics in
terms of Cartesian coordinates were obtained many years later by W. D.
Niven-f-, and the following account of their construction is based on
his researches.

The fundamental ellipsoid is taken to be

x ir z, a 0 c

and any confocal quadric is

* Journal de Math. iv. (1839), pp. 100-125, 126-163. t Phil. Trans.
182 a (1892), pp. 231-278.

23'1, 23-2] ELLIPSOIDAL HARMONICS 537

where is a constant. It will be necessary to consider sets of such
quadrics, and it conduces to brevity to write

U - I 1 = (h)

The equation of any member of the set is then

Qp = 0.

The analysis is made more definite by taking the a;-axis as the
longest axis of the fundamental ellipsoid and the -axis as the
shortest, so that a>b> c.

23'2. The foiir species of ellipsoidal hat monics.

A consideration of the expressions for spherical harmonics in factors
indicates that there are four possible species of ellipsoidal
harmonics to be investigated. These are included in the scheme

1, y, zx, xyz r% %. ...%,

X,

yz,

2/'

zx,

z,

 y,

where one or other of the expressions in is to multiply the product If
we write for brevity

©,(H),...e, = n((H)),

any harmonic of the form IT (0) will be called an ellipsoidal harmonic
of the first species. A harmonic of any of the three forms* 11 (©),
yYi (©), 11 (©) will be called an ellipsoidal harmonic of the second
species. A harmonic of any of the three forms* yzXl (0), '.rll (®),
xyYi (©) will be called an ellipsoidal harmonic of the third species.
And a harmonic of the form xyzYl (0) will be called an ellipsoidal
harmonic of the fourth species.

The terms of highest degree in these species of harmonics are of
degrees 27n, 2m + 1, 2m + 2, 2m + 3 respectively. It will appear
subsequently (§ 23"26) that 2/1 + 1 linearly independent harmonics of
degree n can be constructed, and hence that the terms of degree n in
these harmonics form a fundamental system (§ 18"o) of harmonics of
degree n.

We now proceed to explain in detail how to construct harmonics of the
first species and to give a general account of the construction of
harmonics of the other three species. The reader should have no
difficulty in filling up the lacunae in this account with the aid of
the corresponding analysis given in the case of functions of the first
species.

* The three forms will be distinguished by being described as
different tyj es of the species.

538 THE TRANSCENDENTAL FUNCTIONS [CHAP. XXTII

23'21. Tlie construction of ellipsoidal harmonics of the first
species.

As a simple case let us first consider the harmonics of the first
species which are of the second degree. Such a hal-monic must be
simply of the form @i.

Now the effect of applying Laplace's operator, namely

9- d 9- x y- z"

9 ' 9p' 9? a d,' ¥'+T, ¥Ver

2 2 2

ci" + e, b'- + 0, c' + e, '

and so @i isa harmonic if 6 is a root of the quadratic equation

(0 + ¥) 6 + cO + ((9 + d") (6 + a') + (f) + a') (6 + b') = 0.

This quadratic has one root between - c and - b~ and another between -
6 and - a-. Its roots are therefore unequal, and, by giving 6 the
value of each root in turn, we obtain two* ellipsoidal harmonics of
the first species of the second degree.

Next consider the general product @i02 ... @; this product will be
denoted by n (0) and it will be supposed that it has no repeated
factors - a supposi- tion which will be justified later (§ 23-43).

If we temporarily regard Bj, @, ... (S) j as a set of auxiliary
variables, the ordinary formula of partial differentiation gives

9n (0) \ ' dU ((H)) 90 \ dUi® 2cc dx pZi 9@p dx p i d p ' a- + dp'

and, if we differentiate again,

9 n(@)\ 9n(0) 2 . dm(®) Sx

dx' p=i 9@ ' a'+0p' j,, dBpdB, " (a' + dp) a" + 6 ) '

where the last summation extends over all unequal pairs of the
integers 1, 2, ... m. The terms for which p = q may be omitted because
none of the expressions 0,, @o, ... 0 enters into fl (0) to a degree
higher than the first.

It follows that the result of applying Laplace's operator to 11 (0) is

5 i5II(©) j 2 2 2 )

V S'n (0 ) (\ Sx' 8y' 8f ]

  jZ Wpm, t(a + 0p) (a- + e~,) "*" (6 + 0p) (¥ + dg) (c + Op) (c +
d,)\ '

j\ ow = - •

(.ry,A ci'+0p)(O''+f g) 0g-0p' \ a,b,c )

* The complete set of 5 ellipsoidal harmonics of the second degree is
composed of these two together with the three harmonics yz, zx, xy,
which are of the third species.

23 21] ELLIPSOIDAL HARMONICS 539

and 9n (S)/dSp consists of the product 11 (0) with the factor ®p
omitted, while d U (S)/d®pdSq consists of the product IT (0) with the
factors 0 and 0 omitted. That is to say

3 n(0) an(0) a n(0) 8n(©)

If we make these substitutions, we see that

n(0)

9 9- 3'

 o n

may be written in the form

g 9 n (0) f 2 2 2 V' \ \ ?\

the prime indicating that the term for which q = p has to be omitted
from the summation.

If n (0) is to be a harmonic it is annihilated by Laplace's operator;
and it will certainly be so annihilated if it is possible to choose 0
, O., ... 0,n so that each of the equations

a' + e ' b' + ep c'+dp Op-e

is satisfied, where p takes the values 1, 2, ... m.

Now let i9 be a variable and let Aj (6) denote the polynomial of
degree m in 6

m

n d-e,).

q = \

If A/( ) denotes cZAj (6)/cW, then, by direct differentiation, it is
seen that A/ (6) is equal to the sum of all products of - i, 6 - 6.,,
... 6 - 6,ni - l at a time, and A/' (6) is twice the sum of all
products of the same expressions, m - 2 at a time.

Hence, if 6 be given the special value 6p, the quotient A " (6p)IA
(6p) becomes equal to twice the sum of the reciprocals of 6p-6y, 6p -
6.2, ... 6p - 6, (the expression 6p - 6p being omitted).

Consequently the set of equations derived from the hypothesis that

n (0J,) is a harmonic shews that the expression

'' ' 1 1 1 2A/'( J

a' + 6' ¥+6' c'+d A,' (6)

vanishes whenever 6 has any of the special values 6, 60, ... 6j,.

Hence the expression

(a' + 6) (b' + 6) (c + 6) A," 6) + l\ 1 (¥ + 6) (c' + 6),- A/ 6)

" ( a, h, c '

540 THE TRANSCENDENTAL FUNCTIONS [CHAP. XXIII

is a polynomial in 6 which vanishes when 6 has any of the values 0, d
, •, m, and so it has 6-6, - 6.2, •••. - d,n as factors. Now this
polynomial is of degree m + 1 in and the coefficient of '"+ is m (m +
|), Since m of the factors are known, the remaining factor must be of
the form

m (m + ) 6 + IG,

where is a constant which will be determined subsequently.

We have therefore shewn that

( . + 0) f2 + d) (c + 6) A/' 6) + \ \ 1 h - + 6) \& + )l A/ 6)

= |m(m + *)6' + i-C' Ai(6').

That is to say, any ellipsoidal harmonic of the first species of
(even) degree n is expressible in the form

where 0, 6.,, ..., i are the zeros of a pol Tiomial Ai( ) of degree
n; and this polynomial must be a solution of a differential equation
of the type

4V (a + )(6 + )(c + ) |

 \ \ {a + e) ¥ + e) c - e)] '>

= [n n + l)e + C].\, e).

This equation is known as Lame's differential equation. It will be in-
vestigated in considerable detail in §§ 23"4-23*81, and in the course
of the investigation it will be shewn that (I) there are precisely n +
1 different real values of G for which the equation has a solution
which is a polynomial in d of degree \ n, and (II) these polynomials
have no repeated factors.

The analysis of this section may then be reversed step by step to
establish the existence of -J + 1 ellipsoidal harmonics of the first
species of (even) degree n, and the elementary theory of the harmonics
of the first species will then be complete.

The corresponding results for harmonics of the second, third and
fourth species will now be indicated briefly, the notation already
introduced being adhered to so far as possible.

23'22. Ellipsoidal liarmonics of the second species.

in

We take x II (0 ) as a typical harmonic of the second species of
degree

2m + 1. The result of applying Laplace's operator to it is

r: 311(0) I 6 2 2 I

 Ip i dSp \ a' + dp' b' + Op' c' + dp]

   X dSpde ( a'- + dp) (a + 6, ) (6' + dp) (b' + 6, ) " (c= + 0p) (c +
6,)\ ] '

23-22, 23'23] ellipsoidal harmonics 541

and this has to vanish. Consequently, if

3 = 1

we find, by the reasoning of § 23"21, that A 0) is a solution of the
differential equation

(cc- + d) b'- + e) c' + e)A./'(e)

+ 3 (6 + 0) ic' + e) + (c ~ + d) (a' + 0) + ( + 0) ib' + 0) A/ (0)

= m(m + f) + ia Ao(6'), where Cn is a constant to be determined.

If now we write Ag (0) = A ( )/\/(a- + 0), we find that A (0) is a
solution of the differential equation

4> (a + 0)(h - + 0) c +0)

V((a + )(6- + )(c + )]. '

d0

= (2m + 1) 2ru + 2) + C A 0), where C = 6*2 + 6 + c-.

It will be observed that the last differential equation is of the same
type as the equation derived in § 23"21, the constant n being still
equal to the degree of the harmonic, which, in the case now under
consideration, is 2m + 1.

Hence the discussion of harmonics of the second species is reduced to
the discussion of solutions of Lame's differential equation. In the
case of harmonics of the first type the solutions are required to be
polynomials in multiplied by \/ a + 0); the corresponding factors for
harmonics of the second and third types are \/(b- -+- 0) and (c" + 0)
respectively. It will be shewn subsequently that precisely m + 1
values of C can be associated with each of the three types, so that,
in all, 8???, + 3 harmonics of the second species of degree 2m 4- 1
are obtained.

23*23. Ellipsoidal harmonics of the third species.

m

We take yz U (@ ) as a typical harmonic of the third species of degree
2m + 2. The result of applying Laplace's operator to it is

r ' 31] (Ch)) [ 2 6 6

• Ip' i % ( ' + 0p h + 0p c'- -0,

  dm (@) f 8 %if 8 n

+ j q d%d% ((a + 0p) a + 0q) (¥ + dp) (6 +,) " (c + 0 ) (c' + 0q) i I
' and this has to vanish. Consequently, if

A3( )= n 0-0,;),

3=1

542 THE TRANSCENDENTAL FUNCTIONS [cHAP. XXIII

we find, by the reasoning of § 23'21, that A3 6) is a solution of the
differential equation

(a' + 6) (b' + 6) c' + 6) A," \&)

+ h K '-' + d) (c- + 6') + 3 (c + 6) tr + ) + 3 te + 6) b' + 6)] A,' 0

= [m(vi + )e + iC,]As(e), where Oj is a constant to be determined.

If now we write A3 (6) = A 0)l [ b' + 6) (c-' + 6)], we find that A 6)
is a solution of the differential equation

4 Vl(a- + 0) ¥ + 6) (c + )| ~ s/ ce + 6) ¥ + 6) (c' + 6)

dA(e)'

cie

= (2m -f 2) (2m + 3) + C] A (0),

where C = C3 -F 4 (1 + b- + c-.

It will be observed that the last equation is of the same type as the
equation derived in § 23'21, the constant n being still equal to the
degree of the harmonic, which, in the case now under consideration, is
2m + 2.

Hence the discussion of harmonics of the third species is reduced to
the discussion of solutions of Lame's differential equation. In the
case of harmonics of the first type, the solutions are required to be
j olynomials in multiplied by VK " + ) ( + )j ' ® corresponding
factors for harmonics of the second and third types are J (c" + 6) (a-
+ 6)] and \ J[ a- + 6) (b- + 6)] respectively. It will be shewn
subsequently that precisely m + 1 values of C can be associated with
each of the three types, so that, in all, Sni + 3 harmonics of the
third species of degree 2m + 2 are obtained.

23'24. Ellipsoidal harmonics of the fourth species.

The harmonic of the fourth species of degree m + 8 is expressible in
the

m

form xyz 11 (@p). The result of applying Laplace's operator to it is

r an 0) [ 6 i>\ \ . \ \ L

"l3i % t ' + P ' ' + ' P c' + p

  d'U (B) ( 8a,- \ %'

",i ae aB, [(a- + e ) a? + e,) " (ft + e j¥ + e ) " (c + e,;) (c +
e,,)\ \ '

and this has to vanish. Consequently, if

Hi

A,( )=ll( -,),

7 = 1

we find by the reasoning of § 23'21 that A4( ) is a solution of the
equation

( + )(6' + )(c- + )a;'( )+ I :s (6 + <9)(c + 6 )|a;( )

= m(m + |) + iC, A,( ), where C4 is a constant to be determined.

 'ce~ + e) h - + e)(c' + 0) ~

23*24, 23-25] ellipsoidal harmonics 543

If now we write

A, 6) = A 0)/ (a + 6) (¥ + d) (c' + 0), we find that A (0) is a
solution of the differential equation

= (2m + 3) 2m + 4) + C A (6),

where C = C + 4< (a- + h- + c'-).

It will be observed that the last equation is of the same type as the
equation derived in §23'21, the constant n being still equal to the
degree of the harmonic which, in the case now under consideration, is
2m + 3.

Hence the discussion of harmonics of the fourth species is reduced to
the discussion of solutions of Lame's differential equation. The
solutions are required to be polynomials in 6 multij)lied by \/ a" +
6) b"- + 6) (c + n)]. It will be shewn subsequently that precisely m +
1 values of C can be associated with solutions of this type, so that m
+ 1 harmonics of the fourth species of degree 2rii -f 3 are obtained.

23*25. Nivens expressions fur ellipsoidal harmonics in terms of homo-
geneous harmonics.

If Gn (x, y, z) denotes any of the harmonics of degree n which have
just been tentatively constructed, then Gn ( ", y. z) consists of a
finite number of terms of degrees n, n - 2, w - 4, ... in x, y, z. If
H x, y, z) denotes the aggregate of terms of degree n, it follows from
the homogeneity of Laplace's operator that Hn (, y, z) is itself a
solution of Laplace's equation, and it may obviously be obtained from
Gn x, y, z) by replacing the factors (h), which occur in the
expression of Gn oc, y, z) as a product, by the factors Kp.

It has been shewn by Niven loc. cit., pp. 243-245) that Gn (x, y, z)
may be derived from Hn oc, y, z) by applying to the latter function
the differential operator

2(2?i-l)"*'2.4.(2w-l)(2w-3) 2.4.6(2 1 - l)(27i-3)(27i-5) " " '

where D'- stands for

92 B 92

a - + b -- + c ~ •

da dy dz '

and terms containing powers of D higher than the nth may be omitted
from the operator.

We shall now give a proof of this result for any harmonic of the first
species*.

* The proofs for harmonies of the other three species are left to the
reader as examples. A proof applicable to fuiictious of all four
species has been given by Hobson, Proc. London Math. Soc. XXIV.
(1893), pp. (30-64. In constructing the proof given in the text,
several modifi- cations have been made in Niven's proof.

544 THE TRANSCENDENTAL FUNCTIONS [CHAP. XXIII

For such harmonics the degree is even and we write

p=i p=\

where Sn, Sn-2, Sn-i, ... are homogeneous functions of degrees n, w
- 2, w - 4, . . ., respectively, and

in Sn = Hn (x, y, Z)=U Kp.

p = l

The function Sn-2r is evidently the sum of the products of K, K2, ...
Ki, taken |- n - ?• at a time.

If K, K2, ... /iTi n be regarded as an auxiliary system of variables,
then, by the ordinary formula of partial differentiation

a>s,

doc p=-i dKp dx

in dS 9r

p=i dKp a~ + Bp and, if we differentiate again,

= t - - + t

dx pLx dKp a-'+dp p dKpdKq c(r + dp) a + 6 )'

The terms in d-Sn- rl Kp can be omitted because each of the functions
Kp does not occur in Sn\ 2r to a degree higher than the first.

It follows that

p=i dKp \ a + p 6 + <9p \& + dp\

;, BKpdK, ( a' + Op) (a + 6,) ¥ + Op) b' + 6,) (c + dp) (c + 6,)] It
will now be shewn that the expression on the right is a constant
multiple

OI >Oji\ 2r- 2-

We first observe that

a?x OpKp - dqKq

\ a, b, c)

and that, by the differential equation of § 23-21,

a,T, (• " + 0p a, b, c a- + Op

= 6 - Up 2,,

q = \ tTp - Vq

23-25] so that

ELLIPSOIDAL HARMONICS

545

D' n-or

p=i oKp p=i

+ s t e,

dSn-.

, in

bK

1 6p - 0q

pjpq dKpdKq

d"Sn-2r dpKp- UqKq

% - ( q

Now dSn-or/dKp is the sum of the products of the expressions K, Ko, .
. . iTijj (Kp being omitted) taken n - ?' - 1 at a time; and Kqd
Sn-2rldKpdKq consists of those terms of this sum which contain Kg as a
factor.

Hence

C'0,i\ 2 '

dK.,

~ - K,

is equal to the sum of the products of the expressions K, Ko, ...
-K"i; (Kp and Kq both being omitted) taken hn - r-1 at a time; and
therefore, by sym- metry, we have

dKr.

K,

''dKpdKq

so that

dKpdkq " 1 ~dKr

dK,

dSn-

-K

dK.

P dKpdKq'

\ Kq-Kp).

,p w-i q

On substituting by this formula for the second differential
coefficients, it is found that

UpKp- UqKq

hi \ i i

  q = ( p - Gq q = \ \ {Gp- q) Kp- Kq)\

hi

6-8 i'

p i dKp

= 4:11 - I) Z

K

2 = 1 Kp - Kq\

\ Kp-Kq).

p = \ " -Q-p p <i

Now we may write Sn-2r in the form

 n-ir "I" "- p n--2r-'2 "l -' ' Ji- 2r- 2 "T -"- - r/' Jl- 2;'- 4>

where 5io, denotes the sum of the products of the expressions K, K.,
... /fj,, (Kp and Kq both being omitted) taken in at a time; and we
then see that

dSn-

" dK

 Kp-Kq)Sn-,r-2.

Hence

D'-8n-,r = (4n -2) t

p-l cIVp p q

Now it is clear that the expression on the right is a homogeneous sym-
metric function of K, K.2, •••Kx, of degree n - r-\, and it contains
no power of any of the expressions K, K, ... Ki to a degree higher
than the first. It is therefore a multiple of Sn- r-z- To determine
the multiple we w. M. A. 35

546 THE TRANSCENDENTAL FUNCTIONS [CHAP. XXIII

observe that when Sn- - is written out at length it contains jt C;.+i
terms while the number of terms in

is hn (4n - 2) ., \,C, - 8 . a . i \,C,-i-

The multiple is consequently

and this is equal to (2r + 2) (2?i - 2r - 1). It has consequently been
proved that

D- Sn-.r = (2r + 2) (2n - 2r - 1) \ . \,. It follows at once by
induction that

 n-2r -

2.4... 2/- . 2n - 1) (271 - 3) ... (2n - 2r + 1)' and the formula

Gn (x, y, z) =

I (-YD"-'-

Hn x, y, z)

,.=o2.4...2r.(2n-l)(2n-:3)...(2n-2r + l) is now obvious when Gn x, y,
z) is an ellipsoidal harmonic of the first species.

Example 1. Prove Niven's fonmila when G (.r, y, z) is an ellipsoidal
harmonic of the second, third or fourth species.

Example 2. Obtain the symbolic formula

G x, y, 2) = r(i- ). (iZ))"+ /\ \ (Z)). (.r, y, z).

2326. Ellipsoidal harmonics of degree n.

The results obtained and stated in §§ 23"21-23"24 shew that when n is
even, there are n + 1 harmonics of the first species and |/i harmonics
of the third species; when 7i is odd there are |(?i + l) harmonics of
the second species and n -1) harmonics of the fourth species, so that,
in either case, there are 2n + 1 harmonics in all. It follows from §
18*3 that, if the terms of degree n in these harmonics are linearly
independent, they form a funda- mental system of harmonics of degree n
; and any homogeneous harmonic of degree n is expressible as a linear
combination of the homogeneous harmonics which are obtained by
selecting the terms of degree n from the 2)i + 1 ellip- soidal
harmonics.

In order to prove the results concerning the number of harmonics of
degree n and to establish their linear independence, it is necessary
to make an intensive study of Lame's equation; but before we pursue
this investigation we shall study the construction of ellipsoidal
harmonics in terms of confocal coordinates.

23-26, 23-3]

ELLIPSOIDAL HARMONICS

547

These expressions for ellipsoidal harmonics are of historical
importance in view of Lame's investigations, but the expressions which
have just been obtained by Niven's method are, in some respects, more
suitable for physical applications.

For applications of ellipsoidal harmonics to the investigation of the
Figure of the Earth, and for the reduction of the harmonics to forms
adapted for numerical computation, the reader is referred to the
memoir by G. H. Darwin, Phil. Trans. 197 a (1901), pp. 461-537.

23"3. Confocal coordinates.

If X, Y, Z) denote current coordinates in three-dimensional space, and
if a, b, c are positive (a > 6 > c), the equation

X' Y' Z,

- +7 + = 1 a- 0- c-

represents an ellipsoid; the equation of any confocal quadric is

X2 7-2 2

,+

b' + 0' c + e '

a' + e

and 6 is called the pcuximeter of this quadric.

The quadric passes through a particular point x, y, z) if 6 is chosen
so that

+ -

y-

+

= ].

ar + d ¥ + 6 c- + e Whether 6 satisfies this equation or not, it is
convenient to write

  yl\ z' \ /( )

62 + (9

1-

a--ve ¥ + e c'' + e~ a' + d) (6 + 0) (c- + 6>) '

and, since f d) is a cubic function of 6, it is clear that, in
general, three quadrics of the confocal system pass through any
particular point (x, y, z).

To determine the species of these three quadrics, we construct the
following Table :

e

f \&)

- 00

- 00

-d

-.r2(a2-62)(a2.

-C2)

-62

y2(a2-62)(52.

-C2)

- C

-22(a2-c2) (62-

-C2)

+ 00

+ 00

It is evident from this Table that the equation y*( ) = has three real
roots X,, /i, V, and if they are arranged so that ix>v, then

 -C'>fi> - b->v > - a'-; and also /( ) = ( \ X)( - )((9 - i').

From the values of X, /x, v it is clear that the surfaces, on which
has the respective values X, /z, v, are an ellipsoid, an hyperboloid
of one sheet and an hyperboloid of two sheets.

35-2

548 THE TRANSCENDENTAL FUNCTIONS [CHAP. XXIII

Now take the identity in 6,

\ a? \ j/ \ 2' e- ) e-ii ) e-v)

ar- e b-'+tl C--+6 a' + e) b"- + d) c- + e)' and multiply it, in turn,
by a- + 6, h- + 0, c- + 6; and after so doing, replace 6 by - (('-, -
h-, - c- respectively. It is thus found that

, \ (a- + ) (a + fj.) (a- + v)

, \ (6 + X) (6 + 6- + lO ' '(a--6-)( '-c-) '

  \ (c- X ) (c- + fj,)(c- + v) '~ (a -c2)(6--c=) *

From these equations it is clear that, if (a-, y, z) be any point of
space and if X, /Lt, V denote the parameters of the quadrics confocal
with

X'- Y-' Z-,

- + 1T + - = 1 a- 0 c-

which pass through the point, then (x-, y", z-) are uniquely
determinate in terms of (X, fj,, v) and vice versa.

The parameters (X, fi, v) are called the confocal coordinates of the
point x, y, z) relative to the fundamental ellipsoid

X' Y- Z'-,

a- b- c- It is easy to shew that confocal coordinates form an
orthogonal system; for consider the direction cosines of the tangent
to the curve of intersection of the surfaces (/a) and (v); these
direction cosines are proportional to

/dx dy dz \ d dX' dx

and smce - -- + 4.\ -- = % \ - -\ - \ =

dx dx dy dy dz dz \ a + v

dXdJi Xd l' dXdJjL aXc (a -6')(a2- c ) it is evident that the
directions

fdx dy dz\ /dx dy dz\

Vax' dx' dx)' \ d ' Yfi' dfx)

are perpendicular; and, similarly, each of these directions is
perpendicular to

/dx dy dz\

\ dv ' dv ' dv) ' It has therefore been shewn that the three systems
of surfaces, on which X, jji, V respectively are constant, form a
triply orthogonal system. Hence the square of the line-element, namely

 hxy- hjr + hz)\

is expressible in the form

23-31] ELLIPSOIDAL HARMONICS 549

with similar expressions in /j, and v for H. and H3'. To evaluate Hi-
in terras of (X, u, y), observe that

' " 4 va y " 4 / Ux-y 4 \ dx

 x s; (g + fx) a? + i )

But, if we express

(A, - /Lt) (X, - v)

(a + X) (6 + X) (c" + X) '

qua function of X, as a sum of partial fractions, we see that it is
precisely equal to

  (g + fi) (g" + v)

 .5,c(g' + X)(g2-6 )(a -cO'.

and consequently H - = . -r . il - z-A c •

  ' 4 (a2 + X) (62 + X) (c + X)

The values of H. and Hf are obtained from this expression by cyclical
interchanges of (X, /i, i').

Formulae equivalent to those of this section were obtained by Lame,
Journal de Math. II. (1837), pp. 147-183.

Example 1. With the notation of this section, shew that

.r2+/ + 5- = a2 + 62 + c2 + X + jtx + i'.

Example 2. Shew that

ATTI- - I y" I

   a;' + \ f h + \ f c'' + \ f'

23 '31. Uniformising variables associated ivith confocal coordinates.

It has been seen in § 23'3 that when the Cartesian coordinates (x, y,
z) are expressed in terms of the confocal coordinates (X, fx, v), the
expressions so obtained are not one-valued functions of (X, /x, v). To
avoid the inconvenience thereby produced, we express (X, /i, v) in
terms of three new variables u, v, w) respectively by writing

HP (u) = X + |(a- + 6 + C-),

 j(v) =/i +!( ' + ' + c'),

'i'w) = v \ \ {d + h'"' + c ),

the invariants g.2 and g of the Weierstrassian elliptic functions
being defined by the identity

4>(a-+X) b' + ) c'+X) f u)-g u)-g,.

550 THE TRANSCENDENTAL FUNCTIONS [CHAP. XXIII

The discriminant associated with the L-lHptic functions (cf. § 20-33,

example 3) is

16 a- - b-Y b"- - c'Y- (c- - a'Y,

and so it is positive; and, therefore*, of the periods 2a),, 2\&)o
and 2\&)3, 2g)i is positive while 2a)3 is a pure imaginary; and 2a)2
has its real part negative, since Wj + \&).,+ 0)3 =; the imaginary
part of Wo is positive since / ((o.J(o ) > 0.

In these circumstances e, >e.2>e:;, and so we have

3e, = a- + b-- '2c-, Se = c" + a- - 2b-, 8e, = b- + c- - 2a-.

Next we express (x, y, z) in terms of u, v, w); we have

 a -b ) a -c-')

   f (m) - e,] \ \ (v) - 63 g> (w ) - eal (gj - es) bo - e )

\ a - (u) ( T3- (V) 0-3- (tv) q-- (6)1) 0-- ( O.)

~ o- (it) cr v) a- w) ' 0-3- (a)i) 0-3- (td.) '

by § 20-53, example 4. Therefore, by § 20'421, we have

x.- -. "/ 0-3 (") 0-3 (t') o-s (w) o- (w) o- (t;) o- w)

a-2 (w) (7o (v) cTa (w) and similarly 3/ = ± e- -a-(a,,) (,,) - ( ) )

X -T,co "/ o l(")o-l(y)o l(w) \ 4. g ').'"i cr- ( w ) - -- - --- - - .
- a u) a- (v) a (w)

The effect of increasing each of u, v, w by 2(0 is to change the sign
of the expression given for x while the expressions for y and z remain
unaltered; and simila'r statements hold for increases by 2\&).\ . and
2\&)i; and again each of the three expressions is changed in sign by
changing the signs of u, v, iv.

Hence, if the upper signs be taken in the ambiguities, there is a
unique correspondence between all sets of values of x, y, z), real or
complex, and all the sets of values of (it, v, w) whose three
representative points lie in any given cell.

The uniformisation is consequently effected by taking

., ., o-:, u) 0-3 v) 0-3 (w) cr ii) a v) a (w)

-.u. 0/, o-2 (t/-)o-.,(t;)o-,(w) ' a- a) a v) cr w)

(t u) (T v)cr w)

Formulae which differ from these only by the interchange of the
suffixes- 1 and 3 were given by Halphen, Fonctions Elliptiques, li.
(1888), p. 459.

• Cf. § 20-32, example 1.

23-32]

ELLIPSOIDAL HARMONICS

551

23'32. Laplace's equation referred to confocal coordinates.

It has been shewn by Lame and by W. Thomson* that Laplace's equation
when referred to any system of orthogonal coordinates (X, /j., v)
assumes the form

0,

H, dv where H, H.2, H- are to be determined from the consideration
that

is to be the square of the line-element. Although W. Thomson's proof
of this result, based on arguments of a physical character, is
extremely simple, all the analytical proofs are extremely long and
cumbrous.

It has, however, been shewn by Lamef that, in the special -case in
which (X, /JL, v) represent confocal coordinates, Laplace's equation
assumes a simple form obtainable without elaborate analysis; when the
uniformising variables (u, V, w) of § 23"31 are adopted as
coordinates, the form of Laplace's equation becomes still simpler.

By straightforward differentiation it may be proved that, when any
three independent functions (X, /n, v) of x, y, z) are taken as
independent variables, then

a F a F a F

transforms into

dx- dy

+

dz-"-

t + 2 S

yax

\ dx

J<1

+

dz. d/Ji dv

dfx dv d/ji dv

dx dx dy dy dz dzj d/jidv

d \ d \ d'\

d\ V dx'

A.M. V \ \ x- dy- dz \ In order to reduce this expression, we 'observe
that X satisfies the equation

X- y z-

oMO. "*" PTX " cM

= 1,

and so, by differentiation with x, y, z as independent variables, 2x x
y • 1 ' \ A

a- +X

14a;

,+

+

(a + xy (6- + xy (c + x)- j dx

dx

+ 2

a + X (a + X)- dx a + X)

+

y.

X-

(6 + X)=

y

+

+

(c + X)

+

(a2 + X)- h- + \ y (c + X)*- ] dx

d'X

= 0.

* Cf. the footnote on p. 401.

t Journal de Math. iv. (1839), pp. 133-136.

552 THE TRANSCENDENTAL FUNCTIONS • [CHAP. XXIII

Hence ~- =4<Hi,

a.' + \ da;

a' + X (a-' + \ y //j- 2 1 a' + xy /,7, v (a- + Xf ' d.v' '

\ a, b, c)

with similar equations in fx, v and y, z.

d-V . From equations of the first type it is seen that the coefficient
of - - is

1 d-V . . . .

V5r- and the coefficient of tt-t is zero; and if we add up equations
of the

second type obtained by interchanging x, y, z cyclically, it is found
that

[dx- dy- 0Z-] o, i.c " + with similar equations in [x and v. If, for
brevity, we write

 /[ a + X) h- + ) c' -X)] = K, ' with similar meanings for A and A,
we see that

a;- " a " a - (x - /i) (x - v) 1 a- + x 6- +'x " c + X

4Ax c Ax

(X - /x) (X - I/) rfX ' and so Laplace's equation assumes the form

x.T", >' (' - A*) ( ~ ) L ' dx ax

that is to say

= 0,

(' - ) a-xWI ]+< - ) 'IKa + -'') ' .K'5 = -

The equivalent equation with u, v, lu) as independent variables is
simply or, more briefly,

[i j v) - i w)] 1 + w) - in)] + [iO u) - v)] = 0,

 - >a + < - >-a + - >a = '-

The last three equations will be regarded as canonical forms of
Laplace's equation in the subsequent analysis.

23*33. Ellipsoidal harmonics referred to confocal coordinates. When
Niven's function 0p, defined as

a- + 6, J b'' + 6,, c- +

p p p

23*33] ELLIPSOIDAL HARMONICS 553

is expressed in terms of the confocal coordinates \ [x, v) of the
point x, y, z), it assumes the form

  e ) t,-ej) v-dp )

and consequently, when constant factors of the form

- ( + e,,) (¥ + dp) (c + Op)

are omitted, ellij)soidal harmonics assume the form X, yz \

I m m m

1, y, zx, xyz I n (X - ) H (/i-6'p) H v-B ).

I p = l 73 = 1 p = \

z, xy ]

If now we replace x, y, z b ' their values in terms of X, /*, v, w e
see that any ellipsoidal harmonic is expressible in the form of a
constant multiple of AMN, where A is a function of \ only, and M and N
are the same functions of /Li, and V respectively as A is of X.
Further A is a polynomial of degree nn in \ multiplied, in the case of
harmonics of the second, third or fourth species, by one, two or three
of the expressions \/ a' + X), \/ b- + X), \/(c + )-

m

Since the polynomial involved in A is H (X - 6p), it follows from a
con-

i3 = l

4 si [ a- + X) ¥ + X ) (c + X)

V((a- + X)(6 + X)(c + X)

sideration of §§ 23'21-23-24 that A is a solution of Lame's
differential equation

X

= [n n 4-l)X+C' A, where n is the degree of the harmonic in x, y, z).

This result may also be attained from a consideration of solutions of
Laplace's equation which are of the type*

V= AMN, where A, M, N are functions only of X, /x, v respectively.

For if we substitute this expression in Laplace's equation, as
transformed in § 23'32, on division by V, we find that

  v)-io w) d K ( w) - J u) d'M 0 (u) - ipO ) d'N A dii M dv N " div-

The last two terms, qua functions of u, are linear functions of ip
(u), and

1 d A

so -r- -y 2 must be a linear function of o(u); since it is independent
of the

coordinates v and w, we have

where K and B are constants.

* A harmonic which is the product of three functions, each of which
depends on one coordi- nate only, is sometimes called a normal
solution of Laplace's equation. Thus normal solutions with polar
coordinates are (§ 18-31|

r" P ' (cos d) ° m(p. " ' sin

554 THE TRANSCENDENTAL FUNCTIONS [CHAP. XXIII

If we make this substitution in the differential equation, we get a
linear function of (,) (k) equated (identically) to zero, and so the
coefficients in this linear function must vanish; that is to say

m,. . s) 1 d-M 1 rf'N

and on solving these with the observation that (v) - j J (lu) is not
identically zero, we obtain the three equations

When X. is taken as independent variable, the first equation becomes

4 A. |a. I = [K\ - i? + i/i (a + ¥ + c )] A,

and this is the equation already obtained for A, the degree n of the
harmonic being given by the formula

n (n + l) = K.

We have now progressed so far with the study of ellipsoidal harmonics
as is convenient without making use of properties of Lame's equation.

We now proceed to the detailed consideration of this equation.

234. Various forms of Lame s differential equation.

We have already encountered two forms of Lame's equation, namely

and this may also be written

d \ a" + \ h- + X c- +X d\ ~ 4 (a- + X) b- + X) (c- + X) ' which may
be termed the algebraic form; and

  = n n + l) j(ii) + B]A,

which, since it contains the Weierstrassian elliptic function (u), may
be termed the Weierstrassian form; the constants B and C are
connected by the relation

B + in ( + 1) (a- + b- + c-) = C.

23 "4] lame's equation 555

If we take j (u) as a new variable, which will be called, we obtain
the slightly modified algebraic form (c£ § 10"6)

This differential equation has singularities at gj, eo, e at which the
exponents are 0, in each case; and a singularity at infinity, at
which the exponents are - n,\ \ {n + ) .

The "Weierstrassian form of the equation has been studied by Halphen,
Fonctions Elliptiques, ir. (Paris, 1888), pp. 457-531.

The algebraic forms have been studied by Stieltjes, Acta Math. vi.
(1885), y>V. 321-326, Klein, Vorlesimgen iiber lineare
Diferentielgleichunge/i lithogr\&iAiGd, Gottingen, 1894), and Bocher,
l/ber die Reihenentioickelungen der Potentialtheorie (Leipzig, 1894).

The more general differential equation with foiu* arbitrary
singularities at which the exponents are arbitrary (save that the sum
of all the exponents at all the singularities is 2) has been discussed
by Heun, Math. Ann. xsxiii. (1889), pp. 161-179; the gain in
generality by taking the singularities arbitrary is only apparent,
because by a homographic change of the independent variable one of
them can be transferred to the point at infinity, and then a change of
origin is sufficient to make the sum of the complex coordinates of the
three finite singularities equal to zero.

Another important form of Lame's equation is obtained by using the
notation of Jacobian elliptic functions; if we write

Zl = U V(ei - 63),

the Weierstrassian form becomes

dzi"

n n + 1) I - ' 1 1 -

4-ns2 iV +

€ > e,-e

A,

and putting 2 = a - iK', w here 2iK' is the imaginary period of sn z-
, we obtain the simple form

- i- = \ n (n + 1) k- sn-a + A] A, doP / >

where A is a constant connected with B by the relation

B + e-iii (n + 1) = A e,- e-,).

The Jacobian form has been studied by Hermits, Sur quelques
applications des fonctions elliptiques, Comptes Rendns, Lxxxv. (1877),
published separately, Paris, 1885.

In studying the properties of Lame's equation, it is best not to use
one form only, but to take the form best fitted for the purpose in
hand. For practical applications the Jacobian form, leading to the
Theta functions, is the most suitable. For obtaining the properties of
the solutions of the equation, the best form to use is, in general,
the second algebraic form, though in some problems analysis is simpler
with the Weierstrassian form.

556 THE TRANSCENDENTAL FUNCTIONS [cHAP. XXIII

23'41. Solittioiis in series of Lame s equation.

Let us now assume a solution of Lamp's equation, which may be written

in the form

A= I 6.(1-,) "-''.

)-=0

The series on the right, if it is a solution, will converge (§ 10"31)
for sufficiently small values of \ \ - e.,, but our object will be
not the discussion of the convergence but the choice of B in such a w
ay that the series may terminate, so that considerations of
convergence will be superfluous.

The result of substituting this series for A on the left-hand side of
the differential equation and arranging the result in powers of - go
is minus the series

4 i ( -e.,) ' -r + i[r(n-r-[-hJbr- Se,( n-r + ir--in n+l)e,-lB]br-

r=--0

+ (e, - e,) (e, - e,) ( w - r + 2) (i n - r + f ) 6,-2],

in which the coefficients hj. with negative suffixes are to be taken
to be zero.

Hence, if the series is to be a solution, the relation connecting
successive coefficients is

r (n -r + )br = [Se ( n - r + 1)- - i ?i (n +l)eo-lB\ br-i

- (ei - e ) ( 2 - 63) i n - r + 2) n - r + f) br-.,

and (n - i) 61 = n% -ln n + l)e,- B b,.

If we take \& = 1, as we may do without loss of generality, the
coefficients bj. are seen to be functions of B with the following
properties :

(i) bj. is a polynomial in B of degree r.

(ii) The sign of the coefficient of B'' in br is that of (-)'",
provided that ?' :\$ n; the actual coefficient of B is

i-Y

2.4>...2r(2n-l) 2n-3)... ( 2n - 2r + 1) '

(iii) If 1, e., e-i and B are real and i > 62 > 3, then, if b -i = 0,
the values of 6,. and 6,.\ o are opposite in sign, provided that r < n
+ 'A) and r < n.

Now suppose that n is even and that we choose B in such a way that If
this choice is made, the recurrence formula shews that

hn 2 = 0,

23-41] lame's equation 557

by putting r = 7i + 2 in the formula in question; and if both 6i,j\ i
and 1)1 + 2 6 subsequent recurrence formulae are satisfied hy taking

 1/1 + 3 = i + 4 = ... =0.

Hence the condition that Lame's equation should have a solution which
is a polynomial in is that B should be a root of a certain algebraic
equation of degree n + 1, when n is even.

When n is odd, Ave take 6x j i\ to vanish and then 6i/ 3x also
vanishes, and so do the subsequent coefficients; so that the
condition, when n is odd, is that B should be a root of a certain
algebraic equation of degree \ (n + 1).

It is easy to shew that, when e > e., > e-, these algebraic equations
have all their roots real. For the properties (ii) and (iii) shew
that, qua functions of B, the expressions h, b, b, ... b,. form a
set of Sturm's functions* when r < (n + 3), and so the equation

has all its roots realf and unequal.

Hence, when the constants gj, e,, es are real (which is the case of
practical importance, as was seen in § 23'31), there are ?i + 1 real
and distinct values of B for which Lame's equation has a solution of
the type

tbr( -e -'

Avhen n is even; and there are (n+1) real and distinct values of B for
which Lame's equation has a solution of the type

when n is odd.

When the constants ej, eo, e are not all real, it is possible for the
equation satisfied by B to have equal roots; the solutions of Lame's
equation in such cases have been discussed by Cohn in a Konigsberg
dissertation (1888).

Example 1. Discuss solutions of Lame's equation of the types \ (i) ( -
iF i h; -e. -'--, ' .

r=0

*?l-r-J

(ii) ( -e,) 2 V'd-eo) (iii) ( -e,) -e,)i I b;" -e,) "-'-\

* Mem. pr senUs par les Savans Etrangers, vi. (1835), pp. 271-318.

t This procedure is due to Liouville, Journal de Math. xi. (1846), p.
221.

558 THE TRANSCENDENTAL FUNCTIONS [CHAP. XXIII

ohtainiug the recurrence relations

(i) r n -r + h) W =- Se.-, hi - r + hf + ie -e ) h u - r + |) - n (n +
1) 63 - i 6',.-i

- (ei - 62) ( 2 - 63) hi -r + %) hi -r + l) 6V-2, (ii) r n-r + h)b," =
3e., hn-r+hf- ei-e2) in-r + )- 7i 7i+l)e.,-iB\ b"r\,

- ei- 62) 62-63) n - r+ ) hi-r + l) b",.\ .i, (iii) r n-r + h)b;" =
3e2 n-r + h)--ke2 n' + n + l)- B b"'r\ i

- ( 1 - eo) (62 - e-i) ih)i-r+l) l n -r + h) b". \ 2 . Example 2. With
the notation of example 1 shew that the numbers of real distinct

values of B for which Lamp's equation is satisfied by terminating
series of the several species are

(i) \ \ {n- ) or \ \ {n-2); (ii) \ \ {n- ) ov h n-9.); (iii) n-2) or h
n-Z).

2342. The definition of Lame functions.

When we collect the results which have been obtained in § 23*41, it is
clear that, given the equation

n being a positive integer, there are 2n + 1 values of B for which the
equation has a solution of one or other of the four species described
in §§ 23"21-23'24.

If, when such a solution is expanded in descending powers of, the
coefficient of the leading term | " is taken to be unity, as was done
in § 2341, the function so obtained is called a Lame function of
degree n, of the first kind, of the first (second, third or fourth)
species. The 2n + 1 functions so obtained are denoted by the symbol

E, - ); (m = l, 2, ...2 + l).

and, when we have to deal with only one such function, it may be
denoted by the symbol

Tables of the expressions representing Lame functions for ?i = l, 2,
...10 have been compiled by Guerritore, Giornale di Mat. (2) xvi.
(1909), pp. 164-172.

Example 1. Obtain the five Lame functions of degree 2, namely

V(X + i-)v/(X + c2), v/(X + c2) /(X + a2), v'(X + -)V(X + n Example 2.
Obtain the seven Lame functions of degree 3, namely - V (X + a' )(X +
\&'')(X4-c2), and six functions obtained by interchanges of a, b, c
in the expressions

v'(X + a-') . [X + 1 (a2 + 262 + .2c') ± I J a* + ib* + 4c* - 76V-' -
f' a \ a'b ].

23'43. 27te non-repetition of factors in Lame functions.

It will now be shewn that all the rational linear factors of "' (|)
are unequal. This result follows most simply from the differential
equation which - n™ (?) satisfies; for, if | - ?i be any factor of
'"(|), where |i is not one of

23-42-23-44] lame's equation 559

the numbers gj, eo or 63, then fj is a regular point of the equation
(§ 10"H), and any solution of the equation which, when expanded in
powers of f - 1, does not begin with a term in ( - 1)° or ( - Y must
be identically zero.

Again, if |i were one of the numbers gj, 63 or s. the indicial
equation appropriate to 1 would have the roots and J, and so the
expansion of En"' ( ) in ascending powers of 1 would begin with a term
in (f - j)" or

Hence, in no circumstances has,i'"( ), g'wa function of, a repeated
factor.

The determination of the numbers 0, 6.2, ... 0 introduced in §§
23-21- 23*24 may now be regarded as complete; for it has been seen
that solutions of Lame's equation can be constructed with non-repeated
factors, and the values of 0, 62, ... which correspond to the roots
of En ( ) = satisfy the equations which are requisite to ensure that
Niven's products are solutions of Laplace's equation.

It still remains to be shewn that the '2n + 1 ellipsoidal harmonics
con- structed in this way form a fundamental system of solutions of
degree 71 of Laplace's equation.

23'44. The linear independence of Lame functions.

It will now be shewn that the 2n -i- 1 Lame functions E)/ ) which are
of degree n are linearly independent, that is to say that no linear
relation can exist which connects them identically for general values
of .

In the first j)lace, if such a linear relation existed in which
functions of different species were involved, it is obvious that by
suitable changes of signs of the radicals \/( - i), \/( - 2), \/(f -
s) we could obtain other relations which, on being combined by
addition or subtraction with the original relation, would give rise to
two (or more) linear relations each of which involved functions
restricted not merely to be of the same species but also of the same
type.

Let one of these latter relations, if it exists, be

and let this relation involve r of the functions.

Operate on this identity 7 - 1 times with the operator

The results of the successive operations are

ta,, Bn'-y Err )=0 (5 = 1, 2, ... r - 1), where Bn' is the particular
value of jB which is associated with E "- ).

560 THE TRANSCENDENTAL FUNCTIONS [CHAP. XXIII

Eliminate a-, iu, ... a,, from the r equations now obtained; and it
is found

that

1, 1, 1, ... 1 =0.

! R 1 R i R 3 Tl r

 Dii, J->n, -'-'71 > ••• -'-'(!

 B, y-\ B,;r-\ (B/y-'

Now the only factors of the determinant on the left are differences of
the numbers 5 '", and these differences cannot vanish, by § 23"41.
Hence the determinant cannot vanish and so the postulated relation
does not exist.

The linear independence of the 2m +1 Lame functions of degree n is
therefore established.

2345. The linear independence of ellipsoidal harmonics.

Let Gn', y, z) be the ellipsoidal harmonic of degree n associated
with E, ( ), and let T,/" (cc, y, z) be the corresponding homogeneous
harmonic.

It is now easy to shew that not only are the 2?? + 1 harmonics of the
type Gn" (, y, z) linearly independent, but also the 2?; + 1
harmonics of the type Hn'' x, y, z) are linearly independent.

In the first place, if a linear relation existed between harmonics of
the type Gn'" oc, y, z). then, when we expressed these harmonics in
terms of con- focal coordinates X, /x, v), we should obtain a linear
relation between Lame functions of the type En' i ) where =X+ a- + b-+
c"), and it has been seen that no such relation exists.

Again, if a linear relation existed between homogeneous harmonics of
the type Hn ' x, y, z), by operating on the relation with Niven's
operator

(§ 23-25),

  D\

2 (2n - T) " 2.4 (2 - 1 ) 2n- 3) ' " '

we should obtain a linear relation connecting functions of the type
(z,/'' x, y, z), and since it has just been seen that no such relation
exists, it follows that the homogeneous harmonics of degree n are
linearly independent.

2346. Stieltjes' theorem on the zeros of Lame functions.

It has been seen that any Lam6 function of degree n is expressible in
the form

m

(6 + a'Y' (d + h- r (0 + cy .u d- dp),

;> = 1

where i, k, k- are equal to or and the numbers 6, 6., ... 6, are
real and unequal both to each other and to - a, - b, - c-; and n =
ni + Ki + Kn + k . When /cj, /Co, Ks are given the number of Lame
functions of this degi-ee and type is ru + 1.

23*45, 23*46] lame's equation 561

The remai'kable result has been proved by Stieltjes* that these w + 1
functions can be arranged in order in such a way that the rth function
of the set has r - 1 of its zeros f between - a and - h- and the
remaining m - r + 1 of its eros between - 6- and - c and,
incidentally, that, for all the 7?i + 1 functions, 6-, 6.2, ... 0 lie
between - a- and - c-.

To prove this result, let 0j, 2; ••• 4 m be any real variables such
that

(- a- (jip - h, (p - 1,2,...') - 1)

[-h" ( )p - c (p = r,r+l,... m) and consider the product

 ' J- 4-1 a.1

11 = n [| ((/>p +aO T'+i . i (</),. + h"-) r -'i . j (< + c ) i;3+4] n
] (c/> - <,) |.

p=l pdpq

This product is zero when all the variables 0 have their least values
and also when all have their greatest values; when the variables p
are unequal both to each other and to - a-, - If, - c", then IT is
positive and it is obviously a continuous bounded function of the
variables.

Hence there is a set of values of the variables for which 11 attains
its upper bound, which is positive and not zero (cf § 3"62).

For this set of values of the variables the conditions for a maximum
give

eiogn \ a log n \

that is to say

111

fCi + -r Ko + T ' •3 + T

4 \ 4 4 ', 1 \

<f)p + a <f>p + 6 cf)p + c- fjli <j>p - (j),i

where p assumes in turn the values 1, 2, ... m.

Now this system of equations is precisely the system by which 6, 6,
... 6 are determined (cf |§ 23*21-23'24); and so the system of
equations determining Oi, 02, ... dm has a solution for which

j-a'<dp<-b% (p = l,2, ...r-1)

 - b"< dp < - C-. (jo = ?',?•+ 1, ... to)

Hence, if r has any of the values 1, 2, ... m + 1, a Lame function
exists with r - 1 of its zeros between - a and - h" and the remaining
vi - ?• + 1 zeros between - h- and - c'-.

Since there are m + 1 Lame functions of the specified type, they are
all obtained when r is given in turn the values 1, 2, ... to + 1; and
this is the theorem due to Stieltjes.

* Acta Mathematica, \ i. (1885), pp. 321-326.

t The zeros -a', -b, - c are to be omitted from this enumeration, 6
, d., ..., only being taken into account.

W. M. A. 36

562 THE TRANSCENDENTAL FUNCTIONS [CHAP. XXIII

An interesting statical interpretation of the theorem was given by
Stieltjcs, namely that if wi +3 particles which attract one another
according to the law of the inverse distance

are placed on a line, and three of these particles, whose masses are
Ki + -r, f2+ > ''s + ti '

fixed at points with coordinates -a-, -b-, -c, the remainder being of
unit mass and free to move on the line, then log n is the
gravitational potential of the system; and the positions of
equilibrium of the system are those in which the coordinates of the
moveable particles are 61,6-2, ... 6, i-e. the values of 6 for which
a certain one of the Lame functions of degree 2 m + ki + K2 + <z)
vanishes.

Example. Discuss the positions of the zeros of polynomials which
satisfy an equation of the type

d6' "* rti e-a, d6 * '•,, ~ '

n 6-a,)

s = l

where (f)r-2 6) is a polynomial of degree r - 2 in 6 in which the
coefficient of 6 ~- is

r

- m m + r - 1 - 2 a, .-•=1

m being a positive integer, and the remaining coefficients in <f>r-2
6) are determined from

the consideration that the equation has a polynomial solution.

(Stieltjes.)

23'47. Lame functions of the second kind.

The functions En ( ), hitherto discussed, are known as Lame functions
of the first kind. It is easy to verify that an independent solution
of Lame's equation

du- '

is the function i,i'" ( ) defined by the equation*

/ .. (f) = (2 H-l) '"(f)/;jj;,

and i,i'" (1 ) is termed a Lam6 function of the second kind. From
this formula it is clear that, near u = 0,

F ( ) = (271 + 1) u- 1 + (lOl I " u-'' 1+0 u)] du = u''+' 1 + (u)],

Jo and we obviously have

E,r ) = u-[l + 0 u)].

It is clear from these results that Fn" (|) can never be a Lame
function of the first kind, and so there is no value of Bn" for luhich
Lame's equation is satisfied by two Lame functions of the first kind
of different species or types.

It is possible to obtain an expression for Fn'\ \ ) which is free from
quadratures, analogous to Christoffel's formula for Qni ), given on p.
333, example 29. We shall give the analysis in the case when En" (|)
is of the first species. The only irreducible poles of l/,j"'( )j qua
function of u, are at a set of points u, u.,, ... Un which are none
of them periods or half periods.

* This definitiou of the function i<' '" (4) is due to Heine, Journal
fUr Math. xxix. (1845), p. 194.

23-47, 23-5] lame's equation 563

Near any one of these points we have an expansion of the form

En'" (I) = /.-i U - Ur) + h (W - Urf + '3 u - U,) + ...,

and, by substitution of this series in the differential equation, it
is found that 2 is zero.

Hence the principal part of l/ £'n"* ) near iir is

1

k]- (u - Uff ' and the residue is zero.

Hence we can find constants Ar such that

 E,r( yr- I Ario(u-u,.)

r = l

has no poles at any points congruent to any of the points iif; it is
therefore a constant A, by Liouville's theorem, since it is a doubly
periodic function of M.

fit dn n

Now the points Uf can be grouped in pairs whose sum is zero, since Ey
(I) is an even function of u.

If we take iin-,- = - '';+i, we have

--~--: = Au- 2 Ar u-Ur) + K u + U>)] Jo i n g)i r = l

= AlC - 2 u) t Ar-t - ff' ''],

r=l r = lip y-)- Ur)

and therefore

?• = !

where '?< i,j\ i(| ) is a polj nomial in | of degree n - 1.

Example. Obtain formulae analogous to this expression for i, ™ ) when
E, ( ) is of the second, third or fourth species.

23'5. Lame's equation in association tvith Jacobian elliptic
functions.

All the results \ Yhich have so far been obtained in connexion with
Lame functions of course have their analogues in the notation of
Jacobian elliptic functions, and, in the hands of Hermite (cf. § 23*7
1), the use of Jacobian elliptic functions in the discussion of
generalisations of Lame's equation has produced extremely interesting
results.

Unfortunately it is not possible to use Jacobian elliptic functions in
which all the variables involved are real, without a loss of symmetry.

36-2

564

THE TRANSCENDENTAL FUNCTIONS [CHAP. XXIII

and then the formulae of

X =

The symmetrical formulae may be obtained by taking new variables a,
/3, . 7 defined by the equations

fa = iK' + u V(ei - e- ),

1 7 = iK' + w V(ei - e-,), 23"31 are equivalent to A;- \/ a~ - c") .
sn a sn /3 sn 7, y - - (f -/k') V(ft" - C-) . en a en en 7, \ z =
ijk') \/(a- - C-) . dn a dn /8 dn 7, the modulus of the elliptic
functions being

V W-cV-

The equation of the quadric of the confocal system on which a is con-
stant is

X' Y Z-

(a2-62)sn2a (a - b') cn a ~ (a - c-) dn" a ~ " This is an ellipsoid if
a lies between iK' and K + iK'; the quadric on which yS is constant
is an hyperboloid of one sheet if lies between K + iK' and -fir; and
the quadric on which 7 is constant is an hyperboloid of two sheets if
7 lies between and K; and with this determination of (a,, 7) the
point x, y, z) lies in the positive octant.

It has already been seen (§ 23*4) that, with this notation, Lame's
equation assumes the form

--T-v = n + 1 ) A,'- sn- a + A] A,

and the solutions expressible as periodic functions of a will be
called* -£ '" (a). The first species of Lame' function is then a
polynomial in sn a, and generally the species may be defined by a
scheme analogous to that of § 23*2, sn a, en a dn a,

1, en or, dnasna, sn a en a dn a dn a, sn a en a,,

n (sn- oc - sn-Qp). J

23'6. The integral equation satisfied by Lame functions of the first
and second species' .

We shall now shew that, if En' (a) is any Lamd function of the first
species (n being even) or of the second species (n being odd) with sn
a as a

* There is no risk of confusing these with the corresponding functions
i-'/' (t).

t This integral equation and the corresponding formulae of § 28 -62
associated with ellipsoidal harmonics were given by Whittaker, Froc.
London Math.- Soc. (2) xiv. (1915), pp. 260-268. Proofs of the
formulae involving functions of the third and fourth species have not
been previously published.

23*6] lame's equation 565

factor, then E '"'(a) is a solution of the integral equation

E,r ( ) = X Pn (k sn a sn 6) E,r (0) dd;

J -2K

where \ is one of the 'characteristic numbers' (§ 11-23).

To establish this result we need the lemma that P,i (A: sn a sn ) is
annihilated by the partial differential operator

3 2 - 9 - " ( + 1) '' (sn' a - sn d).

To prove the lemma, observe that, when /x is written for brevity in
place of k sn a sn 0, we have

 a|- i ' '-'""'' >

= k- (en- a dn- a sn- 6 - en- 6 dn- 6 sn- a] Pn" (/i)

+ k" sn a sn (sn a - sn 0) P,,' (/x) = k' (sn a - sn 6) [(fj, - - 1 )
P,/' C/.) + 2/iP,/ ( )] = 2 (sn- a - sn 6) n (n + l) P (jj,),

when we use Legendre's differential equation (§ 16-13). And the lemma
is established.

The result of applying the operator

 2 - n (n + l)k"sn-oc-A '

to the integi'al

r2K

Pn (k sn a sn 6) E T (0) dO

J ~2K

is now seen to be

\ - -n n + l) k' sn a - .4,,' [ P (k sn a sn 6) E " (6) dO J -2K (ca
)

= \, \ \ .- n + ) k-s,n e-AyvPn k nasne) and when we integrate twice
by parts this becomes

E,r e)dd,

-B PAfcsnasnB) \ dE e)

2K \ -2K

+ j Pn ( n asn d) Uj,- n (n + 1) k' sn' d - A . [ En'"" d) .dd = 0.

Hence it follows that the integral riK

Pn (k sn a sn 6) En"" (0) d9

J -2K *

is annihilated by the operator

d

T- 2 - n n + l)k sn- a - J.,l"

566 THE TRANSCENDENTAL FUNCTIONS [CHAP. XXIII

and it is evidently a polynomial of degree n in sn- a. Since Lame's
equation has onlj' one integral of this type*, it follows that the
integral is a multiple of En ' (a) if it is not zero; and the result
is established.

It does not appear to have been proved that the only vahies of A, for
which the equation

f a) = \ [" J\ \ {kHnaKn6)f 6)dd

has a solution, are those which make/(o) a solution of Lame's
equation.

Example 1. Shew that the nucleus of an integral equation satisfied by
Lame functions of the first species n being even) or of the second
species n being odd) with en a as a factor, may be taken to be

/ • (-p en a en 6

Example 2. Shew that the nucleus of an integral equation satisfied by
Lame functions of the first species (?i being even) or of the second
species n being odd) with dn a as a factor, may be taken to be

P Ypdnadn y

23'61. Tlie integral equation satisfied by Lame functions of the third
and fourth species.

The theorem analogous to that of § 23'6, in the case of Lame functions
of the third and fourth species, is that any Lame function of the
fourth species (n being odd) or of the third species (n being even)
with en a dn a as a factor,, satisfies the integral equation r2K E,r
(a) = X en a dn a en 6 dn dPn' k sn a sn 6) E/'' (0) dO.

J -2 A'

The preliminary lemma is that the nucleus

en a dn a en d dn dPn' k sn a sn 6), like the nucleus of § 23"6, is
annihilated by the operator

TT-, - -, - n n + l) A'- (sn- a - sn- 6). dot- ctf-

To verify the lemma observe that

| JcnadnaP/(/ snasn )

= k- cn ' a dn'' a sn- PJ'' (/u.) - 3 - sn a en a dn a sn (dn- a + k'
en- a) Pn" (fi) - en a dn a (dn- a + k- en- a - 4k' sn" a) Pn" (fi),
and so

. \ I . (en a dn a en dn 6Pn k sn a sn 0)\

= kcnadnacne dn 6 (sn a - sn' 0) (,x'' - 1 ) PJ " (/i) -f GfxPn" ( ) +
GPn" (/*)]

d = k- en a dn a en dn (sn- a - sn- ) 3 (m-' - 1) Pv/ (a )

= k- n n + 1 ) en a dn o£ en 6 dn 6 (sn- a. - sn'- 6) Pn" (/a),

* The other solution when expanded in descending powers of sn a begins
with a term in (sn a)- - .

23 '61, 23 "62] ellipsoidal hahmonics 567

and the lemma is established. The proof that En (a) satisfies the
integral equation now follows precisely as in the case of the integral
equation of § 23"6.

Example 1. Shew that the nucleus of an integral equation which is
satisfied by Lame functions of the fourth species ( being odd) or of
the third species ii being even) with sn a dn a as a factor, may be
taken to be

sn a dn a sn 6 dn 6 P " ( -77 en a en ) .

Example 2. Shew that the nucleus of an integral equation which is
satisfied by Lame functions of the fourth species ii being odd) or of
the third species (/i being even) with sn a en a as a factor, may be
taken ta be

sn a en a sn 6 en 6P,l' j, dn a dn | .

Example 3. Obtain the following three integral equations satisfied by
Lame functions of the fourth species n being odd) and of the third
species (?i being even) :

(i) k" ..'- a E.r ( ) -X c,. dn / P. (i. ., sn ) jjJj '- ] M, (ii) -i
c =,.i.V'W = -.,n d / P,.(fcn c ) - / > o(,

(iii) I.-' in' a Er M = Xr- s a c / i>. Q. dn dn ) J ™ M;

in the case of functions of even ordei-, the functions of the
different types each satisfy one of these equations only.

23'62. Integral formulae fur ellipsoidal harmonics.

The integral equations just considered make it possible to obtain
elegant representations of the ellipsoidal harmonic Gn x, y, z) and of
the corre- sponding homogeneous harmonic H x, y, z) in terms of
definite integrals.

From the general equation formula of § 18"3, it is evident that Hn x,
y, z) is expressible in the form

Hn''' (x, y,z)= I (x cost + y sin t + izTf(t) dt,

J -TV

where y( ) is a periodic function to be determined.

Now the result of applying Niven's operator D- to x cos t + y sin t +
t )" is

n (n - 1 ) (a- cos"- 1 + b'- sin t - c'-) (x cos t + y sin t + izy ~,

and so, by Niven's formula (§ 23*25) we find that Gn" x, y, s) is
expressible in the form

n n-l)(n-2)(n-S) )

+ 2.4(2n-l)(2u-3) ' • / t)at,

568 THE TRANSCENDENTAL FUNCTIONS [CHAP. XXIII

where * l = xcost + y sin t + is,

  = V ( ' - C-) cos- t + b-- C-) sin t], so that

Now write sin = cd, the modulus of the elliptic functions being, as
usual, given by the equation

, \ tt" - b' a- - C-

The new limits of integration are - K and K, but they may be replaced
by - 'IK and 'IK on account of the periodicity of the integrand.

It is thus found that

 - T / 'a;sn + ycn + >dn \ .

e " ( . y, ) = j\ . i ( :j \ - j > ( ) ''<'.

where ( ( ) is a periodic function of 6, independent of x, y, z, which
is, as yet, to be determined.

If we express the ellipsoidal harmonic as the product of three Lame
functions, with the aid of the formulae of § 235 we find that

rlK

E (a) E,r (/3) E,r l)=G\ Pa (/x) < (0) dO,

J -2A"

where C is a known constant and

/x = k- sn a sn /3 sn 7 sn - k-/k'-) en a en /? en 7 en

- (l/'k'-) dn a dn y8 dn 7 dn 6.

If the ellipsoidal harmonic is of the first species or of the second
species and first type, we now give /3 and 7 the special values

  = K, y = K + iK', and we see that

C [' Pn(ksnasnd)(f3 e)(W

J -2 A'

is a solution of Lame's equation, and so, by § 23-6, ( ) is a solution
of Lame's equation which can be no other* than a multiple of E,i'"
(6).

Hence it follows that

l"- T. /'k'x sn0+ u en \$ + iz dn \,,, 7

where X is a constant.

* If (p e) involved the second solution, the integral would not
converge.

23-63]

ELLIPSOIDAL HARMONICS

569

If Gn!" x, y, z) be of the second species and of the second or third
type

we put

/3 = 0, 7 = iiT + iK',

or /3 = 0, 7 = /i

respectively, and we obtain anew the same formula.

It thus follows that if (th'" x, y, z) be any ellipsoidal harmonic of
the first or second species, then

GrT oo, y,z) = \ f "" Pn (/ ) E,i- 6) cie,

J -2 A'

  '(x, y, z) = X. ' r r ( ' sn + 3/ en 6? + iz dn df En''' (0) dd,

where [x = (k'x sn6 + y end + iz dn d)/\/ b- - c").

23'63. I nteg7'al formulae for ellipsoidal harmonics of the third and
fom th species.

In order to obtain integral expressions for harmonics of the third and
fourth species, we turn to the equation of § 23'62, namely

£,' (a) En'- (/8) En - (7) = C r Pn (yu) cf> (0) d0,

J -iK

where

fi = A; sn a sn /3 sn 7 sn - k-/k'-) en a en /3 en 7 en - (l/k'-) dn a
dn /S dn 7 dn;

this equation is satisfied by harmonics of any species.

Suppose now that £'/' (a) is of the fourth species or of the first
type of the third species so that it has en a dn a as a factor.

We next differentiate the equation with respect to /3 and 7, and then
put = K,y = K + iK'.

It is thus found that d

En' ict)

d/3

En'"( )

i.-=A-L 7

V = A'+iA"

= C'

2A

Now SO that

dPn(f )

\ 87 \

K L ?/3S7 .

(? = K, y = K+iK'

(0) d0.

y = A'+?A''

- ilk') dn a dn /S dn 0Pn (/x),

d dy J =K,y=K+iK

:')

= - en a dn a en dn 0Pn" (k sn a sn 0).

Hence

r

J -2A

en a dn a en dn 0Pn" (k sn a sn ) ( (0) d0

is a solution of Lame's equation with en a dn a as a factor; and so,
by § 23'61, (ji (0) can he none other than a constant multiple of.E -
(a).

570 THE TRANSCENDENTAL FUNCTIONS [CHAP. XXIII

We have thus found that the equation

G,r oc, y,z) = \ \ Pn (fi) E,r (0) (le

J -IK

is satisfied by any ellipsoidal harmonic which has en a dn a as a
factor; the corresponding formula for the homogeneous harmonic is

Hn'" (x, XL z) = \ J- I ( /. ' X sn e- xi en 6 + t>dn ) E. iO) dd.

Example. Shew that the equation of this section is satisfied by the
ellipsoidal harmonics which have sn a dn a or sn a en a as a factor.

23"7. Generalisations of Lame s equation.

Two obvious generalisations of Lame's equation at once suggest them-
selves. In the first, the constant B has net one of the characteristic
values Bn, for which a solution is expressible as an algebraic
function of f u); and in the second, the degree n is no longer
supposed to be an integer. The first generalisation has been fully
dealt with by Hermite* and Halphenf, but the only case of the second
which has received any attention is that in which n is half of an odd
integer; this has been discussed by Brioschij, Halphen§ and Crawford
II .

We shall now examine the solution of the equation

 \ = [n n + l) io u) - B] i\,

where B is arbitrary and n is a positive integer, by the method of
Lindemann- Stieltjes already explained in connexion with Mathieu's
equation (§§ 19'5- 19-52).

The product of any pair of solutions of this equation is a solution of

'~ ' ' + " " + S ~ -" ' + '' '' = ' by § 19"52. The algebraic form of
this equation is

4 ( - e,) (I - e,) ( \ .3) + 3 (6 - i,) '

- 4 ( 2 + - 3) + i l - -In (;i + 1 ) X = 0. If a solution of this in
descending powers of - go be taken to be

* Comptes liendus, lxxxv. (1877), pp. ti89-(;95, 728-732, 821-826.

t Fonctions Elliptiques, 11. (Paris, 1888), pp. 494-502.

:J: Comptes Remlus, i.xxxvi. (1878), pp. 313-315.

§ Fonctions Elliptiques, ii. (Paris, 1888), pp. 471-473.

II Quarterly Journal, xxvii. (1895), pp. 93-98.

237] lame's equation 571

the recurrence formula for the coefficients c,. is

4r (n, - r + I) (2/1 - ? + 1) Cr

= ( ?i - r + 1) 12e (n - r) (n - r + 2) - 4eo (n- + n - 3) - 4J5 c,\ i
- 2 (n - r + ] ) (n - r + 2) (ei - e ) (e. - 63) (2h - 2r + 3) c,\ 2.

Write r = 71 4- 1, and it is seen that Cn+i =; then write r = n + 2
and Cn+2 =; and the recurrence formulae with r > n + 2 are all
satisfied by taking

Hence Lame's generalised equation always has two solutions tuhose
product is of the form

2c,( -e,)-'-.

r=0

This polynomial may be written in the form

n

n j(w)-\&>( r)l>

where i, a, ... a are, as yet, undetermined as to their signs; and
the two solutions of Lame's equation will be called Aj, A,.

Two cases arise, (I) when A1/A2 is constant, (II) when Aj/A is not
constant.

(I) The first case is easily disposed of; for unless the polynomial

r = l

is a perfect square in, multiplied possibly by expressions of the
type - e, - 2, - 63, then the algebraic form of Lame's equation has
an indicial equation, one of whose roots is |, at one or more of the
points | = i> (a,-); and this is not the case (§ 23'43).

Hence the polynomial must be a square multiplied possibly by one or
more of - e, - 62, - e-s, and then Aj is a Lame function, so that B
has one of the characteristic values 5,/"; and this is the case which
has been discussed at length in §§ 23-1-23-47.

(II) In the second case we have (§ 19*53)

du ' du

where S is a constant which is not zero. Then

d log A2 d log Ai \ 26 du du X

d log A2 d log Ai \ 1 dX du du X du

rf log A, \ 1 dX 6 c logA a 1 dX 6 so that - 2 X ' du 2X du X -

572 THE TRANSCENDENTAL FUNCTIONS

On integration, "sve see that Ave may take

A, = VXexp|-(>[~|. Again, if we differentiate the equation

[chap. XXIII

we find that

1 dA,\ 1 dX (i Ai du ~ '2X du X '

1 d'A, 1 dA,\ \ 1 d'X 1 fdXy g dX

A, dir- ( Ai du \ 2X dti? 2X' \ du ) X du '

and hence, with the aid of Lame's equation, we obtain the interesting
formula

If now,. = jf) a,), we find from this formula (when multiplied by
X-), that, \ i u be given the special value cir, then

'dXV

m-

We now fix the signs of aj, an, ... cin by taking 'dX\ 2g

fdX\ \ 2(

( .)

And then, if we put 26/X, g-ita function of |, into partial fractions,
it is seen that

2(J (a' (n \ '

and therefore

Ai =

n[i (H)-li>(a.)i

X exp 2 - log a (ctr + it) - log a (a, - n) - 2m (a,)|, whence it
follows that (§ 20'53, example 1)

Ai= n

a- (ar + -m)

;x[exp|-?t 2 (a,)|,

and

r=l lO" (z<) (7 (a,

 2= n I - J-exp|M 2 ?(a,.)h

,.=1 |o-(iOo-(a,) =1 '

The complete solution has therefore been obtained for arbitrary values
of the constant B.

2371] lame's EQUATION 573

23'71. The Jacobian form of tJie generalised Lame equation. We shall
now construct the solution of the equation

-- = [n n + 1) k' sn a + A,

for general values of A, in a form resembling that of § 23*6.

The solution which corresponds to that of § 23"6 is seen to be*

where p, a, a, ...cun are constants to be determined. On
differentiating this equation it is seen that

Ada,-=i|H(a + a,) B\ a)\ \

n

= X Z (a + a, + iK') - Z (a) + p + \ mrilK,

r=l, 1 C A (1 dA, 1 .w -rr i,,

   ' A - JA 1 = .-X "" " + ° + - ' " ""

and therefore, since A is a solution of Lame's equation, the constants
p, a, Oo, ... a,i are to be determined from the consideration that
the equation

2 Z(a + ar+ iK') - Z a ] -f p + niri/K

r = l

n n + 1) k- sn- a - A = X dn- (a + a,- + iK') - dn- a

r=l

+ is to be an identity; that is to say

n

n'k" sn- a + 71 + A + 2 cs-(a + a,-)

2 Z (a + a, + t7f ') - Z (a)] + p + mrilK

r=l

Now both sides of the proposed identity are doubly periodic functions
of a with periods 2K, 2iK', and their singularities are double poles
at points congruent to -iK', - Qi, - a.,, ... - a; the dominant terms
near -iK' and -;. are respectively

n- 1

(a+TKy ' ~ (a + arf

in the case of each of the expressions under consideration.

The residues of the expression on the left are all zero and so, if we
choose p, !, Ko, ... On SO that the residues of the expression on the
right are zero,

* This solution was published in 1872 in Hermite's lithographed notes
of his lectures delivered at the Ecole polytecbnique.

574

THE TRANSCENDENTAL FUNCTIONS [CHAP. XXIII

it will follow from Liouville's theorem that the two expressions
differ by a constant which can be made to vanish by proper choice of
A.

We thus obtain n+ 2 equations connecting p, a, a.., ... a,,, with A,
but these equations are not all independent.

It is easy to prove that, near - o,

2 Z a + a, + iK') - Z a)\ + p + i iri/K

r=l

= - + I' Z (Op - a, + Hi') + nZ (a,) + p + i (n - 1) -nilK + (a +
a,-), where the prime denotes that the term for which jj = ?' is
omitted; and, near

I [Z (a + a, + iK') - Z (a)] + p + i n-ni\ K

r = \

= + i Z(,) + p+0(a+27r).

ft + til >. = !

Hence the residues of

i Z (a + a, + i7i ') - Z (a) + p + | ??.7rt7A" will all vanish if p,
a,, a., ... ft are chosen so that the equations ' i' Z (ft,, - ft, +
iK') + '/iZ (ft,) + p + ( - 1) 7r?:/ii = 0,

I Z(a,) + P = V -=1

are all satisfied.

The last equation merely gives the value of p, namely

- 2 Z (ft,),

and, when we substitute this value in the first system, we find that

2' [Z (ftp - ft, -f- iK') + Z (ft,) - Z (a,,) + t TTzyZ] = 0,

where r = 1, 2, ... n. By § 22*735, example 2, the sum of the
left-hand sides of these equations is zero, so they are equivalent to
n - 1 equations at most; and, when i, a.,, ... a have any values
which satisfy them, the difference

7t-A''- sn- a + ?i + -4 + 51 cs- ( + ft;-)

S |Z (ft + ft, + IK') - Z (ft) - Z (ft,) + \ irijK\

2371] lame's equation 575

is constant. By taking a = 0, it is seen that the constant is zero if

n + A + S cs- Of,. =

r = l

2 [Z (a, + iK') - Z (a,) + Tri/K]

)• = !

i.e. if \ S en Of,, ds cif,.[- - 2 ns- ot,. = .

ij-=l j r=l

We now reduce the system of n equations; with the notation of § 22
'2, if functions of Op, a,, be denoted by the suffixes 1 and 2, it is
easy to see that

Z up -ar + iK') + Z (a,) - Z (op) + Tri'/A'

= Z (a - Or + iK') + Z(ar) -Z ap + iK') + Cj 0?i /Si = k' sn (op +
iK') sn a sn (ap + iX' - o ) + Ci ci?i /si

 2

+

Cidi

Sj sn (op- n,.) Si

\ SiCiO?i+g2C2C 2 ~ V- 2

Consequently a solution of Lame's equation

-- -= 71. (n + l) -sn-a+ A] A act- I '

IS

A= n

@( ) exp-aZ(a,)

provided that !, a2> ••• n be chosen to satisfy the n independent
equations comprised in the system

/, sn ttp en ttp dn ctp + sn of en o dn;. \

I J i sn Op - sn ar

2 en ttr ds a,.

2 ns a,. = A;

r = l

and if this solution of Lame's equation is not doubly periodic, a
second solution is

"H(a-a,)

n

@ (a)

exp [otZ (,.)]•

= 0.

The existence of a solution of the system of n - l equations follows
from § 23-7.

REFERENCES.

G. Lam, Journal de Math. ii. (1837), pp. 147-188; iv. (1839), pp.
100-125, 126-163, 351- 385; vin. (1843), pp. 397-434. Lecons sur les
fonctions inverses des transcendantes et les surfaces isothermes
(Paris, 1857). Lemons sur les coordonnees curvilignes (Paris, 1859).

E. Heine, Journal fUr Math. xxix. (1845), pp. 185-208. Theorie der
Kugelfunctionen, ii. (Berlin, 1880).

C. Herhite, Comptes Rendus, lxxxv. (1877), pp. 689-695, 728-732,
821-826; Ann. di Mat. (2) IX. (1878), pp. 21-24. Oeuvres
Mathematiques (Paris, 1905-1917).

576 THE TRANSCENDENTAL FUNCTIONS [CHAP. XXIII

G. H. Halphen, Fonctions Elliptiques, il. (Paris, li

F. LixuEMANN, Math. Ann. xix. (1882), pp. 323-386.

K. Hecn, JIath. Ann. xxxili. (1889), pp. 161-179, 180-196.

L. Crawford, Quarterly Journal, xxvii. (1895), pp. 93-98; xxix.
(1898), pp. 196-201.

W. D. NiVEN, Phil. Trans, of the Royal Society, 182 a (1891), pp.
231-278.

A. Cayley, Phil. Trans, of the Royal Society, 165 (1875), pp. 675-774.

G. H. DAR VIN, Phil. Trans, of the Royal Society, 197 A (1901), pp.
461-557; 198 a (1901),

pp. 301-331.

Miscellaneous Examples.

1. Obtain the formula

Gn X, y, =, /J " Pn (J) e-' dn.Hn x, y, z).

(Niven, Phil. Travis. 182 a (1891), p. 245.)

2. Shew that

rr fl l\ \ (-r. 2n)l Hn x,y,z)

(Hobson, Proc. London Math. Soc. xxiv.)

3. Shew that the 'external ellipsoidal harmonic' F '" (\$) En'" (rj)
E,,''" (C) is a constant multiple of

"Xcx' dy' 82; V 2.(271 + 3) 2. 4 (2yi + 3) (2?i + 5) --J x +y + z )'

(Niven; and Hobson, Proc. London Math. Soc. xxiv.)

4. Discuss the confluent form of Lame's equation when the invariants 2
and g of the

Weierstrassian elliptic function are made to tend to zero; express the
solution in terms of

Bessel functions.

(Haentzschel, Zeitschrift fiir Math, und Phys. xxxi.)

5 If t! denotes - - exp [ X - Z (/x) a], where X and /x are constants,
shew that

G(a)

Lamd's equation has a solution which is expressible as a linear
combination of

dn-l cln-3 dn-b

where X- and sn p. are algebraic functions of the constant A.

(Hermite.)

6. Obtain solutions of

- =12Psn22-4(l + F)±5V(I-F + Z-'). w dz-

(Stenberg, Acta Math, x.)

7. Discuss the solution of the equation

2(2-l)(2-a) + [(a + + l)s' - n + /3-g + l+(y + S)a i+ay] + (5-j)3/ =

in the form of the series

Gn q) hY

l+a,3 2

i=i !y(y+l)...(y+ )' where Gi q) = q, (?2(?) = a/3?2 + (a + /3-8+ l) +
(-y + 8)a j-ay,

6-' + i (?) = [ (a + /3-8 + n) + (y + S + -l)o + a/3j]6' (g)

-(a + -l)(/3 + ?i-l)(y + n-l)?i .6' \ i(j).

(Heun, Math. Ann. xxxiii.)

LAME S EQUATION 577

8. Shew that the exponents at the singularities 0, \, a, cc of Heun's
equation are

(0, 1-y), (0, 1-S), (0,1-0, (, ), where y + 8 + e = a + l3 + l.

(Heun, Math. Ann. xxxiii.)

9. Obtain the following group of variables for Heun's equation,
corresponding to the group

\ I z 2-1

for the hypergeometric equation :

 )

1-z,

1

z '

1

1-2'

z z-l'

z

-1

z '

z

a - z

  

a

z

z-a

a'

a '

a-z'

z-a'

z

z-

-a

z-l

1-

 a

a-

1 z-

 a

z

-1

1-a' a - 1' 2-a' z-V s-1' 2 '

2-a (a-l)2 a(2 - 1) a(2-l) z - a (l-a)2

a (2-1)' rt(2-l)' 2-a ' (a -1)2' (I-o)!;' 2-a '

(Heun, J/a . 4 ?;. xxxiii.)

10. If the series of example 7 be called

F a, q; a, jS, y, 8; 2),

obtain 192 solutions of the differential equation in the form of
powers of 2, 2- 1 and z - a. multiplied by functions of the type F.

[Heun gives 48 of these solutions.]

11. If ic=2v, shew that Lame's equation

r/2 A

- = n (71 + 1) iHu) + B A

may be transformed into

by the substitution

12. If C = P (v), shew that a formal solution of the equation of
example 11 is

provided that (a - 2?i) (a - ?i + 5) =

and that

4:(a-r-2n) a-r-n + h)b,+ [l2e.2ia-r + l) a-r-2n+l) + 'ie2n
2n~l)-4B]br-i

- 4( 1 - 62) (6-2 - 63) (a-r + 2) (a-r - n+l) h,.\ 2 = 0., (Brioschi,
Comptes Bendus, lxxxvi. (1878), pp. 313-3,15 and Halphen.)

13. Shew that, if n is half of an odd positive integer, a solution of
the equation of example 11 expressible in finite form is

i= 6,(C-e2)'"-'-, r=0

W. M. A. 37

578 THE TRANSCENDENTAL FUNCTIONS [CHAP. XXIII

p o ded that

 ir u-r + )l>,. + [' -2e.> -2n-i- + ) r-l)-4ein '2n-l) + -iB]b,\ i

+ 4(ei-e2)( '2-t'3)(2 -r + 2)( -/-+S)6r-2 = 0,

and B is so determined that,j .j = 0-

(Brioschi and Halphen.)

14. Shew that, if n is half of an odd integer, a sohition of tlie
equation of example 11 xpressible in finite form is

Z'="i'V(C-eo)"- -

provided that

4/) (n+jt? + A) bp - [Ue., n -p+ [ +p - *) - 1<'2 (2 - 1) + 45] //p\ i

+ 4 (ei - f 2) ( -.' - 3) in-p + ' ) p-l) ?>',, -2 =

and ?>' 1=0 is the equation which determines B.

(Crawford.)

15. "With the notation of examples 13 and 14 shew that, if

V = ( - )" ( 1 - '2)'' e. - es)" c \ \ .j. the equations which
determine Cq, Cj, .••< \ j!. ' I'e identical with those which
determine /)o, 61, ... > \ i; and deduce that, if one of the
solutions of Lame's equation (in which n is half of an odd integer) is
expressible as an algebraic function of p v\ so also is the other.

(Crawford.)

16. Prove that the values of B determined in example 13 are re;\ l
when e, e and 6-3 are real.

17. Shew that the complete solution of

is A = iO' U)r - Ap iu) + B],

where A and B are arbiti-aiy constants.

(Halphen, Jle'm. par divers savants, xxviii. (i), (1880), p. 105.)

18. Shew that the complete solution of

ig = |Fsn a-i(l4-F)

is \ = sn C-a)cQ C-a)dn C-a)r A + Bsnn.iC-a),

where .-1 and B are arbitrary constants and C=2K+iK'.

(Jamet, Comptes Rendus, cxi.)