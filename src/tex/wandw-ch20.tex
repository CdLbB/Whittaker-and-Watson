\chapter{Elliptic Functions. General Theorems and the 
Weierstrassian Functions} 

20"1. Doubly -periodic functions.

A most important property of the circular functions sin, cos, tan,
... is that, if/( ) denote any one of them,

f z + 2 )=f z),

and hfmce f z- '2mr)=f z), for all integer values of n. It is on
account of this property that the circular functions are frequently
described as periodic functions with period 27r. To distinguish them
from the functions which will be discussed in this and the two
following chapters, they are called singly-periodic functions.

Let Q)i, 0)2 be any two numbers (real or complex) whose ratio* is not
purely real. A function which satisfies the equations

f(z + 2a,0 =f z), f(z + 2a,,) =f(z),

for all values of z for which f(z) exists, is called a doubly-periodic
function of z, with periods 2\&)i, 2co2. A doubly-periodic function
which is analytic (except at poles), and which has no singularities
other than poles in the finite part of the plane, is called an
elliptic function.

[Note. What is now known as an elliptic integral occurs in the
researches of Jakob Bernoulli on the Elastica. Maclaurin, Fagnano,
Legendre, and others considered such integrals in connexion with the
problem of rectifying an arc of an ellipse; the idea of 'inverting'
an elliptic integral (§ 21 '7) to obtain an elliptic function is due
to Abel, Jacobi and Gauss.]

The periods 2( i, 2\&,o play much the same part in the theory of
elliptic functions as is played by the single period in the case of
the circular functions.

Before actually constructing any elliptic functions, and, indeed,
before establishing the existence of such functions, it is convenient
to prove some general theorems (\hardsubsectionref{20}{1}{1}-20'14) concerning properties
common to all elliptic functions; this procedure, though not strictly
logical, is convenient

* If w.,/wj is real, the parallelograms defined in\hardsubsectionref{20}{1}{1} collapse,
and the function reduces to a singly-periodic function when Wg/wj is
rational; and when w /wj is irrational, it has been shewn by Jacobi,
Journal fiir Math. xiii. (183.5), pp. 55-56 [Ges. Werke, 11. (1882),
pp. 25-26] that the function reduces to a constant.

t A brief discussion of elliptic integrals will be found in §§
22-7-22*741.

%
% 430
%

because a large number of the properties of particular elliptic
functions can be obtained at once by an appeal to these theorems.

Example. The diflferential coefficient of an elliptic function is
itself an elliptic function.

2011. Pei-iod-parallelograms.

The study of elliptic functions is much facilitated by the geometrical
representation afforded by the Argaud diagi-am.

Suppose that in the plane of the variable z we mark the points 0,
2(Wi, 2(02, 2a)i + 2\&)2, and, generally, all the points whose complex
coordinates are of the form 2mo)i + 2h\&j.,, where m and n are
integers.

Join in succession consecutive points of the set 0, 2(Oi, 2(Oi + 2o
.2, 2ei).,, 0, and we obtain a parallelogram. If there is no point a
inside or on the boundary of this parallelogram (the vertices
excepted) such that

f z + <o)=f z)

for all values of z, this parallelogram is called \& fundamental
period-parallelo- gram for an elliptic function with periods 2a)i, 2(0
.

It is clear that the -plane may be covered with a network of
parallelo- grams equal to the fundamental period-parallelogram and
similarly situated, each of the points 2?/ia)i + 2na), being a vertex
of four parallelograms.

These parallelograms are called peHod-parallelograms, or meshes; for
all values of z, the points z, z- 2w, ... z +2m(o +2nw. ...
manifestly occupy corresponding positions in the meshes; any pair of
such points are said to be congruent to one another. The congruence of
two points z, z is expressed by the notation / = (mod. 2\&)i, 2w.2).

From the fundamental property of elliptic functions, it follows that
an elliptic function assumes the same value at every one of a set of
congruent points; and so its values in any mesh are a mere repetition
of its values in any other mesh.

For purposes of integration it is not convenient to deal with the
actual meshes if they have singularities of the integrand on their
boundaries; on account of the periodic properties of elliptic
functions nothing is lost by taking as a contour, not an actual mesh,
but a parallelogram obtained by translating a mesh (without rotation)
in such a way that none of the poles of the integrands considered are
on the sides of the parallelogi-am. Such a parallelogram is called a
cell. Obviously the values assumed by an elliptic function in a cell
are a mere repetition of its values in any mesh.

A set of poles (or zeros) of an elliptic function in any given cell is
called an irreducible set; all other poles (or zeros) of the function
are congruent to one or other of them.

%
% 431
%

20'12. Simple properties of elliptic functions.

(I) The number of poles of an elliptic function in any cell is finite.

For, if not, the poles would have a limit point, by the
two-dimensional analogue of\hardsubsectionref{2}{2}{1}. This point is \hardsubsectionref{5}{6}{1}) an
essential singularity of the function; and so, by definition, the
function is not an elliptic function.

(II) The number of zeros of an elliptic function in any cell is
finite.

For, if not, the reciprocal of the function would have an infinite
number of poles in the cell, and would therefore have an essential
singularity; and this point would be an essential singularity of the
original function, which would therefore not be an elliptic function,
[This argument presupposes that the function is not identically zero.]

(III) The sum of the residues of an elliptic function, f z), at its
poles in any cell is zero.

Let C be the contour formed by the edges of the cell, and let the
corners of the cell be, + 2\&ji, t + 2(yi + 2\&).,, t + 2\&)2.

[Note. In future, the periods of an elliptic function will not be
called 2ci)i, 2(B2 indifferently; but that one will be called 2wi
which makes the ratio o),/©! have a positive imaginary part; and
then, if C be described in the sense indicated by the order of the
corners given above, the description of C is counter-clockioise.

Throughout the chapter, we shall denote by the symbol C the contour
formed by the edges of a cell.]

The sum of the residues oif z) at its poles inside C is

 .\ f z)dz=-~\ \ . + + + \ f z)dz.

In the second and third integrals write z+2a)i, z + 2(Oo respectively
for 2, and the right-hand side becomes

 . /( ) -/ + 2a>.)) dz - - f z) -f z + 20,01 dz,

and each of these integrals vanishes in virtue of the periodic
properties of f z); and so I f z) dz = 0, and the theorem is
established.

(IV) Liouvilles theorem*. An elliptic function, f z), luith no poles
in a cell is merely a constant.

For if f(z) has no poles inside the cell, it is analytic (and
consequently bounded) inside and on the boundary of the cell \hardsubsectionref{3}{6}{1}
corollary ii); that is to say, there is a number K such that f(z), <
K when z is inside or on the boundary of the cell. From the periodic
properties oi f z) it follows that

* This modification of the theorem of\hardsubsectionref{5}{6}{3} is the result on which
Liouville based his lectures on elliptic functions.

%
% 432
%

f z) is analytic and \ f z) \ < K for all values of z; and so, by §
5-63, f z) is a constant.

It will be seen later that a very large number of theorems concerning
elliptic functions can be proved by the aid of this result.

2013. The order of an elliptic function.

It will now be shewn that, \ \ f z) be an elliptic function and c be
any constant, the number of roots of the equation

f( ) = c

which lie in any cell depends only on f z), and not on c; this number
is called the order of the elliptic function, and is equal to the
number of poles off(z) in the cell.

By\hardsubsectionref{6}{3}{1}, the difference between the number of zeros and the number
of poles o f(z) - c which lie in the cell G is

1 ' /'< > d..

27ri J r f z) - c

Since /' z + 2\&)i) =/' z + 2(Wo) = /" z), by dividing the contour
into four parts, precisely as in\hardsubsectionref{20}{1}{2}(III), we find that this
integral is zero.

Therefore the number of zeros of /'( )- c is equal to the number of
poles oi f z) - c\ but any pole oif z) - c is obviously a pole of f(z)
and conversely; hence the number of zeros of f(z)- c is equal to the
number of poles of f(z), which is independent of c; the required
result is therefore established.

[Note. In determining the order of an elliptic function by counting
the number of its irreducible poles, it is obvious, from\hardsubsectionref{6}{3}{1}, that
each pole has to be reckoned according to its multiplicity.]

The order of an elliptic function is never less than 2; for an
elliptic function of order 1 would have a single irreducible pole;
and if this point actually were a pole (and not an ordinary point) the
residue there would not be zero, which is contrary to the result of §
201 2 (III).

So far as singularities are concerned, the simplest elliptic functions
are those of order 2. Such functions may be divided into two classes,
(i) those which have a single irreducible double pole, at which the
residue is zero in accordance with\hardsubsectionref{20}{1}{2} (III); (ii) those which
have two simple poles at which, by § 20"! 2 (III), the residues are
numerically equal but opposite in sign.

Functions belonging to these respective classes will be discussed in
this chapter and in Chapter xxii under the names of Weierstrassian and
Jacobian elliptic functions respectively; and it will be shewn that
any elliptic function is expressible in terms of functions of either
of these types.

%
% 433
%

20"14. Relation hetiueen the zeros and poles of an elliptic function.

We shall now shew that the sum of the affixes of a set of irreducible
zeros of an elliptic function is congruent to the sum of the affixes
of a set of irreducible poles.

For, with the notation previously employed, it follows, from\hardsubsectionref{6}{3}{1},
that the difference between the sums in question is

27ri J C f 2) TTl [J t J +2<o, J f+2a,,+2a,2 J t+2ojJ J 2:)

  j rt+2. j ) \ z+2oy,)f'(z + 2co. ] ~27ri]t \ f ) f z + 2ay,) \ '

27rij, \ f z) f z+2c.,) r

27ri [ Jt f (z) J t f z)

J.j-2.,[log/(.)

t+iu

+ 2(0,

log/( )

t

on making use of the substitutions used in § 2012 (III) and of the
periodic properties off z) and f' z).

Now f z) has the same values at the points t + 2(Oi, t 4-20), as at t,
so the values of \ ogf(z) at these points can only differ from the
value of f z) at t by integer multiples of 27ri, say - 2n7ri,
2///'7rt; then we have

2'mJc fiz)

and so the sum of the affixes of the zeros minus the suin of the
affixes of the poles is a period; and this is the result which had to
be established.

20'2. The construction of an elliptic function. Definition of z).

It was seen in | 20'1 that elliptic functions may be expected to have
some properties analogous to those of the circular functions. It is
therefore natural to introduce elliptic functions into analysis by
some definition analogous to one of the definitions which may be made
the foundation of the theory of circular functions.

One mode of developing the theory of the circular functions is to
start

from the series S z-m7r)~-; calling this series (sin ')~-, it is
possible

JW= - CO

to deduce all the known properties of sin z; the method of doing so
is briefly indicated in\hardsubsubsectionref{20}{2}{2}{2}.

W. M. A. 28

%
% 434
%

The analogous method of founding- the theory of elliptic functions is
to define the function (,.> ( ) by the equation*

 " m.n \ \ {2; - 2vio)i - '2n(02y (2? \&)i+2?ieu2)-j ' Avhere (o, Wo
satisfy the conditions laid down in §§ 0*1, 20"12(III); the summation
extends over all integer values (positive, negative and zero) of m and
n, simultaneous zero values of ni and n excepted.

For brevity, we write Qm,n in place of 'Iinw + 'Inwo, so that

m, n

When m and n are such that \ \ m,n\ is large, the general terra of the
series defining ip (z) is 0(\ flm,n\~% and so \hardsectionref{3}{4}) the series
converges absolutely and uniformly (with regard to z) except near its
poles, namely the points Clm,n-

Therefore \hardsectionref{5}{3}), J (z) is analytic throughout the whole 2 -plane
except at the points n,\, where it has double poles.

The introduction of this function p z) is due to Weierstrassf; we now
proceed to discuss properties of (z), and in the course of the
investigation it will appear that j z) is an elliptic function with
penods 2\&)i, 2a)2.

For purposes of numerical computation the series for p (z) is useless
on account of the slowness of its convergence. Elliptic functions free
from this defect will be obtained in Chapter xxi.

Example. Prove that

P(2)=U- I - + 2 cosec2 =7r - 2 cosec- - "tt .

20*21. Periodicity and other properties of z).

Since the series for z) is a uniformly convergent series of analytic
functions, term-by-terra differentiation is legitimate \hardsectionref{5}{3}), and so

ij' z) =,j z)=-2 t

  m, n \ \ - '>, n)

The function ' z) is an odd functio7i of z\ for, from the definition
of y( ), we at once get

 '(- ) = 2 ( 4-n,, )- .

m, n * Throughout the chapter 2 will be written to denote a summation
over all integer values of m and n, a prime being inserted (2') when
the term for which j =:n = has to be omitted

m, n

from the summation. It is also customary to write ' z) for the
derivate of 4> z). The use of the prime in two senses will not cause
confusion.

t Werke, ii. (1895), pp. 245-2.55. The subject-matter of the greater
part of this chapter is due to Weierstrass, and is contained in his
lectures, of which an account has been published by Schwarz, Formeln
und Lehrsatze zum Gebrauche der elUptischen Fiinktionen, Xach
Vorle.tnngen und Aufzeichnungen des Herrn Prof. K. Weierstrass
(Berlin, 1893). See also Cayley, Journal de Math. X. (1845), pp.
385-420 [Math. Papers, i. pp. 156-182], and Eisenstein, Journal fUr
Math. XXXV. (1847), pp. 137-184, 18-5-274.

%
% 435
%

But the set of points - D, (\ is the same as the set Qm, n and so the
terms of ' (- z) are just the same as those of - ' (z), but in a
different order. But, the series for j' (z) being absolutely
convergent \hardsectionref{3}{4}), the derangement of the terms does not affect its
sum, and therefore

 y -z) = - y (z).

In like manner, the terms of the absolutely convergent series

ni,n

are the terms of the series

y Hz - n - n~ i

m, n

in a different order, and hence

i (- z) = i U); that is to say, (z) is an even functipn\ oj[j.

Further, j' z + 2co,) = - 2 1 ( - D, + 2a,r;

m, n

but the set of points n, - 2\&ji is the same as the set Q,,, so the
series for y (z + 'Zcoi) is a derangement of the series for ' (z). The
series being absolutely convergent, we have

 y (z + 2ft)i) = y (z);

that is to say, '/ (z) has the period 2fOi; in like manner it has the
period 2ft)o.

Since y' (z) is analytic except at its poles, it follows from this
result that y (z) is an elliptic function.

If now we integrate the equation ' z + '2(o ) = < ' z), we get

<p z + 2oy,) = < i z) + A,

where A is constant. Putting z = - w and using the fact that < z) is
an even function, we get = 0, so that

iO z + 2ft)i) = J (z); in like manner j z + 'Icoo) = (jp (z).

Since p (z) has no singularities but poles, it follows from these two
results that J(2) is an elliptic function.

There are other methods of introducing both the circular and elliptic
functions into analysis; for the circular functions the following may
be noticed :

(1) The geometrical definition in which sin z is the ratio of the side
opposite the angle 2 to the hypotenuse in a right-angled triangle of
which one angle is z. This is the definition given in elementary
text-books on Trigonometry; from our point of view it has various
disadvantages, some of which are stated in the Appendix.

(2) The definition by the power series

Z 2

sm2 = 2--, + -

28-2

%
% 436
%

(3) The definition by the product

(4) The definition by 'inversion' of an integral

fBim J

The periodicity properties may be obtained easily from (4) by taking
suitable paths of integration (of. Forsyth, Theory of Functions,
(1918), 104), but it is extremely difficult to prove that sin z
defined in this way is an analytic function.

The reader will 'see later (§§ 22-82, 22-1, 20-42, 20-22 and\hardsubsectionref{20}{5}{3}
example 4) that elliptic functions may be defined by definitions
analogous to each of these, with corre- sponding disadvantages in the
cases of the first and fourth.

Example. Deduce the periodicity of .> (z) directly from its definition
as a double series. [It is not difficult to justify the necessary
derangement.]

2022. The differential equation satisfied hy j z).

We shall now obtain an equation satisfied by < z), which will prove to
be of great importance in the theory of the function.

The function j (2) -,2" which is equal to S' (2 - n,,,i)~2 - H, is

m, n

analytic in a region of which the origin is an internal point, and it
is an even function of s. Consequently, by Taylor's theorem, we have
an expansion of the form

valid for sufficiently small values of | |. It is easy to see that

m, n ni, n

Thus (z) = Z-' + 20 g z' + 28 3 + ( ') '

differentiating this result, we have

 y (z) = - 2z-' + g,z 1 g,z + (z ). Cubing and squaring these
respectively, we get

f z) = z-' + .l~ g,z- + l g..+ 0(z ),

Hence ' (z) - 4 =* (z) = - g.z" -gs + z'),

and so '- z) - 4f (z) + g (z) + gs=0 (z').

That is to say, the function z) - 4 (z) + g (z) + g-s, which is
obviously an elliptic function, is analytic at the origin, and
consequently it is also analytic at all congruent points. But such
points are the only possible singularities of the function, and, so it
is an elliptic function luith no singularities; it is therefore a
constant \hardsubsectionref{20}{1}{2}, lY).

On making z- 0, we see that this constant is zero.

%
% 437
%

Thus, finally, the function ( z) satisfies the differential equation

where 2 and g (called the invariants) are given by the equations

g, = m T n f, 5r3 = i4o 1' n-.

), n m, n

Conversely, given the equation if numbers \&)i, (o can he determined*
such that

m, n i, n

then the general solution of the differential equation is

y = iO ±z + a), where a is the constant of integration. This may be
seen by taking a new dependent variable u defined by the equationf y =
< (u), when the differential

equation reduces to [ \ =1.

Since g? z) is an even function of z, we have y = 0 z ± a), and so the
solution of the equation can be written in the form

y = j z + a) without loss of generality.

Example. Deduce from the differential equation that, if

)i = l

then C2=5r2/22 . 5, Ci=g l . 7, CQ=g ij2i.' . 52,

 ~2*.5.7.11' "~25..3.5M.3" 2*.72.13' '' 2 . 3 . 52 . 7 . 11 "

20"221. The integral formula for < z). Consider the equation

z=\ (U'-g,t-g,)- dt,

  i

determining z in terms of if; the path of integration may be any
curve which does not pass through a zero of M - g. t - g . On
differentiation, we get

(i/ = 4?'-< f-...

and so =Sf> ( + oc),

where a is a constant.

* The difficult problem of establishing the existence of such numbers
w and Wj when g and g are given is solved in \hardsubsectionref{21}{7}{3}.

t This equation in it always has solutions, by\hardsubsectionref{20}{1}{3}.

%
% 438
%

Make - > x; then - >0, since the integral converges, and so a is a
pole of the function; i.e., a is of the form n,,i, and so = j(2 + n,
n) = (z).

The result that the equation z=l 4:t' - gd - (/ i)~-dt is equivalent
to

u

the equation = z) is sometimes written in the form

= f,

20*222. An illustration from the theory of the circular functions.

The theorems obtained in §>; 20'2-20'221 may be illustrated by the
corresponding results in the theory of the circular functions. Thus we
may deduce the properties

of the function cosec- z from the series 2 z- mTT)~ in the following
manner :

m=-a)

Denote the series by/ (2); the series converges absolutely and
uniformly* (with regard to z) except near the points mn at which it
obviously has double poles. Except at these points, /(j) is analytic.
The effect of adding any multiple of tt to is to give a series whose
terms are the same as tho.se occurring in the original series; since
the series converges absolutely, the sum of the series is unaffected,
and 80/(2) is a periodic function of z icith period n.

Now consider the behaviour oi f z) in the .strip for which -\ 7r R z)
\ n. From the periodicity of f z), the value off z) at any point in
the plane is equal to its value at the corresponding point of the
strip. In the strip/ (2) has one .singularity, namely 2 =; and /(2)
is bounded as 2-*-oc in the strip, becau.se the terms of the series
for/ (2 are

ac

small compared with the corresponding terms of the comparison series
2' m~ .

m= - 00

In a domain including the point z=0, f z)-z~ is analytic, and is an
even function; and consequently there is a Maclaurin expansion

/(2)-2-2= 2 a, z valid when ' z' <n. It is easily seen that

a2 = 27r-- (2 + l) 2 jn-'"~-,

m = l

and so o= a2 = 67r~* 2 in'* .

m=l

Hence, for small values of | 2 |,

f z) = z- + + \ z + 0(z*). Differentiating this result twice, and also
squaring it, we have

f"(z)=ez- + +0 z%

 Hz)=z-*+p- +n+o z ).

It follows that /" (2) -e/'-i (2) + 4/(2)= (z ).

That is to say, the function /" (2) - 6/ (2) + 4/" (2) is analytic at
the origin and it is obviously periodic. Since its only possible
singularities are at the points mn, it follows from the periodic
property of the function that it is an integral function.

* By comparison with the series 2' m~' .

m = - 30

%
% 439
%

Furthei*, it is bounded as z-s-qc in the strip - ir R z) 7r, since f
z) is bounded and so is* f" (z). Hence/" (2) - 6/ (3) + 4/(2) is
bounded in the strip, and therefore from its periodicity it is bounded
everywhere. By Liouville's theorem \hardsubsectionref{5}{6}{3}) it is therefore a
constant. By making z ~0, we see that the constant is zero. Hence the
function cosec z satisfies the equation

f"(z) = 6fHz)-4f z).

Multiplying by 2/' (s) and integrating, we get

f z) = 4P z) f(z)-l +c,

where c is a constant, which is easily seen to be zero on making use
of the power series

for/' (2) and/ (2).

We thence deduce that 2s = I f' t-l)~i dt,

J fiz)

when an appropriate path of integration is chosen.

Example 1. If j/ = (2) and primes denote difierentiations with regard
to 2, shew that

4 -|j3=tV (y- i)-H0/-.'.2)-H(3/-.3)- -|y(3/- i)-'(y- 2)-V.y- 3)--\

where ej, e, e are the roots of the equation Afi - got - gz = - [AVe
have y" = 4f-go,y-g

Differentiating logarithmically and dividing by y\ we have

r=\

Differentiating again, we have

2y"' 4?/"2 3,,

y y r=i '

Adding this equation multiplied by j to the square of the preceding
equation, multiplied by, we readily obtain the desired result.

It should be noted that the left-hand side of the equation is half the
Schwarziaii derivative t of z with respect to y; and so z is the
quotient of two solutions of the equation

c + Ire,.!, (//-O -gy n (.-..) j .=0.]

Example 2. Obtain the 'properties of homogeneity ' of the function
(2); namely that

 H>''')= '' f 'l'"')'; ~'92, X-V3) = X-'\&?(--; <72,5'3), \ 1 A 2/ \
i CO2/

where (s M denotes the function formed with periods 2q)i, 2a)2 and (2;
g, g )

denotes the function formed with invariants g-n g -

[The former is a direct consequence of the definition of z) by a
double series; the latter may then be derived from the double series
defining the g invariants.]

* The series for /" (s) may be compared with 2' t~'*.

m= -

t Cayley, Gamh. Phil. Tram. xiii. (1883), p. 5 [Math. Papers, xi. p.
148].

%
% 440
%

20"3. The addition-theorem for the function z).

The function z) possesses what is known as an addition-theorem; that
is to say, there exists a formula expressing J z + y) as an algebraic
function oi z) and (y) for general values* <) z and ?/.

Consider the equations

i ' z) = Af z) + B, <,j' y) Af j) + B,

which determine A and B in terms of and y unless z) = < y), i.e.
unlessf z = ±y (mod. 2wi, 2\&)2),

Now consider \&>' (0 - iP (t) - >

 i/a function of f. It has a triple pole at = and consequently it has
three, and only three, irreducible zeros, by\hardsubsectionref{20}{1}{3}; the sum of these
is a period, by\hardsubsectionref{20}{1}{4}, and as =z, =y are two zeros, the third
irreducible zero must be congruent to - - y. Hence - z - y h a. zero
of ' ( ) - Af ( ) - B,

and so

 o' -z-y)=A< -z-y) + B.

Eliminating A and B from this equation and the equations by which A
and B were defined, we have

  z) f' z) 1 =0.

i iz y) - ' z + y) 1

Since the derived functions occurring in this result can be expressed
algebraically in terms of z), i> y), <fP (z + y) respectively (§
20"22), this result really expresses .> z + y) algebraically in terms
of i z) and < (y). It is therefore an addition-theorem.

Other methods of obtaining the addition-theorem are indicated in §
20-311 examples 1 and 2, and\hardsubsubsectionref{20}{3}{1}{2}.

A symmetrical form of the addition-theorem may be noticed, namely
that, if u + V -I- w = 0, then

! ij(u) i iu) 1 =0.

\&(v) '(v) 1 (lu) ' (lu) 1

20*31. Another form of the addition-theorem.

Retaining the notation of\hardsectionref{20}{3}, we see that the values of, which
make ' ( ) - A ( ) - B vanish, are congruent to one of the points z,
y,-z - y.

* It is, of course, unnecessary to consider the special cases when y,
or z, or ij +z is n period.

t The function z)- (y), qua function of z, has double poles at points
congruent to 2 = 0, and no other singularities; it therefore (§
20-13) has only two irreducible zeros; and the points congruent to z=
y therefore give all the zeros of (z) - J (y).

%
% 441
%

Hence ' (0- l- fr (O + B]- vanishes when t is congruent to any of the
points z, y, - z - y. And so

4j.nr) - AY K) - AB + g. f 0 - (B-' + 9s)

vanishes when j ( ) is equal to any one of (z), y), j (z + y).

For general values of z and y, j (z), o (y) and z + y) are unequal and
so they are all the roots of the equation

4>Z - A-Z-- - -lAB + g )Z- (B- +g,) = 0.

Consequently, by the ordinary formula for the sum of the roots of a
cubic equation,

 ( ) + iHy) + i (2 + y) = lA%

and so i. + y)=li - -,). o(y),

on solving the equations by which A and B were defined.

This result expresses j (z + y) explicitly in terms of functions of z
and of 2/.

20'311. T e duplication formula for < z).

The forms of the addition-theorem which have been obtained are both
nugatory when y = z. But the result of\hardsubsectionref{20}{3}{1} is true, in the case of
any given value of z, for general values of y. Taking the limiting
form of the result when y approaches z, we have

From this equation, we see that, if 22- is not a period, we have

/ox Ir W (z) - i ' + h) o / X g (2z) = -J hm V- -x - -7 ~-y - 2p (z)

= -. 11m

on applying Taylor's theorem to J z + h), j' z + h)\ and so

 < >=i |<f- ( >'

unless 2z is a period. This result is called the duplicatioih formula.
Example 1. Prove that

qua function of j, has no singularities at points congruent with 2 =
0, ±i/; and, by making use of Liouville's theorem, deduce the
addition-theorem.

%
% 442
%

Example 2. Apply the process indicated in example 1 to the function

I (2/) ¥iy) 1 I,

and deduce the addition-theorem. Example 3. Shew that

  z .y) + z-y) = W z)-<p :y) -'-[ 2 z)ip y)-\ g ] <§> z)- y) -g,\ [By
the addition-theorem we have

Replacing ' z) and p (y) by 4 ( )\ (2)\ 3 and <\& y) - g y) - 9
respec- tively, and reducing, we obtain the required result.] Example
4. Shew, by Liouville's theorem, that

j W z-a) z-h)] = <p a-h) <p' z-a) + ' ' z-h) -iy a-b) z-a)- z-h)].

\addexamplecitation{Trinity, 1905.} 20"312. Abel's* method of proving the
addition-theorem for p z).

The following outline of a method of establishing the addition-theorem
for p (z) is instructive, though a completely rigorous proof would be
long and tedious.

Let the invariants of p z) be go, g ', take rectangular axes OX, OY in
a plane, and consider the intersections of the cubic curve

y - = Ax -g.2X-g with a variable line y = mx- n.

If any point ( i, y ) be taken on the cubic, the equation in z

P z)-x, = Q has two solutions +2i, - i \hardsubsectionref{20}{1}{3}) and all other
solutions are congruent to these two.

Since P' z) = ip z)-g2P z)-gz, we have P"- z)=yi; choose z- to be the
solution for which p' (2i)= -Hyi, not -y .

A number Zi thus chosen will be called the parameter of ( i, j/i) on
the cubic.

Now the abscissae i, x.2, x. of the intersections of the cubic with
the variable line are the roots of

< ix) ~ 4x gi -gz- (" - +nY= 0,

and so ( x)=.A x- x ) (x - x, (x - x ).

The variation bx in one of these abscissae due to the variation in
position of the line consequent on small changes hn, 8n in the
coefficients m, n is given by the equation

(b' (Xr) S*v+ - 8m + ? 8?t = 0, Cm en

and so * (f)' (Xr) 8 r = 2 (mx + n) x tm - 8n),

TODO

whence 2 - =2 2

=1 mjCr + n r=l 0' (- r) '

provided that Xi, x, x are unequal, so that < ' Xr)+0.

* Journal fiir Math. ii. (1827), pp. 101-181; iii. (1828), pp. 160-190
[Oeuvres, i. (Christiania, 1839), pp. 141-2.52].

%
% 443
%

Now, if we put X (x 8)n + 8n)/4> (s), qua function of .r, into partial
fractions, the result is

3 r=\

where,.= lim .r ( Sm + Sw) -p

= x x . 8m + 8n) lim (x - av)/ ( )

by Taylor's theoi-em.

:=.c, x,8m + 8n)l(f)' x \

3 3

Putting x=0, we get 2 S.r,./j . = 0, i.e. 2 62,. = 0.

r=l r=l

That is to say, the sum of the parameters of the points of
intersection is a constant independent of the position of the line.

Vary the line so that all the points of intersection move off to
infinity (no two points coinciding during this process), and it is
evident that 21 + 22 + 23 is equal to the sum of the parameters when
the line is the line at infinity; but when the line is at infinity,
each parameter is a period of p (2) and therefore 21 + 22 + 23 is a
period of (2).

Hence the sum of the parameters of three collinear points on the cubic
is congruent to zero. This result having been obtained, the
determinantal form of the addition-theorem follows as in\hardsectionref{20}{3}.

2032. The constants e-, 62, 63.

It will now be shewn that J (coi), j((Oo), i ifOs), (where 0)3= - co -
coo), are all unequal; and, if their values be e, e, e, then e-,
e.2, e-, are the roots of the equation 4 * - g t - g O.

First consider < ' (\&)i). Since ' z) is an odd periodic function, w e
have \&>' i i) = ~ i ' (- i) = - i ' (2<yi - i) = - < ' (ft)i), and so
fr''(\&>i) = 0.

Similarly < ' w. = ( ' (0)3) = 0.

Since < ' z) is an elliptic function whose only singularities are
triple poles at points congruent to the origin, < ' z) has three, and
only three \hardsubsectionref{20}{1}{3}), irreducible zeros. Therefore the only zeros of
f' z) are points congruent to

COj, \&),, 6)3.

Next consider i z) - ei. This vanishes at (o and, since < ' wy) = 0,
it has a double zero at (o . Since i z) has only two irreducible
poles, it follows from\hardsubsectionref{20}{1}{3} that the only zeros of ( z) - ei are
congruent to coj. In like manner, the only. zeros of z)- e-., J(z) -
es are double zeros at points con- gruent to (1)2, ois respectively.

Hence i 62 = s- For if gj = e, then (z) - e has a zero at w., which
is a point not congruent to Wj.

Also, since ' (z) = 4 (z) - gz (z) - g and since '' (z) vanishes at
coj, to,, 6)3, it follows that 4 =* (z) - g2 (z) - g vanishes when p
(z) = e, 62 or e .

That is to say, e, 62, e are the roots of the equation

4>f-g2t-gs = 0.

%
% 444
%

From the well-known formulae connecting roots of equations with their

coefficients, it follows that

e, + e. + 3 = 0,

Example 1. When g and rg are real and the discriminant g.? - %1g- is
positive, shew that ex, e2 s are all real; choosing them so that e-
>eo> 63, shew that

. 00 0,1=1 t -g t-g- ~ dt,

and 0)3 = - 1 / ' g:i+g.it - 4fi) ~ dt,

so that 0)1 is real and 003 a pure imaginary.

Example 2. Shew that, in the circumstances of example 1, p z) is real
on the peri- meter of the rectangle whose corners are 0, 0)3, wi +
cos, coi.

20*33. T/ie addition of a half-period to the ai-gument of p (z). From
the form of the addition-theorem given in\hardsubsectionref{20}{3}{1}, we have

,(,)., +,K)=H|-l;!- r.

3

and so, since J'-(z) = 4 n P z) - er,

r=l

we have (-'+o,.) = ' <";!l' i''"'" - W- .

 J Z) - Bi

3 on using the result 2 6 =0;

r=l

this formula expresses (s + coi) in terms of (2).

Example 1.' Shew that

  (|6,i)=ei± (61-62) (61-63) - 

Example 2. From the formula for z + at. combined with the result of
example 1, shew that

  ( CO, + Q).,) = 61 + (61 - 62) (61 - 63) .

\addexamplecitation{Math. Trip. 1913.}

Example 3. Shew that the value of '(2) '(z-t-wi) ' (s-l-wo) (2 + W3)
is equal to the discriminant of the equation 4t - got - g3=0.

[Differentiating the result of\hardsectionref{20}{3}3, we have

r (2 -f l) = - (61 - 62) (61 - 63) ' (2) if> (2) - 6, -2;

from this and analogous results, we have

f (2) ' (2 + Wi) g)' (2 + CO2) p' (2 -I- CO3)

= (61 -62)2 (62 -63)- (63-61)2 (2) n p(2)-6, -2

r=l = 16(6l-62)M 2- 3)M 3-e,)

which is the discriminant g - Tig- in question.]

%
% 445
%

Example 4. Shew that, with appropriate interpretations of the
radicals,

 ' (icoi)= -2 e,-e.,) e,-e )]i l ei-e ) + e -e )h .

\addexamplecitation{Math. Trip. 1913.}

Example 5. Shew that, with appropriate interpretations of the
radicals, p 2z) - e p 2z) - 63 * + W (22) - 3 * P (22) - e

+ iP m~e, i ip 2z)-e.2 = p z)-p 2z).-

20'4. Quasi-periodic functions. The function* (z).

We shall next introduce the function (z) defined by the equation

dz - ' ' coupled with the condition lim [ z) - z~ ]=0.

Since the series for p z) - z~' converges uniformly throughout any
domain from which the neighbourhoods of the points f 'm,n are
excluded, we may integrate term-by-term \hardsectionref{4}{7}) and get

l;iz)-z- = -n z)-z-,dz J

= -S' r z-n,,nr'- 7n%]dz, 711, n J

andso ( )= +N !\ \ + +

The reader will easily see that the general term of this series is

0( n i,;-= ) as in,,,, - co;

and hence (cf\hardsectionref{20}{2}), z) is an analytic function of z over the Avhole
sr-plane except at simple poles (the residue at each pole being -H 1)
at all the points of the set i m, ti- lt is evident that

Z m.n \ Z -T i-m,n iii,n I2, j,j)

and, since this series consists of the terms of the series for (z),
deranged in the same way as in the corresponding series of\hardsubsectionref{20}{2}{1}, we
have, by\hardsubsectionref{2}{5}{2},

  -z) = - (z),

that is to say, z) is an odd function of z.

* This function should not, of course, be confused with the
Zeta-funetion of Eieraann, discussed in Chapter xni.

t The symbol il', is used to denote all the points fl,,,i with the
exception of the origin (cf.\hardsectionref{20}{2}).

%
% 446
%

Following up the analogy of J 20-222, we may compare z) with the
function cot 2 defined by the series 2~'+ 2' (2- i7r)~' + (m7r)~i,
the equation -7; cot 2 = - cosec 2

cori-e ponding to f (2) = - (2).

2041 . 77/ e quasi-periodicity of the function i z).

The heading of\hardsectionref{20}{4} was an anticipation of the result, which will
now be proved, that i z) is not a doubly-periodic function .of z; and
the effect on z) of increasing z by 2( i or by 2w.. will bo
considered. It is evident from\hardsubsectionref{20}{1}{2} (III) that z) cannot be an
elliptic function, in view of the fact that the residue of z) at every
pole is + 1.

If now we integrate the equation

ip z- 2\&)i) = (j) z\

weget ( +2a,,)=r(~0 + 277i,

where 27 1 is the constant introduced by integration; putting z= - ai
, and taking account of the fact that z) is an odd function, we have

In like manner, z + 2\&)o)= z)- 2??2,

where 1-1 = K ( 2)-

Example 1. Prove by Liouville's theorem that, if A--i-?/-|-2 = 0, then

(Frobenius u. Stiekelberger, Journal fur Math, lxxxviii.)

[This result is a pseudo-addition theorem. It is not a true
addition-theorem since

C i- ), C y\ C (2) ai'e not algebraic functions of ( (x), f (.?/), C
(2)-]

Example 2. Prove by Liouville's theorem that

2 I 1 X) p (x)

' 1 PLy) PHy)

1 (2) (2)

1 p x) p' x)\ \ = ax+y + z)-i x)-( jj)-C .z).

1 (y) F(i )

1 (2) F'(2) I Obtain a generalisation of this theorem involving n
variables.

\addexamplecitation{Math. Trip. 1894.}

20'411. Tlie relation hetiueen ij and rj.,.

We shall now shew that

1 .

To obtain this result consider i z)dz taken round the boundary of a
cell. There is one pole of i z) inside the cell, the residue there
being -I- 1.

Hence 1: (z) dz = 27ri.

J r

%
% 447
%

Modifying the contour integral in the manner of\hardsubsectionref{20}{1}{2}, we get

 -Ki = K z) - r( + 2a),)i dz - ( ) - ( + 2 o (

= - 27?2 rf + 27/1 fZt

and so 27rt = - iVi i + ' Vi -z,

which is the required result.

20-42. The function a z).

We shall next introduce the function o z), defined by the equation

j- \ og(7 z)=l; z)

coupled with the condition lim [a z)!z] = 1.

On account of the uniformity of convergence of the series for t, z),
except near the poles of z), we may integrate the series term-by-term.
Doing so, and taking the exponential of each side of the resulting
equation, we get

a(z) zn'\ [ l- ' ' ( z z

( ) = n' l- -)exp

+

--m.n 2S2T.

the constant of integration has been adjusted in accordance with the
condition stated.

By the methods employed in §§ 202, 20-21, 20-4, the reader will easily
obtain the following results :

(I) The product for a (z) converges absolutely and uniformly in any
bounded domain of values of z.

(II) The function a(z) is an odd integral function of with simple
zeros at all the points flm,n'

The function a (z) may be compared with the function sin z defined by
the product

the relation -j- log sin z = cot z corresponding to -i- log o- (z) =
(z).

20-421. The quasi-periodicity of the function cr z).

If we integrate the equation

  z + 2(o,) = !: z +2v we get a (z + 2(Wi) = ce- ' -a (z),

where c is the constant of integration; to determine c, we put z = -
coi, and

then

a (coi) = - ce~-''''">o- (coi).

z

J = - CO (\

%
% 448
%

Consequently c = - e-'''"',

and o- (z + 2\&)j) = - e2''i( +'-i) a z).

In like manner (t(z + 2<i)o) = - g i 'z-i-'xj) a (z).

These results exhibit the behaviour of (t z) when z is increased by a
period of (z).

If, as in\hardsubsectionref{20}{3}{2}, we wTite ais = - co - ay.,, then three other
Sigma-functions are defined by the equations

a, z) = e-'i'-'o- (z + \&),.) V (wr) (? = !, 2, 3).

The four Siguia-functions are analogous to the four Theta-functions
dis- cussed in Chapter XXI (see\hardsectionref{21}{9}).

Example 1 . Shew that, if m and n are any integers,

<r z + 277i(Oi + 2na>2) = ( - )"' + " a- (z) exp 27nrii + 2nr).>) z +
2m-r) <3>i + 4? ?i;j,a)2 + 2n' r\ < ia>, and deduce that r]xi>ii-
r]-i<Ji\ is an integer multiple of \ ni.

Example 2. Shew that, if 5' = exp iriu)-!! o>i)i so that y < 1, and if

then / (2) is an integi-al function with the same zeros as (t z) and
also F z)la- z) is a doubly-periodic function of 2 with periods 2a)i,
2a).2-

Example 3. Deduce from example 2, by using Liouville's theorem, that

Example 4. Obtain the result of example 3 by expressing each factor on
the right as a singty infinite product.

20"5. Formulae expressing any elliptic function in terms of
Weie7'strassian functions ivith the same periods.

There are various formulae analogous to the expression of any rational
fraction as (I) a quotient of two sets of products of linear factors,
(II) a sum of partial fractions; of the first type there are two
formulae involving Sigma- functions and Weierstrassian elliptic
functions respectively; of the second type there is a formula
involving derivates of Zeta-functions. These formulae will now be
obtained.

20*51. The expression of any elliptic function in terms of z) and ( >'
z).

Let f z) be any elliptic function, and let i z) be the Weierstrassian
elliptic function formed with the same periods 2\&)i, 2\&)o.

We first ite

fiz) = I Uiz) +/(- z)] + \ [ f z) -f - z)\ Wizfr] ' z).

%
% 449
%

The functions

f z) +/(- z\ [f z) -/(- )] Wi )]-' are both even functions, and they
are obviously elliptic functions when/( ) is an elliptic function.

The solution of the problem before us is therefore effected if we can
expi ess any even elliptic function < (2), say, in terms of i ) (z).

Let a be a zero of (f> (z) in any cell; then the point in the cell
congruent to - a will also be a zero. The irreducible zeros of (z) may
therefore be arranged in two sets, say a, a, ...an and certain
points congruent to - a-,

 2 >    - Cln 

In like manner, the irreducible poles may be arranged in two sets, say
hi, b.., ... bn, and certain points congruent to -61, - 60, ... - 6 .
Consider now the function*

1 fi H')( )-iP("r)

d> z)-, = i iiO Z) - ) br)

It is an elliptic function of z, and clearly it has no poles; for the
zeros of (b (z) are zeros f of the numerator of the product, and the
zeros of the denominator oi" the product are polesf of 4>(z).
Consequently by Liouville's theorem it is a constant, A, say.

Therefore < (.) = A H | ? 1,

that is to say, <f) (z) has been expressed as a rational function of J
(z).

Carrying out this process with each of the functions

f z) +fi-z), f(z) -f -z) ip\ z) -\

we obtain the theorem that any elliptic function f (z) can be
expressed in terms, of the Weierstrassian elliptic functions (z) and
p' z) luith the same periods., the expression being rational in z) and
linear in < ' z).

20"52. The expression of any elliptic function as a linear combination
of Zeta functions and their derivates.

Let f z) be any elliptic function with periods 2coi, 2(iJo: Let a set
of irreducible poles of f(z) be i, c/o, ... a, and let the principal
part \hardsubsectionref{5}{6}{1}) off z) near the pole a be

 k,i . Ck,2 Ck, rje

z-ajc (z- a f '" z- ttkY '

* If any one of the points a,, or h is congruent to the origin, we
omit the corresponding factor ii> (2) - .' (rt,.) or J ( ) - (\&,.).
The zero (or pole) of the product and the zero (or pole) of (j) [z) at
the origin are then of the same order of multiplicit.y. In this
product, and in that of\hardsectionref{20}{5}o, factors corresponding to multiple
zeros and poles have to be repeated the appropriate number of times.

f Of the same order of multiplifity.

W. M. A, 29

%
% 450
%

Then we can shew that

f z) = .4, + i \ c,,, 2 - ( ) -Ct,,r ( - "X") + . c - l (.

k:

where A., is a constant, and (z) denotes -r- z).

Denoting the summation on the right by F z), we see that F z +
-2co,)-F z)= I 27J,Ck,,

k = l

by\hardsubsectionref{20}{4}{1}, since all the derivates of the Zeta-functions are
periodic.

n

But S C/c i is the sum of the residues of y'(2 ) at all of its poles
in a cell,

k = l

and is consequently \hardsubsectionref{20}{1}{2}) zero.

Therefore F(z) has period 2\&)i, and similarly it has period 2coo; and
so f z) - F (z) is an elliptic function.

Moreover F z) has been so constructed that f z) - F z) has no poles at
the points nj, o, ... a; and hence it has no poles in a certain cell.
It is consequently a constant, A.,, by Liouville's theorem.

Thus the function f z) can be expanded in the form

A=is=i(s - i; ! This result is of importance in the problem of
integrating an elliptic function f(z) when the principal part of its
expansion at each of its poles is known; for we obviously have

f(z)dz = A.z+ 2 k=i

Ck,i\ og a (z - ak)

where C is a constant of integration.

Example. Shew by the method of this article that

and deduce that

where C is a constant of integration.

2053. TJie expression of any elliptic function as a quotient of Sigma-
fiinctions.

Let f z) be any elliptic function, with periods 2\&)i and 2a)2, and
let a set of irreducible zeros of /( ) be a,, a, ... an. Then
\hardsubsectionref{20}{1}{4}) we can choose a

%
% 451
%

set of poles bi,bo, ... hn such that all poles 0 /(2) are congruent to
one or other of them andf

a + a2+ ... + an = bi + b.2 + ... +bn.

Consider now the function

  (r(z- ar)

This product obviously has the same poles and zeros as f z); also the
effect of increasing z by 2(0j is to multiply the function by

  exp [2771 z - g,.),.=1 exp 277i ( - br)] The function therefore has
period 2\&)i (and in like manner it has period 2\&)2), and so the
quotient

is an elliptic function with no zeros or poles. By Liouville's
theorem, it must be a constant, A- say.

Thus the function/' (2 ) can be expressed in the form

r=l(T Z-b,)

An elliptic function is consequently determinate (save for a
multiplicative constant) when its periods and a set of irreducible
zeros and poles are known. Example 1. Shew that

Example 2. Deduce by difFerentiation, from example 1, that

and by further differentiation obtain the addition-theorem for (z).

II

2 b,., shew that

=1

I (r ar-bi)(r(ar-b. ...(T ar-bJ \,.=1 o- (a,. - ai) o- (a,. - a J . .
. . . . o- (a - a ) ' the * denoting that the vanishing factor o- (a -
a ) is to be omitted. Example 4. Shew that

  z)-e,. = a/ z)la' z) (r=l,2, 3).

[It is customary to define g? z) - e to mean o-,. (2)/(r (2), not -
o-,. (s)/'o- (2).] Example 5. Establish, by example I, the '
three-term equation,' namely,

d z + a) a- z - a) a- b + c) a- b - c) + <T z + b) a- z - b) a- c + a)
a c - a)

+ 0- z + c)(t z-c) (r a + b)a a-b) = 0.

t Multiple zeros or poles are, of course, to be reckoned according to
their degree of multi- plicity; to determine b, h-i, ...b, we
choose 6i, bo,, ... b, \ i, 6,/ to be the set of poles in the cell
in which ai, a-i, ...a lie, and then choose \&, congruent to 6, in
such a way that the required equation is satisfied.

29-2

Example 3. If 2 a,.= 2 6,., shew that

r-. r=]

%
% 452
%

[This result is due to Weierstrass; see p. 47 of the edition of his
lectures by Schwarz.] The equation is characteristic of the
Sigma-function; it has been proved by Halphen, Fonctions ElUptiques,
I. (Paris, 1886), p. 187, that no function essentially diflferent from
the Sigma-function satisfies an equation of this type. See p. 461,
example 38.

20"54. The connexion between ani/ hvo elliptic functions with the same
periods.

We shall now prove the important result that an algebraic relation
exists bettveen any ttuo elliptic functions, f z) and <f)(z), with the
same periods.

For, by\hardsubsectionref{20}{5}{1}, we can express f(z) and (z) as rational functions of
the Weierstrassian functions (z) and ' (z) with the same periods, so
that

f(z) = R, [p (z), io' (z)], < (z) = R, (z), ' (z)], where Ri and jB,
denote rational functions of two variables.

Eliminating (z) and ' (z) algebraically from these two equations and
'Uz) = 4>f' z)-go\ \& z)-g

we obtain an algebraic relation connecting f z) and (f> (z); and the
theorem is proved.

A particular case of the proposition is that every elliptic function
is con- nected with its derivate by an algebraic relation.

If now Ave take the orders of the elliptic functions f(z) and 4> z) to
be m and n respectively, then, corresponding to any given value of/( )
there is (§ 2018) a set of m iiTeducible values of z, and consequently
there are m values (in general distinct) of cf) (z). So, corresponding
to each value off, there are ni values of cf) and, similarly, to each
value of (f) correspond n values of /.

The relation between f(z) and cf) (z) is therefore (in general) of
degree m in (f) and n in f

The relation may be of lower degree. Thus, iff(z) = p (z), of order 2,
and (j) (2) = 2 of order 4, the relation is/- = cf).

As an illustration of the general result take f(z) = o z), of order 2,
and (j> (z) = y (z), of order 8. The relation should be of degree 2 in
(/> and of degree 3 in f; this is, in f;ict, the case, for the
relation is < - = 4/'-' - .2/- 3.

Example. If u, v, ic are three elliptic functions of their argument of
the second order with the same periods, shew that, in general, there
exist two distinct relations which are linear in each of l, v, w,
namely

A uvtv+Bvw + Cicu + Btiv + E u + F r + G iv+ H =0, A'uvw + B'vw +
C'lvti + D'u v + E'u + F'v + G'w + H' = 0, where 1, B, . . ., IT'
are constants.

20"6. On the iy\ tegration of [aoX*' + a af' + Qa x + 4ea x + a ~ .

It will now be shewn that certain problems of integration, which are
insoluble by means of elementary functions only, can be solved by the
intro- duction of the function < z).

%
% 453
%

Let ttoX + a af + Qa. x- + a x + a =f(x) be any quartic polynomial
which has no repeated factors; and let its invariants* be

g = ciocti - 4aia3 + 3a.

gs = a aoCti + 2aia2a3 - ai - a a - a-ca .

/'* -1

Let z = /(t)] 'dt, where a-'o is any root of the equation/(a;) =;
then,

if the function z) be construe tedf with the invariants g and g, it
is possible to express x as a rational function of <,p z; go, g ).

[Note. The reason for assuming tbat/(.r) has no repeated factors is
that, when/(.r) has a repeated factor, the integration can be eftected
with the aid of circular or logarithmic functions only. For the same
reason, the case in which aQ = ai = need not be considered.]

By Taylor's theorem, we have

f t) = 4 3 t - X,) + QA. (t - x,y + 4>A, (t - x,y + 0 ( - 'oY,

(since / (xq) = 0), where

A 2 = UoXq + 2aiXQ + a.,, As = o V'' + ' UiXq- + a.2Xo + ttj. On
writing (t - Xa)~ = t, (x - Xq)~'- =, we have

2 = 1 4.43T= + QA,T + 4A,T + Ao] ~ irfr.

' s

To remove the second term in the cubic involved, write

r = Ar(cr-h . =A,- (s-iA,), and we get

r z ['ia'- SA. - A,A,)a- 2A,A.,A,-A, -AoAs')]~ -d(T.

. s

The reader will verify, without difficulty, that

3i4.;- - 4 1 3 and iA- A. A - A - A A

are respectively equal to g. and g-, the invariants of the original
quartic, and so

s=io z;go,g.;).

Now X = Xq + Az \ s - A. ~',

and hence x = Xo + \ f' ( o) z ', 92, Qi) - -hf" ( 'o) ~S

so that X has been expressed as a rational function of ( (z; g.,, g .

* Burnside and Pauton, Tlieonj of Equations, ii. p. 113. + See\hardsubsectionref{21}{7}{3}.

J This substitution is legitimate since A3 + O; for the equation -13 =
involves /(x) = having x - Xq as a repeated root.

%
% 454
%

This formula for cc is to be regarded as the integral equivalent of
the relation

z

Example 1. With the notation of this article, shew that Example 2.
Shew that, if

I a

where a is ani/ constant, not necessarily a zero of f x), and / (x) is
a quartic polynomial with no repeated factors, then

,,, /( ) F( )+i/'W ( -)- V/"( ) +A/( )/'"( )

the function p (z) being formed with the invariants of the quartic
/(:c).

(Weierstrass.)

[This result was first published in 1865, in an Inaugural-dissertation
at Berlin by Biermann, who ascribed it to Weierstrass. An alternative
result, due to Mordell, Messengery XLIV. (1915), pp. 138-141 is that,
if

\ C' v y dx - X dy

where /(:f, y) is a homogeneous quartic whose Hessian is h x, y), then
we may take

x=ap' z) ff+ip z)f,+ h

y-hp- z) lf-\ p z)f,-Ua, where /and h stand for /(a, 6) and A (a, 6),
and suffixes denote partial dMferentiations.] Example 3. Shew that,
with the notation of example 2,

(. M /w/( ) +/( ) I /'( ),rw

 'x-af 4(.r-a) 24 '

and F( )= - 1, 3 - rr l /( ) - j/- 3 +/ U /(-) *

* ' x-af A x-a)-y-' ' x-af A x-ay] ' '

20"7. The uniformisation* of curves of genus unity. The theorem of §
20*6 may be stated somewhat differently thus : If the variables x and
y are connected by an equation of tlie form y"- = a id + a-i a? + a x
+ a x + a,

then they can be expressed as one-valued functions of a variable z by
the

equations, .,,, . .

  x=x,+\ r x,)\ p z) - j-j" x,yr

y = -U' o)p' z) p z)-i-J" xo)]

where f x) = a x + 4aia + Qa.,x + 'ia x -t- a, Xq is any zero of f
x), and the function < z) is formed with the invariants of the quartic
; and z is such that

z=r [f t)]- dt.

* This term employs the word uniform in the sense one-valued. To
prevent coufusion with the idea of uniformity as explained in Chapter
in, tliioughout the present work we have used the phrase 'one-valued
function' as being preferable to 'uniform function.'

!07]

%
% 455
%

It is obvious that y is a two-valued function of x and a; is a
four-valued function of y; and the fact, that x and y can be
expi'essed as one-valued functions of the variable z, makes this
variable z of considerable importance in the theory of algebraic
equations of the type considered; z is called the uniformising vm
iahle of the equation

y" = a x* -f- 4ai.'C -f a. x" a x + a .

The reader who is acquainted with the theory of algebraic plane curves
will be aware that they are classified according to their deficiency/
or genus*, a number whose geometrical significance is that it is the
difterence between the number of double points possessed by the curve
and the maximum number of double points which can be possessed by a
curve of the same degree as the given curve.

Curves whose deficiency is zero are called tmicursal curves. If/ (x,
y) = is the equation of a unicursal curve, it is well known t that x
and ?/ can be expressed as rational functions of a 'parameter. Since
rational functions are one-valued, this parameter is a uniformising
variable for the curve in question.

Next consider curves of genus unity; let /(.r, ?/) = be such a curve;
then it has been shewn by CIebsch| that x and y can be expressed as
rational functions of and ?; where ry' is a polynomial in | of degree
three or four. Hence, by\hardsectionref{20}{6}, and r\ can be expressed as rational
functions of (2) and ' z\ (these functions being formed with suitable
invariants), and so x and y can be expressed as one-valued (elliptic)
functions of z which is therefore a uniformising variable for the
equation under consideration.

When the genus of the algebraic curve /(.r, 3/) = is greater than
unity, the uniformi- sation can be effected by means of what are known
as automorphic functions. Two classes of such functions of genus
greater than unity have been constructed, the first by Weber,
Oottinger Nach. (1886), pp. 359-370, the other by Whittaker, Phil.
Trans, cxcii. (1898), pp. 1-32. The analogue of the
period-parallelogram is known as the 'fundamental polygon.' In the
case of Weber's functions this polygon is ' mviltiply-connected,' i.e.
it consists of a. region containing islands which have to be regarded
as not belonging to it; whereas in the case of the second class of
functions, the polygon is ' simply-connected,' i.e. it contains no
such islands. The latter class of functions may therefore be regarded
as a more immediate generalisation of elliptic functions. Cf. Ford,
Introduction to theory of Auto- morphic Functions., Edinburgh Math.
Tracts, No. 6 (1915).

REFERENCES.

K. Weierstrass, Werke, i. (1894), pp. 1-49, 11. (1895), pp. 245-255,
257-309.

C. Briot et J. C. Bouquet, Theorie des fonctions elliptiques. (Paris,
1875.)

H. A. ScHWARZ, Formeln und Lehrsdtze zuni Gehrauche der elliptischen
Funktionen. Nach

Vorlesungen und Aufzeichnungen des Herrn Prof. K. Weierstrass.
(Berlin, 1893.) A. L. Daniels, 'Notes on Weierstrass' methods,'
American Journal of Math. vi. (1884),

pp. 177-182, 253-269; vn. (1885), pp. 82-99. J. LiouviLLE (Lectures
published by C. W. Borchardt), Journal fiir Math. Lxxxviii.

(1880), pp. 277-310. A. Enneper, ElliptiS'-he Funktionen. (Zweite
Auflage, von F. Miiller, Halle, 1890.) J. Tannery et J. Molk,
Fonctions Elliptiques. (Paris, 1893-1902.)

* French genre, German Geschlecht.

t See Salmon, Higher Plane Curves (Dublin, 1873), Ch. 11.

X Journal f fir Math. lxiv. (1865), pp. 210-270. A proof of the result
of Clebsch is given by Forsyth, Theory of Functions (1918), § 248. See
also Cayley, Proc. London Math. Sac. iv. (1873), pp. 347-352 [Math.
Papers, vni. pp. 181-187].

%
% 456
%

Miscellaneous Examples.

1. Shew that

9 'ry)- z-y)=-\&' z) ' y) ip z)- y)]-\

2. Prove that

where, on the right-hand side, the subject of diflferentiation is
symmetrical in 2, y, and w.

\addexamplecitation{Math. Trip. 1897.} ti. Shew that

n -y) r\ y- ' r'o - )

\ 1

55 2

r"(2-y) r"(3/-"') r'( '-2)

  (-- y) (y- O (w'- ) 1 1 1

\addexamplecitation{Trinity, 1898.}, dy

4. If y= (2)-ei, .y'=

shew that y is one of the vahies of

if \ d' \ i 1 i

|/(y-4 2logyj +(ei-e2)(ei-e3)| 

\addexamplecitation{Math. Trip. 1897.}

5. Prove that

2 W (2) - (P (y) - P ( ') ' IP (y + ) - 4* (P (y - ' ) - )* = O'

where the sign of summation refers to the three arguments 2, y, v.;
and e is any one of the

roots Ci, Co, So.

\addexamplecitation{Math. Trip. 1896.}

6. Shew that

P' z + <oi)\ (P (i i)-£( i)r

P'( ) ........

\addexamplecitation{Math. Trip. 1894.}

7. Prove that

P (22) - p (a,,) = IP' (2) -2 [ (2) - (|a ) 2 (2) - (a>2 +*o,i) .

\addexamplecitation{Math. Trip. 1894.}

8. Shew that

p u + v)iO(u-v) =

iP (u) p (v) + ig l+giilMllMl

 p u)-p v)

,,.V2

\addexamplecitation{Trinity, 1908.}

9. If p(u) have primitive periods 2a)i, 2( 2 and f u) = p u) - p co- '
, while g:)i (?0 And/i (m) are similarly constructed with periods
2a)i/yi and 2a)2, prove that

Pi ii) = P ti)+"'2 p u + '2ma>iln) -p 2mcoi/7i),

m=l

n-1 n f u + 2ma)iin)

 and /i (w) = " i 

n f 2mailn)

m=r

(Math. Trip. 1914; the first of the formulae is due to Kiepert,
Journal fur Math. Lxxvi. (1873), p. 39.)

%
% 457
%

10. If .r = p u + a), y = <p u-a),

where a is constant, shew that the curve on which (.r, y) lies is

 xy + ex + cy + g. f = 4 ( +y + c) [cxy - lg \ where c = p(2a).

(Burnside, Messenger xxi.)

11. Shew that

2 "3 (.0 - Zg.£" u) +gi = 21 ' u)+g,Y.

12. If z=r xi + 6cx + e -)-idx,

verify that x=,

the elliptic function being formed with the roots -c, c + e), (c - e).

\addexamplecitation{Trinity, 1909.}

\addexamplecitation{Trinity, 1905.}

F(i/)i z)-<P y) " '' (2)- (y)

13. If m be any constant, prove that

1 / e' (2)-S'(2')fp(3)(;2, e MS'(2)-S'(y) y

-I? IT

 P( )- r P(y)-er '

where the summation refers to the values 1, 2, 3 of;; and the
integrals are indefinite.

\addexamplecitation{Math. Trip. 1897.}

1 4. Let R x) = Ax + Bx" + Cx"- + Dx + ',

and let |= (.r) be the function defined by the equation

. '=| ( ) - /, where the lower limit of the integral is arbitrary.
Shew that

20' ( ) \ <j>' a- y) + ' ia ) 0' (a-y)+< '(a) \ < '(a+y)-< '( )

( (A'+2/)-0(a) < (o+ )-0(a) (i> a-y)-<f> a) (]i a+y)-(j) x)

0'( -y)-0'(- ')

c ) a-y)-cl) x) '

[Hermite, Proc. J/ / Congress (Chicago, 1896), p. 105. This formula is
an addition-formula which is satisfied by every elliptic function of
order 2.]

15. Shew that, when the change of variables

is applied to the equations

r + .(l+i> ) + l = 0, a.- - = 0, they transform into the similar
equations

Shew that the result of performing this change of variables three
times in succession is a retiirn to the original variables, r; and
hence prove that, if and r) be denoted as functions of by -Ei u) and F
u) respectively, then

where A is one-third of a period of the functions E u) and F u). Shew
that E (u) = -P it; g2, g ),

where 92 = '2p + - p\ 93= -'i-- qP - P -

(De Brun, Ofversigt af K. Vet. Akad., Stockholm, Liv.)

%
% 458
%

16. Shew that

[chap. XX

aud

\&' =

F ( )=

2o- (2 + Wj) (T (2 + <t>2) <T (Z- <i>i - 0)9) O- (s) C (ft)i) (T (0)2)
O" (©i +0)2)

60- (z + a) a z-a)(r z + c) a z - c)

(r*(2)<r2(a)o-2(c)

where

17. Prove that

+ \&'(a-h) C z-a)-a'-b) + C <')-ah)] IS. Shew that

\addexamplecitation{Math. Trip. 1913.}

\addexamplecitation{Math. Trip. 1895.}

2l (M)- (t ) gJ(i;)- (M')J ' ''

\addexamplecitation{Math. Trip. 1910.}

19. Shew that

C( i) + C,W2) + C("3)-r(W] + *2 + 3)

2 (? i) - g> (M2) ( 2) - 9 n )] P ( 3) - P ( <l)

F ( l) W ( 2) - (Ws) + ' ( -2) P ( 3) - ( l) + i;-'' (%) ( l) - ( 2)

\addexamplecitation{Math. Trip. 1912.}

20. Shew that

a x+y+z) o- x-y) <r 1/ - z) a z - x) 1 I 1 P (-0 ' i- )
a3(.r)cr3(y)cr3(2) 2,

1 p z) p' z) Obtain the addition-theorem for the function p z) from
this result.

21. Shew by induction, or otherwise, that

1 ( i) '( i)... <"- )(2i)

\ /\ xiM(n-l)j I 2 ! ... ?i :

, O- (20 + 2 i + . . . + 2n) no- (2; - Z )

'<T z )...a"*' z )

' 1 PizJ P'(2 )...p-1)(2 )

where the product is taken for pairs of all integral values of X and
/i from to 71, such

that X < /I.

(Frobenius u. Stickelberger*, Journal fiir Math, lxxxiii. (1877), p.
179.)

22. Express

1 p x) P( ) p' x) I

1 Piy) F(y) r(3') I 1 (2) P(2) F( ) I 1 p u) p( ) f (w);

as a fraction whose numerator and denominator are products of
Sigma-functions.

* See also Kiepert, Journal filr Math. L.xs.yi. (1873), pp. 21-33;
Hermite, Journal fiir Math. Lxxxn. (1877), p. 346.

%
% 459
%

Deducethat if a = p(.r), 8 = p y), y=p z), 8 = p (ii), where x +
9/+z+u = 0, then ( 2 - 63) (a - ei) (/3 - ei) (y - e ) 8 - ej) + (es -
ei) (a - e ) (/3 - e,) (7 - e ) (S - 63) *

+ ( 1 - ea) (a - 63) (/3 - e ) (y - 63) (8 - 63) * = e - 63) ( 3 - ei)
(ej - eg)-

\addexamplecitation{Math. Trip. 1911.}
23. Shew that

2C(2 )-4aiO = |J' ''

3t(32i)-9C(20 =

\addexamplecitation{Math. Trip. 1905.}

24. Shew that

and prove that a- nu)j cr ( ) "' is a doubly-periodic function of ?6.

\addexamplecitation{Math. Trip. 1912.}

25. Prove that

a- (z-2a + b) a- (z -2b + a)

  z-a)-( z-b)-C(a-b) + C 2a-2h)-

a- (26 - 2a) o- (2 - a) o- ( - 6) '

\addexamplecitation{Math. Trip. 1895.}

26. Shew that, if Sj + S2 + 3 + i = 0, then

 2C (2,.) = 3 2t (2,.) SP (,.) + 2 ' (.-.), the summations being
taken for r = l, 2, 3, 4. \addexamplecitation{Math. Trip. 1897.}

27. Shew that every elliptic function of order n can be expressed as
the quotient of two ex2)ressions of the form

aiPiz + b) + a,p' z + b) + ...+a,,p( -')(z + b),

where b, rtj, 02, ... a are constants. (Painleve, Bulletin de la Soc.
Math, xxvii.)

28. Taking e >e2>e;, p a>) = ei, p(co') = es, consider the values
assumed by

C( )-Mf ( ')/ '

as u passes along the perimeter of the rectangle whose corners are -
co,, w + w', -a> + w

\addexamplecitation{Math. Trip. 1914.}

29. Obtain an integral of the equation

1 d w,, ., - -Ty = 6 (2) + 36

in the form

dzla z)a c) ' \ b-2p c) ' fj'

where c is defined by the equation

(62- 39 2) (0) = 3 (63 + 3).

Also, obtain another integral in the form

 f j exp -sf(ai)-2aa2),

where ( i) + P ( 2) = \&, F ( i) + P ( 2) = 0,

and neither ai + a2 nor ai-a-i is congruent to a period.
\addexamplecitation{Math. Trip. 1912.}

%
% 460
%

30. Prove tliat

, .\ a z + Zi) (r z + Z 2) a- z + Z3) a- (z + Zj)

  '~ ' r 22 + i(2i + 22 + £3 + 24)

is a doubly-periodic function of z, such that

  (2) +5r (2 + coj) + (/ (2+ o).,) + <7 (2+ 0)1 + C02)

= - 20- | (20 + 23 - 2i - 24) or \ (23 + 2i - Zo - 24) (T J (21 + 22 -
23 - 24) .

\addexamplecitation{Math. Trip. 1893.}

31. If f z) be a doubly-periodic function of the third order, with
poles at 2 = Cj, 2 = 02, 2=C3, and if (2). be a doubly-periodic
function of the second order with the same periods and poles at 2 = 0,
: = 3, its value in the neighbourhood of 2 = being

( z) = + \ \ z-a) + <o z-af - ...,

z - a

prove that

iX-' /" (a) -/" (3) - \ /' ( ) +/' m 2</) ( 1) + / (a) -fm sXXi + 2ct>
c,) cj> (C3)| = 0.

\addexamplecitation{Math. Trip. 1894.}

32. If X (2) be an elliptic function with two poles aj, a, and if z,
z-j,, ... 22n be 2n constants subject only to the condition

Zl + Z2 + ...+Z. = 7l ai + Cto),

shew that the determinant whose ith row is

1, \ \ {Zi), X2(2,.), ... \ \ Zi), X,(2,), X(2i)X,(2i), X2 (2,) Xi
(2,), . .. X -M t) l (2f)

[where Xj (zi) denotes the result of writing 2 for 2 in the derivate
of X (2)], vanishes identically. \addexamplecitation{Math. Trip. 1893.}

33. Deduce from example 21 by a limiting process, or otherwise prove,
that

\ F z) P" z) ...p-')(2) =(-)"-Ml! 2!... ( -l)! 2cr(;m)/ a-(?0 ' .

p" z) r'(2) -P' H ) :

\ 2(01 "'1/ <"! ©i

P-1)(2) p)(2)...F""'K2) I

(Kiepert, Journal fur Math, lxxvi.)

34. Shew that, provided certain conditions of inequality are
satisfied,

  z)(T y) ' 2coi

where the summation applies to all positive integer values of m and n,
and j = exp (7ria)2/< i)

\addexamplecitation{Math. Trip. 1895.}

35. Assuming the formula

>)i s? 1 - 2o- cos - + q*" 2a), 2<x)i . nz ° tO] a-(2) = e ' . - sm
r- n 7 :r-r,,

prove that

(P (2)= - - + U- cosec2 2 - 2, o cos

when 2 satisfies the inequalities

-2li( )<R( )<2R(?A, \ la>l/ \ i(Oi/ V(Oi/

\addexamplecitation{Math. Trip. 1896.}

%
% 461
%

36. Shew that if 2 is- be any expression of the form 2ma)i + 2n(02 and
if

then X is a root of the sextic

,r - 55'2a;*-405'3A-3 - 5g. x - Sg.2gzX-bg = 0, and obtain all the
roots of the sextic. \addexamplecitation{Trinity, 1898.}

37. Shew that

where

/ . - )(. -.)ri..=-iiog:-|-;-f; i.og:jj >,

(Dolbnia, Darhoux' Bulletin (2), xix.) 38. Prove that every analjiiic
function/ (3) which satisfies the three-term equation

2 /(2 + a)/(2-a)/(6+c)/(6-c) = 0,

for general values of, 6, c and, is expressible as a finite
combination of elementary functions, together with a Sigma-function
(including a circular function or an algebraic

function as degenerate cases).

(Hermite, Fonctions elliptiqnes, i. p. 187.)

[Put 0=a = 6 = c=O, and then/(0) = 0; put 6 = c, and then /( -
6)+/(6-(x) = 0, so that/ (2) is an odd function.

If F [z) is the logarithmic derivate of /'(2), the result of
differentiating the relation with respect to 6, and then putting 6 =
c, is

Differentiate with respect to 6, and put 6 = 0; then /(. + a)/( -a) /'
(0)F \

 /( )/(a)F " -

If/' (0) "were zei'o, / ' z) would be a constant and, by integration,/
(2) would be of the form A exp (Bz+Cz ), and this is an odd function
only in the trivial case when it is zero.

If /' (0) 0, and we write F' (s)= - 4> (z), it is found that the
coefficient of a* in the expansion of

l2f z+a)f z-a)/ fiz)Y

is 6 \$ (z) - " (j), and the coefficient of a* in 12 /( ) * (a) - *
(2) is a linear function of * (2). Hence 4>" (2) is a quadratic
function of \$(2); and when we multiply this function by \$' (2) and
integrate we find that

 <!>' (2) 2 = 4 * (2) 3+ 12J * (2)j-'+ 12 \$ (2)4-46',

where A, B, C are constants. If the cubic on the right has no
re2:)eated factors, then, by\hardsectionref{20}{6}, <I> z) = z + a) + A, where a is
constant, and on integration

f z) = (r z + a) exp - Az'-Kz-L),

where K and L are constants; since/ (2) is an odd function a = K=0,
and

/(2) = (r(2)exp -iJ22-Z .

If the cubic has a repeated factor, the Sigma-function is to be
replaced (cf.\hardsubsubsectionref{20}{2}{2}{2}) by the sine of a multiple of z, and if the
cubic is a perfect cube the Sigma-function is to be replaced by a
multiple of 2.]

