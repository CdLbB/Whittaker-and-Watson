%
% 125
%
\chapter{The Expansion of Functions in Infinite Series}

\Section{7}{1}{A formula due to Darhoux*TODO.  Journal de Math. (3), ii. (1876), p. 271.}

Let $f(z)$ be analytic at all points of the straight line joining $a$
to $z$, and let $\phi(t)$ be any polynomial of degree $n$ in $t$.

Then if $0 \leq t \leq 1$, we have by differentiation
\begin{align*}
  \frac{\dd }{\dd t}
  \sum_{m=1}^{n} (-)^{m} (z-a)^{m} \phi^{(n-m)}(t) f^{(m)}(a + t(z-a))
  \\
  =
  -(z-a) \phi^{(n)}(t) f'(a + t(z-a))
  +
  (-)^{n} (z-a)^{n+1} \phi(t) f^{(n+1)}(a + t (z-a)).
\end{align*}

Noting that $\phi^{(n)}(t)$ is constant $= \phi^{(n)}(0)$, and integrating
between the limits $0$ and $1$ of $t$, we get
\begin{align*}
  \phi^{(n)}(0) \thebrace{ f(z) - f(a) }
  \\
  =&
  \sum_{m=1}^{n} (-)^{m-1} (z-a)^{m}
  \thebrace{
    \phi^{(n-m)}(1) f^{(m)}(z) - \phi^{(n-m)}(0) f^{(m)}(a)
  }
  \\
  &
  + (-)^{n} (z-a)^{n+1}
  \int_{n}^{1} \phi(t) f^{(n+1)}(a + t(z-a)) \dmeasure t,
\end{align*}
which is the formula in question.

Taylor's series may be obtained as a special case of this by writing
$\phi(t) = (t-1)^{n}$ and making $n\rightarrow\infty$.

%\begin{smalltext}
Example. By substituting $2n$ for $n$ in the formula of Darboux, and
taking $\phi(t) = t^{n} (t-1)^{n}$, obtain the expansion (supposed convergent)
$$
f(z) - f(a)
=
\sum_{n=1}^{\infty}
\frac{ (-)^{n-1} (z-a)^{n} }{2^{n} n!}
\thebrace{ f^{(n)}(z) + (-)^{n-1} f^{(n)}(a) },
$$
and find the expression for the remainder after $n$ terms in this
series.
%\end{smalltext}
\Section{7}{2}{The \Bernoulli an numbers and the \Bernoulli an polynomials.}

The function $\half z \cot \half z$ is analytic when $\absval{z} <
2\pi$,
and, since it is an even function of $z$, it can be expanded into a
Maclaurin series, thus
$$
\half z \cot \half z
=
1
- B_{1} \frac{z^{2}}{2!}
- B_{2} \frac{z^{4}}{4!}
- B_{3} \frac{z^{6}}{6!}
\ cdots
;
$$
then $B_{n}$ is called the $n$th \emph{\Bernoulli an
  number}\footnote{TODO These numbers were introduced by Jakob BernouUi in bis Ars
Conjectandi, p. 97 (published posthumously, 1713).}.
It is found that\footnote{TODO Tables of the first sixty-two \Bernoulli an numbers have been given by
Adams, Brit. A.is. ReiJorts, 1877.}
$$
B_{1} = \frac{1}{6},
\quad
B_{2} = \frac{1}{30},
\quad
B_{3} = \frac{1}{42},
\quad
B_{4} = \frac{1}{30},
\quad
B_{5} = \frac{5}{66},
\quad
\ldots.
$$

%
% 126
%
These numbers can be expressed as definite integrals as follows:

We have, by example TODO:2 (p. TODO:122) of Chapter TODO:VI,
\begin{align*}
  \int_{0}^{\infty}
  \frac{\sin px \dmeasure x}{e^{\pi x} - 1}
  =&
  -\frac{1}{2p} + \frac{i}{2} \cot ip
  \\
  =&
  -\frac{1}{2p}
  +
  \frac{i}{2} \thebrace{
    1
    + B_{1} \frac{(2p)^{2}}{2!}
    - B_{2} \frac{(2p)^{4}}{4!}
    + \cdots
    }.
\end{align*}

Since
$$
\int_{0}^{\infty}
\frac{x^{n} \sin \theparen{px + \half n \pi}}{e^{\pi x} - 1}
\dmeasure
$$
converges uniformly (by de la Vall\'ee Poussin's test) near
$p=0$ we
may, by \hardsubsectionref{4}{4}{4} corollary, differentiate both
sides of this equation any number of times and then put $p = 0$; doing
so and writing $2t$ for $x$, we obtain
$$
B_{n}
=
4n
\int_{0}^{\infty}
\frac{t^{2n-1} \dmeasure t}{e^{2\pi t} - 1}.
$$
%\begin{remark}
A proof of this result, depending on contour integration, is given by
Carda, Monatshefte fur Math, v.nd Phys. v. (1894), pp. 321-4.
%\end{remark}

TODO:fixexample
Example. Shew that
$$
B_{n}
=
\frac{2n}{\pi^{2n} (2^{2n}-1)}
\int_{0}^{\infty}
\frac{x^{2n}-1}{\sinh x}
> 0.
$$

Now consider the function $t \frac{e^{zt}-1}{e^{t}-1}$, which may be
expanded into a
Maclaurin series in powers of $t$ valid when $\absval{t} < 2\pi$.

\emph{The \Bernoulli an polynomial\footnote{TODO The name was given by
    Raabe, Journal filr Math. xlii. (1851), p.348.} of order $n$} is
defined to be the coefficient of $\frac{t^{n}}{n!}$
in this expansion. It is denoted by $\phi_{n}(z)$, so that
$$
t \frac{e^{zt}-1}{e^{t}-1}
=
\sum_{n=1}^{\infty}
\frac{\theta_{n}(z) t^{n}}{n!}.
$$

This polynomial possesses several important properties. Writing $z+1$
for $z$ in the preceding equation and subtracting, we find that
$$
t e^{zt}
=
\sum_{n=1}^{\infty}
\thebrace{
  \phi_{n}(z+1) - \phi_{n}(z)
}
\frac{t^{n}}{n!}.
$$

On equating coefficients of $t^{n}$ on both
sides of this equation we obtain
$$
n z^{n-1}
=
\phi_{n}(z+1) - \phi_{n}(z),
$$
which is a difference-equation satisfied by the function $\phi_{n}(z)$.

%
% 127
%

An explicit expression for the \Bernoulli an polynomials can be obtained
as follows. We have
$$
e^{zt} - 1
=
zt
+ \frac{z^{2}t^{2}}{2!}
+ \frac{z^{3}t^{3}}{3!}
+ \cdots,
$$
and
$$
\frac{t}{e^{t}-1}
=
\frac{t}{2i} \cot \frac{t}{2i} - \frac{t}{2}
=
1
- \frac{t}{2}
+ \frac{B_{1} t^{2}}{2!}
- \frac{B_{2} t^{4}}{4!}
+ \cdots.
$$

Hence
$$
\sum_{n=1}^{\infty}
\frac{\phi_{n}(z) t^{n}}{n!}
=
\thebrace{
  zt
  + \frac{z^{2} t^{2}}{2!}
  + \frac{z^{3} t^{3}}{3!}
  + \cdots
}
\thebrace{
  1
  - \frac{t}{2}
  + \frac{B_{1} t^{2}}{2!}
  - \frac{B_{2} t^{4}}{4!}
  + \cdots
}.
$$

From this, by equating coefficients of $t^{n}$
(\hardsubsectionref{3}{7}{3}), we have
$$
\phi_{n}(z)
=
z^{n}
- \half n z^{n-1}
+ \binomialcoeff{n}{2} B_{1} z^{n-2}
- \binomialcoeff{n}{4} B_{2} z^{n-4}
+ \binomialcoeff{n}{6} B_{3} z^{n-6}
- \cdots,
$$
the last term being that in $z$ or $z^{2}$ and
$\binomialcoeff{n}{2}, \binomialcoeff{n}{4},\ldots$  being the
binomial coefficients; this is the Maclaurin series for the $n$th
\Bernoulli an polynomial.
%\begin{Remark}
When $z$ is an integer, it may be seen from the difference-equation that
$$
\phi_{n}(z)/n
=
1^{n-1}
+ 2^{n-1}
+ \cdots + (z-1)^{n-1}.
$$

The Maclaurin series for the
expression on the right was given by \Bernoulli.

%\begin{Remark}
\begin{wandwexample}
  Shew that, when $n > 1$,
  $$
  \phi_{n}(z) = (-)^{n} \phi_{n}(1-z).
  $$
\end{wandwexample}
%\end{Remark}
\Subsection{7}{2}{1}{The Euler-Maclaurin expansion.}

In the formula of Darboux \hardsectionref{7}{1}) write $\phi_{n}(t)$
for $\phi(t)$, where $\phi_{n}(t)$ is the $n$th \Bernoulli an polynomial.

Differentiating the equation
$$
\phi_{n}(t+1) - \phi_{n}(t) = n t^{n-1}
$$
$n - k$ times, we have
$$
\phi_{n}^{(n-k)}(t+1) - \phi_{n}^{(n-k)}(t)
=
n (n-t) \cdots k t^{k-1}
$$
Putting $t=0$ in this, we have
$\phi_{n}^{(n-k)}(1) = \phi_{n}^{(n-k)}(0).$

Now, from the Maclaurin
series for $\phi_{n}(z)$, we have if $k > 0$
\begin{align*}
  \phi_{n}^{(n-2k-1)}(0) = 0,
  &
  \quad
  \phi_{n}^{(n-2k)}(0) = \frac{n!}{(2k)!} (-)^{k-1} B_{k},
  \\
  \phi_{n}^{(n-1)}(0) = -\half n!,
  &
  \quad
  \phi_{n}^{(n)}(0) = n!.
\end{align*}

Substituting these values of $\phi_{n}^{(n-k)}(1)$ and
$\phi_{n}^{(n-k)}(0)$ in Darboux's
result, we obtain the Euler-Maclaurin sum
formula\footnote{TODO A history of the formula is given by Barnes, Proc. London Math. Soc.
(2), iii. (1905), p. 253. It was discovered by Euler (1732), but was
not published at the time. Euler communicated it (June 9, 1730) to
Stirling who replied (April 16, 1738) that it included his own theorem
(see \hardsubsectionref{12}{3}{3}) as a particular case, and also that the more general
theorem had been discovered by Maclaurin; and Euler, in a lengthy
reply, waived his claims to priority. The theorem was published by
Euler, Comm. Acad. Imp. Petrop. vi. (1732), [Published 1738], pp.
68-97, and by Maclaurin in 1742, Treatise on Fluxions, p. 672. For
information concerning the correspondence between Euler and Stirling,
we are indebted to Mr C. Tweedie.},

%
% 128
%
\begin{align*}
(z-a) f'(a)
=&
f(z) - f(a)
- \frac{z-a}{2} \thebrace{ f'(z) - f'(a) }
\\
&
+ \sum_{m=1}^{n-1} \frac{ (-)^{m-1} B_{m} (z-a)^{2m}}{(2m)!}
\thebrace{f^{(2m)}(z) - f^{(2m)}(a)}
\\
&
-\frac{(z-a)^{2n+1}}{(2n)!}
\int_{0}^{1} \phi_{(2n)}(t) f^{(2n+1)}\thebrace{ a + (z-a) t }
\dmeasure t.
\end{align*}

In certain cases the last term tends to zero as
$n \rightarrow \infty$, and we can
thus obtain an infinite series for $f(z) - f(a)$.

If we write $\omega$ for $z - a$ and $F(x)$ for $f'(x)$, the last formula becomes
\begin{align*}
  \int_{a}^{a+\omega} F(x) \dmeasure x
  =
  &
  \half \omega \thebrace{ F(a) + F(a + \omega) }
  \\
  &
  + \sum_{m=1}^{n-1}
  \frac{(-)^{m} B_{m} \omega^{2m}}{(2m)!}
  \thebrace{F^{(2m-1)}(a+\omega) - F^{(2m-1)}(a)}
  \\
  &
  + \frac{\omega^{2n+1}}{(2n)!}
  \int_{0}^{1} \phi_{2n}(t) F^{(2n)}(a + \omega t) \dmeasure t.
\end{align*}

Writing $a + \omega, a + 2\omega, \ldots, a + (r-1) \omega$
for $a$ in this result and adding up, we get
\begin{align*}
\int_{a}^{a + r\omega} F(x) \dmeasure x
=
\omega
 &
\thebrace{
  \half F(a) + F(a+\omega) + F(a+2\omega)
  + \cdots + \half F(a + r\omega)
}
\\
&
+ \sum_{m=1}^{n-1}
\frac{(-)^{m} B_{m} \omega^{2m}}{(2m)!}
\thebrace{
  F^{(2m-1)}(a + r\omega)
  -
  F^{(2m-1)}(a)
}
+ R_{n}
\end{align*}
where
$$
R_{n}
=
\frac{\omega^{2n+1}}{(2n)!}
\int_{0}^{1} \phi_{2n}(t)
\thebrace{ \sum_{m=0}^{r-1} F^{(2n)}(a + m\omega + \omega t)}
\dmeasure t.
$$

This last formula is of the utmost importance in connexion with the
numerical evaluation of definite integrals. It is valid if $F(x)$ is
analytic at all points of the straight line joining
$a$ to $a + r \omega$.

%\begin{Remark}
\begin{wandwexample}
  If $f(z)$ be an odd function of $z$, shew that
  $$
  z f'(z)
  =
  f(z)
  +
  \sum_{m=2}^{n}
  (-)^{m}
  \frac{B_{m-1} (2z)^{2m-2}}{(2m-2)!}
  f^{(2m-2)}(z)
  -
  \frac{2^{2n} z^{2n+1}}{(2n)!}
  \int_{0}^{1}
  \phi_{2n}(t)
  f^{(2n+1)}(-z + 2zt)
  \dmeasure t.
  $$
\end{wandwexample}
\begin{wandwexample}
  Shew, by integrating by parts, that the remainder after $n$
  terms of the expansion of $\half z \cot \half z$ may be written in the form
  $$
  \frac{ (-)^{n+1} z^{2n+1} }{ (2n)! \sin z }
  \int_{0}^{1} \phi_{2n}(t) \cos (zt) \dmeasure t.
  $$
\addexamplecitation{Math. Trip. 1904.}
\end{wandwexample}
%\end{Remark}
\Section{7}{3}{\Burmann's theorem*TODO * Memoires de VInstitut, ii. (1799), p. 13. See also Dixon, Proc.
London Math. Soc. xxxiv. (1902), pp. 151-153..}

We shall next consider several theorems which have for their object
\emph{the expansion of one function in powers of another function.}

%
% 129
%

Let $\phi(z)$ be a function of $z$ which is analytic in a closed
region $S$ of which $a$ is an interior point; and let
$$
\phi(a) = b.
$$

Suppose also that $\phi'(a) \neq 0$. Then Taylor's theorem furnishes the
expansion
$$
\phi(z) - b
=
\phi'(a) (z-a)
+ \half \phi''(a) (z-a)^{2}
+ \cdots,
$$
and if it is legitimate to revert this series we obtain
$$
z - a
=
\frac{1}{\phi'(a)}
\thebrace{ \phi(z) - b }
-
\half \frac{\phi''(a)}{\thebrace{\phi'(a)}^{3}}
\thebrace{ \phi(z) - b }^{2}
+ \cdots,
$$
which expresses $z$ as an analytic function of the variable
$\thebrace{ \phi(z) - b }$,
for sufficiently small values of $\absval{z-a}$. If then $f(z)$ be
analytic near $z = a$, it follows that $f(z)$ is an analytic function
of $\thebrace{ \phi(z) - b }$
when $\absval{z - a}$ is sufficiently small, and so there will be an
expansion of the form
$$
f(z)
=
f(a)
+ a_{1} \thebrace{ \phi(z) - b }
+ \frac{a_{2}}{2!} \thebrace{ \phi(z) - b }^{2}
+ \frac{a_{3}}{3!} \thebrace{ \phi(z) - b }^{3}
+ \cdots.
$$

The actual coefficients in the expansion are given by the following
theorem, which is generally known as \emph{\Burmann's theorem}.

\emph{Let $\psi(z)$ be a function of $z$ defined by the equation
$$
\psi(z) = \frac{z-a}{ \phi(z) - b };
$$
then an analytic function $f(z)$ can, in a certain domain of values of
$z$, be expanded in the form
$$
f(z)
=
f(a)
+
\sum_{m=1}^{n-1}
\frac{ \thebrace{\phi(z)-b}^{m} }{m!}
\frac{\dd^{m-1}}{\dd a^{m-1}}
\thebracket{
  f'(a) \psi^{m}(a)
}
+
R_{n},
$$
where
$$
R_{n}
=
\frac{1}{2 \pi i}
\int_{a}^{z}
\int_{\gamma}
\thebracket{
  \frac{\phi(z) - b}{\phi(t) - b}
}^{n-1}
\frac{ f'(t) \phi'(z) }{\phi(t) - \phi(z)}
\dmeasure t \dmeasure z,
$$
and $\gamma$ is a contour in the $t$-plane, enclosing the points $a$
and $z$ and such that, if $\zeta$ be any point inside it, the equation
$\phi(t) = \phi(\zeta)$ has no roots on or inside the contour
except\footnote{It is assumed that such a contour can be chosen if
  $\absval{z - a}$ be sufficiently small;
  see\hardsubsectionref{7}{3}{1}.} a simple root $t=\zeta$.
}

To prove this, we have
\begin{align*}
  f(z) - f(a)
  = &
  \int_{a}^{z} f'(\zeta) \dmeasure \zeta
  = \frac{1}{2 \pi i} \int_{a}^{z} \int_{\phi} TODO
\end{align*}

%
% 130
%

But, by\hardsectionref{4}{3},
\begin{align*}
  &
  \frac{1}{1 \pi i}
  \int_{a}^{z} \int_{\gamma}
  \thebracket{ \frac{\phi(\zeta) - b}{\phi(t) - b} }^{m}
  \frac{f'(t)\phi'(\zeta)}{\phi(t) - b}
  \dmeasure t \dmeasure \zeta
  \\
  &=
  \frac{ [\phi(z) - b]^{m+1}] }{2 \pi i (m+1)}
  \int_{\gamma} \frac{f'(t)}{[\phi(t) - b]^{m+1}} \dmeasure t
  \\
  &=
  \frac{ [\phi(z) - b]^{m+1}] }{2 \pi i (m+1)}
  \int_{\gamma}
  \frac{ f'(t) [\psi(t)]^{m+1} }{ (t-a)^{m+1} }
  \dmeasure t
  \\
  &=
  \frac{ [\phi(z) - b ]^{m+1} }{ (m+1)! }
  \frac{ \dd^{m} }{ \dd a^{m} }
  \thebracket{ f'(a) \thebrace{\psi(a)}^{m+1} }.
\end{align*}
Therefore, writing
$m - 1$ for $m$,
\begin{align*}
  f(z) = f(a) +
  &
  \sum_{m=1}^{n-1}
  \frac{ [\phi(z) - b]^{m} }{m!}
  \frac{\dd^{m-1}}{\dd a^{m-1}}
  [ f'(a) \thebrace{ \psi(a) }^{m} ]
  \\
  &
  \frac{1}{2 \pi i}
  \int_{a}^{z} \int_{\gamma}
  \thebracket{
    \frac{\phi(\zeta) - b}{\phi(t) - b}
  }^{n-1}
  \frac{f'(t) \phi'(\zeta)}{\phi(t) - \phi(\zeta)}
  \dmeasure t \dmeasure \zeta.
\end{align*}

If the last integral tends to zero as $n \rightarrow \infty$, we may write the
right-hand side of this equation as an infinite series.
\begin{wandwexample}
Prove that
$$
z
=
a
+ \sum_{n=1}^{\infty}
\frac{(-)^{n-1} C_{n} (z-a)^{n} e^{n (z^{2} - a^{2}}}{n!},
$$
where
$$
C_{n}
=
(2na)^{n-1}
- \frac{n(n-1)(n-2)}{1!} (2na)^{n-3}
+ \frac{n^{2}(n-1)(n-2)(n-3)(n-4)}{2!} (2na)^{n-5}
- \cdots.
$$
To obtain this expansion, write
$$
f(z) = z,
\quad
\phi(z) - b = (z-a) e^{z^{2} - a^{2}},
\quad
\psi(z) = e^{a^{2} - z^{2}}
$$
in the above expression
of \Burmann's theorem; we thus have
$$
z
=
a
+ \sum_{n=1}^{\infty}
\frac{1}{n!}
(z-a)^{n}
e^{n(z^{2} - a^{2})}
\thebrace{
  \frac{\dd^{n-1}}{\dd z^{n-1}}
  e^{n(a^{2} - z^{2})}
}_{z=a}.
$$

But, putting $z = a + t$,
\begin{align*}
  \thebrace{
    \frac{\dd^{n-1}}{\dd z^{n-1}}
    e^{n(a^{2} - z^{2})}
  }_{z=a}
  =&
  \thebrace{
    \frac{\dd^{n-1}}{\dd t^{n-1}}
    e^{-n(2at + t^{2})}
  }_{t=0}
  \\
  =&
  TOOD
\end{align*}
The highest value of $r$ which gives a term in the summation is
$r = n-1$.
Arranging therefore the summation in descending indices $r$,
beginning with $r = n-1$, we have
\begin{align*}
  \thebrace{
    \frac{\dd^{n-1}}{\dd z^{n-1}}
    e^{n(a^{2} - z^{2})}
  }_{z=a}
  =&
  TODO
  \\
  =& (-)^{n-1} C_{n},
\end{align*}
which gives the required result.
\end{wandwexample}
\begin{wandwexample}
  Obtain the expansion
  $$
  z^{2}
  =
  \sin^{2} z
  + \frac{2}{3} \half \sin^{4} z
  + \frac{2 \cdot 4}{3 \cdot 5} \frac{1}{3} \sin^{6} z
  + \cdots.
  $$
\end{wandwexample}
%
% 131
%
\begin{wandwexample}
Let a line $p$ be drawn through the origin in the $z$-plane,
perpendicular to the line which joins the origin to any point $a$. If
$z$ be any point on the $z$-plane which is on the same side of the line
$p$ as the point $a$ is, shew that
$$
\log z
=
\log a
+ 2 \sum_{m=1}^{\infty}
\frac{1}{2m+1} \theparen{\frac{z-a}{z+a}}^{2m+1}.
$$
\end{wandwexample}
\Subsection{7}{3}{1}{\Teixeira's extended form of \Burmann's theorem.}

In the last section we have not investigated closely the conditions of
convergence of \Burmann's series, for the reason that a much more
general form of the theorem will next be stated; this generalisation
bears the same relation to the theorem just given that Laurent's
theorem bears to Taylor's theorem: viz., in the last paragraph we
were concerned only with the expansion of a function in \emph{positive}
powers of smother function, whereas we shall now discuss the expansion
of a function in \emph{positive and negative} powers of the second
function.

The general statement of the theorem is due to
\Teixeira\footnote{TODO Journal f\"ur Math, cxxii. (1900), pp.
  97-123.}, whose exposition we shall follow in this section.

Suppose (i) that $f(z)$ is a function of $z$ analytic in a ring-shaped
region $A$, bounded by an outer curve $C$ and an inner curve $c$;
(ii) that $\theta(z)$ is a function analytic on and inside $C$,
and has only one zero a
within this contour, the zero being a simple one;
(iii) that $x$; is a given point within $A$;
(iv) that for all points $z$ of $C$ we have
$$
\absval{\theta(x)} < \absval{\theta(z)},
$$
and for all points $z$ of $c$ we have
$$
\absval{\theta(x)} > \absval{\theta(z)}.
$$

The equation
$$
\theta(z) - \theta(x) = 0
$$
has, in this case, a single root $z = x$ in the interior of $C$, as is
seen from the equation\footnote{The expansion is justified
  by\hardsectionref{4}{7}, since
  $\sum_{n=1}^{\infty} \thebrace{\theta(x)/\theta(z)}^{n}$
  converges uniformly when $z$
  is on $C$.}
\begin{align*}
  \frac{1}{2 \pi i}
  \int_{C} \frac{\theta'(z)}{\theta(z) - \theta(x)} \dmeasure z
  =&
  \frac{1}{2 \pi i}
  \thebracket{
    \int_{C} \frac{\theta'(z)}{\theta(z)} \dmeasure z
    +
    \theta(x)
    \int_{C} \frac{\theta'(z)}{ \thebrace{\theta(z)}^{2}}
    \dmeasure z
    + \cdots
  }
  \\
  =&
  \int{1}{2 \pi i}
  \int_{C} \frac{\theta'(z)}{\theta(z)} \dmeasure z
\end{align*}
of which the left-hand and right-hand members represent respectively
the number of roots of the equation considered
(\hardsubsectionref{6}{3}{1}) and the
number of the roots of the equation $\theta(z) = 0$ contained within
$C$.

Cauchy's theorem therefore gives
$$
f(x)
=
\frac{1}{2 \pi i}
\thebracket{
  \int_{C} \frac{f(z) \theta'(z)}{\theta(z) - \theta(x)} \dmeasure z
  -
  \int_{c} \frac{f(z) \theta'(z)}{\theta(z) - \theta(x)} \dmeasure z
}.
$$

%
% 132
%
The integrals in this formula can be expanded, as in Laurent's
theorem, in powers of $\theta(x)$, by the formulae
\begin{align*}
  TODO
\end{align*}

We thus have the formula
$$
TODO
$$
where
$$
TODO
$$

Integrating by parts, we get, if $n \neq 0$,
$$
TODO
$$

This gives a development of $f(x)$ in positive and negative powers of
$\theta(x)$, valid for all points $x$; within the ring-shaped space $A$.

If the zeros and poles of $f(z)$ and $\theta(z)$ inside $C$ are known,
$A_{n}$ and $B_{n}$ can be evaluated by\hardsubsectionref{5}{2}{2} or
by \hardsectionref{6}{1}.

\begin{wandwexample}
  Shew that, if $\absval{x} < 1$, then
  $$
  x
  =
  \half \theparen{ \frac{2x}{1+x^{2}} }
  +
  \frac{1}{2 \cdot 4} \theparen{ \frac{2x}{1+x^{2}} }^{3}
  +
  \frac{1\cdot 3}{2 \cdot 4 \cdot 6} \theparen{ \frac{2x}{1+x^{2}} }^{5}
  +
  \cdots.
  $$

  Shew that, when $\absval{x} > 1$, the second member represents $x^{-1}$.
\end{wandwexample}
If $S^{(m)}_{2n}$ denote the sum of all combinations of the numbers
$$
2^{2}, 4^{2}, 6^{2}, \ldots, (2n-2)^{2},
$$
taken $m$ at a time, shew that
$$
\frac{1}{z}
=
\frac{1}{\sin z}
+
\sum_{n=0}^{\infty}
\frac{(-)^{n+1}}{ (2n+2)! }
\theparen{
  \frac{1}{2n+3}
  - \frac{S^{(1)}_{2(n+1)}}{2n+1}
  + \cdots
  + \frac{ (-)^{n} S^{(n)}_{2(n+1)}}{3}
}
(\sin z)^{2n+1}
$$
the expansion being valid for all values of $z$ represented by points
within the oval whose equation is $\absval{\sin z} = 1$ and which contains the
point $z = 0$. \addexamplecitation{\Teixeira.}

\Subsection{7}{3}{2}{Lagrange's theorem.}

Suppose now that the function $f(z)$ of \hardsubsectionref{7}{3}{1}
is analytic at all points in the interior of $C$, and let
$\theta(x) = (x - a) \theta_{1}(x)$. Then $\theta_{1}(x)$ is
analytic and not zero on or inside $C$ and the contour $c$ can be
dispensed with; therefore the formulae which give $A_{n}$ and
$B_{n}$ now become, by\hardsubsectionref{5}{2}{2} and \hardsectionref{6}{1},
\begin{align*} % TODO: multiline?
  A_{n}
  =&
  \int_{C} \frac{f'(z)}{(z-a)^{n} \theta_{1}^{n}(z)} \dmeasure z
  = \frac{1}{n!} \frac{\dd^{n-1}}{\dd a^{n-1}}
  \thebrace{
    \frac{f'(a)}{\theta_{1}^{n}(a)
    }
  }
  \quad (n \geq 1),
  \\
  A_{0}
  =&
  \int_{C} \frac{f(z) \theta'(z)}{\theta_{1}(z)}
  \frac{\dmeasure z}{z-a}
  =
  f(a),
  \\
  B_{n}
  =&
  0.
\end{align*}

%
% 133
%

The theorem of the last section accordingly takes the following form,
if we write $\theta_{1}(z) = 1 / \phi(z)$:

\emph{Let $f(z)$ and $\phi(z)$ he functions of $z$ analytic on and inside a
contour $C$ surrounding a point $a$, and let $t$ be such that the inequality
$$
\absval{t \phi(z) } < \absval{z - a}
$$
is satisfied at all points $z$ on the perimeter of $C$;
then the equation
$$
\zeta = a + t \phi(\zeta),
$$
regarded as an equation in $\zeta$, has one root in the interior
of $C$; and further any function of $\zeta$ analytic on and inside
$C$ can be expanded as a power series in $t$ by the formula
$$
f(\zeta)
=
f(a)
+
\sum_{n=1}^{\infty}
\frac{t^{n}}{n!}
\frac{\dd^{n-1}}{\dd a^{n-1}}
\thebracket{
  f'(a) \phi^{n}(a)
}.
$$
}
This result was published by Lagrange\footnote{TODO Mem. de VAcad. de Berlin, xxiv.; Oeuvres, ii. p. 25.} in 1770.
\begin{wandwexample}
Within the contour surrounding $a$ defined by the inequality
$\absval{z (z - a)} > \absval{a}$, where
$\absval{a} < \half \absval{ a }$, %TODO: verify
the equation
$$
z - a - \frac{a}{z} = 0
$$
has one root $\zeta$, the expansion of which is given by Lagrange's theorem
in the form
$$
\zeta
=
a
+
\sum_{n=1}^{\infty}
\frac{(-)^{n-1} (2n-2)!}{n! (n-1)! a^{2n-1}} a^{n}
%TODO: verify
$$

Now, from the elementary theory of quadratic equations, we know that
the equation
$$
z - a - \frac{a}{z} = 0
%TODO: verify
$$
has two roots, namely $TODO$ and $TODO$; and our
expansion
\emph{represents the former\footnote{The latter is outside the given
    contour.} of these only}---an example of the need for
care in the discussion of these series.
\end{wandwexample}
\begin{wandwexample}
  If $y$ be that one of the roots of the equation
  $$
  TODO
  $$
  which tends to $1$ when $z \rightarrow 0$, shew that
  $$
  TODO
  $$
  so long as $\absval{z} < \frac{1}{4}$.
\end{wandwexample}
\begin{wandwexample}
If $x$ be that one of the roots of the equation
$$
x = 1 + y x^{a}
$$
which tends to $1$ when $y \rightarrow 0$, shew that
$$
TODO
$$
the expansion being valid so long as
$$
\absval{y}
<
\absval{
  (a-1)^{a-1} a^{-a}
}.
$$
\addexamplecitation{McClintock.}
\end{wandwexample}

%
% 134
%

\Section{7}{4}{The expansion of a class of functions in rational fractions*'.}
Consider a function $f(z)$, whose only singularities in the finite
part of the plane are simple poles $a_{1},a_{2},a_{3},\ldots$, where
$\absval{a_{1}} \leq \absval{a_{2}} \leq \absval{a_{3}} \leq \cdots$;
let $b_{1},b_{2},b_{3},\ldots$, be the residues at these
poles, and let it be possible to choose a sequence of circles $C_{m}$ (the
radius of $C_{m}$ being $R_{m}$) vvith centre at $O$, not passing through any
poles, such that $\absval{f(z)}$ is bounded on $C_{m}$. (The function
$\cosec z$ may
be cited as an example of the class of functions considered, and we
take $R_{m} = (m + \half)\pi$.) Suppose further that
$R_{m} \rightarrow \infty$ as $m \rightarrow \infty$ and that
the upper bound\footnote{Which is a function of $m$.}
of $\absval{f(z)}$ on $C_{m}$ is itself bounded
as\footnote{Of course $R_{m}$ need not (and frequently must not) tend to infinity
  continuously; e.g. in the example taken
  $R_{m} = (m+\half)z$, where $m$ assumes only integer values.}
$m\rightarrow\infty$; so
that, for all points on the circle $C_{m}$, $\absval{f(z)} < M$, where $M$ is
independent of $m$.

Then, if $x$ be not a pole of $f(z)$, since the only poles of the
integrand are the poles of $f(z)$ and the point $z = x$, we have, by\hardsectionref{6}{1},
$$
\frac{1}{2 \pi i} \int_{C_{m}} \frac{f(z)}{z-x} \dmeasure z
=
f(x) + \sum_{r} \frac{b_{r}}{a_{r}-x}.
$$
where the summation extends over all poles in the interior of C, .

But
\begin{align*}
  \frac{1}{2 \pi i}
  \int_{C_{m}} \frac{f(z)}{z-x} \dmeasure z
  =&
  \frac{1}{2 \pi i}
  \int_{C_{m}} \frac{f(z)}{z} \dmeasure z
  +
  \frac{x}{2 \pi i}
  \int_{C_{m}} \frac{f(x)}{z(z-x)} \dmeasure z
  \\
  =&
  f(0) + \sum_{r} \frac{b_{r}}{a_{r}}
  +
  \frac{x}{2 \pi i}
  \int_{C_{m}} \frac{f(z)}{z(z-x)} \dmeasure x
\end{align*}
if we suppose the function $f(z)$ to be analytic at the origin.

Now as $m \rightarrow \infty$,
$\int_{C_{m}} \frac{f(z)}{z(z-x)} \dmeasure z$ is
$\bigo(R_{m}^{-1})$, and so tends to zero as
$m$ tends to infinity.

Therefore, making $m \rightarrow \infty$, we have
$$
0
=
f(x) - f(0)
+
\sum_{n=1}^{\infty}
b_{n} \theparen{
  \frac{1}{a_{n}-x} - \frac{1}{a_{n}}
}
-
\lim_{m\rightarrow\infty}
\frac{x}{2 \pi i}
\int_{C_{m}} \frac{f(z)}{z(z-x)} \dmeasure x,
$$
\ie
$$
f(x) = f(0)
+
\sum_{n=1}^{\infty}
b_{n}
\thebrace{
  \frac{1}{x-a_{n}} + \frac{1}{a_{n}}
  %TODO:consistent with above?
}
$$

which is an expansion of f(x) in rational fractions of x; and the
summation extends over all the poles of f x).

If a |<| n + i| this series converges uniformly throughout the region
given by I A' I a, where a is any constant (except near the points ).
For if R, be the radius of the circle which encloses the points |
ai'\, ... | a |, the modulus of the remainder of the terms of the
series after the first ?i is

I *' /" / ) I  '* \ 2iri J Cm (z - ') \ Rm-( by\hardsubsectionref{4}{6}{2}; and, given
t, we can choose n independent of a; such that Maj Rm - CL) < e.

* Mittag-Leffler, Acta Soc. Scient. Fennicae, xi. (1880), pp. 273-293.
See also Acta Math. iv. (1884), pp. 1-79.

%
% 135
%

The convergence is obviously still uniform even if ! I < | + 1 1
provided the terms of the series are grouped so as to combine the
terms corresponding to poles of equal moduli.

If, instead of the condition \ f(z) \ < M, we have the condition |
z~'>f z) | < M, where M is independent of ra when z is on (7,, and is
a positive integer, then we should have to

expand / - - by writnig

1 I .V XP"

= - + -, + ...+-

z-x z z '" zJ'+ iz-xy

and should obtain a similar but somewhat more complicated expansion.

Example 1. Prove that

cosec2=- + 2(-)"( H ),

z \ \ ~ " 1 ' /

the summation extending to all positive and negative values of n.

To obtain this result, let cosec z -- = f z). The singularities of
this function are at the

z ' '

points z=nir, where Ji is any positive or negative integer.

The residue of f(z) at the singularity rnr is therefore (-)", and the
reader will easily see that \ f(z) | is bounded on the circle \ z\ = n
+ ) ir as - -x .

Applying now the general theorem

/W=/(0)+.o,.[ -L.+l],

where c is the residue at the singularity a, we have

f( )-m+ (-r 7 + i]-

But /(0)= lim 14 - =0.

Therefore cosec s = - + 2 ( - ) + -,

z \ \ z-mr nnj

which is the required result.

Example 2. If < a < 1, shew that

e" 1 22 cos 27iair - 4n7r sin 2na7r

Example 3. Prove that

TODO

The general term of the series on the right is

(grTT \ e-rTT) 1( )4 + 1 .4 '

which is the residue at each of the four singularities r, -r, ri, - ri
of the function

(ttV+j ) (e' - e-'"2) sin nz

%
% 136
%

The singularities of this latter function which are not of the type r,
- r, ri, - 7-i are at the five points

 ±l±i)a:

2

At 2 = the residue is - j :

at each of the four points 2= ~, the residue is

 27r 2 (cos . \ cosh .r) ~ '. Therefore

  (-l)*-?- 12 2

 j grn- \ g-rn- (;. )4 + \ 4 jr\ .2 (cOsh .V - COS )

1,. /" nzdz

hm

Stti,;J J c (7rV + J *) (e' - e-' ) sin tt '

where Cis the circle whose radius is + -, (?i an integer), and whose
centre is the origin.

But, at points on C, this integrand is 0 \ z \~ ); the limit of the
integral round C is therefore zero.

From the last equation the required result is now obvious.

Ea:ample 4. Prove that sec.r=4. ( - -A + J- - .. j .

Example 5. Prove that cosech x = - 2.v ( - - - --5 - <, + tt-s - -
...).

Example 6. Prove that seeh .v=47r ( -. - -- - r- - -5 + - r- - -
,-...). Vtt +4.t-2 Q-n- + ix 257r2 + 4A-2 /

Example 7. Prove that coth x = - + 2x( -x 5 + - + \ - 5 + .).

Example 8. Prove that 2 2 -. - 5,, - rirr = -7 coth tto coth nb.

 m + a ) [n + 0 ) ao

m=-x )! = -cc

\addexamplecitation{Math. Trip. 1899.}
\Section{7}{5}{The expansion of a class of functions as
infinite products.}

The theorem of the last article can be applied to the expansion of a
certain class of functions as infinite products.

For let f(z) be a function which has simple zeros at the points* Oi,
tts. 3, > where lim | an I is infinite; and let/(2') be analytic
for all values

of z.

f'iz) Then /' (z) is analytic for all values of z \hardsubsectionref{5}{2}{2}), and so "
,., can have

singularities only at the points a, a, a, -

Consequently, by Taylor's theorem,

f z) = z- ar)f' ( .) + - V" M +

and /' z) = /' a.r) + z - a,) /" (a,) + . . . .

* These being the only zeros oif(z); and a 4=0.

/ /
%
% 137
%
It follows immediately that at each of the points a, the function .
has a simple pole, with residue + 1.

If then we can find a sequence of circles C\ n of the nature described
in

f (z) % 7"4, such that; is bounded on C as ni->cc, it follows, from
the

expansion given in § 7 '4, that

f( )\ f'(0) I [1,1

f z) f 0) n = \ [Z - an Clr.

Since this series converges uniformly when the terms are suitably
grouped \hardsectionref{7}{4}), we may integrate term-by-term \hardsectionref{4}{7}). Doing so, and
taking the exponential of each side, we get

/( ) = ce/( ) " n (1 e 4,

where c is independent of z.

Putting z = 0, we see that /(O) = c, and thus the general result
becomes

This furnishes the expansion, in the form of an infinite product, of
any function /( ) which fulfils the conditions stated.

Example 1. Consider the function f(z) =, which has simple zeros at
the points

rrr, where r is any positive or negative integer.

In this case we have / (0) = 1, /' (0) = 0,

and so the theorem gives immediately sins

1-1 ]e ''

Z n-l IV mr J I IV iTT

for it is easily seen that the condition concerning the behaviour of
-fFS ' 1"*"°° ' fulfilled.

Example 2. Prove that

H ) -G.- J H r\ H )) H ))

\ cosh k - cos X

1 - cos X

\addexamplecitation{Trinity, 1899.}

\Section{7}{6}{The factor theorem of Weierstrass*.}

The theorem of\hardsectionref{7}{5} is very similar to a more general theorem in
which the character of the function f z), as j j- >x, is not so
narrowly restricted,

* Berliner Alh. (1876), pp. 11-60; Math. Werke, 11. (1895), pp.
77-124.
%
% 138
%
Let f(z) be a function of z with no essential singularities (except at
' the point infinity'); and let the zeros and pol6s oi f(z) be at aj,
ag, as, ....where

< I ai I 1 a.2 ] i 3 1 Let the zero* at be of (integer) order ?? .ft.

If the number of zeros and poles is unlimited, it is necessary that
ja,i - >X), as ?i- >oo; for, if not, the points a would have a limit
point i*, which would be an essential singularity oi f z).

We proceed to shew first of all that it is possible to find
polynomials gn z) such that
$$
TODO
$$
converges for all J finite values of z.

Let K be any constant, and let \ z\ < K; then, since |a l- >oo, we can
find N such that, when n> N, \ cin \ > K.

The first N factors of the product do not affect its convergence :j:;
consider any value of ?i greater than N, and let

1 /

+ ... +

kn - 1 \ ttn

An-1

Then

zy

- t - (- ) +gn( )\ =

=

1 1 fzy

<

<2\ \ {Kan-')''"\,

since U a,ri \ < k

Hence

z \ ) '**

1 e5 (-) i = e" (

aJ J

where
$$
TODO
$$

Now Mn and a are given, but k is at our disposal; since Kan' <, we
choose kn to be the smallest number such that 2 | vin Kan~' y"' 1 <
bn, where

00

S bn is any convergent series§ of positive terms.

Hence

n

n = N+l

 ( l - - le H' ) I'"'"

where \ Un z)\ < bn', and therefore, since b is independent of z, the
product converges absolutely and uniformly when 1 j < K, except near
the points a .

* We here regard a pole as being a zero of negative order.

t From the two-dimensional analogue of\hardsubsectionref{2}{2}{1}.

X Provided that z is not at one of the points for which m is negative.

§ E.g. we might take 6 =2-".
%
% 139
%

Now let F(2)= U ( l--) eOn i )

Then, ii f(z) F(z) = Gi z), Gi z) is an integral function \hardsubsectionref{5}{6}{4}) of
z and has no zeros.

It follows that p, -J- G z) is analytic for all finite values of z;
and

\ Ti \ Z J ctz

00

so, by Taylor's theorem, this function can be expressed as a series S
nbnZ ''~ converging everywhere; integrating, it follows that

where G z)= S 6,1 " and c is a constant; this series converges
everywhere,

M = l

and so G (z) is an integral function. Therefore, finally,

/( )=/(0)e '-) n r|(i\ f),.n..)

where G (z) is some integral function such that G (0) = 0.

[Note. The presence of the arbitrary element G (2) which occurs in
this formula for f(z) is due to the lack of conditions as to the
behaviour of /(s) as ' s |-9-oo .] Corollary. If wi =l, it is
sufficient to take k - n, by\hardsubsectionref{2}{3}{6}.

\Section{7}{7}{The expansion of a class of periodic functions in a series of
cotangents.}

Let $f(z)$ be a periodic function of $z$, analytic except at a certain
number of simple poles; for convenience, let tt be the period of f z)
so that f z)=f(z + 'rr).

Let z = x + iy and let f z)->l uniformly with respect to x as y->+ x,
when O x tt; similarly let/( )- uniformly as - > - x .

Let the poles of f(z) in the strip < x tt he nt a, a., ... an', and
let the residues at them be Ci, Cj, ... c,i.

Further, let ABGD be a rectangle whose corners are* -ip, ir - ip, IT +
ip and ip in order.

Consider - . I f(t) cot t - z) dt

ZlTl J

taken round this rectangle; the residue of the integrand at a,, is
c,. cot (a,. - z), and the residue at z is f z).

Also the integrals along DA and CB cancel on account of the
periodicity of the integrand; and as p->y:>, the integrand on AB tends
uniformly to I'i, while as p'->oo the integrand on CD tends uniformly
to -li; therefore

Ul'-l)=f z)+i c, cot (a, -4

r=l

* If any of the poles are on x = 7r, shift the rectangle slightly to
the right; p, p' are to be taken so large that ai, a-i, ... a are
inside the rectangle.
%
% 140
%

That is to say, we have the expansion

f z) = \ V -0+1 c, cot z - ar).

  r=l

Example 1.

n,

cot x - ftj) cot x - a.,) . . . cot x - aj = 2 cot (a - ai) . . .*. .
. cot (a - a ) cot (a; - a ) + ( - ),

r=l n

or =2 cot (c? - ai)...*...cot (,. -a )cot (.V - a,.),

r=l

according as n is even or odd; the * means that the factor cot (a -
a,) is omitted. Example 2. Prove that

sin x - bi) sin x -h. ... sin x - 6 ) \ sin ( ! - 6i) . . . sin (ai -
6 )

sin x - a-i) sin (0 - 02) ... sin (a' - a ) sin (aj - 2)  sin ( ! -
a )

sin (q.2 - 1)    sin a - \& ) sin (a2 - i)  sin (a2 - a )

cot (a;-ai) cot (.r - 02)

+

+ cos (ai + a2 + ...+a -6i-62- ... - n)-

\Section{7}{8}{Borel's theorem- .}

Let/(2-) = S ttji " be analytic when \ z\ \ r, so that, by
\hardsubsectionref{5}{2}{3},
$
TODO
$
where If is independent of n.

CtnZ"

Hence, if < z)= S, < z) is an integral function, and

and similarly | </)<' ' z) \ < il/el l/''/r

Now consider f(z) =1 e~ zt) dt; this integral is an analytic function

Jo

of z when \ z\ < r, by\hardsubsectionref{5}{3}{2}.

Also, if we integrate by parts,

/ z) = -e-'(f> (zt) +z e-' 4>' zt) dt

JO

= 2 2

 t=0

+ "+1 e- (f> ''+' (zt) dt.

But lim -< </)" > ( ) = a i; and, when \ z\ < r, lim e"' </>'*"' (zt)
= 0.

f,(z)= t a.nZ' + Rn, m=0

Therefore

t Lemons sur les series divergentes (1901), p. 94. See also the
memoirs there cited.
%
% 141
%
/no

where TODO

Consequently, TODO
where TODO is called Borel' s function associated with S
an :**.

 l = *  M =

If *S'= 2 and (f) z)= 2 - and if we can establish the relation S= I
e~'0 ( ) o?, n=0 re=0 ''  Jo

the series <S is said \hardsubsectionref{8}{4}{1}) to be ' summable (BY; so that the
theorem just proved shews that a Taylor's series representing an
analytic function is summable (B).

\Subsection{7}{8}{1}{Borel's integral and analytic continuation.}

We next obtain Borel's result that his integral represents an analytic
function in a more extended region than the interior of the circle \
z\ = r.

This extended region is obtained as follows : take the singularities
a,b,c,... of f(z) and through each of them draw a line perpendicular
to the line joining that singularity to the origin. The lines so drawn
will divide the plane into regions of which one is a polygon with the
origin inside it.

Then Borel's integral represents an analytic function (which, by\hardsectionref{5}{5}
and\hardsectionref{7}{8}, is obviously that defined by f(z) and its continuations)
throughout the interior of this polygon. The reader will observe that
this is the first actual formula obtained for the analytic
continuation of a function, except the trivial one of § 5 '5, example.

For, take any point P with affix ( inside the polygon; then the circle
on GF as diameter has no singularity on or inside it*; and
consequently we can draw a slightly

* The reader will see this from the figure; for if there were such a
singularity the correspond- ing side of the polygon would pass between
and P; i.e. P would be outside the polygon.
%
% 142
%

larger concentric circle* € with no singularity on or inside it. Then,
by\hardsectionref{5}{4},

" 2771 J CZ"*

but 2 C -Llf converges imiformly \hardsubsectionref{3}{3}{4}) on C since f(z) is bounded
and l 2 j 8 > 0, where S is independent of z; therefore, by\hardsectionref{4}{7},

c (C0 = 2 - I -\ f .z) exp Ctz-') dz,

and so, when t is real, | (CO i < (0 < where F (0 is bounded in any
closed region lying wholly inside the polygon and is independent of t
; and \ is the greatest value of the real part of (jz on C.

If we draw the circle traced out l)y the point 2/, we see that the
real part of f/2 is gi-eatest when z is at the extremity of the
diameter through (, and so the value of X is

1C|. ICI + S - <1.

We can get a similar inequality for ( ' (CO and hence, by\hardsubsectionref{5}{3}{2}, |
e~' (CO t is

.' analytic at ( and is obviously a one-valued function of C-

This is the result stated above.
\Subsection{7}{8}{2}{Expansions in series of inverse factorials.}

A mode of development of functions, which, after being used by Nicole
f and Stirling:]: in the eighteenth century, was systematically
investigated by Schlomilch§ in 1863, is that of expansion in a series
of inverse factorials.

To obtain such an expansion of a function analytic when \ z r, we let

the function he f z)= X "h "", and use the formula /( ) = I ze~ t)dt,
=o '0

00

where (f> t)= S anV j n !); this result may be obtained in the same
way as

 =o that of\hardsectionref{7}{8}. Modify this by writing e"' = 1 - f, </> (0 = F ( )
; then

Jo

Now if t = u + iv and if t be confined to the strip - 7r<w<7r, Hsa
one- valued function of | and F is an analytic function of; and | is
restricted so that - TT < arg (1 - |) < w. Also the interior of the
circle | | = 1 corresponds

* The difforeuce of the radii of the circles beioR, say, 5.

t Mem de VAcad. des Sci. (Paris, 1717); see Tweedie, Proc. Edin. Math.
Soc. xxxvi. (1918).

J Methodus Dijferentialis (Londou, 1730).

§ Compendium der h'dheren AnalysU. More recent investigations are due
to Kluyver, Nielsen and Pincherle. See Comptes liendiis, cxxxiii.
(1901), cxxxiv. (1902), Annales de I'Ecole norm, sup. (3), XIX.,
XIII., xxiii., JRendiconti del Lincei, (5), xi. (1902), and Palermo
Rendiconti, xxxiv. (1912). Properties of functions defined by series
of inverse factorials have been studied in an important memoir by
Norlund, Acta Math, xxxvii. (1914), pp. 327-H87.
%
% 143
%

to the interior of the curve traced out by the point t = - log (2 cos
6) + id, (writing = exp [i (6 + ir) ); and inside this curve

\ t R(t) [ R (t)Y + TT'f - R (0- 0, as R ( )- >oc .

It follows that, when || 1 1, | i ( ) | < Me"" * < M, | e*-* |, where
J/j is in- dependent of t; and so F(i) < M \ (1 - l

Now suppose that < 1; then, by\hardsubsectionref{5}{2}{3}, | F"*' (|) | < ifa . n\ p-'\
where M2 is the upper bound of \ F z)\ on a circle with centre and
radius P<l-l

Taking p = =-(1 - ) and observing that* (1 4-;i~ )' < e we find that

1 n + 1

i- f+ Tlf

<M,e n + iy .111(1- )->-'\ Remembering that, by\hardsectionref{4}{5}, means lim, we
have, by repeated

Jo e +O J

integrations by parts,

$$
TODO
$$

where

+ 1 z + l)(z + 2) z + l)(z+2)... z + n)

= F "" (0), if the real part of + w - 7' - n > 0, i.e. if R (z) > r;
further

+ Rn,

bn = lim

l-e

Rn\ \

<

<

(2 + l)(z + 2)...(z + n)

M,e(n + 2y.n\

\ \ {z+l) z + 2)...(z + n). R z-r)

M,e(7i + 2y.nl

lim f '|(l- y+ i <"+i'(| )|cZ

e O J

(r+l + B) r + 2 + 8)... r + n + 8).S' where 8 = R z - r).

* (l + x-i) increases with x; for :; >ey, when ?/<l, and so log ( :,
)>'/ That is to

say, putting y-' - l + x, - a; log (l + a;~ ) = ]og (l+a;~ ) -z >0.

%
% 144
%

Now n (i + '- le

n|(i

>+

tends to a limit \hardsubsectionref{2}{7}{1}) as n->oo, and so | Rn >0 if (n + 2ye '" '
"' tends to zero; but

1 l/m> = log(n + l),

 +i dx [/m > I

by\hardsubsectionref{4}{4}{3} (II), and (n + '2y(n+ l)-'- -*0 when 8 >0; therefore K->0 as
n- x, and so, when R (z) > r, we have the convergent expansion

Example 1. Obtain the same expansion by using the results

[ fi =.[ dt j f(t) i-uy-'- du.

Example 2. Obtain the expansion

iog(i+J)=!--"'

z z+l) z 2 + l) z + 2) '

where u= I t I- t) 2-t) ... n-l-t) dt,

J

and discuss the region in which it converges. \addexamplecitation{Schlomilch.}

REFERENCES. E. Goursat, Cours d' Analyse (Paris, 1911), Chs. xv, xvi.
E. BoREL, Lecons sur les series divergentes (Paris, 1901). T. J. I'a.
Bromwich* Theory of Infinite Series (1908), Chs. viii, x, xi. 0.
Schlomilch, Compendium der hoheren Analysis, ii. (Dresden, 1874).

Miscellaneous Examples.

1. If y -x - (i> y) = 0, where is a given function of its argument,
obtain the expansion

/(,)-/(.) + 1 ± H wr (i 1,)7(. ),

where / denotes any analytic function of its argument, and discuss the
range of its validity. (Levi-Civitk, Bertd. dei Lincei, (5), xvl
(1907), p. 3.)

2. Obtain (from the formula of Darboux or otherwise) the expansion

/(.) -/( )= J /~J;" jy" /"' (2) - '"/<"> ( );

find the remainder after n terms, and discuss the convergence of the
series.

* The expansions considered by Eromwich are obtained by elementary
methods, i.e. without the use of Cauchy's theorem.

%
% 145
%

3. Shew thcat

n 13 5 (2m - ) h

OT=1 V"' )

where

J

and shew that y ( ) is the coefficient of ?i ! P in the expansion of
(1 - t.v) (1+ - Lv) 2 [ i ascending jiowers of t.

4. By taking

< ( + l) =,-

' d"- f(l-r)e ' ( 1 l-re-

in the formula of Darboux, shew that

/( + A)-/(.r)=- I,;|/(-)(,; + A)-l/M(4 m=l Vl. [ r )

\ i.(\ ) A"+i r (ji(t)f(n+ (x+ht)dt,

J

where . \ --=i-a, ~ + a,~-asy, + ....

5. Shew that

/(.)-/( ) = J (-)-! - 2 ! /'"""" ( )+/<''"- ' ( )

6. Prove that

/( 2) -/(--l) = l ( 2 - --1)/' Z2) + C2 iZ2-Ziyf" ( l) - C3 Z2 " OV" (
2)

-< 4( 2- l)V*M2l) + --- + (-)M 2---l) / ' | (e'''SechH)| \ /( + l)(2l
+ to.- .-l);

in the series phis signs and minus signs occur in pairs, and the last
term l)efore the integral is that involving Z'>-h)"", also Cn is the
coefficient of 5" in the expansion of

cot ( T - o ) '" ascending powers of z. \addexamplecitation{Trinity, 1899.}

7. If Xi and x are integers, and <f) (z) is a function which is
analytic and'bounded for all values of z such that x R (z) X2, shew
(by integrating

f <t>( )dz

je= 27r \ l

round indented rectangles whose corners are Xi, 2? 2± < i, Xi±co i)
that i( (a-i) + < (*-i + l) + ( ( i + 2)+... + 0( 2-l) + 2< (. 2)

'"'A. i~\, 1 /'" (/) ( 2 + 3/) - (- 1 +iy)-(i> ( 2 - y) + < (A i -
iy)

1 I ( x +iy)-(pi-vi+iy)-(p ( 2 - .y) + <?> ( i - y) ..

Mo e2f i/ - 1  "

Hence, by applying the theorem

/ ',2n-l

4?i -1 - du = Bn,

W. M. A. 10

%
% 146
%

where Bi, B., ... are \Bernoulli's numbers, shew that

(where C is a constant not involving w), provided that the last series
converges.

(This important formula is due to Plana, Mem. della R, Accad. di
Torino, xxv. (1820), pp. 403-418; a proof by means of contour
integration was published by Kronecker, Journal fur Math. cv. (1889),
pp. 345-348. For a detailed history, see Lindelof, Le Calcul des
Residus. Some applications of the formula are given in Chapter xii.)

8. Obtain the expansion

X "",,, 1.3...(2;i-3) .r

"=2+ !.(->"~ n\ 2

for one root of the equation x = 2u + u\ and shew that it converges so
long as | jp | < 1.

' all combinations of th 12, 32, 5%... 2n-ir,

9. If .S'™', denote the smii of all combinations of the numbers

2n.+l

taken m at a time, shew that

1 oc (\ ) + ]

sins =o(2?i + 2)

cos.\ 1 |\ (-) |2 ( )\ U, \ +r-)n < ) -Isin -i.

\addexamplecitation{Teixeira.}

10. If the function f(z) is analytic in the interior of that one of
the ovals whose equation is | sin s | = C (where C< 1), which includes
the origin, shew that f(z) can, for all points z within this oval, be
expanded in the form

0. /(2 )(o)+Mi'/C' -2)(o)+...+>s;'"V'(o) '

/(.)=/(0)+ S- " - sin z

I /' "- ' (0) +4;; /(2n-i) (0)+ ... +<v, /' (0)

where *S' is the sum of all combinations of the numbers

2n

22, 42, 62, ... (2/i-2)2, taken m at a time, and *S ",, denotes the
sum of all combinations of the numbers

' 2(1 + 1

12, 32, 52,...(2;i-l)2, taken mi at a time. \addexamplecitation{Teixeira.}

11. Shew that the two series

2z 1z 2 +32" + "52 +'

2z 2\ / 2z Y 2 / 2z

 "r 1.32 1- 2; +3.52V1-1

represent the same function in a certain region of the z plane, and
can be transformed into each other by Biirmaun's theorem.

(Kapteyn, Nieuw Archief, (2), iii. (1897), p. 225.)

12. If a function f(z) is periodic, of period 27r, and is analytic at
all points in the infinite strip of the plane, included between the
two branches of the curve jsin2| = C (where C> 1), shew that at all
points in the strip it can be exjianded in an infinite series of the
form

/(2) = o + ylisin2+...+ sin"2 +

+ cos2(5i + 52sin2+... + sin"-is + ...); and find the coefl\&cients
-4 and B . *

%
% 147
%

13. If ( and /are connected bj the equation of which one root is a,
shew that

 ( )= -T '/ '+i 3

0" PP') I 1! 2! 3 X"'

j< ' (( 2)' (/3/") " ct>T ipF')"

+ ...,

the general term being ( - )'",, ", ( f\ imim + i) iiaultiplied by a
determinant in which

the elements of the first row are ', ( )', (f) )', ..., (< '"~ )', /'"
P') and each row is the diflferential coefficient of the jjreceding
one with respect to a; and P, /", P', ... denote F a),f a\ P' a)

(Wronski, Philosophie de la Technie, Section ii. p. 381. For proofs of
the theorem see Cayley, Quarterly Journal, xil. (1873), Transon, Nouv.
Ann. de Math. xill. (1874j, and C Lagrange, Brux. Mem. Couronnes, 4",
xlvii. (1886), no. 2.)

14. If the function W a, b, x) be defined by the series

T.r / 7 N a-b (a - b)(a~2b), W a, b, x) = x + j-x + a- +...,

which converges so long as

x\ <

w

d

-p W (a, b, x) = l+ a- b) W a - b, b, .v);

shew that

and shew that if y= W a, b, .r),

then x= yV b, a, y).

Examples of this function are

Jr(l,0, A-) = e*-l, lf(0, 1, .r) = log(l+.r), + xY-\

15. Prove that

W'(a, \,x) =

1 \ 1; -Yx

\addexamplecitation{Jezek.}

2 a :ir"

n=0

where

Gn =

2ai 4a2 Gag

3ai

5ao

2ao 4ai

3ao

(2 -2)a \ i ( -l)ao

nan (?i - 1 ) a \ x a

and obtain a similar expression for

h

I 2 a .xA' U=o I

(Mangeot, Ann. de VEcole norm. sup. (3), xiv.

16. Shew that

2 a x"

r=0

- 2

1\ cSr i

=o?'+l cai

10-2

%
% 148
%

where Sr is the sum of the rth powers of the reciprocals of the roots
of the equation

r=0

\addexamplecitation{Gambioli, Bologna Memorie, 1892.}

17. If / (z) denote the nth derivate of f z), and if /\ (z) denote
that one of the nth integrals oif z) which has an w-ple zero at 2 = 0,
shew that if the series

  = - X

is convergent it represents a function oi z + x; and if the domain of
convergence includes the origin in the .r-plane, the series is equal
to

\addexamplecitation{Guichard.}

2 /\ (2 + .t-)5r (0).

n =

Obtain Taylor's series from this result, by putting g z) = 1.

18. Shew that, if jr be not an integer,

, =-;, n=-v x-¥mf x + nf

as 1/ - QC, provided that all terms for which in = n are omitted from
the summation.

\addexamplecitation{Math. Trip. 1895.}

19. Sum the series

  / 1 1'

n -q \ \ {-Yx-a-n n) ' where the value n = is omitted, a ud p, q are
positive integers to be increased without

limit.

\addexamplecitation{Math. Trip. 1896.}

20. If i (.r) = Jo*" °*( ")'', shew that

F x)=< :t iJ L,

and that the function thus defined satisfies the relations

F -x) =, F(x)F l-x) = 2sm.V7r.

Further, if = + 2 + ' + -= - \ logC -Oy,

shew that when

21. Shew that

ll-e-2' ' Kl.

\addexamplecitation{Trinity, 1898.}

where and

[-©i-( -.)i-(i .)"][-( -...)i'-G,y"]

n l-2e- aCOs(a; + /3 ) + e-2 p l-2e-a(,cos( -i3 )+e-2 (, g=i,

2*'*(l-cos.r)* e-' ''''*'''"

,20-1, 2af-l ag=k?,\ r - TT, Pg=KCoa- tt,

0<a.'<27r.

\addexamplecitation{Mildner.}

%
% 149
%

22. If I 0.' I < 1 and a is not a positive integer, shew that

 =i n~a ~ 1 - e2 ' ' 1 - e2a7r/ j ~ J '

where C is a contour in the plane enclosing the points 0, x.

(Lerch, Casopis, xxi. (1892), pp. 65-68.)

23. If ( 1(2), < 2 ( 2), .. are any polynomials in z, and if F(z) be
any integrable function, and if yj i (z), yjr (z), ... be polynomials
defined by the equations

J a Z - X

f F( ) c i (x) <i>, (x) ... (,\, ( 0 'l> - z)-< (x)

J a Z X

Shew that r (f ) = Ml) + J2j + V-3( )

2 (2) 1 (2) ( 2 (2) 3 (2)

dx

z - x'

24. A system of functions p, (z), pi (2), po (z), ... is defined by
the equations

 0 (S) = 1, Pn + l z) = (2- + n2 + K) Pn z\

where a and 6 are given functions of n, which tend respectively to the
limits and - 1

as 71-*- cx) .

Shew that the region of convergence of a series of the form 2e pn z),
where ej, 621  are independent of 2, is a Cassini's oval with the
foci +1, - 1.

Shew that every function / (2), which is analytic on and inside the
oval, can, for points inside the oval, be expanded in a series

/(2)=2(c + 20f'n(2),

where

Cn = 27 . j ( + 2) ? (2) f (2) dz, Cn' = . j qn z)f z) dz.

the integrals being taken round the boundary of the region, and the
functions q (2) being defined by the equations

?0 ( ) = yl n A-h ' " + I ") XTT TTT 2'n ( )-

2 + ao2 + Oo 2'' + rt +i + 0 + l

(Pincherle, Rend, dei Lincei, (4), v. (1889), p. 8.)

\ 25. Let C be a contour enclosing the point a, and let (f) z) and/
(2) be analytic when z is on or inside C. Let | < | be so small that

\ t(f) z) \ < .\ z - a\ when 2 is on the periphery of C. By expanding

± [ f(.) l- < (2) .

Stti j c 2 - a - 0 (2)

in ascending powers of <, shew that it is equal to

f a)+ I [/'(a)(0(a) ].

Hence, by using §§ 6"3, 6"31, obtain Lagrange's theorem.

