\chapter{Integral Equations} 

11"1. Definition of an integral equation.

An integral equation is one which involves an unknown function under
the sign of integration; and the process of determining the unknown
function is called solving the equation*.

The introduction of integral equations into analysis is due to Laplace
(1782) who considered the equations

f x) = [e f (f) (t) dt, g x) = [ - 4> (t) dt

(where in each case (f) represents the unknown function), in connexion
with the solution of differential equations. The first integral
equation of which a solution was obtained, was Fourier's equation

f(.v) = l cos (xt) (f> (t) dt,

of which, in certain circumstances, a solution isf

2 f

(f)( x) = - I cos (ux)f(u) du,

TTJo

f x) being an even function of a;, since cos xt) is an even function.

Later, Abel+ was led to an integral equation in connexion with a
mechanical problem and obtained two solutions of it; after this,
Liouville investigated an integral equation which arose in the course
of his researches on differential equations and discovered an
important method for solving integral equations §, which will be
discussed in § 11 '4.

In recent years, the subject of integral equations has become of some
importance in various branches of Mathematics; such equations (in
physical problems) frequently involve repeated integrals and the
investigation of them naturally presents greater difficulties than do
those elementary equations which will be treated in this chapter.

To render the analysis as easy as possible, we shall suppose
throughout that the constants a, h and the variables x, y, are real
and further that

* Except in the case of Fourier's integral \hardsectionref{9}{7}) we practically
always need continuom solutions of integral equations.

t If this value of (p be substituted in the equation we obtain a
result which is, effectively, that of\hardsectionref{9}{7}.

+ Solution de quelques problemes a Vaide d'integrales definies (1823).
See Oeuvres, i. pp. 11, 97.

§ The numerical computation of solutions of integral equations has
been investigated recently by Whittaker, Proc. Roijal Soc. xciv. (A),
(1918), pp. 367-383.

U-2

%
% 212
%

O', y, i>' also that the given function* K (x, y), which occurs under
the integral sign in the majority of equations considered, is a real
function of a; and y and either (i) it is a continuous function of
both variables in the range (a x b, a i/ b), or (ii) it is a
continuous function of both variables in the range a y x b and K (x,
y) = when y > x; in the latter case K x, y) has its discontinuities
regularly distributed, and in either case it is

easily proved that, iff(y) is continuous when a y b, f y) K x, y) dy
is a

J a

continuous function of x when a x b.

11 'll. An algebraical lemma.

The algebraical result which will now be obtained is of great
importance in Fredholm's theory of integral equations.

Let (a'i, 3/1, Zy), .V2, i/2, 22)) ( '35 3) 3) be three points at
unit distance from the origin. The greatest (numerical) value of the
volume of the parallelepiped, of which the lines joining the origin to
these points are conterminous edges, is +1, the edges then being
perpendicular. Therefore, if Xr + 7/ + Zr = l r = l, 2, 3), the upper
and lower bounds of the determinant

 2 2/2 22 I

 3 3 23 I

are ±1.

A lemma due to Hadamardt generalises this result.

Let

 21? 0 22, ... a n

A

 nl) )!2'  < n

n

where a,. is real and 2 a" r = l ( i = l, 2, ... n); let A j. be the
cofactor of a r in D and let A be the determinant whose elements are A
j., so that, by a well-known theorem |,

Since Z) is a continuous function of its elements, and is obviously
bounded, the ordinary theory of maxima and minima is applicable, and
if we consider variations in

" dD

air ( '=!) 2, ... n) only, D is stationary for such variations if 2 -
8aij. = 0, where Saj,...

r=l VOlir

n

are variations subject to the sole condition 2 a.ir8air=0; therefore
§

r = l

n

but 2 airAir=I), and so X'2a\ r = D; therefore Air = Da.ir-

r=l

* Bocher in his important work on integral equations (Camb. Math.
Tracts, No. 10), always considers the more general case in which A'
(x, y) has discontinuities regularly distributed, i.e. the
discontinuities are of the nature described in Chapter iv, example 10.
The reader will see from that example that the results of this chapter
can almost all be generalised in this way. To make this chapter more
simple we shall not consider such generalisations.

+ Bulletin des Sci. Math. (2), xvii. (1893), p. 240.

J Burnside and Panton, Theory of Equations, ii. p. 40.

§ By the ordinary theory of undetermined multipliers.

%
% 213
%

Considering variations in the other elements of D, we see that D is
stationary for variations in all elements when Amr=Damj. (m=l, 2, ...
n; r=l, 2, ... n). Consequently A = D". D, and so I)' ' - D ~ . Hence
the maximum and minimum values of I) are +1.

Corollary. If a r be real and subject only to the condition | a I < j
since

2 a,,/( ii/) 2 1,

r=l

we easily see that the maximum value of | i9 | is (/i J/)" = ?i "i/".

11'2. Fredholms* equation and its tentative solution. An important
integral equation of a general type is

J a

where f(x) is a given continuous function, X is a parameter (in
general complex) and K (oo, ) is subject to the conditions f laid down
in § ll'l. K (x, ) is called the nucleusX of the equation.

This integral equation is known as Fredholms equation or the integral
equation of the second kind (see § 11 "3). It was observed by Volterra
that an equation of this type could be regarded as a limiting form of
a system of linear equations. Fredholm's investigation involved the
tentative carrying out of a similar limiting process, and justifying
it by the reasoning given below in\hardsubsectionref{11}{2}{1}. Hilbert (Gottinger Nach.
1904, pp. 49-91) justified the limiting process directly.

We now proceed to write down the system of linear equations in
question, and shall then investigate Fredholm's method of justifying
the passage to the limit.

The integral equation is the limiting form (when 5-*-0) of the
equation

<i> x) =f x) + \ i K X, X,) (t> (.r,) 8, where x - Xq\ i = 8, XQ = a,
.r = b.

Since this equation is to be true when a x b, it is true when x takes
the values Xi, Xi, ... Xni and so

n

-\ 8 2 K x.p, x ) (.rg) + (f) (Xj,) =/ (Xj,) ip = l,2,... n).

q=l

* Fredholm's first paper On the subject appeared in the Ofversigt af
K. Vetenskaps-Akad. Forhandlingar (Stockholm), lvii. (1900), pp.
39-46. His researches are also given in Acta Blath. xxvii. (1903), pp.
365-390.

t The reader will observe that if K x, |) = |>x), the equation may be
written

<p (x) =f (x) +\ Tk (x, I) ( ) d|.

This is called an equation with variable upper limit.

1;. Another term is kernel; French noyau, German Kern.

214

THE PROCESSES OF ANALYSIS

[chap. XI

This system of equations for < (.rp), p=l, 2, ... 7i) has a unique
solution if the determinant formed by the coefficients of x ) does not
vanish. This determinant is

Z> (X) = 1 - X8 K (. 1,xi) -X8K .vi, x. ) ... -UK x, .v ) -\ 8K x.,
, A'l) 1 - XS K x.,, .r.,) ... - X S A' (.rg, . )

-\ 8K (.r, xi) - \ 8 K x, x.,) ... 1-X8K (x, x ) = 1 - X 2 8A' x,
.Vp) +- 2 S2 1 ''''' '  '" " '

p = l 2 !p, q=l I K X, Xp) A [Xq, Xg)

X3 )i

';>> <I, r=l

+ ...

 "(" 'p' ' P' V" p " '9/ ' (." p> " r)

K Xq,Xp) K Xg,.Vq) K Xg,Xr)

lK x,.,Xp) K Xr,Xq) K Xr, Xr)

on expanding* in powers of X.

Making S-3-O, ti-s-qo, and writing the summations a-s integrations, we
are thus led to consider the series

/6 x2 [ [

i)(X) = l-X| A' (li, 0 1 + 2] / j

 ( 2, 1) />:( 2, I2)

d idi;-...

Further, if Z) (.r, x,) is the cofactor of the term in I)n ) which
involves K x\, x J, the solution of the system of linear equations is

,, /( i)i) (.r, xi)+f x.2) Z) (av, .v.i) + ...+f xJD x, Xn)

Now it is easily seen that the appropriate limiting form to be
considered in association with Da Xf, x ) is D ); also that, if /n +
i',

Dn (AV, -iV) = U\ K x, x ) - XS 2

K(x,x ) K x,Xp)

A Xpi Xi,) A (A'p, Xp)

+,\ \ 8 2

 ( A'- V) Xf,.Vp) K x,Xg)

K Xp, .r ) K (Xp, .?7p) (Xp,; g)

A' (.Pg, X ) K (Xg, Xp) K Xg, Xg)

So that the limiting form for 8~ I) x ., x ) to be considered t is

D (.r,Xt,; X) = X A" x

>, Av)-X2 j

2! ja Ja A:( i,

'' I K x,x ) A'(.r, 1) I

K 2,X;) K -2, l) ( 2,6)

 Il' l2-----

Consequently we are led to consider the possibility of the equation
<P(x)=f x) + J \ [ d .x,; ) fii)d giving the solution of the integral
equation.

* The factorials appear because each determinant of s rows and columns
occurs s ! times as p, q, ... take all the values 1, 2, ... n, whereas
it appears only once in the original determinant for D (X).

t The law of formation of successive terms is obvious from those
written down.

] 1 -21] INTEGRAL EQUATIONS

Example 1. Shew that, in the case of the equation

  x) = x->r\ \ xii(l) y)dy, J

D ) = 1\, D (.r, y; X) = \ xy Zx

215

we have

and a solution is

0( ) =

3-X'

we have

Example 2. Shew that, in the case of the equation

<j> x) = x + \ I (xy+y )<f>(i/)dy,

J

D x, y; X) = X xy + ?/2) +X xy -,xy - If + y\ and obtain a sohition
of the equation.

11"21. Investigation of Fredholms solution.

So far the construction of the solution has been purely tentative; we
now start ah initio and verify that we actually do get a solution of
the equation; to do this we consider the two functions D ( ), D x,
y; ) arrived at in\hardsectionref{11}{2}.

We write the series, by which D ( -) was defined in § 11 '2, in the
form 1 + 2 so that

 =i n\

rb fb rb J a -1 a J a

d,d 2  d n;

since K x, y) is continuous and therefore bounded, we have \ K x,y)\ <
M, where M is independent of x and y; since K x, y) is real, we may
employ Hadamard's lemma \hardsubsectionref{11}{1}{1}) and we see at once that

Write n 'M'' (b - a)" = n ! 6; then

lim(6.WM=lim f-" - f|(l+jYf

u oo M oo (71+1)5 (\ nj )

since ( 1 +

The series 2 bnX"- is therefore absolutely convergent for all values
of X,;

w = l

and so \hardsubsectionref{2}{3}{4}) the series 1+2 - ~ converges for all values of X and
there- fore \hardsubsectionref{5}{6}{4}) represents an integral function of X.

Now Tfrite the series for D (x, y;X) in the form 2 - - '- - .

%
% 216
%

Then, by Hadamard's lemma § 1111),

and hence ' < Cn, where c is independent of a; and y and 2 Cn> ' is n\
=o

absohitely convergent.

Therefore D x, y; ) is an integral function of \ and the series for D
x, y\ \ )- \ K (x, y) is a uniformly convergent \hardsubsectionref{3}{3}{4}) series of
continuous* functionsj|of x and y when a x b, a y b.

Now pick out the coefficient of A" (a;, y) in D(x, y;X); and we get

D x,y;X) \ D ) K x, y) + 2 (-) X +', where

Expanding in minors of the first column, we g t Q x, y) equal to the
integral of the sum of n determinants; writing | i, o, ... m-i, |,
,n,   n-\ in place of fi, o, ... in the ??ith of them, we see that
the integrals of all the determinants + are equal and so

Qn x,y) = -n\ K x, y) Pnd d i    d n-i,

J a J a J a

where

Pn=\ K X, I), K X,,), ... K(x, | \ 0

i di.a K A ... A-(inf -o

It follows at once that

D(x,y\ X) = \ D(X)K(x,y)+\ ( D (x, \ X)K ly)dl

. a

Now take the equation

<i> )=fi ) + \' K ly)<f> y)dy,

J a

multiply by D (x,; X) and integrate, and we get

l'f(BD(,; )d

J a

= I % (I) 0 -. ?; ) ? - r [' (' '; ) < ' ) > y* '

.'a . a J i(

the integrations in the repeated integral being in either order.

* It is easy to verify that every term (except possibly the first) of
the series for D (x, y; ) is a continuous function under either
hypothesis (i) or hypothesis (ii) of\hardsectionref{11}{1}.

h

t The order of integration is immaterial \hardsectionref{4}{3}).

t

%
% 217
%

That is to say

J a

= r<j>( )D(w,; ) d -r D w,y; ) -\ D( ) K x,y)]<j>(y)dy

J a J a

= \ D ) \' K x,y)<i> y)dy

in virtue of the given equation.

Therefore if D X) 0 and if Fredholm's equation has a solution it can
be none other than

< X) =f x) +jy( ) d;

and, by actual substitution of this value of <f> x in the integral
equation, we see that it actually is a solution.

This is, therefore, the unique continuous solution of the equation if
i)(X)4=0.

Corollary. If we put /( ) = 0, the ' homogeneous ' equation

J a

has no continuous solution except (j) (.i')=0 unless D X) = 0.

Example 1. By expanding the determinant involved in \$ (.r, y) in
minors of its first row, shew that

D x,y; ) = \ D K)K x,y) + \ \ ' K x, )D,y; ) d .

J a

Example 2. By using the formulae

2)(X) = 1+ i ""fi, D x,y; ) = \ D X) K(x, y)+ i ( - )" " ' ' "/'' \
/i=i 'I 11=1 " 

shew that f " i> (,; X) f l = - X .

Example 3. If K x, y) = 1 (y 0) i \ y) = (y > ' )-> shew that D ) = b-
a) X.

Example 4. Shew that, if K (.r, y)=fi (x) .f [y), and if

-6

fi )f2 x)dx = A,

then

D ) = A\ D x,y; ) = \ h x)f., (y),

and the solution of the corresponding integral equation is

%
% 218
%

Example 5. Shew that, if

K x, y) =/i x) gi y) +f x) 2 y), then D (X) and D x, y; X) are
quadratic in X; and, more generally, if

II K x,y)= 2 f x)g,n ),

m = l

then i)(X) and D x, y, X) are polynomials of degree n in X.

11*22. Volterra's reciprocal functions.

Two functions K x, y), k x, y; X) are said to be reciprocal if they
are bounded in the ranges a- x, y - 1), any discontinuities they may
have are regularly distributed (§ ll'l, footnote), and if

K x,y)+k x,y] ) = \ \ k x, \ \ ) K, y)d .

J a

We observe that, since the right-hand side is continuous*, the surn of
two reciprocal functions is continuous.

Also, a function K x, y) can only have one reciprocal if Z) (X) 4=;
for if there were two, their difference k x, y) would be a continuous
solution of the homogeneous equation

h x,y; X) = X f >(-,(.r, )K k,y)di,

(where x is to be regarded as a parameter), and by § 11 '21 corollary,
the only continuous solution of this equation is zero.

By the use of reciprocal functions, Volterra has obtained an elegant
reciprocal relation between pairs of equations of Fredholm's type.

We first observe, from the relation

B x,y; ) = \ D X)K x,y) + \ \ D x, -X) K H y) d

proved in § 11 "21, that the value of A;(, y; X) is

-D x,y;X)l[\ D ) ], and from\hardsubsectionref{11}{2}{1} example 1, the equation

k x, y; ) + K x, y) = X f K x, ) k, y;X)d

J a M

is evidently true.

Then, if we take the integral equation

< > x)=f(x) + xl'K x, )<f>( )d,

J a

when a'> x h, we have, on multiplying the equation

J a * By example 10 at the end of Chapter iv.

%
% 219
%

by k (x,; ) and integrating,

J a

= r k(x, : ) f( )d + xf' I' k(x, : ) K 1,)<f>(,)d,dl

J a J II J a

Reversing the order of integration* in the repeated integral and
making use of the relation defining reciprocal functions, we get

\ \ x, : ) 4>( )di

J a

= !'k(w, : ) f( )d +r K(x,,) + k w,,;X) < (|0 i,

J a J a

and so X f '(, X)/( )fZ = -X l K (x, ) <f> (,) d,

J a n

= -<P(x)+f x). Hence f(x) = ( ) + \ f ' (a;, f; ) /( ) d;

. a

similarly, from this equation we can derive the equation

4> x)=f(x) + \ f'K(x, )4>( )d,

J II

so that either of these equations with reciprocal nuclei may be
regarded as the solution of the other.

11"23. Homogeneous integral equations.

rb The equation < x) = X ) K x, ) </> ( ) d is called a homogeneous
integral

- n

equation. We have seen (§ 1121 corollary) that the only continuous
solution of the homogeneous equation, when D (X) 4= 0, is ( x) = 0.

The roots of the equation D (X) = are therefore of considerable
importance in the theory of the integral equation. They are called the
characteristic numbers of the nucleus.

It will now be shewn that, when D (X) = 0, a solution which is not
identically zero can be obtained.

Let+ X = Xo be a root m times repeated of the equation D (X) = 0.

Since D (X) is an integral function, we may expand it into the
convergent series

D (X) = Cm (X - Xo)" + c,n, (X - Xor +1 + . . . (m > 0, c, + 0).

* The reader will have no diflSculty in extending the result of\hardsectionref{4}{3}
to the integral under consideration.

t French valeurs caracteristiques, German Eigenicerthe.

J It will be proved in\hardsubsectionref{11}{5}{1} that, if K (x, ij) = K y, x), the
equation D (X) = has at least one root.

%
% 220
%

Similarly, since D x, y; X) is an integral function of \, there
exists a Taylor series of the form

by\hardsubsectionref{3}{3}{4} it is easily verified that the series defining g (cc, y), (n
= 1, 1 + 1, ...) converges absolutely and uniformly when a x b, a y -
b, and thence that the series for D (x, y; ) converges absolutely and
uniformly in the same domain of values of x and y.

But, by\hardsubsectionref{11}{2}{1} example 2,

L

i)(f.f;X)df=-X >

now the right-hand side has a zero of order m - 1 at A,o, while the
left-hand side has a zero of order at least I, and so we have m-\' I.

Substituting the series just given for D ( ) and D x,y\ X) in the
result of\hardsubsectionref{11}{2}{1} example 2, viz.

D x,y- ) = XD (X) K x, y) + \ \ ''K x, ) D I y; ) d

J a

dividing by (X, - XoY and making X -* X,o. we get

9i, y) = \ K x, I) gi (, y) d .

J a

Hence if y have any constant value, gi x, y) satisfies the homogeneous
integral equation, and any linear combination of such solutions,
obtained by giving y various values, is a solution.

Corollary. The equation

<l> G ) =/(. -) + Xo f ' iT (.r, ) < ( ) d

J a

has no solution or an infinite number. For, if ( x) is a solution, so
is x) + c gi (x, y),

y where Cy may be any function of y.

Example 1. Shew that solutions of

< x) = \ I cos'<(.r-|)( ( )c/

are </> (.r) = cos (?i - 2r) .r, and ( (.r) = sin (% - 2?-) .r; where
r assumes all positive integral values (zero included) not exceeding
hi.

Example 2. Shew that

  x) = \ y cos'' (.r-h I) ( ( ) o?|

has the same solutions as those given in example 1, and shew that the
corresponding values of X give all the roots of D (X) = 0.

%
% 221
%

11*3. Integral equations of the first and second kinds. Fredholra's
equation is sometimes called an integral equation of the second kind;
while the equation

f x) = \ \ \ \ {x, ), )d

J a

is called the integral equation of the first kind.

In the case when K x, ) = Q ii > x, we may write the equations of the
first and second kinds in the respective forms

f x) = \ [ K x, ) )dl

J a

4> a )=f x) + xrK x, )cf>( )d .

J a

These are described as equations with variable upper limits.

11"31. Volterra's equation.

The equation of the first kind with variable upper limit is frequently
known as Volterra's equation. The problem of solving it has been
reduced by that writer to the solution of Fredholm's equation.

Assuming that K x, ) is a continuous function of both variables when
\$ X, we have

f x) = \ \ K x, )< )dl

J a

The right-hand side has a differential coefficient \hardsectionref{4}{2} example 1)
if -T- exists and is continuous, and so

ox

f (x) = \ K (x, x) <p X) + Xj -cf> ) dl

This is an equation of Fredholm's type. If we denote its solution by
x), we get on integrating ft-om a to x,

f x)-f a)=\ \ K x, )<l> )dl

J a

and so the solution of the Fredholm equation gives a solution of
Volterra's equation if /(a) = 0.

The solution of the equation of the first kind with constant upper
limit can frequently be obtained in the form of a series *.

11"4. The Liouville- Neumann method of successive substitutions "f. A
method of solving the equation

cf> x)=f(x) + \ \ ' K(x, )cf ( )d,

J a

which is of historical importance, is due to Liouville.

* See example 7, p. 231; a solution valid under fewer restrictions is
given by Bocher. t Journal de Math. ii. (1837), iii. (1838). K.
Neumann's investigations were later (1870); see his Untersuchungen
ilber das logarithmische und Newton'sche Potential.

%
% 222
%

It consists in continually substituting the value of 4) oo) given by
the right-hand side in the expression < ( ) which occurs on the
right-hand side.

This procedure gives the series S x)=f x) + \ \ ' K x, aA?) + S X- f K
x, 0 [* K I,)

J a m = 2 J a J a

J a

Since | K x, y) \ and \ f x) \ are bounded, let their upper bounds be
M, M'. Then the modulus of the general term of the series does not
exceed

|\ [ ' l/' ir(6-a)'". The series for S x) therefore converges
uniformly when

\ \ \ < M-' h-a)-; and, by actual substitution, it satisfies the
integral equation.

If Kix, y) = when y>x\ we find by induction that the modukis of the
general term in the series for S x) does not exceed

X I * Jf'" M' x - a)>"l m l) \ X\'>' i/' M' (6 - ay>'jm !,

and so the series converges uniformly for all values of X; and we
infer that in this case Fredholm's solution is an integral function of
X.

It is obvious from the form of the solution that when | X. | < M~ (b -
a)~\ the reciprocal function k (x,; X) may be written in the form

k x, X)=.-K x, )- t X-- f' K(x, 0 f /l (I,, eO

m=2 J a J a

for with this definitipn of k x,; A,), we see that

S x)=f x)-\ \ ' k(x,; ) f( )d,

J a

so that k x, |;X) is a reciprocal function, and by\hardsubsectionref{11}{2}{2} there is
only one reciprocal function.

Write

K (X, I) = K, X, ), f K X, r) Kn (r, B r - J n+r (, ),

and then we have

-k x,;X)= S V K, +,(x, ),

m =

while f K (x, r ) Kn (r, e r = A .+ (*'. )>

J a

as may be seen at once on writing each side as an (m + n - l)-tuple
integral. The functions K, (x, f) are called iterated functions.

J a

%
% 223
%

11*5. Symmetric nuclei.

Let Ki x, y) = K (y, x); then the nucleus K x, y) is said to be
symmetric. The iterated functions of such a nucleus are also
symmetric, i.e. Kn, y) = Kn (y, x) for all values of n; for, if Kn
x, y) is symmetric, then

Kn+r (, y) = f K,, ) Kn I y) d = j K, (|, o:) K, y, ) d

J a 'a

= Kn(y, )K,i,x)d = Kn, y,x),

J a

and the required result follows by induction.

Also, none of the iterated functions are identically zero; for, if
possible, let Kp (x, y) = 0; let n be chosen so that 2"" <]) - 2",
and, since Kp (x, y) = 0, it follows that K n x, y) = 0. from the
recurrence formula.

But then = K n x, x) = I, \ i x, ) K -i (, ) d

J a

= f [Kn-.i . rr-d,

J a

and so K n-i i, ) - j continuing this argument, we find ultimately
that Ki (x, y) = 0, and the integral equation is trivial.

11"51. Schmidt's* theorem that, if tfoe nucleus is symmetric, the
equation J) ( ) = has at least one root.

To prove this theorem, let

Un = Kn (X, X) dx,

J a

SO that, when \ \ \ < M~ (b - a)~\ we have, by\hardsubsectionref{11}{2}{1} example 2 and §
114,

\ 1 dDiX)

D ) dx ri " 

Now since I I fjLKn+i (x, |) +,i\ i (x, )]- d dx

J a J a

for all real values of fi, we have

tl'U . + 2/JLU.>n + U,n-, > 0, and so U n+i U.n-- U n-, Uon-1 > 0.

Therefore U.,, Ui, ... are all positive, and if- JZ /C, = i, it
follows, by in- duction from the inequality ?7o +2 C/2,1-2 C 2 i'.
that Unn+n/U.n > v -

X

Therefore when j X- v~\ the terms of S Un'. ' ~ do not tend to zero;

and so, by\hardsectionref{5}{4}, the function, - -7- - has a singularity inside or
on the

* The proof given is due to Kneser, Palermo Bendiconti, xxii. (1906),
p. 236.

%
% 224
%

circle \ \ = v~ \ but since D( ) is an integral function, the only
possible

sinerularities of -r., . - j- are at zeros of D(X); therefore D (X)
has a zero ° 1) (X,) d\

inside or on the circle \ \ \ = v~ .

[Note. By\hardsubsectionref{11}{2}{1}, Z> (X) is either an integral function or else a
mere polynomial; in the latter case, it has a zero by\hardsubsectionref{6}{3}{1} example
1; the point of the theorem is that in the former case D (X) cannot
be such a function as e ', which has no zeros.]

11'6. Orthogonal functions.

The real continuous functions (jj (x), (f)o (x), . . . are said to be
orthogonal and normal* for the range a, b) if

If Ave are given n real continuous linearly independent functions u-
x), Uo x), ...Un(x), we can form n linear combinations of them which
are orthogonal.

For suppose we can construct m - 1 orthogonal functions </>!, ...
(f>m-i such that (f>p is a linear combination of u-, ii, ... Up
(where p = 1, 2, ... ni - I); we shall now shew how to construct the
function, such that c i, <p.2,  (f>m are all normal and
orthogonal.

Let (f>m (x) = Ci, m (f>i ( ) + C2, m < 2 ( ) +  + Cm-i <pm-i (i )
+ Urn ),

so that i( is a function of Mj, lu, ... w, . Then, multiplying by and
integrating,

i< ni x) (t>p oc) dx = Cp,, + I Um ) <j>p (x) dx p < m). Hence I i4>m
i ) 4>p oc) dx =

J a

if Cp m == - Uin (jc) <pp ( ) dx;

J a

a function i, (x), orthogonal to (f) (x), (f>. (x), . . . <pm-i (x),
is therefore con- structed.

rb

Now choose a so that a- I [i(f>r,i (x)]- dx=l;

J a

and take <p,n (x) = a. i<, . (x).

Then j </) (x) 4>p (x) dx |~ '

We can thus obtain the functions (/)i, c/).., ... in order.

* They are said to be orthogonal if the first equation only is
satisfied; the systematic study of such functions is due to Murphy,
Camb. Phil. Trans, iv. (1833), pp. 353-408, and v. (1835), pp.
113-148, 31.5-394.

%
% 225
%

The members of a finite set of orthogonal functions are linear! '
inde- pendent. For, if

a,(j), (x) + a.2(f>. (x)+ ... + an(f>n ( v) = 0,

we should get, on multiplying by (f>p (a;) and integrating, a =;
therefore all the coefficients a vanish and the relation is nugatory.

It is obvious that tt ~ cost/u;, tt ~ - sin mx form a set of normal
orthogonal functions foifthe range ( - tt, tt).

Example 1. From the functions 1, .r, x-, ... construct the following
set of functions which are orthogonal (but not normal) for the range (
- 1, 1) :

1, X, - i, x -'jx, x* - x- + - g, ....

Example 2. From the functions 1, x, x, ... construct a set of
functions

which are orthogonal (but not normal) for the range (a, h); where

[A similar investigation is given in >5 15-14.]

11 "61. The connexion of orthogonal functions with homogeneous
integral equations.

Consider the homogeneous equation

<l> x) = \,\% )K x, )dl

  a

where Xq is a real* characteristic number for K(x, ); we have already
seen how to construct solutions of the equation; let n linearly
independent solutions be taken and construct from them n orthogonal
functions (pi, < .\,, ... (/> .

Then, since the functions, are orthogonal, rr i <l>Uy)f K x, )cf>, (
)di\'dy = i f\ < p.n(y)fK(x, )<f>, )Ydi

J a |\ t = 1 J a J m = lJ a [\ J a J

and it is easily seen that the expression on the right may be written
in the form

rb

i If K(x. )4>, ( )d

H = 1 (. J a

>H = 1

on performing the integration with regard to y; and this is the same
as i r K x, y) cf>,n (y) dyi' K x, |),, ( ) rff

m. = \ J a J a

Therefore, if we write K for K x, y) and A for

i 4>Ay)\' K, )<l>.nX )dl

m=\ J a

* It will be seen immediately that the characteristic numbers of a
symmetric nucleus are all real.

W. M. A. 15

Therefore

%
% 226
%

rb rb

we have A?dy = \ KAdy,

J a J a

rb rb rb

and so 1 A-dij = K'dy - (K - Kfdy.

la J a J a

a \ m = \ f o ) J a

and SO Xo~' <, ( ) [ [K x,y)Ydy.

w = 1 J a

Integrating, we get

n <: Xo'- I / [K x, y)Y-dydx.

J a J a

This formula gives an upper limit to the number, 7i, of orthogonal
functions corresponding to any characteristic number Xo-

These n orthogonal functions are called characteristic functions (or
auto- functions) corresponding to Xq.

Now let '"* (x), < <i' (x) be characteristic functions corresponding
to different characteristic numbers Xq, Xj.

Then </>' > (x) </><!' (x) = xJ K x, ) ( c) x) </)' ' ) d,

J a

and so

[ (f> ' >(x)(f) ' (x)dx = \,\ i K(x, )4> '> (x)cf)'' )d dx ...(1),

and similarly

[ (/)' ' (x) </) w (x) dx = \ of I K x, ) 0 " ( ) </)(!> (x) d dx

la J a J a

= X [ [ K(lx)cf> '' ix)(f> ' ( )dxd ...(2),

J a ' a

on interchanging x and .

We infer from (1) and (2) that if \ \ and if K x, ) = K, x),

I 4> ' (x)(f> ' (x)dx=0,

J a

and so the functions < "'' (x), '" (x) are mutually orthogonal.

If therefore the nucleus be symmetric and if, corresponding to each
characteristic number, we construct the complete system of orthogonal
functions, all the functions so obtained will be orthogonal.

Further, if the nucleus be symmetric all the characteristic numbers
are real; for if Xq, Xj be conjugate complex roots and if* Uq x) = v
(x) + iw (x) be

* V (x) and w (x) being real.

117] INTEGRAL EQUATIONS 227"

a solution for the characteristic number X, then iii (x) = v (x) -
iiu (x) is a solution for the characteristic number \; replacing 0*''
(x), 'i* (x) in the equation

I ( 'o' (x) </>< ' (x) dx =

J a

by V (x) + iw (x), v x) - iw (x), (which is obviously permissible), we
get

f [\ v(x)Y+ \ w x)Y]dx = 0,

J a

which implies v (x) = w (x) = 0, so that the integral equation has no
solution except zero corresponding to the characteristic numbers Xq.
il this is

contrary to § 11 "23; hence, if the nucleus be symmetric, the
characteristic

numbers are real.

11 7. The development* of a symmetric nucleus.

Let < j x), (f)., (x), 3 (x), ... be a complete set of orthogonal
functions satisfying the homogeneous integral equation with symmetric
nucleus

cf>(x) = xfK(x, )<f>( )dl

J a

the corresponding characteristic numbers beingf Xj, X, X, -

XT J. 1, f' < n (* ) < i(v)  -r 1

Now supposel that the series - - - - is umiormly convergent

when a x b, a y %b. Then it luill he shewn that

K(x.y)=i M My). =i

For consider the symmetric nucleus

71 = 1 / n

If this nucleus is not identically zero, it will possess (§ 11 '51) at
least one characteristic number /j,.

Let ylr(x) be any solution of the equation

 jr x)=fJ,f H(x.. )ylri )dl

' a

which does not vanish identically.

Multiply by cf)n (x) and integrate and we get

ry!r(x)cf>,, x)dx = f,f Hxix, )- I "t if l l ( ) cj U ) dxd;

J a a ' a [ n = l 7i )

* This investigation is due to Schmidt, the result to Hilbert.

t These numbers are not all different if there is more than one
orthogonal function to each characteristic number.

X The supposition is, of course, a matter for verification with any
particular equation.

15-2

>t\

%
% 228
%

since the series converges uniformly, we may integrate term by term
and get

f ir (.v) 4> x) dx = \ \ ( ) (/> i )d - r < (I) f ( ) d

= 0.

Therefore t/ (x) is orthogonal to < i (x), < 2 ( ),  ', and so
taking the equation

J a [ n = l n )

rb

we have yjr (x) = y"- / K (x, )' ) d .

J a

Therefore /i, is a characteristic number of K (x, y), and so -v/r x)
must be a linear combination of the (finite number of) functions < n
(*') corresponding to this number; let

m

Multiply by (j>m (x) and integrate; then since -yfr (x) is orthogonal
to all the functions (j)n (x), we see that a, = 0, so, contrary to
hypothesis, yjr (x) = 0.

The contradiction implies that the nucleus H (x, y) must be
identically zero; that is to say, K (x, y) can be expanded in the
given series, if it is uniformly convergent.

Example. Shew that, if Xq be a characteristic number, the equation

4> x)=f .v) + \ J'' E x, ) )d

J a

certainly has no sokition . unless f x) is orthogonal to all the
characteristic functions corresponding to .

11 "71. The solution of Fredhohns equation by a series. Retaining the
notation of § 11 '7, consider the integral equation

   x) =f x) +\ f K (x, )< ) dl

-' a

where K x, ) is symmetric.

If we assume that ( ) can be expanded into a uniformly convergent

00

series 2 an4>n (1), we have

00 00 A

2 an(f)n x!)=f(x)+ 1 -an(f>n( ), ' w = l M=l n

SO that f x) can be expanded in the series

00 A -A

S an -\ (f)n x).

Hence if the function f x) can be expanded into the convergent series

< . " b \

2 bn< >n(ix), then the series 2 . " \ <f>n ), if it converges
uniformly in

n = l n-l n

the range a, b), is the solution of Fredhohns equation.

%
% 229
%

X

To determine the coefficients 6, we observe that S bn (f)n ( )
converges uni-

n=l

formly by\hardsubsectionref{3}{3}{5}*; then, multiplying by (f)n(x) and integrating, we
get

J a

11 "8. Solution of AbeVs integral equation. This equation is of the
form

  -llw

d\$ 0<fi<l, a x b),

where/' (x) is continuous aud/(a)=0; we proceed to find a continuous
solution u (x)

Let (f)(x)= I u (I) o?|, and take the formula t

J a

7T n dx

sin/iTT ~ J z-xY~> x - y

multiply by u ( ) and integrate, and we get, on using Dirichlet's
formula \hardsubsectionref{4}{5}{1} corollary\

  f' f x)dx

Since the original expression has a continuous derivate, so has the
final one; therefore the continuous solution, if it exist, can be
none other than

"" ''V~dzj z-xy-> '

and it can be verified l)y substitution J that this function actually
is a solution.

11 "SI. Schlomilch's integral equation.

Let f x) have a continuous differential coefficient when - tt .r tt.
Then the equation

2 /"i" f x)=- (f) xsmd)de

  J n

has one solution with a continuoiis differential coefficient when - it
x it, namely

/ in < x)=f 0) + x ' f' xsm6)d6. Jo From\hardsectionref{4}{2} it follows that

f (.v) = - sin Oct)' x sin (9) dd j

(so that we have (0)=/(0), ' (0) = |7r/' (0)).

* Since the numbers X, are all real we may arrange them in two sets,
one negative the other positive, the members in each set being in
order of magnitude; then, when ' X ! > X, it is evident that X /(X -
X) is a monotonic sequence in the case of either set.

t This follows from\hardsubsectionref{6}{2}{4} example 1, by writing (z - x)l(x - f ) in
place of x.

X For the details we refer to Bocher's tract.

§ Zeitschrift filr Math, und Phys. ii. (1857). The reader will easily
see that this is reducible to a case of Volterra's equation with a
discontinuous nucleus.

%
% 230
%

Write .r sin // for x\ and we have on multiplying by x and integrating

X j /' x sin \//-) d\ lr - - I < I sin ( ' (.i- sin 6 sin \//-) c? \
dyjr.

Change the order of integration in the repeated integral \hardsectionref{4}{3}) and
take a new variable x in iDlace of \ f defined by the equation sin; =
sin 6 sin yj/.

rr,. fi" .,, .s,, 2.V fi ( f (b' (x sin v) cos Y dv],

Then xj f'ixsn.i.)d = j \ j -~J -~ - \ de.

Changing the order of integration again \hardsubsectionref{4}{5}{1}),
$$
TODO
$$
and so x \ f x sin y\ r) d = x \ </>' (.r sin x) cos;>( o?;

Jo J

= 0Gt-)-( (O).

Since (f) (0) =/ (0), we must have

(j) (.r) =/ (0)+x ' f (x sin;/.) c a/;

y

and it can be verified by substitution that this function actually is
a solution.

REFERENCES.

H. Bateman, Report to the British Association*, 1910.

M. BocHER, Introditction to Integral Equations (Cambridge Math.
Tracts, No. 10, 1909).

H. B. Heywood et M. Fr chet, Liquation de Fredkolm. (Paris, 1912).

V. Volte ra, Lego7is sur les equations integrales et les equations
integro-differentielles (Paris, 1913).

T. Lalesco, Introduction a la the'orie des equations integrales
(Paris, 1912).

I. Fredholm, Acta Mathematica, xxvii. (1903), pp. 365-390.

D. HiLBERT, Orundzilge einer allgemeinen Theorie der linearen
Integralgleichungen (Leipzig, 1912).

E. Schmidt, Math. Ann. lxiii. (1907), pp. 433-476.

E. Goursat, Cours d' Analyse, iii. (Paris, 1915), Chs. xxx-xxxill.

MiSCELT.ANEOUS EXAMPLES.

1. Shew that if the time of descent of a particle down a smooth curve
to its lowest point is independent of the starting point (the particle
starting from rest) the curve is a cycloid. \addexamplecitation{Abel.}

* The reader will find a more complete bibliugrapby in this Report
than it is possible to give here.

%
% 231
%

2. Shew that, if/( ) is continuous, the sohition of

(f) (x) =f x) + \ j COS (2.rs) <f) (s) ds

/(.r) + X I f s)cos 2jcs)ds

'" *( >= Hrjxv .

assuming the legitimacy of a certain change of order of integration.

3. Shew that the Weber-Hermite functions

satisfy 4> x) = \ I e '*- (s) ds

for the characteristic values of X. (A. Milne.)

4. Shew that even periodic solutions (with period Stt) of the
diflferential equation

- - + ( 2 + >l-2 cos2 x) cf) (.r) = satisfy the integral equation

<f> x) = \ j e < oo'=< '(f, s)ds. (Whittaker; see\hardsubsectionref{19}{2}{1}.)

5. Shew that the characteristic functions of the equation

are <f (x) = cos m.v, sin inx,

where \=m' and m is any integer.

6. Shew that <p (x) = / " - -f (|) d\$

has the discontinuous solution <f) (x) = Lr ~ . \addexamplecitation{Bocher.}

7. Shew that a solution of the integral equation with a symmetric
nucleus

f :v)=rK x, )<t>i\$)di

J a

is 0( )= 2 a \ \ (f) (x),

n=l

provided that this series converges uniformly, where X, 0 (x) are the
characteristic numbers and functions of K x, |) and 2 a 0 (x) is the
expansion off(x).

n=l

8. Shew that, if | A ] < 1, the characteristic functions of the
equation

'f' '' =LI[ i-2hcL i- Hh'' "

are 1, cos?n.r, sin mx, the corresponding characteristic numbers being
1, l/A'", Ijh'", where m takes all positive integral values.