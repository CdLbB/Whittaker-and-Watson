\chapter{The Theory of Riemann Integration} 

4"1. Tlie concept of integration.

The reader is doubtless familiar with the idea of integration as the
operation inverse to that of differentiation; and he is equally well
aware that the integral (in this sense) of a given elementary function
is not always expressible in terms of elementary functions. In order
therefore to give a definition of the integral of a function which
shall be always available, even though it is not practicable to obtain
a function of which the given function is the differential
coefficient, we have recourse to the result that the integral* o f x)
between the limits a and h is the area bounded by the curve y =f(x),
the axis of cc and the ordinates x = a, x = b. We proceed to frame a
formal definition of integration with this idea as the starting-point.

4"11. Upper and lower integrals'f.

Let f(x) be a bounded function of x in the range a, b). Divide the
interval at the points Xi,Xo, ... Xn-iia x - x ... Xn-i b). Let U, L
be the bounds of /(a;) in the range (a, b), and let Ur, L be the
bounds of f x) in the range (xr-i, Xr), where Xq = a, Xn=b.

Consider the sums:|:

Sn = U, (iCi - a) +Uo Xn-X,)+ ...+ Un (b - Xa-,), Sn = L, X - a) + Z/2
( 2 - i) + • • • + Ln (b - Xn-i).

Then U(b- a) Sn s., L b- a).

For a given n, Sn and s are bounded functions of x, x, ... n-i- Let
their lower and upper bounds § respectively be S, Sn, so that Sn, s
depend only on n and on the form of f x), and not on the particular
way of dividing the interval into n parts.

* Defined as the (elementary) function whose differential coefficient
is/(x).

t The following procedure for establishing existence theorems
concerning integrals is based on that given by Goursat, Cours d'
Analyse, i. Ch. iv. The concepts of upper and lower integrals are due
to Darboux, Ann. de I'Ecole norm. sup. (2) iv. (1875), p. 64.

+ The reader will find a figure of great assistance in following the
argument of this section. Sn and s represent the sums of the areas of
a number of rectangles which are respectively greater and less than
the area bounded by y=f(x), x-a, x-h and ?/ = 0, if this area be
assumed to exist.

§ The bounds of a function of n variables are defined in just the same
manner as the bounds of a function of a single variable (§ 3-62).



62 THE PROCESSES OF ANALYSIS [CHAP. IV

Let the lower and upper bounds of these functions of n be S, s. Then

Sn > S, Sn S.

We proceed to shew that s is at most equal to S; i.e. S' s.

Let the intervals (a, x ), x-, x., ... be divided into smaller
intervals by new points of subdivision, and let

a,y,y, ... yu-i, yk = i), yk+i, • yi-i, yi = 2), yi+1, • Vm-i, h be
the end points of the smaller intervals; let U L,.' be the bounds of
/( ) in the interval y,-i, yr)-

m m

Let T,, = X yr- yr-i) UJ, t,n = t y,. - yr-i) L,!.

r=\ r=l

Since Ui, U.2, ... Uk do not exceed JJi, it follows without difficulty
that

Now consider the subdivision of (a, b) into intervals by the points
x-y, X2, ... Xn-\, and also the subdivision by a different set of
points a? x, ... x'n'-i- Let S'n',s'n' be the sums for the second
kind of sub- division which correspond to the sums *S, Sn for the
first kind of subdivision. Take all the points x, ... Xn-i] x, ...
x'n'\ y as the points y, y., ... y .

Then 8n T, t,, Sn,

and S'n' T n tm s'n' .

Hence every expression of the type Sn exceeds (or at least equals)
every expression of the type s'n'; and therefore S cannot be less
than s.

[For \ i S<s and s - S - 27] we could find an Sn and an s'n' such that
g - S<r], s - s'n'<'n and so s'n'>Sn, which is impossible.]

The bound S is called the upper integral off(x), and is written 1 f(x)
dx;

J a

s is called the lower integral, and written I / x) dx.

J a

If S = s, their common value is called the integral of f(x) taken
between the limits* of integration a and b.

The integral is written I f x)dx. .

ra rb

We define | f x)dx, when a< b, to mean - I f(x)dx. Exam'ple 1, I /
(•*') + (•*-') dx = \ f x) dx+ l (x) dx.

J a J a J a

Example 2. By means of example 1, define the integral of a continuous
complex function of a real variable.

* ' Extreme values ' would be a more appropriate term but ' limits '
has the sanction of custom. 'Termini' has been suggested by Lamb,
Infinitesimal Calculus (1897), p. 207.



4*12, 4" 13] THE THEORY OF RIEMANN INTEGRATION 63

412. Riemann's condition of integrahility*.

A function is said to be ' integrable in the sense of Riemann ' if
(with the notation of §4"11) Sn and s,i have a common limit (called
the Riemann integral of the function) when the number of intervals
x,.\ -, x ) tends to infinity in such a way that the length of the
longest of them tends to zero.

The necessa7-y and sufficient condition that a hounded function should
he integrahle is that S - Sn should teiid to zero luhen the numher of
intervals (xr-i, Xr) tends to infinity in such a way that the length
of the longest tends to zero.

The condition is obviously necessary, for if S and s,i have a common
limit n - Sn - ► as 71 - > 30 . And it is sufficient; for, since Sn
S' s s,i, it follows that if lim (Sn - Sn) = 0, then

lim *S' = lim s = S = s.

Note. A continuous function f x) is 'integrable.' For, given e, we can
find 8 such that \ f(af) - f x")\ < (l b-a) whenever \ x' - x"\ < 8.
Take all the intervals Xg\ i, x-g) less than 8, and then Ug- Lg<€/ b -
a) and so >S' - s <e; therefore <S' - s - -0 under the circumstances
.specified in the condition of integrahility.

Corollary. If <S' and s have the same limit S for one mode of
subdivision of (a, h) into intervals of the specified kind, the limits
of .S' and of s for any other such mode of subdivision are both 8.

Example I. The product of two integi'able functions is an integrable
function.

Example 2. A function which is continuous except at a finite number of
ordinary discontinuities is integrable.

\ li f x) have an ordinary discontinuity at c, enclose c in an
interval of length S,; given f, we can find 8 so that j f x')-f x) \
< e when i x' -x \ <8 and x, x' are not in this interval.

Then *S' -s,j f (6-a-8i) + /'Si, where k is the greatest value of \ f
x')-f x)\, when X, x' lie in the interval.

When Si- 0, y(-- |/(c + 0)-/(c-0), and hence lim (.S' -s )=0.]

 -*

Example 3. A function with limited total fluctuation and a finite
number of ordinary discontinuities is integrable. (See § 3"64 example
2.)

413. A general theorem on integration.

Let /(a;) be integrable, and let e be any positive number. Then it is
possible to choose S so that

n rh

t xp - Xj,\ i)f(x'p\ i) - f x)dx < 6,

p=\ J a

provided that Xy - Xp\ i- h, Xp\ i x'p\ i%Xp.

* Biemann (Ges. Math. Werke, p. 239) bases his definition of an
integral on the limit of the sum occurring in § -i-lS; but it is then
difficult to prove the uniqueness of the limit. A more general
definition of integration (which is of very great importance in the
modern theory of Functions of Real Variables) has been given by
Lebesgue, Annali di Mat. (3) vn. (1902), pp. 231-359. See also his
Lecons sur V integration (Paris, 1904).



64 THE PROCESSES OF ANALYSIS [CHAP. IV

To prove the theorem we observe that, given e, we can choose the
length of the longest interval, B, so small that S - Sn < e.

w

Also Sn> S ( - \ i)/(a?' -i) Sn,

p = l

  a

Therefore



  rb

2 xp - Xp\ )f(a;'p\,) - f x) dx

9 = 1 J a



 s.,



  f-Jn ' n



< 6.

As an example* of the evaluation of a definite integral directly from
the theorem of this section consider I " - j, where X<1.

Jo (1- 2)2

Take S= - arc sin X and let,?,= sin s8, (0 <s8 <h tt), so that

 s+i-A's=2 sin |S cos (s+ ) 8< S;

also let Xg = sin (s + i) 8.

, P . c - - s-i sin sS - sin (5 - 1)S

Then 2 - '- = 2 \,.

s=i(i\ y2 \ j)i .=1 cos(s-A)S

= 2/9 sin 2 S

= arc sin X. sin |S/(JS) .

By taking p sufficiently large we can make



P dx I Xg-Xs-i

Jo (l\,,.2)i,s=l(l\ . '2 \ j)4



arbitrarily small.

We can also make arc sin X . < -j-| 1

arbitrarily small.

That is, given an arbitrary number f, we can make



P dx

/ i~



<e



arc sin X

by taking p sufficiently large. But the expression now under
consideration does not

depend on p; and therefore it must be zero; for if not we could take
c to be less than it,

and we should have a contradiction.

rx f g That is to say I '- - =arc sin X.

Jo (i\ .' )2

Example 1. Shew that

X 2x (n - l)x

I+COS- + COS f-...+cos -

,. 71 n n sm

Iim - - - . - - - - - - = .

n-*-'x> *'' -

Example 2. If f x) has ordinary discontinuities at the points aj, 02,
..., then

fb ( fu -S, fa.,-S., [b 1

f x)dx = \ \ m\ \ + +...+ f x)dx\,

J a \ J a J a, +6, J ax + <c J

where the limit is taken by making 81, S2, ... §, ei, t i ••• f* tend
to +0 independently. * Netto, Zeitschriftfilr Math, und Phys. xl.
(1895).



4-14] THE THEORY OF RIEMANX INTEGRATION 65

Example 3. If /( ) is integrable when i x i and if, when Oj a < 6 < 6i
, we write

/ f x)dx = <i> a, b), and if/(6 + 0) exists, then



lim < (,\& + fi)-> K\&) ( o)

Deduce that, i f(x) is continuous at a and b,

d\

da



jjix) dx= -f a), -I jjix) dx=f b). Bxample 4. Prove by differentiation
that, if (f> x) is a continuous function of ./; and



dx

-y- a continuous function of t, then

at



fxi l ft fir

 < x)dx=\ \ x)'j dt.



Example 5. If /' x) and < ' x) are continuous when a x b, shew from
example 3 that

r / (x) <i> o;) dx + J'l 4>' x)f x) dx=f b) < (b) -f a) cf> (a).

Example 6. If/(.r) is integrable in the range (a, c) and 6 c, shew
that I f x) dx

J a is a continuous function of b.

414. J/c ?i Fa we Theorems.

The two following general theorems are frequently useful.

(I) Let U and L be the upper and lower bounds of the integrable
function /(.r) in the range (a, b).

Then from the definition of an integral it is obvious that

J' U-f x)] dx, j ' fj(x) L) dx

are not negative; and so

U b-a)- l f x)dx L b-a).

This is known as the First Mean Value Theorem.

li' f x) is contimious we can find a number | .such that a- b and such
that/( ) has any given value lying between U and L (§ 3-63). Therefore
we can find | such that



rf x)dx = b-a)f \$).

J a



If F x) has a continuous differential coefficient F' (x) in the range
(a, 6), we have, on

writing F' (x) for f(x),

F b)-F d) = b-a)F' )

for some value of such that a b.

Example. lif x) is continuous and ( x)' 0, shew that can be found such
that



' fix) cf> (x) dx =f (I) / % (x) dx.

a J a.



W. M. A.



66 THE PROCESSES OF ANALYSIS [CHAP. IV

(11) Let /(.? ) and 4> .v) be integrable in the range (o, b) and let
(a-) be a positive decreasing function of .r. Then Bonnefs* form of
the Second Mean Value Theorem is that a number exists such that a | 6,
and

 " f x)ci> x)dx 4> a) \ \ f x)dx. \ y'

J a J a

For, with the notation of §§ 4'1-4'13, consider the sum

p ♦ .;S'= 2 Xs-x,\ i)f x,\ )(i> x,\ ).

s=l

Writing x - x y) f x,\ i) = a,\ i, Xs-i) = 4>s-\, o + i + --- + 08 =
s, 'e have

Each term in the summation is increased by writing b for 6g\ i and
decreased by writing b for ftg\ i, if b, b be the greatest and least
of 6o, 6i, ... 6p\ i; and so b(j)(, S b(Po-

m

Therefore S lies between the greatest and least of the sums ( ) xq) 2
(xg-Xg\ i)f Xg\ i)

s=l

where m = l, 2, 3, ... p. But, given e, we can find 8 such that, when
Xg-Xg\ i<d,

p f p I

2 x, - x,\ i)f Xs i) (t> (.r,\ i) - I f x) (f) (x) dx < e,

S=l J 0 I

m Cxra [

< (:ro) 2 X, -Xs-i)/ (X, \ i) - < ( o) / / (• ) < *- < f, s=l a I

and so, writing a, b for a'q, . p, we find that / f x)(j> x)dx lies
between the upper and

J a'

lower bounds ott < (a) I ' f .v)d.v±2e, where j may take all values
between a and ?>.

Let C and L be the upper and lower bounds of </> (a) j f x) dx.

J a

fh

Then U+ 2e I /'(.r) < ( 0 dx' L-2e for a jjositive values of e;
therefore

r I f (x) cfi (x) dx L.

Since (j) a) I \ f x) dx qua function of j takes all values between
its upper and lower J

bounds, there is some value, .say, of |i for which it is equal to I f
(x) (f> (x) dx. This proves the Second Mean Value Theorem.

E.rarnple. By writing ( x) -(f> b)\ in place of (f) (.v) in Bonnet's
form of the mean value theorem, .shew that if < (x) is a monotonic
function, then a number | exists such that a \$ b and

\ f x)(l> x)dx = 4> a) j f x)dx + 4) b) j f x)dx.

(Du Bois Reymond.)

* Journal de Math. xiv. (1849), p. 249. The proof given is a modified
form of an iuvestigatiou due to Holder, Gdtt. Nach. (1889), pp. 38-47.

+ By § 413 example 6, since /(.r) is bounded, I ' f(x) d.v is a
continuous function of fj.



4 '2] THE THEORY OF RIEMANN INTEGRATION 67

42. Differentiation of integrals containing a parameter.

The equation* f x, a)dx=\ dx is true if f(x, a) possesses a

ace J (I (I vCL

Riemann integral with respect to x and fa. = \ is a continuous
function of hoth-f the variables x and a.

For I- '/(., ) <fa = lim f .Aj;.\ ° + A)-/( . )

eta j a h-f-O .' a h

if this limit exists. But, by the first mean value theorem, since / is
a continuous function of a, the second integrand is fa x, a + 6h),
where

But, for any given e, a number 8 independent of x exists (since the
con- tinuity of fa is uniform]: with respect to the variable x) such
that

\ fa x, a) -fa (x, a) I < e/(6 - a), whenever | a' - a | < S.

Taking j A | < S we see that ! 6h | < 8, and so whenever \ h < 8,

[H x, a + h) -fix, a) J i *,,,, [ ., m x w m

• - -! y - J \ ' / dx- \ /a (, a) dx I fa (x, a + Oh) - / (x, a) \ dx

< e.

Therefore by the definition of a limit of a function (§ 3"2), lim
i'f(-. + h)-f(, .a)

h O J a h

I"*

exists and is equal to fadx.

J a

Example 1. If a, b be not constant.s but functions of a with
continuous differential coefficients, shew that

 |'/(.- a)d.v=f b, a) -/(a, ) +/ fj:.:

Example 2. If /(.r, a) is a continuous function of both variables, / f
x, a)dx is a

J a continuous function of a.

* This formula was given by Leibniz, without specifying the
restrictions laid on/(.r, a).

t (p x, y) is defined to be a continuous function of both variables
if, given e, we can find 5 such that | 4> x', y') - (/> (x, y)\ < €
whenever (x' - x)' + (y' - y)' -<S. It can be shewn by § 3-6 that if
(x, y) is a continuous function of both variables at all points of a
closed region in a Cartesian diagram, it is uniformly continuous
throughout the region (the proof is almost identical with that of §
3-61). It should be noticed that, if (.r, y) is a continuous function
of each variable, it is not necessarily a continuous function of both
; as an example take

 [x,y)=. p!, 0(0,0) 1;

this is a continuous function of x and of y at (0, 0), but not of both
x and y.

X It is obvious that it would have been sufficient to assume that /
had a Riemann integral and was a continuous function of a (the
continuity being uniform with respect to x), instead of assuming that
/ was a continuous function of both variables. This is actually done
by Hobson, FuJictions of a Real Variable, p. 599.

5-2



68 THE PROCESSES OF ANALYSIS [CHAP. IV

4"3. Double integrals and repeated integrals.

'Letf x, y) be a function which is continuous with regard to both of
the variables x and y, when a x h, a y - /3,

By § 4'2 example 2 it is clear that

j 1 1 /( ' y) dy\ dx, j U J x, y) dx\ dy both exist. These are called
repeated integrals.



Also, as in § 3"62, f(x, y), being a continuous function of both
variables, attains its upper and lower bounds.

Consider the range of values of x and y to be the points inside and on
a rectangle in a Cartesian diagram; divide it into nv rectangles b -
lines parallel to the axes.

Let Z7 i,, L,n, be the upper and lower bounds of f x, y) in one of
the smaller rectangles whose area is, say, Ayn,,j.', and let

71 V n V

- w - m,n -"-m,!!- n,v y -i - j,n - m,/u h, I'- j/t = 1 jj. = 1 ) = 1
;u. = 1

Then *S', >•?, \,, and, as in § 4"11, we can find numbers h,.-, s,,,
which are the lower and upper bounds of Sn,v, V" respectively, the
values of Sn,v, Sn,v depending only on the number of the rectangles
and not on their shapes; and n, s, . We then find the lower and upper
bounds S and s) respectively of,, Sn,v qua functions of n and v;
and S v S s s,, as in §411.

Also, from the uniformity of the continuity of f(x, y), given e, we
can find B such that

'J m,iJ. - Hij/i '" f>

(for all values of m and /i) whenever the sides of all the small
rectangles are less than the number h which depends only on the form
of the function f x, y) and on e.

And then >S', - Sa, < e (6 - a) (/3 - a),

and so S - s < e h - a) (/3 - a).

But S and s are independent of e, and so S = s.

The common value of S and s is called the double integral of f x, y)
and is written

f(x, y) (dxdy).



It is easy to shew that the reijeated integrals and the double
integral are all equal when f :c, y) is a continuous function of both
variables.



4 '3, 4*4] THE THEORY OF RIEMANN INTEGRATION 69

For let Y j, A, be the uppei- and lower bounds of

as .V varies between x -i and a:, .

Then 2 Y, (x, - x, -i) > \ f x,y)dy\ dx A, (a;,,, - x,,,\ .

"1 = 1 <' \ 3 a. ) i = l

But* 2 Ura,,.i y.-y,.-i) \,n .\ n 2 Z,,m (y/x - m-i)-

M = l /n = l

Multiplying these last inequalities by x -Xm x, using the preceding
inequalities and summing, we get

2 2 r .M,. / ]/ f x,y)dy\ dx 2 2 L,, A,;

 ( = 1 pi = l J a \ J a. ) m = l /x = l

and so, proceeding to the limit,

'S' £ [jy x,y)dy dx s.

But ' =s=l lfix,y) dxdy),

and so one of the repeated integrals is equal to the double integral.
Similarly the other repeated integral is equal to the double integral.

Corollary. If/(.v, y) be a continuous function of both variables,

44. Infinite integrals.

If lini 1 /(.r)c?j;j exists, we denote it by f x)dx; and the limit in
question is called an infinite integral;. Examples.

,. r d'x,. (I \ \ 1

  j ( ' + a')' " b V 2 (62 + a') + 2a2y' 2a '

(3) By integrating by parts, shew that / t"e~ dt = n. (Euler.)

J

Similarly we define / f(x)dx to mean lim / f(x)dx, if this limit
exists; and

J -X a- - -o J a

/f x)dx is defined as / f(x)dx+l f .v)dx. In this last definition the
choice -00 J -<x>' J a

of a is a matter of indifference.

* The upper bound of f x, y) in the rectangle --i,, is not less than
the upper bound of /(.r, y) on that portion of the line .r = | which
lies in the rectangle.

t This phrase, due to Hardy, Proc. London Math. Soc. xxxiv. (1902), p.
16, suggests the analogy between an infinite integral and an infinite
series.



70 THE PROCESSES OF ANALYSIS [CHAP. IV

4"41. Infinite integrals of continuous functions. Conditions for con-
vergence.

A necessary and sufficient condition for the convergence of f x)dx is

J a

that, corresponding to any positive number e, a positive number X
should exist such that f x) dec \ < e whenever

The condition is obviously necessary; to prove that it is sufficient,
suppose

ra+n

it is satisfied; then, ii n X -a and n be a positive integer and Sn =
f(so),

. a

we have j Sn+p - Sn\ < €.

Hence, by § 2*22, Sn tends to a limit, S; and then, if > a + n,

S-i f x)da;\ \ S-\'' ''f(a;)dx + t f(x)da;\

J a ' -a \ J a+n I

<26;

and so lim f x) dx = S; so that the condition is sufficient.

4"42. Uniformity of convergence of an infinite integral.

The integral f x, a) dx is said to converge uniformly with regard to a

J a

in a given domain of values of a if, corresponding to an arbitrary
positive number e, there exists a number X independent of a such that

J /( > a)dx\ \ <€

for all values of a in the domain and all values of x X.

The reader will see without difficulty on comparing §§ 2'22 and 3'31
with § 4"41 that a necessary and sufficient condition that f x, a) dx
should

.' a

converge uniformly in a given domain is that, corresponding to any
positive number e, there exists a number X independent of a such that

I f(x, a)dx \ < e

\ J x' I

for all values of a in the domain whenever x" x X.

4'43. Tests for the convergence of an infinite integral.

There are conditions for the convergence of an infinite integral
analogous to those given in Chapter II for the convergence of an
infinite series.

The following tests are of special importance.






4-41-4*43] THE THEORY OF RIEMANN INTEGRATION 71



(I) Absolutely convergent integrals. It may be shewn that f x)dx

J a

certainly converges if \ f(a;) | dx does so; and the former integral
is then said to be absolutely convergent. The proof is similar to that
of § 2-32.

Example. The comparison test. If \ f x) g x) and / g x) dx converges,
then / f x) dx converges absolutely.

[Note. It was observed by Dirichlet* that it is not necessary for the
convergence of I f x)dx that f x)-a'0 as x- cc : the reader may see
this by considering the function

/( ) = ( n x n + l- n + l)-' ),

f(x) = n + iyin + l-x) x- n+l) + n+l)- n + l -(n + l)- x n + l),

where n takes all integral values.

For / f(x)dx increa.'ses with and / f x)dx=l n+l)-; whence it follows

without difficulty that / f x)dx converges. But when a- = n + l -i
(?n-l)-2, y'(.t>) =; and so f x) does not tend to zero.]

(II) The Maclaurin-Cauchyf test. Tf/(A-)>0 and/(x')-*0 steadily,

J "00 00

f x) dx and S /( ) converge or diverge together. 1 M = l

fm + l

For A -- /( 0 > fix) dx f m + 1 ),

J m n fn+] n+1

and SO 2 f m) l f x)dx' 2 /(m).

  m = l J 1 wi=2

The first inequality shews that, if the series converges, the
increasing sequence / f x)dx converges (§ 2-2) when - -oo through
integral values, and hence it follows

fx'

without difficulty that / f(.v)dx converges when .r'-*-x; also if the
integral diverges, so does the series.

The second shews that if the series diverges so does the integral, and
if the integral converges so does the series (§ 2'2).

(III) Bertrand'sX test. 1 f x) = 0 x ~' ), f x)dx converges when X <;
and \ if x) - x~' loga; " ), | f x) dx converges when X, < 0.

. a

These results are particular cases of the comparison test given in
(I).

* Dirichlet's example was/(.r) = sin .r'-; Journal fiir Math. xvii.
(1837), p. 60. t Maclaurin Flit.vions, i. pp. 289, 290) makes a verbal
statement practically equivalent to this result. Cauchy's result is
given in his Oeuvrcs (2), vii. p. 269. X Journal de Math. vii. (1842),
pp. 38, 39.



72 THE PROCESSES OF ANALYSIS [CHAP. IV

(IV) Chartiers test for integrals involving periodic functions. If /(
) - steadily as x and if < x) dx is bounded as x <x>,

1 ' a

then f x) (x) dx is convergent.

J a

For if the upper bound of I (.r) dx \ he A, we can choose X such that
f x)<e/2A

\ J a

when .X > A'; and then by the second mean vahie theorem, when .v" .v'
A', we have I /" " f x) 6 (x) dx =\ f x') f x) dx =f x') \ (f) x)dx-
cf) x) dx 2Af x') < f,

I \ / a;' • I J x' \ J a J a

which is the condition for convergence.

I ** Sill

Example I. I dt' converges.

J *'

Example 2. I a; - 1 sin x - ax) dx converges.

4-431. Tests for uniformity of convergence of an infinite integral f.

(I) De la Vallee Poussins test . The reader will easily see by using

("00

the reasoning of § 3-34 that f x, a) dx converges uniformly with
regard

to a in a domain of values of a if \ f x, a) | < fi x), where fM x) is
independent

fee r "

of a and /jl (x) dx converges. [For, choosing X so that fx(x)dx<e

rx" when x' x' X, we have f x, a)dx < e, and the choice of X is inde-

J x'

pendent of a.]

/•oo

Example. \ x' ~' e~''dx converges uniformly in any interval A, B) such
that

(II) The method of change of variable. This may be ilkist rated by an
example.

Consider / '- dx where a is real.

y " sin ax, /" " sin y,

We have / - 7- ' =, ~ 7~ •

] 3c' X J ax' y

   dy converges we can find Y such that / - - dy <e when y" y' Y.

y J y y






dx



< 6 whenever | a ' | F; if | a | S > 0, we therefore get

I /"*" sin ax, \ I dx \ < f

I y a;' •-' I



* Journal de Math, xviii. (1853), pp. 201-212. It is remarkable that
this test for conditionally convergent integrals should have been
given some years before formal definitions of absolutely convergent
integrals.

t The results of this section and of § 4-44 are due to de la Valine
Poussin, Ann. de la Soc. Scientifique de Bruxelles, xvi. (1892), pp.
150-180.

X This name is due to Osgood.



4-431, 4-44] THE THEORY OF RIEMANN INTEGRATION 73

when .'. " .// X= Y/8; and this choice of X is independent of a. So
the convergence is uniform when a S > and 'when a - 8 < 0.

Example. I j/ sm \& a:: )d > dx is uniformly convergent in any range
of real values of a. (de la Vallee Poussin.)

2- i sin zdz does not exceed a constant inde-

!

l endent uf a and .v since / z-i sin z dz converges.] J

(III) T/ie method of integration by parts.

If / / (x, a) dx <p x, a)+ X (•* > ) d-''

and if ( f.r, a)-*-0 uniformly as x -X3 and /;( (.r, a)o?jp converges
uniformly with regard

J <t

to a, then obviously / f x, a) dx converges uniformly with regard to
a.

(IV) The method of decomposition.

Example. |J c\ os.r sin a |J sin (a +l) . |J sin (o - l)a. .

loth of the latter integrals converge uniformly in any closed domain
of real values of a from which the points a= ± 1 are excluded.

4"44. Theorems concerning uniformly convergent infinite integrals. (I)
Let f x, a) dx converge uniformly luhen a lies in a domain S.

. a

'Then, if f x, a) is a continuous function of both variables ivJien x
a and a lies in S, f x, a)dx is a continuous function* of a.

J a

I r*

For, given e, we can find X independent of a, such that ' I f(x, a)dx
<e whenever X.

Also we can find 8 independent of x and a, such that \ f x,a)-f(x,a')\
< el X-a) whenever a - a.' < B.

That is to say, given e, we can find 8 independent of a, such that

f x,a.')dx-\ f x,a)dx \$ f x,a)-f x,a')]dx\

. a J a \ J a 1

+ I f x, a') dx +\ I fix, a) dx

\ Jx \ Jx

<3e,

whenever | a' - a | < S; and this is the condition for continuity.

* This result is due to Stokee. His statement is that the integral is
a continuous function of a if it does not ' converge infinitely
slowly.'



5



74



THE PROCESSES OF ANALYSIS



[chap. IV



(II) If f x, a) satisfies the same conditions as in (I), and if a.
lies in S when A <a<B, then



I \ f x, (x)dx\ doi= \ fix, a)da[dx.



For, by § 4-3,



Therefore



If

A [J a £ A



f x, a) dx r da= \ \ \ f(x, a) day dx. \ I f x, a) dx[ da- I \ i f x,
a) da)- dx

< f eda<e B-A),

J A



for all sufficiently large values of .

But, from §§ 2'1 and 4"41, this is the condition that



lim I \ i fix, a)da\ dx



should exist, and be equal to



f x, a) dx\ da. Corollary. The equation -r- \ rb .v, a)dx=l ~ dx is
true if the integral on the

da J a . J a va

right converges uniformly and the integrand is a continuous function
of both variables, when x' a and a lies in a domain >S', and if the
integral on the left is convergent.

Let A be a point of S, and let S=f x, a), so that, by § 4-13 example
3,

va

/ f x, a) da = (.r, a) - x, A).

Then / J / /(.r, a) day dx converges, that is / 4> x, a)-( ) x, A) dx
converges,

and therefore, since / cf) x, a)dx converges, so does i x, A) dx. J a
J a

I (f) x, a) dx \=j- / 0 (•*-', n) - (- j -'1 ) •



Then



da



d da



/ \ j f x,a)daydx\

= T I \ l f(x.a)dx\ da daj A [J a- ' ' J

= fjix,a)dx=fy dx, which is the required result; the change of the
order of the integrations has been justified above, and the
differentiation of / with regard to a is justified by § 4'44 (I) and §
4-13 example 3.



4*5, 4-51] THE THEORY OF RIEMANN INTEGRATION 75

- 4*5. Imjiroper integrals. Principal values.

If I /(x) - >cc as X - a + 0, then lim f(x) dx may exist, abd is

i- + O J a+\&

written simply I f x) dx; this limit is called an improper integral.

J a

If \ f x) I - 00 as a; - > c, where a< c <b, then

/•e-5 rb

lim I /( ) c?j: + lim I /(- O c?

S +O J a S' +oJ C+\&'

may exist; this is also written I f(x)dx, and is also called an
improper

J a

integral; it might however happen that neither of these limits exists
when 8, S' - > independently, but

lim ]/ f x)dx+i f(x)dxy exists; this is called 'Cauchy's principal
value of I f x)dx' and is written

J a

for brevity P I f(x) dx.

J a

Results similar to those of §§ 4-4-4-44 may be obtained for improper
integrals. But all that is required in practice is (i) the idea of
absolute convergence, (ii) the analogue of Bertrand's test for
convergence, (iii) the analogue of de la Vallee Poussin's test for
uniformity of convergence. The construction of these is left to the
reader, as is also the consideration of integrals in which the
integrand has an infinite limit at more than one point of the range of
integration*.



Examples. (1) / x - cos .v <ilr is an improper integral. J



- b



(2) r ./" (1 -.rf " dx is an improper integral if <X < 1, </i< 1. It
does not converge for negative values of X and /x.

dx is the principal value of an improper mtegral when

1 -A'

0<a<l. . 4-51. The inversion of the order of integration of a certain
repeated integral. ?r General conditions for the legitimacy of
inverting the order of integration when the

integrand is not continuous are difficult to obtain.

The following is a good example of the difficulties to be overcome in
inverting the order of integration in a repeated improper integral.

* For a detailed discussion of improper integrals, the reader is
referred either to Hobson's or to Pierpont's Functions of a Real
Variable. The connexion between infinite integrals and improper
integrals is exhibited by Bromwich, Infinite Series, § 164.



' Q THE PROCESSES OF ANALYSIS [CHAP. IV

Let f x,y) he a continuous function of both variables, and let 0<X 1,
0</x l, < v < 1; then



This integral, which was first employed by Dirichlet, is of
iraijortance in the theory of integral equations; the investigation
which we shall give is due to W. A. Hurwitz*.

Let x''~ i/' ~ (1 -x-yy~' f x,y) = (li x,y); and let M be the upper
bound of \ f x,y) |. Let S be any positive number less than .

Draw the triangle whose sides ave x = b, y = b, x+y = l-b\ at all
points on and inside this triangle ( x, y) is continuous, and hence,
by § 4-3 corollary,

Now r~''dx 11 "'' ct> X, y) dy =l'~' dx |P"' < (•, V) l + f]"' hdx+j'
'* Ldx,

where /i = / </> x, y) dy, L= (f> x, y) dy.

Jo J i-a;-6

But I /i I < r i/.r - y - 1 (1 - .r - y)" - 1 o y

since (l- -?/r-i<(l- '-S)''- .

Therefore, writing x = i\ -S)a'i, we havet

T"" /i dr I J/S',x- 1 p~ ./ - 1 ( 1 - . • - S)" - 1 c

  \& I Jo

 i/r -1 (1 - bt '- C x, - (1 -A-i)""' dx

The reader will prove similarly that Ly- 0 as 8- 0.

Hence I / o?a' i / 4) x,y)dy\= hm / t/ j / 4> x,y)dy\



= lim /



1-25 C fl-x-S

dx J: i ( ) x, y) dy



= lim

5H..0



fl-2S ( fi-y-S ]



* Annals of Mathematics, ix. (1908), p. 183.

t I a.-i ~ (l-.ri)''- dxi = £(A, ) exists if 0, j'>0 (§4-5example2).

t The repeated integral exists, and is, in fact, absolutely
convergent; for

/"i ri

writing 2/ = (1- a.-) s; and/ il/ - (1 - .r)' + ''-i cZx . I ' s' -1
(1 -s)''" £?s exists. And since the

• " fl-e - •' fl-ZS

integral exists, its value which is lim I may be written lim I

5, e O J S S O J S



4 6] THE THEORY OF RIEMANN INTEGRATION 7*7

by what has beeu already proved; but, by a precisely similar jjiece
of work, the last integral is

We have consequently proved the theorem in question.

Corollary. Writing = a + h-a) x, rj = h - b-a)y, we see that, if 4>
i\$j ) i con- tinuous,

//I f (|-; ~' ib-vf' iv-\$r'' a, V) dv

-J/V fli -af-Hb-rir-Hn- )"-' c >, rj)d Y

This is called Dirichlet's formula.

[Note. What are now called infinite and improper integrals Avere
defined by Cauchy, Leco?iS sur le calc. inf. 1823, though the idea of
infinite integrals seems to date from Maclaurin (1742). The test for
convergence was employed by Chartier (1853). Stokes (1847)
distinguished between 'essentially' (absolutely) and non-essentially
convergent integrals though he did not give a formal definition. Such
a definition was given by Dirichlet in 1854 and 1858 (see his
I'orlesiingen, 1904, p. 39). In the early part of the nineteenth
century improper integrals received more attention than infinite
integrals, probably because it was not fully realised that an infinite
integral is really the Iwiit of an integi'al.]

4"6. Complex inteyration* .

Integration with regard to a real variable x may be regarded as
integration along a particular path (namely part of the real axis) in
the Argand diagram. \ \ Qtf z), (= 7 -f iQ), be a function of a
complex variable z, which is continuous along a simple curve i in the
Argand diagram.

Let the equations of the curve be

x = x (t), tj = y it) (a t b).

Let X (a) + iy a) = Zq, x (b) + iy (b) = Z.

Then if-f x(t), y(t) have continuous difierential coefficients J Ave
define z f z)dz taken along the simple curve AB to mean



/



dx . dy Mt dt



 F + iQ)( + i' ]dt.



The 'length' of the curve AB will be defined as I \/ (--f) +( ) ' •

It obviously exists if -7-, -~ are continuous; we have thus reduced
the • dt dt

discussion of a complex integral to the discussion of four real
integrals, viz.



! \ A-' />!- />



dt



* A treatment of complex integration based on a different set of ideas
and not making so many assumptions concerning the curve AB will be
found in Watson's Complex Integration and Cauchy's Theorem.

f This assumption will be made throughout the subsequent work.

X Cp. § 4-13 example 4.



78 THE PROCESSES OF ANALYSIS [cHAP. IV

By § 4'13 example 4, this definition is consistent with the definition
of an integral when AB happens to be part of the real axis.

Examples, l f z) dz= - l " f z) dz, the paths of integration being the
same (but in opposite directions) in each integral.

/:- /.w:f-s-4f--(4'-4) *\

4*61. The fundamental theorem of complex integration.

From § 4"13, the reader will easily deduce the following theorem :

Let a sequence of points be taken on a simple curve z Z; and let the
first n of them, rearranged in order of magnitude of their parameters,
be called i<">, gC", . . . z <"' (iTo"*' = z +i'"' = Z); let their
parameters be j"*', J"', . . . <'", and let the sequence be such that,
given any number h, we can find N such that, when n > N, +i<"' - <"' <
B, for r = 0, 1, 2, ...,n; let,."" be any point whose parameter lies
between < < '', r+i*"'; then we can make

I (,+i'") - Zr fiC/- ) - I \ f z) dz

arbitrarily small by taking n sufficiently large.

4*62. An upper limit to the value of a complex integral.

Let M be the upper bound of the continuous function \ f z) .

Then jJVw,.j £:/(.)sj(| + 4y)],,

 Ml, where I is the ' length ' of the curve z yZ.

That is to say, I f z) dz cannot exceed Ml.

4'7. Integration of infinite series.

We shall now shew that if S z) = u z) + ii. z)- ... is a uniformly
con- vergent series of continuous functions of z, for values of z
contained within some region, then the series

I III z) dz + I lu z) dz + ..., J c J c

(where all the integrals are taken along some path C in the region) is
con- vergent, and has for sum I S (z) dz.

J c



4 "6 1-4 7] THE THEORY OF EIEMANN INTEGRATION 79

For, writing

>Sf Z) = Ml Z) + 2 ( ) + • • • + Ihi Z) + Rn Z),

we have

I S z) dz =1 u z)dz + ... -\ Un (z) dz + I R,, (z) dz.

J c J c J c J c

Now since the series is uniformly convergent, to every positive number
e there corresponds a number r independent of z, such that when n r we
have I Rn (z) I < f> for all values of z in the region considered.

Therefore if I be the length of the path of integration, we have (§
4'62)



f Rn z) J C



<el



dz

Therefore the modulus of the difference between / S z) dz and

J c n r

S I Um (z) dz can be made less than any positive number, by giving n
any

m = lJ c

sufficiently large value. This proves both that the series 2 Um z)dz
is

OT = 1 J c

convergent, and that its sum is / S(z)dz.

J c

Corollary. As in § 4-44 corollary, it may be shewn that*

0? ",, " d, .

if the series on the right converges uniformly and the series on the
left is convergent.

Example 1. Consider the series

" 'ix n n +1) 11x x -1 cos x" ?i 1+ 2 sin2 x ] 1 + n + 1)2 sin- .r '
in which x is real.

The Jith term is

2.rw cos x 2x (n + 1) cos x



1 + 71 sin"- X- 1 + (n + 1 )2 sin- x ' and the sum of n terms is
therefore

2x cos x' 2x(n + l) cos x

1+Sin2 2~ l + (7i+l)2sin2a;2*

Hence the series is absolutely convergent for all real values of x
except ± /(mrr) where ?>i = 1, 2, . . .; but

r> .,\ 2x(n+l)cosx -'"''-l+(K + l)2sin2, :2'

and if n be any integer, by taking x = n + l)~' this has the limit 2
as n <X) . The series is therefore non-uniformly convergent near x=0.

* - - ' means lira " where h- 0 along a definite simple curve; this
definition

is modified slightly in § 5-12 in the case when/(z) is an analytic
function.



80 THE PROCESSES OF ANALYSIS [CHAP. IV

Now tbe sum to infinity of the series is - - -, and so the integral
from to,r of •' 1 + sm-' x

the sum of the series is arc tan sin r . On the other hand, the sum of
the integrals from

to .V of the first n terms of the series is

arc tan sin .r - arc tan ( + !) sin x',

and as /i- X this tends to arc tan sin x-] - hir.

Therefore the integral of the sum of the series difiers from the sum
of the integi-als of the terms by tt.

Example 2. Discuss, in a similar manner, the series

== 2e .p l- (e-l) + e" + U'2

for real values of x.

Example 3. Discuss the series

7(1 + 11-2 + U3+...,

where

Ui = ze-'\ Un=nze-'"' - ( - 1) se-l"-!), for real values of z.

The sum of the first n terms is ?i2e~" ", so the sum to infinity is
for all real values of z. Since the terms Un are real and ultimately
all of the same sign, the convergence is absolute.



In the series



/ Uidz+ I ti.2dz+ I v.3dz + ...,



the sum of Ji. terms is -g (1 - e"" ), and this tends to the limit |
as n tends to infinity; this is not equal to the integral from to 2
of the sum of the series 2?<n-

The explanation of this discrejjancy is to be found in the
non-uniformity of the convergence near 2 = 0, for the remainder after
n terms in the series Ui + 112 + ...is - -aze~ "; and by taking z =
7i~' we can make this equal to e' '"-, which is not arbitrarily small;
the series is therefore non-uniformly convergent near z = 0.

Example 4. Compare the values of

/\ 2 tin[ dz and 2 / Undz, U=l J n=lj

where

2n z 2 n- l) z



"l+n'z )\ og n + l) l + n + -iyh'- log n + 2)'

(Trinity, 1903.)

REFERENCES.

G. F. B. RiEMANN, Ges. Math. Werke, pp. 239-241.

P. G. Lejeuxe-Dirichlet, Yorlesungen. (Brunswick, 1904.)

F. G. Meyer, Bestimmte Integrale. (Leipzig, 1871.)

E. GoDRSAT, Cours d Analyse (Paris, 1910, 1911), Chs. iv, xiv.

C. J. DE LA Vall e Poussix, Cours d' Analyse In fiyiite'stmale (Payis
and Louvaiu, 1914),

Ch. VI. E. W. HoBSON, Functions of a Real Variable (1907), Ch. v. T.
J. I'a. Bromwich, Theory of Infinite Series (1908), Appendix ill.



THE THEORY OF RIEMANN INTEGRATION 81

Miscellaneous Examples.

1. Shew that the integrals

I sin x )dx, I cos (.r-) dx, I x exp ( - x sin- x) dx Jo Jo Jo

converge. (Dirichlet and Du Bois Eeymoud.)

2. If a be real, the integral

f °° cos (ax),

Jo 1+ is a continuous function of a. (Stokes.)

3. Discuss the uniformity of the convergence of j x sin x - ax) dx.

Jo

3 /.rsin x -ax)dx= -f - -l-ir-j) cos (x- -ax)

/"/I a\,,,, 1 fsin(x -ax), ~\

- JU + : j " ' -" ) ' '+3°"j - x - ""-J

(de la Vallee Poussin.)

4. Shew that / ex ) [-e'< x -nx)]dx converges unifonuly in the range -
hr, hir) of values of a. (Stokes.)

r " x' dx

5. Discu.ss the convergence of I, - . when u, p, p are positive.

* Jo l+JT" |smjp|P - )/- r

(Hardy, Messenger, xxxi. (1902), p. 177.)

6. Examine the convergence of the integrals

Jo V 2 l-e'J .r ' jo X"

(Math. Trip. 1914.)

7. Shew that / - exists.

J " x' (sin x)

8. Shew that I .r -"e"" ' sin 2.rc/.r converges if a >0, a >0. (Math.
Trip. 1908.)

J a

9. If a series (7(2)= 2 (c - (? + i)sin (2i/+l) Tri:, (in which C(, =
0), converges uniformly

v=0

TT . . . . C

in an interval, shew that g z) -. is the derivative of the series/
(2)= 2 - sin 2vTrz.

sm irZ v=i V

(Lerch, Ann. de VEc. norm. sup. (3) xii. (1895), p. 351.)

10. Shew that r r... r |L i tff2- n d r r r dx,dx,...dx

converge when a>hi' and a~i + /3~i + ...+X~' < 1 respectively. (Math.
Trip. 1904.)

11. Iff x, ) be a continuous function of both.r andy in the ranges (a
x b), (a ?/ 6) except that it has ordinary discontinuities at points
on a finite number of curves, with continuously turning tangents, each
of which meets any line parallel to the coordinate axes

fb

only a finite number of times, then I f x, y) dx is a continuous
function of y.

/a, - Sj Ca.2-\&2 [b

+ 1 +...+ I fC 'j y + h)-f(x, y)]dx, where the numbers

Sj, §2 5 ••• fi, f2? ••• are so chosen as to exclude the
discontinuities ot f x, y + h) from the range of integration; Oj, 02,
... being the discontinuities off x, y).] (Bocher.)

W. M. A. 6

