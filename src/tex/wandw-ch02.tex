%
% 11
%
\chapter{The Theory of Convergence} 

\Section{2}{1}{The definition* of the limit of a sequence.}

Let Zi, Zo, z-i, ... be an unending sequence of numbers, real or
complex. Then, if a number I exists such that, corresponding to every
positive f number e, no matter how small, a number ??o can be found,
such that

\ Zn-l'.< e

for all values of n greater than o. the sequence (z,,) is said to tend
to the limit I as n tends to infinity.

Symbolic forms of the statement;]: ' the limit of the sequence Zn), as
n tends to infinity, is / ' are :

lim Zn=l, \ unzn=l, 2,1- * I " -> 

If the sequence be such that, given an arbitrary number N (no matter
how large), we can find 7?o such that |,i | > iV for all values of n
greater than /lo, we say that '\ Zn\ tends to infinity as n tends to
infinity,' and we write

kn i -> 

In the corresponding case when -x,i>N when n> n we say that i/;,j - *
- oc,

If a sequence of real numbers does not tend to a limit or to co or to
- x, the sequence is said to oscillate.

\Subsection{2}{1}{1}{Definition of the phrase ' of the order of}

If ( ) and zn) are two sequences such that a number n exists such
that i (Ku/ n) I < jK" whenever n > n, where K is independent of n,
we say that n is ' of the order of Zn, and we write§

thus 15!i±l = 0fl

1 + n" \ n-

If lim( / ) = 0, we write n = o Zn).

* A definition equivalent to this was first given by John Wallis in
1655. [Opera, i. (1695), p. 382.]

t The number zero is excluded from the class of positive numbers.

J The arrow notation is due to Leathern, Camb. Math. Tracts, No. 1.

§ This notation is due to Bachmann, Zahlenthcorie (1894), p. 401, and
Landau, Primzahlen, I. (1909), p. 61.

%
% 12
%

\Section{2}{2}{The limit of an increasing sequence.}

Let (a7) be a sequence of real numbers such that Xn+i' Xn for all
values of w; then the sequence tends to a limit or else tends to
infinity (and so it does not oscillate).

Let X be any rational-real number; then either :

(i) Xn X for all values of n greater than some number /?o depending on
the value of x.

Or (ii) Xn < X for every value of n.

If (ii) is not the case for any value of x (no matter how large), then

Xn- OO.

But if values of x exist for which (ii) holds, we can divide the
rational numbers into two classes, the Z-class consisting of those
rational numbers x for which (i) holds and the i -class of those
rational numbers x for which (ii) holds. This section defines a real
number a, rational or irrational.

And if € be an arbitrary positive number, a- e belongs to the i-class
which defines a, and so we can find n such that Xn>oi - \ e whenever n
> n; and a + e is a member of the i?-class and so Xn<a + e.
Therefore, whenever n > n,

TODO

Therefore Xn -> a.

Corollary. A decreasing sequence tends to a limit or to - oo . Example
1. If lim4:, =, lims,' = r, then lim(, + 3,') = + '. For, given
e, we can find n and ' such that

(i) when TODO, (ii) when TODO. Let TODO be the greater of n and n';
then, when TODO

and this is the condition that Urn z,n + Sm')='i + i'-

Example 2. Prove simikrly that lim(s,- 2,,/) =;-/", l[m z,z,') =
ll', and, if /' + 0,

lim zjz,') = l/l'.

Example 3. If < x < 1, x" 0. For if A- = (l+a)-i, a > and

TODO

by the binomial theorem for a positive integral index. And it is
obvious that, given a positive number f, we can choose no such that
TODO

\Subsection{2}{2}{1}{Limit-points and the Bolzano- Weierstrass* theorem.}

Let (xn) be a sequence of real numbers. If any number G exists such

* This theorem, frequently ascribed to Weierstrass, was proved by
Bolzano, Ahh. der k. bohmischen Ges. der Wiss. v. (1817). [Eeprinted
in Klassiker der E.mkten Wiss., No. 153.] It seems to have been known
to Cauchy.

1-

%
% 13
%

that, for every positive value of e, no matter how small, an unlimited
number of terms of the sequence can be found such that

G - 6 < Xn < G + e,

then G is called a limit-point, or cluster-point, of the sequence.

Bolzano's theorem is that, if X '<cp, ivhere X, p are independent of
$n$, then the sequence TODO Juts at least one limit-point.

To prove the theorem, choose a section in which (i) the i -class
consists of all the rational numbers which are such that, if A be any
one of them, there are only a limited number of terms Xn satisfying
Xn>A; and (ii) the Z-class is such that there are an unlimited number
of terms a-',i such that x a for all members a of the Z-class.

This section defines a real number G; and, if e be an arbitrary
positive number, G - e and G + e are members of the L and R classes
respectively, and so there are an unlimited number of terms of the
sequence satisfying

G- e< G - he Xn G + €<G- - e, and so G satisfies the condition that it
should be a limit-point.

\Subsubsection{2}{2}{1}{1}{Definition of 'the greatest of the limits.'}

The number G obtained in\hardsubsectionref{2}{2}{1} is called ' the greatest of the limits
of the sequence xn)' The sequence x ) cannot have a limit-point
greater than G] for if G' were such a limit-point, and e = i ((r' -
(r), G' - e is a member of the jR-class defining G, so that there are
only a limited number of terms of the sequence which satisfy Xn>G' -
€. This condition is incon- sistent with G' being a limit-point. We
write

G= \ imxn. The ' least of the limits,' L, of the sequence (written lim
x ) is defined to be

- lim (- Xn).

\Subsection{2}{2}{2}{Cauchy's* theorem on the necessary and sufficient
  condition for the existence of a limit.}

We shall now shew that the necessary and sufficient condition for the
existence of a limiting value of a sequence of numbers z, Zn, z-i,
... is that, corresponding to any given positive number e, hoiuever
small, it shall be possible to find a number n such that

for all positive integral values of j)- This result is one of the most
important and fundamental theorems of analysis. It is sometimes called
the Principle of Convergence.

* Analyse Algebrique (1821), p. 125.

%
% 14
%

First, we have to shew that this condition is necessary, i.e. that it
is satisfied whenever a limit exists. Suppose then that a limit I
exists; then \hardsectionref{2}{1}) corresponding to any positive number e, however
small, an integer n can be chosen such that

for all positive values of p; therefore

  i Zn+p -l\ + \ Zn-l\ <,

which shews the necessity of the condition

I ii+p ~ -2'rt I < > and thus establishes the first half of the
theorem.

Secondly, we have to prove* that this condition is suficient, i.e.
that if it is satisfied, then a limit exists.

(I) Suppose that the sequence of real numbers xn) satisfies Cauchy's
condition; that is to say that, corresponding to any positive number
e, an integer n can be chosen such that

for all positive integral values of p.

Let the value of n, corresponding to the value 1 of e, be 7n.

Let Xj, pi be the least and greatest of a;i, a-g, ... av; then

Xi-1 <Xn< p, + l,

for all values of n; write Xj - I = X, pi + 1 = p.

Then, for all values of n, X < Xn < p- IVierefore by the theorem of
§2'21, the sequence (xn) has at least one liniit-point G.

Further, there cannot be more than one limit-point; for if there were
two, G and H (H < G), take e < I G - H). Then, by hypothesis, a number
n exists such that j Xn+p - Xn | < e for every positive value of p.
But since G and H are limit-points, positive numbers q and r exist
such that

I G - X, +q \ < €, \ H - Xn+r \ < l-nen | Cr X- q | -j- | Xn q X | -|-
| X n+r I ~r I Xji f. Ji ] < '±6.

But, by\hardsectionref{1}{4}, the sum on the left is gi-eater than or equal to j G -
H .

Therefore G - H < 4e, which is contrary to hypothesis; so there is
only one limit- point. Hence there are only a finite number of terms
of the sequence outside the interval G -, (j + S), where 8 is an
arbitrary positive number;

* This proof is given by Stolz and Gmeiner, Theoretische Arithmetik,
ii. (191)2), p. 144.

%
% 15
%

for, if there were an unlimited number of such terms, these would have
a limit-point which would be a limit-point of the given sequence and
which would not coincide with G; and therefore G is the limit of x ).

(II) Now let the sequence Zn) of real or complex numbers satisfy
Cauchy's condition; and let Zn = Xn -f- iyn, where Xn and yn are real
; then for all values of n and p

I n-irp - i I n+p 2'n i, | yn+p ~ yn \ | 2'n+p n\-

Therefore the sequences of real numbers x ) and (yn) satisfy Cauchy's
condition; and so, by (I), the limits of (.r) and y ) exist.
Therefore, by\hardsectionref{2}{2} example 1, the limit of (2) exists. The result is
therefore established.

\Section{2}{3}{Convergence of an infinite series.}

Let Wj, ?/2, u.i, ... Kit, ... be a sequence of numbers, real or
complex. Let the sum

Ml 4- Uo + . . . -I- tin

be denoted by Sn-

Then, if *S,i tends to a limit S as /; tends to infinity, the
infinite series

"i + Hi + 3 -f- <4 + . . . is said to he convergent, or to converge to
the sum S. In other cases, the infinite series is said to be
divergent. When the series converges, the expression S-Sn, which is
the sum of the series

" +l+ " +2+ Un+-,+ ...,

is called the remainder after n terms, and is frequently denoted by
the symbol R .

The sum Un+ + 11 +.. + ...-\- Un+p

will be denoted by Sn,p.

It follows at once, by combining the above definition with the results
of the last paragraph, that the necessary and sufficient condition for
the convergence of an infinite series is that, given an arbitrary
positive number e, we can find n such that I /S'\ | < e for every
positive value of p.

Since Un+i = n,l, it follows as a particular case that lim Un+i = - in
other words, the 7?th term of a convergent series must tend to zero as
// tends to infinity. But this last condition, though necessary, is
not sufficient in itself to ensure the convergence of the series, as
appears from a study of the series

In this series, Sn,n =, +, + - +  + . 

' n+ i n + 2 n + 6 2n

The expression on the right is diminished by writing (2?i)~ in place
of each term, and so Sn,,1 > 

%
% 16
%

Therefore S n+i = 1 + /Si, i + 2, 2 + 'S14, 4 + >S'8\ g + >Si6, le + 
  + S.n 

> ( + 3) -> X;

so the series is divergent; this result was noticed by Leibniz in
1673.

There are two general classes of problems which we are called upon to
investigate in connexion with the convergence of series :

(i) We may arrive at a series by some formal process, e.g. that of
solving a linear differential equation by a series, and then to
justify the process it will usually have to be proved that the series
thus formally ob-' tained is convergent. Simple conditions for
establishing convergence in such circumstances are obtained in
§§2'31-2"61.

(ii) Given an expression S, it may be possible to obtain a development

S= X i(,n + Rn, valid for all values of n; and, from the definition
of a limit,

00

it follows that, if we can prove that Rn - 0, then the series 2 u,,
converges

m = l

and its sum is S. An example of this problem occurs in § 54.

Infinite series were used* by Lord Brouncker in Phil. Trans. 11.
(1668), pp. 645-649, and the expressions convergent and disergent were
introduced by James Gregory, Professor of Mathematics at Edinburgh, in
the same year. Infinite series were used systematically by Newton in
1669, De anali/si per aequat. num. term, inf., and he investigated the
con- vergence of hypergeometric series (§ 14- 1) in 1704. But the
great mathematicians of the eighteenth century used infinite series
freely without, for the most part, considering the question of their
convergence. Thus Euler gave the sum of the series

1 1 1,, .. ... + -,+ - + -+l+2 + 2- + r'+ a)

as zero, on the ground that

2 +,2 +,3 + ... (6)

\-z

1, 1 1 3

and 1+- -t- -, + ... = - - (c).

z z~ z- V

The eiTor of course arises from the fact that the series h) converges
only when | 2 j < 1, and the series (c) converges only when | s j > 1,
so the series (a) never converges.

For the history of researches on convergence, see Pringsheim and Molk,
Encyclope'die des Sci. Math., i. (1) and Keifi", Geschichte der
unendlichen Reihen (Tiibingen, 1889).

\Subsubsection{2}{3}{0}{1}{AheVs inequality.}

I "* Let fn fn+ for all integer values of n. Then 2 /

A is the greatest of the sums

1 ! i, i ! + 2 1 > I o-i + 0.2 + a, !, . . ., i ! + rto +   . + am

  Af, ivhere

* See also the note to\hardsectionref{2}{7}.

t Journal fiir Math. i. (1826), pp. 311-339. A particular case of the
theorem of\hardsubsectionref{2}{3}{1}, Corollary (i), also appears iu that memoir.

%
% 17
%

For, writing Ui + ao + . . . + a = Sn, we have

rii

t ilnfn = 5i/i + S.2 - Si)fo + (Ss - S jfs + . + (s - Sm-i)fm 11 = 1

- Si (/i -J 2) + S-2. (/2 ~y 3) + . . . + S i\ i \ Jm-i ~Jm) + Smfm-

Since /i - /o,/2 -fs,  are not negative, we have, when n = 2, 3,
... m,

I Sn-i \ (fn-i -fn) (fn-i - fn) ', alsO j S,n \ fm < fm,

and so, summing and using\hardsectionref{1}{4}, we get

n=l I

Corollari). If i, a ... ?Pi, Wj, ... are any numbers, real or complex,

2 a w \ <,A\ 2 i M' + 1 - w, i + 1 u-

'/'

where J is the greatest of the sums

2 a

n=l

, p = \, 2, ... m).

\addexamplecitation{Hardy.}

\Subsection{2}{3}{1}{Dirichlet's* test for convergence.}

Let

Z < K, luhere K is independent of p. Then, if fn >fn\-\ >

and lim/ = O-f*, ((nfn converges.

M = l

For, since lim i = 0, given an arbitrary positive number e, we can
find m such that +! < e/2/i .

Then

m + q

m + q

2 an\ i X a j + ! X cirt t < 2/1', for all positive values of; so

that, by Abel's inequality, we have, for all positive values of j;;,

where A <2K.

I m+p

Therefore 2 cinfn

I n = m+l

m+p Z CLnJn - Jin+i >

n=m+l

< 2Kfm+i < e; and so, by\hardsectionref{2}{3}, S cinf 71 converges.

Corollary (i). Jfte 's test for convergence. If 2 converges and the
sequence (m ) is

n = l

nionotouic (i.e.,i ?t + i always or else M Wrt + i always) and j?(
|<k, where k is independent of /(, then 2 i/ converges.

For, by\hardsectionref{2}{2}, tends to a limit u; let |w- |=/ . Then i- 0 steadily;
and

therefore 2 ' converges. But, if (m ) is an increasing sequence, /
=w-m, and so

rt=i

2 li-n Un converges; therefore since 2 ua converges, 2 converges. If
(?<J is

 =1 ii=\ n=\

a decreasing sequence / = e<,i -, and a similar proof holds.

* Journal de Math. (2), vii. (1862), pp. 253-255. Before the
publication of the 2nd edition of Jordan's Cours d' Analyse (1893),
Dirichlet's test and Abel's test were frequently jointly described as
the Dirichlet-Abel test, see e.g. Pringsheim, Math. Ann. xxv. (1885),
p. 423.

t In these circumstances, we say j -0 steadily.

W. M. A. 2

%
% 18
%

Corollary (ii). Taking a = (-) -i in Dirichlet's test, it follows
that, if / /; i and lim / = 0, /i -ft +/3 -/i + . .  converges.

p

Example 1. Shew that if 0< <27r, I 2 sin (9 <coseci(9; and deduce
that, if

f - Q steadily, 2 / sin nO converges for all real values of 6, and
that 2 / cos nd converges

n=l "=i

if 6 is not an even multiple of tt.

Example 2. Shew that, if fn- 0 steadily, 2 (-)"/' cos (9 converges if
6 is real and

n = \

not an odd multiple of tt and 2 -)"-fnS\ nne converges for all real
values of 6. [Write 7r + for in example 1.]

\Subsection{2}{3}{2}{Absolute and conditional convergence.}

QO

In order that a series X Un of real or complex terms may converge, it
is

n=l

sufficient (but not necessary) that the series of moduli S Un \ should

n = \

00

converge. For, if <Tn,p = Un+i, + | Un+2 i +    + | Un+p \ and if
2 | w | converges,

71 = 1

\ ve can find n, corresponding to a given number e, such that cTn p <
e for all

values of jj. But ] Sn,p cr \ < e, and so S converges.

  = i

we see that t - 9 + o~4+--- converges, though \hardsectionref{2}{3}) the series of
moduli

The condition is not necessary; for, writing i = 1/n in\hardsubsectionref{2}{3}{1},
corollary (ii), see that

i\ -4- + +...is known to diverge.

1 2 O -i

In this case, therefore, the divergence of the series of moduli does
not entail the divergence of the series itself.

Series, which are such that the series formed by the moduli of their
terms are convergent, possess special properties of great importance,
and are called absolutely convergent series. Series which though
convergent are not abso- lutely convergent (i.e. the series themselves
converge, but the series of moduli diverge) are said to be
conditionally convergent.

" 1
\Subsection{2}{3}{3}{The geometric series, and the series TODO.}

w = l

The convergence of a particular series is in most cases investigated,
not by the direct consideration of the sum Sn p, but (as will appear
from the following articles) by a. comparison of the given series with
some other series which is known to be convergent or divergent. We
shall now investigate the convergence of two of the series which are
most frequently used as standards for comparison.

%
% 19
%

(I) The geometric series. The geometric series is defined to be the
series z + z-- z + z'+ .... Consider the series of moduli

l+\ z\ + \ z ' + \ z' + ...\ for this series Sn,p =\ z\'' ' + \ z, "+-
+ ...->r z v

l-\ z\ P

= izr+

l-\ z\ '

Hence, if .g- < 1, then S,i,p< : -; for all values of j), and, by §
22,

example 3, given any positive number e, we can find n such that

[ |n+i l\ |2i|-i<e.

Thus, given e, we can find n such that, for all values of p, Sa,p<€.
Hence, by\hardsubsectionref{2}{2}{2}, the series

is convergent so long as 2 | < 1, and therefore the geometric series
is absolutely convergent if\ z\ < .

When z ' 1, the terms of the geometric series do not tend to zero as n
tends to infinity, and the series is therefore divergent.

TT VV .11111

(II) Ihe series + - + - + - + - + ....

" 1 Consider now the series,i = 2 -, where s is greater than 1.

7n = l i*

112 1 'e have 2" + 3 - < 2*- = 2 i '

11114 1

- I 1 1 - < - =

4 5*' 6' 7* 4 4*-i ' and so on. Thus the sum of 2 -1 terms of the
series is less than 1 J\ J\ J\ 1 1

] s-i 2*~i 4*~i 8 ~  I 2(2>-i) (s-1) 1 - 2 ~* ' and so the sum of
ani/ number of terms is less than (1 - 2 ~*)~ Therefore

n

the increasing sequence S m~ cannot tend to infinity; therefore, by §
2'2,

w = l

=0 1 .

the series S - is convergent if s>\ \ and since its terms are all real
and

M = 1 'i

positive, they are equal to their own moduli, and so the series of
moduli of the terms is convergent; that is, the convergence is
absolute.

2-2

%
% 20
%

If s = 1, the series becomes

1 + 1 + 1 + 1 + ...,

which we have already shewn to be divergent; and when 5 < 1, it is a
fortiori divergent, since the effect of diminishing s is to increase
the terms of the

< 1 .

series. The series S - is therefore divergent if s 1. n = l ''

\Subsection{2}{3}{4}{The Comparison Theorem.}

We shall now shew that a series; i + i, + 2/3+ ... is absolutely
con- vergent, provided that \ u \ is always less than G \ vn\, ivhere
C is some number independent of n, and v,i is the nth term of another
series which is known to be absolutely convergent.

For, under these conditions, we have

I ' ?i+i I + I 'i' i+2 1 +  .  + I Un p I < C I j Vn+i I + I f,iJ-2
: + ... + 1 V +p \ \,

where n and p are any integers. But since the series Si',i is
absolutely convergent, the series S | Vn \ is convergent, and so,
given e, we can find n such that

I " n+i I + i /i+2 I + .-..+ I Vn- p I < e/ C,

for all values oi p. It follows therefore that we can find /; such
that

1 Un+i I + I Wn+2 i + . . + 1 Un+p \ < e,

for all values of p, i.e. the series S | Un \ is convergent. The
series %Un is therefore absolutely convergent.

Corollary. A series is absolutely convergent if the ratio of its th
term to the nih. term of a series which is known to be absolutely
convergent is less than some number indej)endent of n.

Example 1. Shew that the series

COS,Z +- 2COS22 + .-J5COS 32 + -T, COS42+...

li" o" 4"

is absolutely convergent for all real values of z.

iCOS TliZ 1

- 5- -, . The moduli of n I ji-

the terms of the given series are therefore less than, or at most
equal to, the corresponding

terms of the series

n 1 1 1

1 + 2-2 + .3- + 4 2+-' .

which by\hardsubsectionref{2}{3}{3} is absolutely convergent. The given series is
therefore absolutely convergeut.

Example 2. Shew that the series
$$
TODO
$$
where 2 = e'", (?i=l, 2, 3,

is convergent for all values of 2, which are not on the circle [ ] =
1.

%
% 21
%

The geometric representation of complex numbers is helpful in
discussing a question of this kind. Let values of the complex number z
be represented on a plane; then the numbers Z\ t z, Zz, ... will
give a sequence of points which lie on the circumference of the circle
whose centre is the origin and whose radius is unity; and it can be
shewn that every point on the circle is a limit-point \hardsubsectionref{2}{2}{1}) of the
points z . --

For these special values z-, of, the given series does not exist,
since the denomi- nator of the nth term vanishes when 2 =,,. For
simplicity we do not discuss the series for any point z situated on
the circumference of the circle of radius unity.

Suppose now that [zj + l. Then for all values of 7i, | z - 2 (1 - [2 '
|>c~i, for

some value of c; so the moduli of the terms of the given series are
less than the corre- sponding terms of the series

c c c c

which is known to be ab.solutely convergent. The given series is
therefore absolutely convergent for all values of z, except those
which are on the circle | 2 | = 1.

It is interesting to notice that the area in the i-plane over which
the series converges is divided into two parts, between which there is
no intercommunication, by the circle

1 1 = 1.

Example 3. Shew that the series

SsinJ-f 4sin |-f8 sin -f- ... -l-2"sin .f- + ...

 J .J it i o

converges absolutely for all values of 2.

Since* lim 3" sin (2/8") = 2, we can find a number /, independent of
n (but depending on 2), such that | 3" sin (2/3") \ < i:; and
therefore

2 sin < U) .

3" I V3.

  /2\" Since 2 i' i ) converges, the given series converges
absolutely.

\Subsection{2}{3}{5}{Gauchy's test for absolute convergence']'.}

i/' lim M ]"" < 1, S ? i converges absolutely. >i -*. * jj = 1

For Ave can find m such that, when n ni, j m | '" p < 1, where p is
independent of u. Then, when n > ni, '. Un \ < p' ] and since 2 p"
converges,

n = >n+l

it follows from\hardsubsectionref{2}{3}{4}! that 2 Un (and therefore 2 m,J converges ab-
solute I '.

[Note. If lim \ u, \ \ ' >\, u does not tend to zero, and, by\hardsectionref{2}{3}, 2
m does not converge.]

* This is evident from results proved iu the Appendix. t Analyse
Algehrique, pp. 132-135.

%
% 22
%

\Subsection{2}{3}{6}{D'Alembert's* ratio test for absolute convergence.}

We shall now shew that a series

?/i+ U2+ u-i + 1/4+ ... is absolutely convergent, provided that for
all values of n greater than some fixed value r, the ratio ' is less
than p, tvhere p is a positive number independent of n and less than
unity.

For the terms of the series

j Ur+i i + i Ur+-2 I + i Ur+i \ + ...

are respectively less than the corresponding terms of the series

which is absolutely convergent when p < 1; therefore S Un (and hence

  = / + !

the given series) is absolutely convergent.

A particular case of this theorem is that if lim (Un+Jun) \ = l <1,
the

series is absolutely convergent.

For, by the definition of a limit, we can find r such that

I ! \ 4 i < i (1 - 0, when n > r,

and then P±-' <l(l+l)<l,

when 11 > r.

[Note. If lim \ u, + l i<,, >1, % does not tend to zero, and, by §
2-3, 2 ?<,, does not

H = l

converge.]

Example 1. If 1 c |<1, shew that the series

/t=i converges absolutely for all values of z.

[For w + i/?i,i = c( + i)--''-e c-" + ie - 0, as ??- x, if;Cj<l.]

Example 2. Shew that the senes

a-6, (a- 6) (a- 26) a-h) a-2h)ia- h) . Z+-2J-.-+ 3j + j z +...

converges absolutely if \ z\ < \ b~' .

[For ±1 = - '1-z- -hz, as k- oo; so the condition for absolute
convergence is Un n + \ ' '

\ hz\ < \,. Q.\ z\ < h- .'\

* Opuscules, t. V. (1768), pp. 171-182.

%
% 23
%

Example 3. Shew that the series 2 - - \, converges absolutely if
i2|<l.

[For, when;2|<1, | 2 -(l 4-n-i)" I (1 + -!)"- I 2" i 1 + 1 + + ... -
1>1, so the moduli of the terms of the series are less than the
corresponding terms of the series 2 n Is""! I; but this latter series
is absolutely convergent, and so the given series con- verges
absolutely.]

\Subsection{2}{3}{7}{A general theorem on series for luhicli lim |- =1.}

It is obvious that if, for all values of n greater than some fixed
value r, I i/-,i+i I is greater than | Un\, then the terms of the
series do not tend to zero as

?i - > X, and the series is therefore divergent. On the other hand,
if i j

is less than some number which is itself less than unity and
independent of n (when n > r), we have shewn in\hardsubsectionref{2}{3}{6} that the series
is absolutely con- vergent. The critical case is that in which, as n
increases, - ' tends to the value unity. In this case a further
investigation is necessary.

We shall now shew that* a series u + u.. + u.i+ .. .,dn which lim -" =
1 will be absolutely convergent if a positive number c exists such
that

For, compare the series S ] Un j with the convergent series Ivn, where
and y1 is a constant; we have

Vn \ n + 1/ V nj n \ n

. \'Vn+i T I 1

and hence we can find m such that, when n > m,

u,

By a suitable choice of the constant A, we can therefore secure that
for all values of n we shall have

i Un I < V,v

As Si'n is convergent, 2 | Un j is also convergent, and so Sm,i is
absolutely convergent.

* This is the second (D'Alembert's theorem given in\hardsubsectionref{2}{3}{6} being the
first) of a hierarchy of theorems due to De Morgan. See Chrystal,
Algebra, Ch. xxvi. for an historical account of these theorems.

%
% 24
%


Corollary. If then the series is absolutely convergent if J.i < - 1

= 1 H - + - ], where i is independent of n,

n \ n-J

00 / " 1 \

Example. Investigate the convergence of 2 '' exp ( - -2 - ), when r>k
and when n=l \ 1 "v

r<k.

\Subsection{2}{3}{8}{Convergence of the hijpergeometric series.}

The theorems which have been given may be illustrated by a discussion
of the convergence of the hypergeometric series

  a.h a(a+l)6(6+l) . a(a+ l)(a + 2) 6(6 + 1)(6 + 2)

  + 17 + 1.2.c(c + l) " 1.2.3.c(c + l)(c + 2) " '

which is generally denoted (see Chapter XIV) by F a, b; c; z).

If c is a negative integer, all the terms after the (1 - c)th have
zero

denominators; and if either a or 6 is a negative integer the series
will

terminate at the (1 - a)th or (1 - 6)th term as the case may be. We
shall

suppose these cases set aside, so that a, h, and c are assumed not to
be

negative integers.

  In this series

i Un+i i ( "+ n - l)(b + n - 1)1 II

, - 7 -, V <s I * >

\ Ufi \ I n c + n - l) I

as ?? - > 00 .

We see therefore, by\hardsubsectionref{2}{3}{6}, that the series is absolutely convergent
ivhen \ z\ < l, and divergent ivhen | | > 1.

When U I = 1, we have *

1 +

0-1]

ii+'

n

a + b-

-c-1

6-1

n \ n /

Let a, b, c be complex numbers, and let them be given in terms of
their real and imaginary parts by the equations

a = a + ia", 6 = 6' + ib", c = c + ic". Then we have

 ' + 6' - c' - 1 + z (a" + 6" - c")

Un

1 +

= 1 +

a +b' - c

ly fa" + b"-c'

+ 0(1

\ n'

+

  

By\hardsubsectionref{2}{3}{7}, Corollary, a condition for absolute convergence is

a' + b' -c' < 0. * The symbol (l/>i-) does not denote the same
function of n throughout. See\hardsubsectionref{2}{1}{1}.

%
% 25
%

Hence ivhen \ z\ = \, a sufficient condition* for the absolute
convergence of the hyper geometric series is that the real part of a +
b - c shall be negative.

\Section{2}{4}{Effect of changing the order of the terms in a series.}

In an ordinary sum the order of the terms is of no importance, for it
can be varied without affecting the result of the addition. In an
infinite series, however, this is no longer the casef, as will appear
from the following example.

T.,11111111

Let 2 = H-3-2+5 + 7-| + 9 + n-o+---

1 cr -. 1 1 1 1 1 .

and,S'=l-2 + 3-4 + 5-6 + --->

and let 2,i and Sn denote the sums of their first n terms. These
infinite series are formed of the same terms, but the order of the
terms is different, and so - and Sn are quite distinct functions of n.

Let

11 1 I . a 0-w = j; + 2 +   . + > SO that bn = (T-.n - a

Then

 11 111 1

-3n-i +3 +  + 4,i\ i 2~4  2),

1 1 n 2 -'* 2 """

= (0"4/i - CT-jn) + 2 iti ~ n)

= n "r 9 211 

Making n->-y:, we see that

S = S + lS;

and so the derangement of the terms of /i' has altered its sum.

Example. If in the series

-. 1 1 1

1-2+3-4+---

the order of the terms be altered, so that the ratio of the number of
positive terms to the number of negative terms in the first n terms is
ultimately a-, shew that the sum of the series will become log (2a).
\addexamplecitation{Manning.}

\Subsection{2}{4}{1}{The fundamental property of absolutely convergent series.}
We
shall shew that the sum of an absolutely convergent series is not
affected by changing the order in which the terms occur.

Let 8 = uy Un + 3 + ...

* The coudition is also necessary. See Bromwicb, Infinite Stnies, pp.
202-204.

t We say that the series S !' consists of the terms of S m,j in a
different order if a law

)! = 1 )t = l

is given by which corresponding to each positive integer x) we can
find one (and only one) integer q and vice versa, and Vq is taken
equal to Up. The result of this section was noticed by Dirichlet,
Berliyier Abh. (1837), p. 48, Journal de Math. iv. (1839), p. 397. See
also Cauchy, Resumes analytiques (Turin, 1833), p. 57.'

%
% 26
%

be an absolutely convergent series, and let S' be a series formed by
the same terms in a different order.

Let e be an arbitrary positive number, and let n be chosen so that

I I I I I I '

for all values of p.

Suppose that in order to obtain the first n terms of S we have to take
m terms of S'; then if k > m,

 k = n + terms of S with suffices greater than n, so that

>SV - S = Sn - S + terms of S with suffices greater than n.

Now the modulus of the sum of any number of terms of S with suffices
greater than n does not exceed the sum of their moduli, and therefore
is less

than 2 e-

Therefore | >S a;' - 'S' j < Sn- S\ + €.

But j - >Sf I < lim 11 Un+i I + I Un+-2 I +    + i Uri+p |

p-*-x

1 Therefore given e we can find m such that

\ Si:-s\ < €

when k > m; therefore /S,,/- > S, which is the required result.

If a series of real terms converges, but not absolutely, and if >S be
the sum of the first p positive terms, and if an be the sum of the
first n negative terms, then 8p->cc, cr - >- oo; and lim (>Sp + cr,j)
does not exist unless we are given some relation between p and n. It
has, in fact, been shewn by Riemann that it is possible, by choosing a
suitable relation, to make lim(>Sf 4- (7n) equal to cnii/ given real
number*.

\Section{2}{5}{Double series.}

Let ii i,n be a number determinate for all positive integral values of
m and ??; consider the array

 ],l) Wj O) '**I,3j  ' ' 2,1 J '2,2) *'2,3)   ';i, 1 ) h, 2 ) 3,
3 )   

* Ges. Werke, p. 221.

t A complete theory of double series, on which this account is based,
is given by Pringsh\&iui, Miinchcner Sitzunysberichte, xxvii. (1807),
pp. 101-152. See further memoirs by that writer, Math. Ann. liii.
(1900), pp. 289-321 and by London, ibid. pp. 322-370, and also
Bromwich, Infinite Series, which, m addition to an account of
Pringsheim's theory, contains many develop- ments of the subject.
Other important theorems are given by Bromwich, Proc. London Math.
Sac. (2), I. (1904), pp. 176-201.

%
% 27
%

Let the sum of the terms inside the rectangle, formed by the first m
rows of the first n columns of this array of terms, be denoted by
S,n,n-

If a number *S' exists such that, given any arbitrary positive number
e, it is possible to find integers m and n such that

whenever both /a > ru and v > n, we say* that the double series of
luliich the general element is u,,. converges to the sum S, and we
write

lira *S' = S.

 fi,V

If the double series, of which the general element is ] w, |, is
convergent, we say that the given double series is absolutely
convergent.

Since w = (/SV, - >SV, -i) - ('S' -], -'Sm-i, -iX it is easily seen
that, if the double series is convergent, then

lim Uf, = 0.

Stolz necessary and suffi cientf condition for convergence. A
condition for convergence which is obviously necessary (see\hardsubsectionref{2}{2}{2}) is
that, given e, we can find m and n such that | S +p, y - *%, / < e
whenever fi > m and v > n and p, a may take any of the values 0, 1, 2,
.... The condition is also sufficient; for, suppose it satisfied;
then, when fM> m + n, >S' +p, +p - S; < e.

Therefore, by\hardsubsectionref{2}{2}{2}, S has a limit *S'; and then making p and a tend
to infinity in such a way that p, + p = v + a, v,'e see that \ S - 8,,
, e when- ever p > m and v> n; that is to say, the double series
converges.

Corollary, An absolutely convergent double series is convergent. For
if the double series converges absolutely and if ty,t n he the sum of
m rows of n columns of the series of moduli, then, given f, we can
find fx such that, when p>ra>fi and q>/i>fi, i,j,q - t,n,n< - But \ Sp
q-S, n\ \ ip,q-tm,H and so \ Sp g-S, n\ < e when jc ??i>/x, q>n>fi;
and this is the condition that the double series should converge.

\Subsection{2}{5}{1}{Methods of summing double series.}
TODO

Let us suppose that S u, converges to the sum S . Then S >S' is
called the sum by rows of the double series; that is to say, the sum
by rows

OC/X\ 30/00\

is 5 ( S li, ). Similarly, the sum by columns is defined as 2 ( 2
V,*')- That these two sums are not necessarily the same is shewn b the
example

Su V =, in which the sum by rows is - 1, the sum by columns is + 1;

' ' p + v - -

and S does not exist.

* This definition is practically due to CsiU.chy, Analyse Algehrique,
p. 540. t This condition, stated by Stolz, Math. Ann. xsiv. (1884),
pp. 157-171, appears to have been first proved by Pringsheim.

J These methods are due to Cauchy.

%
% 28
%

Pringsheim's theorem* : If S exists and the sums by rows and columns
exist, then each of these suyns is equal to S.

For since S exists, then we can find m such that

I *S /i, p - S < e, if yu. > 7u, V > m.

And therefore, since lim >Sf exists, Mvn 8, ) - S %e; that is to say,

  Sp - S \ i e when fx > m, and so (§ 222) the sum by rows converges
to S. In like manner the sum by cohimns converges to S.

\Subsection{2}{5}{2}{Absolutely convergent double series.}

We can prove the analogue of \hardsubsectionref{2}{4}{1} for double series, namely that if
the terms of an absolutely convergent double series are taken in any
order as a simple series, their sum tends to the same limit, provided
that every term occurs in the summation.

Let cr be the sum of the rectangle of fx rows and v columns of the
double series whose general element is | m, |; and let the sum of
this double series be cr. Then given e we can find m and n such that
o- - cr < e whenever both fi > m and v> n.

Now suppose that it is necessary to take iV terms of the deranged
series (in the order in which the terms are taken) in order to include
all the terms of >S j/-fi,3/+i, and let the sum of these terms be ty

Then a' - 'Sj/-t-i, j/+i consists of a sum of terms of the type Up,q
in which p > m, q >n whenever M > m and M > n; and therefore

I h' - Sm+1,3T+1 O" - 0-j/+i\ 3/+1 < 2 f-

Also, S- Sji+i ijj i consists of terms iip q in which j) > m, q> n;
therefore I S - Sjfi+ijf+i I <r - o-jf+ M+i < 2 ' therefore | S-ty j <
e; and, corresponding to any given number e, we can find X; and
therefore ty- S.

Example 1. Prove that in an absolutely convergent double series, 2 !<
(, exists, and

H = l

thence that the sums by rows and columns respectively converge to S.

[Let the sum of fi rows of v columns of the series of moduli be t,
and let t be the sum of the series of moduli.

Then 2 \ ?i, \ < t, and so 2 m,;, converges; let its sum be 6;
then

\ bi\ + \ b.i\ + ... + \ b \ lim t t,

and so 2 b converges absolutely. Therefore the sum by rows of the
double series

exists, and similarly the sum by columns exists; and the required
result then follows from Pringsheim's theorem.]

* Loc. cit. p. 117.

%
% 29
%

Example 2. Shew from first principles that if the terms of au
absokitely convergent double series he arranged in the order

 M + ( 2,l + l,2) + ( :i,l + 2,2 + "l,3') + (*/4,l + --- + '<l,4) +
---5

tliis series converges to S*.

\Subsection{2}{5}{3}{Cauchy's theorem* on the multiplication of absolutely convergent series.}

We shall now shew that if two series

S = u + V. + (/3 + . . . and T =Vi + Vo + V3+ ...

are absolutely convergent, then the series

P = U Vi + 1 2 1 + "i V-i + . . .,

formed by the products of their terms, written in any order, is
absolutely con- vergent, and has for sum ST.

Let Sn = Ui + U., + . . . + Un,

Tn=V, + V.,+ ...+ Vn.

Then ST = lim S,, lini T = lim (SnTn)

by example 2 of\hardsectionref{2}{2}. Now

SnTn = Ui I'l + U>Vi + . . . 4- HnVi + l/il'2+ U.,V. + ... + UnV.

+

But this double series is absolutely convergent; for if these terms
are replaced by their moduli, the result is a Tn, where

<7" = 1 "1 I + I /2 I +    + 1 n I, rn = \ Vi \ + \ v.,\ + ...+\
Vn\,

and cTnTn is known to have a limit. Therefore, by \hardsectionref{2}{2}TODO:verifyref, if the
elements of the double series, of which the general term is u,nVn, be
taken in any order, their sum converges to ST.

Example. Shew that the series obtained by multiplying the two series 2
22 23 4,111

i + + + 2;. + 24 + -' 1 + 1 + + + -'

and rearranging accoi'ding to powers of z, converges so long as the
representative point of z lies in the ring-shaped region bounded by
the circles \ z\ = l and | a | = 2.

\Section{2}{6}{Fower-Seriesf.TODO}

A series of the type

Uo + aiZ + a z- + a-iZ" - ...,

in which the coefficients a, a,a2, a, ... are independent of z, is
called a series proceeding according to ascending powers of z, or
briefly a poiver-series.

* Analyse Algebrique, Note vii.

t The results of this section are due to Cauchy, Analyse Algebrique,
Ch. ix.

%
% 30
%

We shall now shew that if a power-series converges for any value z of
z, it ivill he absolutely convergent for all values of z whose
representative points are luithin a circle luhich passes through z and
has its centre at the origin.

00

For, if z be such a point, we have j | < i 'o | . Now, since S, V
converges,

anZo must tend to zero as ?i->oo, and so we can find M (independent
of n) such that

I 0.nZ,'' \ < M.

Thus i anz'' \ < M \ .

00

Therefore every term in the series S | n " | is less than the
corresponding term in the convergent geometric series

2i)/ -

the series is therefore convergent; and so the power-series is
absolutely convergent, as the series of moduli of its terms is a
convergent series; the result stated is therefore established.

Let lim | a | ~ "* = r; then, from\hardsubsectionref{2}{3}{5}, 2 anZ converges absolutely
when

00

\ z\ < r\ if \ z r, anZ" does not tend to zero and so 2 a z diverges
\hardsectionref{2}{3}).

The circle \ z\=r, which includes all the values of z for which the

power-series

tto + aiZ + a.iZ"- + ag -f- . . .

converges, is called the circle of convergence of the series. The
radius of the circle is called the radius of convergence.

In practice there is usually a simpler way of finding r, derived from
d'Alembert's test \hardsubsectionref{2}{3}{6}); r is lim ( / + 1) if this limit exists.

A power-series may converge for all values of the variable, as
happens, for instance, in the case of the series*

z z

which represents the function sin z; in this case the series
converges over the whole r-plane.

On the other hand, the radius of convergence of a power-series may be
zero; thus in the case of the series

I + 1\ z + 2\ z- - Z\ z + 4>\ z' - ...

we have

U

= n\ z

* The series for c, sin z, cos z and the fundamental properties of
these functions and of log z will be assumed throughout. A brief
account of the theory of the functions is given in the Appendix.

%
% 31
%

which, for all values of n after some fixed value, is greater than
unity when z has any value different from zero. The series converges
therefore only at the point z = 0, and the radius of its circle of
convergence vanishes.

A power-series may or may not converge for points which are actually
on the periphery of the circle; thus the series

z z- z'-' z* -'- + p + 25 + 3 + 4i +   '

whose radius of convergence is unity, converges or diverges at the
point z = 1 according as s is greater or not greater than unity, as
was seen in\hardsubsectionref{2}{3}{3}.

Corollary. If ( ) be a sequence of positive terms such that lim(a +i/a
) exists, this limit is equal to lim '"'.

\Subsection{2}{6}{1}{Convergence of series derived from a power-series.}

Let cify + i2 + a.iZ- + a z + a r* + . . .

be a power-series, and consider the series

ai + 1a.,z - a z" + 4a42-' + . . ., which is obtained by
differentiating the power-series term by term. We shall now shew that
the deHved series has the same circle of convergence as the original
series.

For let 3 be a point within the circle of convergence of the
power-series; and choose a positive number 7'i, intermediate in value
between \ z\ and r the

X

radius of convergence. Then, since the series S n i" converges
absolutely, its

terms must tend to zero as n - > x; and it must therefore be possible
to find a positive number M, independent of, such that ! I < Mr ~ for
all values of n.

 X

Then the terms of the series 2 ?i j a | 1 2- 1 "~ are less than the
corre- sponding terms of the series

 =i

M 71:2,"-'

n =i n"~'

But this series converges, by\hardsubsectionref{2}{3}{6}, since \ z\ < r- . Therefore, by\hardsubsectionref{2}{3}{4}, the series

 ncin \ z\''-

X

converges; that is, the series 2 la "" converges absolutely for all
points z

n = l

X

situated within the circle of convergence of the original series 2
OnZ'' . When

n =

I I > ? anZ does not tend to zero, and a fortiori na z '' does not
tend to zero; and so the two series have the same circle of
convergence.

%
% 32
%

Corollary. The series 2 - - - > obtained by integrating the original
power-series term by term, has the same circle of convergence as 2 a
z""-.

n=0

\Section{2}{7}{Infinite Products.}

We next consider a class of limits, known as infinite products.

Let 1 + ai, 1 + a.2, 1 + as,    be a sequence such that none of its
members vanish. If, as n oo, the product

(1 + ai) (1 + 2) (1 + as)   . (1 + a ) (which we denote by Tin)
tends to a definite limit other than zero, this limit is called the
value of the infinite product

n = (l + a0(l + a,)(l + a3)...,

and the product is said to be convergent *. It is almost obvious that
a necessary condition for convergence is that lim cin = 0, since lim
Un-i = lim Un + 0.

00

The limit of the product is written II (1 4- cin).

n-l m ( rn \

Now n (l+a ) = exp- S log(l +a,,)K

w=l I n-\ . J

andf exp lim, ] = lim exp ii,n]

if the former limit exists; hence a sufficient condition that the
product

00

should converge is that 2 log(l + a ) should converge when the
logarithms

n = l

have their principal values. If this series of logarithms converges
absolutely, the convergence of the product is said to be absolute.

The condition for absolute convergence is given by the following
theorem : in order that the infinite product

(l+a,)(l+a2)(l + a3)... may he absolutely convergent, it is necessary
and sufiicient that the series

tti + 02 + as + . . . sJiould be absolutely convergent.

For, by definition, 11 is absolutely convergent or not according as
the

series

log (1 + Oi) + log (1 + Oo) + log (1 + ag) + ...

is absolutely convergent or not.

* The convergence of the product in which rt,i\ i= - l/n was
investigated by WaUis as early as 1655.

t See the Appendix, § A-2.

%
% 33
%

Now, since lim a = 0, Ave can find m such that, when n > ?n, | a | < |
; and then

  "' log (1 + Un) - 1

2- + 2 +    - 2 

<2 . + 23+--- =

And thence, when n> m, - h 5 therefore, by the comparison

theorem, the absolute convergence of 2 log (1 + ) entails that of Sa
and

conversely, provided that a 4= - 1 for any value of n.

This establishes the result*.

If, in a product, a finite number of factors vanish, and if, when
these are suppressed, the resulting product converges, the original
product is said to converge to zero. But such

a product as n (!- "') is said to diverge to zero.

n=2

Corollary. Since, if Sn- 'l, exp ( S' )- -exp, it follows from\hardsubsectionref{2}{4}{1}
that the factors of an absolutely convergent product can be deranged
without aftecting the value of the product.

Example 1. Shew that if n (1 + ) converges, so does 2 log (1 +a ), if
the logarithms

H=l )l=l

have their principal values.

Example 2. Shew that the infinite product

sin z sin \ z sin \ z sin z z ' \ z ' \ z ' \ z '" is absolutely
convergent for all values of z.

[For (sin-j /(-) can be written in the form 1 - |, where | X |<X- and
/ i.s inde-

l endent of n; and the series 2 - is aVjsolutely convergent, as is
seen on comparing

it with 2 - . The infinite product is therefore absohitely
convergent.] w = l '

\Subsection{2}{7}{1}{Some examples of infinite products.}
Consider the infinite
product

  - ) -m-£)

which, as will be proved later (§ 7 '5), represents the function z sin
z.

In order to find whether it is absolutely convergent, we must consider
the

series 2 -, or -- S - : this series is absolutelv convergent, and so
the

product is absolutely convergent for all values of z. Now let the
product be written in the form

* A discussion of the convergence of infinite products, in which the
results are obtained without making use of the logarithmic function,
is given by Pringsheim, Math. Ann. xxxm. (1889), pp. 119-154, and also
by Bromwich, Infinite Series, Ch. vi.

W. M. A. 3

%
% 34
%

The absolute convergence of this product depends on that of the series

z z z z IT IT 27r 27r

But this series is only conditionally convergent, since its series of
moduli

\ z\ \ z\ \ z\ \ z\ IT IT 27r 27r

is divergent. In this form therefore the infinite product is not
absolutely

convergent, and so, if the order of the factors [ 1 + - ] is deranged,
there is

a risk of altering the value of the product.

Lastly, let the same product be written in the form

in which each of the expressions

1 + ) e mn

miTj

is counted as a single factor of the infinite product. The absolute
convergence of this product depends on that of the series of which the
(2?n - l)th and (2m)th terms are

1 + e mn - 1.

But it is easy to verify that

V mTTJ \ m-/

and so the absolute convergence of the series in question follows by
comparison

with the series

111111

l + l+2 + 2, + 3. + 3, + 4. + p+....

The infinite product in this last form is therefore again absolutely

convergent, the adjunction of the factors e '*"" having changed the
con- vergence from conditional* to absolute. This result is a
particular case of the first part of the factor theorem of Weierstrass
\hardsectionref{7}{6}).

Example 1. Prove that n ](l - ) e" is absolutely convergent for all
values of

n=i l.\ c-f-?i/ )

z, if c is a constant other than a negative integer.

For the infinite product converges absolutely with the series

n=i t\ c + nj J

%
% 35
%

Now the general term of this series is

But 2 - converges, and so, by\hardsubsectionref{2}{3}{4}, 2 ](l je"-!- converges
absolutely,

n=l n=\ l\ C + %/ J

and therefore the product converges absolutely.

Example 2. Shew that n jl-H--] z~'\ converges for all points z
situated

outside a circle whose centre is the origin and radius unity.

For the infinite product is absolutely convergent provided that the
series

=o / ] yn

2 1 S-"

is absolutely convergent. But lim (l -- ) =e, so the limit of the
ratio of the (w + l)th

term of the series to the Jith term is -; there is therefore absolute
convergence when

z

1

Example 3. Shew that

- < 1, i.e. when ] 2 | > 1. 1.2.3...(m-l)

2 2 ""U'

(2+l)(2 + 2)...(2 + 7ft-l)

tends to a finite limit as ??j- -x, unless 2 is a negative integer.

For the expression can be written as a product of which the nth factor
is

2 + 7i \ n ) ~\ n) V / I

This product is therefore absolutely convergent, provided the series

* 1 is absolutely convergent; and a comparison with the convergent
series 2 - shews that

this is the case. When 2 is a negative integer the expression does not
exist because one of the factors in the denominator vanishes.

Example 4. Prove that For the given product

,|o..(,-i)(i-i)(i.|)...(.-, )(i- )(i.£)

(,77 \ 2 3 "''2 ' 2k-l 2k k/

--log 2 .

= e sin 2.

= lim

X 2 1 - - e

 -27'

1-

2/C7

,2kn

1 +

2 \ -1-

klT

= lim e"-V 2+3"-+2A-i 2k) Ji\ \ en fi + \ e~ (l- e (1+ e <!-...,

3-2

%
% 36
%

since the product whose factors are

1 - - ) e

is absohitely convergent, and so the order of its factors can be
altered.

Since log2 = l-HJ-i + *---M

this shews that the given product is equal to

--logs . e " sin 2.

\Section{2}{8}{Infinite Determinants.}

Infinite series and infinite products are not by any means the only
known cases of limiting processes which can lead to intelligible
results. The researches of G. W. Hill in the Lunar Theory* brought
into notice the possibilities of infinite determinants.

The actual investigation of the convergence is due not to Hill but to
Poincare, Bull, de la Soc. Math, de France, xiv. (1886), p. 87. We
shall follow the exposition given by H. von Koch, Acta Math, xvi,
(1892), p. 217.

Let Aik be defined for all integer values (positive and negative) of
i, k, and denote by

the determinant formed of the numbers Aik i,k = - m, ... +m); then if,
as m - cc, the expression D,n tends to a determinate limit D, we
shall say that the infinite determinant

[- >i-J?,i-=-< ...+oo

is convergent and has the value D. If the limit D does not exist, the
deter- minant in question will be said to be divergent.

T he elements An, (where i takes all values), are said to form the
principal diagonal of the determinant D; the elements Aik, (where i is
fixed and k takes all values), are said to form the 7'ow i; and the
elements A c, (where k is fixed and i takes all values), are said to
form the column k. Any element A-iy; is called a diagonal or a
non-diagonal element, according as = A; or i \$ k. The element udo.o
is called the origin of the determinant.

\Subsection{2}{8}{1}{Convergence of an infinite determinant.}

We shall now shew that an infinite determinant converges, provided the
product of the diagonal elements converges absolutely, and the sum of
the non-diagonal elements converges absolutely.

For let the diagonal elements of an infinite determinant I) be denoted
by l+a, and let the non-diagonal elements be denoted by ajj., i=¥k),
so that the determinant is

* Reprinted in Acta Mathematica, viii. (1886), pp. 1-36. Infinite
determinants had previously occurred in the researches of Fiirstenau
on the algebraic equation of the 7ith degree, Darstellung der reellen
Wurzeln alyebraincher Gleiclnmgen durch Determinanten der
Coeffizienten (Marburg, 1860). Special types of infinite determinants
(known as continuants) occur in the theory of infinite continued
fractions; see Sylvester, Math. I'apers, i, p.~504 and in, p.~249

%
% 37
%

Then, since the series  2

i,k=-

is convergent.

Now form the products

m / m \ m / m

P, = n 1+ 2 au-, P,= n 1+ 2

i"=-ni\ fc= - m / i = -n( \ fc=-m

then if, in the expansion of P i, certain terms are replaced by zero
and certain other terms have their signs changed, we shall obtain i),
; thus, to each term in the expansion of 2>, there corresponds, in
the expansion of P,, a term of equal or greater modulus. Now An + p -
An represents the sum of those terms in the determinant i), + p which
vanish when the numbers TODO are replaced by zero; and to each of
these terms there corresponds a term of equal or greater modulus in Pm
+ p- m-

Hence

B

,-D \ < R,

-P.

Therefore, since P,n tends to a limit as ni-*-cc, so also Z), tends
to a limit. This establishes the proposition.

\Subsection{2}{8}{2}{The rearrangement theorem for convergent infinite determinants.}

We shall now shew that a determinant, of the convergent form already
co)isidered, remains convergent when the elements of any row are
replaced by any set of elements whose moduli are all less than some
fixed positive mimber.

Replace, for example, the elements

-" 0, - >i

'0

A,

of the row through the origin by the elements

.../!\,,... jiQ ... ix,n ...

which satisfy the inequality

I M>- I < Mj where /x is a positive number; and let the new values of
Z) i ' nd D be denoted by Dm and D'. Moreover, denote by /* / and P'
the products obtained by suppressing in P,n and P the factor
corresponding to the index zero; we see that no terms of 2) ' can
have a greater modulus than the corresponding term in the expansion of
nP '; and consequently, reasoning as in the last article, we have

which is sufficient to establish the result stated.

Example. Shew that the necessary and sufficient condition for the
absolute conver- gence of the infinite determinant

lim 1 a, ...

 2

a., ... 1 as ...

is that the series

shall be absolutely convergent.

,0 ... /3, 1

ai/3i + ao/32 + 03/33 + ...

(von Koch.)

%
% 38
%

REFERENCES. Convergent series.

A. Pringsheim, Math. Ann. xxxv. (1890), pp. 297-394.

T. J. I'a. Bromwich, Theory of Infinite Series (1908), Chs. Ii, ill,
iv.

Conditionally convergent series.

G. F. B. Riemann, Ges. Math. Werke, pp. 221-225. A. Pringsheim, Math.
Ami. xxii. (1883), pp. 455-503.

Double series.

A. Pringsheim, MUnchener Sitzungsherichte, xxvii. (1897), pp. 101-152.

    Math. Ann. hill. (1900), pp. 289-321.

G. H. Hardy, Proc. London Math. Soc. (2) i. (1904), pp. 124-128.

Miscellaneous Examples.

1. Evaluate litn (e """ ), lim (?i~ ogn) when a>0, b>0.

2. Investigate the convergence of

3. Investigate the convergence of

(l.3...2n-l 4/1 + 3

\addexamplecitation{Trinity, 1904.}

- 1 . . . . \addexamplecitation{Peterhouse, 1906.}

 =i\ 2. 4, ..2% 2>i + 2j '

4. Find the range of vakies of z for which the series

2sin2s-4sin- 2 + 8sin''s-... + (-)™ + i2 sin2' s+... is convergent.

5. Shew that the series

1 l\ \ 1 1\

Z 2+1 2 + 2 2 + 3

is conditionally convergent, except for certain exceptional values of
z; but that the series

11 11

1 Jl\ 1

Z 2+1 Z + p-\ Z+p Z+p + l

+

+ ..

2 + 2p + -l 2 + 2 + 5- in which (p + q) negative terms always follow p
positive terms, is divergent. \addexamplecitation{Simon.}

6. Shew that

l-i - 1 + 1-1- 1+1- =ilof 2

7. Shew that the series

is convergent, although

8. Shew that the series is convergent although

1111

1" 23 3" 4

 2n + l/W2n- a>-

a + /3- + a3 + /3< + ...

\addexamplecitation{Trinity, 1908.} (l<a</3)

\addexamplecitation{Ceskro.}

(0<a</3<l)

\addexamplecitation{Cesaro.}

%
% 39
%

9. Shew that the series

   z"-i (H-%-i) -l

39

 =l(3"-l) 3 -(l+ ->)

converges absolutely for all values of z, except the values

Z-(\ j gtkKilm

(a = 0, 1; k = 0, 1, ... m-\; i = l, 2, 3, ...).

10. Shew that, when s> 1,

I i= + i r + j-j- i Lii

and shew that the series on the right converges when <s< 1.

(de la Vallee Poussin, Mem. de VAcad. de Belgique, liii. (1896), no.
6.)

11. In the series whose general term is

   = y - yi'"'+l', (0<9<1< )

where v denotes the number of digits in the expression of n in the
ordinary decimal scale of notation, shew that

lim u = q,

and that the series is convergent, although lim Mn+i/wn==c 

12. Shew that the series

where !? v = j' ""'"""', (0<5'<1)

is convergent, although the ratio of the (?t + l)th term to the nth is
greater than unity

when n is not a triangular number. \addexamplecitation{Ceskro.}

13. Shew that the series

2

 =o( <' + )*' where w is real, and where w + nY is understood to mean
e iog(w + n) the logarithm being taken in its arithmetic sense, is
convergent for all values of s, when 1 x) is positive, and is
convergent for values of s whose real part is positive, when x is real
and not an integer.

14. If Un>0, shew that if 2?< converges, then lim nu, = 0, and that,
if in addition

M,i w,i + i, then lim (?i?< ) = 0.

15. If

shew that

m - n m + n - 1) !

"MLn - om+n

2m+n m\ n\ '

am,o = 2-'", ao, =-2-, ao,o=0,

n!=0 i=0 / =0 \ m=0

(m, n>0)

\addexamplecitation{Trinity, 1904.}

16. By converting the series

1+:

16g2 24g3

+ .2 "f" 1 l3 +    ?

l-q 1+ 2 l-jS (in which | g- 1 < 1), into a double series, shew that
it is equal to

1 +

X2 +

2 8(73

  qf (1+ 2)2 (1\ 3)2

\addexamplecitation{Jacobi.}

%
% 40
%

17. Assuming that sin2 = 3 n (I--5-,),

shew that if ? - qo and 3t-*-co iu such a way that lim (in\ n) = k,
where k is finite, then

hm n' 1 + - =F'',

the prime indicating that the factor for which r = is omitted. (Math.
Trip., 1904.)

18. If Uq-=Ui = U2 = 0, and if, when n>\,

W2 -i- -

 7l'

1 1 1

si 11 n n y/n

then n (1 +?< ) converges, though 2 and 2 2 are divergent.

n=0 n=0 ?i=0

\addexamplecitation{Math. Trip. 1906.}

19. Prove that

n - 1 - - exp 2

where k is any positive integer, converges absolutely for all values
of z.

20. If 2 a,i be a conditionally convergent series of real terms, then
n (l4-,i) con-

n=l n=l

verges (but not absolutely) or diverges to zero according as 2 a 2
converges or diverges.

11=1

\addexamplecitation{Cauchy.}

21. Let 2 dn be an absolutely convergent series. Shew that the
infinite determinant

11=1

A C)-

(C- 4)2- 0

~e.

-00

42- 0

- 8

42- 0

- 4

   42- 0

42- 0

42- 0 -

-e,

(c- 2)2- 0

22- 0

-01

-02

2- -00

- 3

    22- 0

22- 0

22- 0

-( 2

02- 0

02- 0

-0,

02- 0

-02

'" 02- 0

02- 0

-e.

-02

2'-0o

- 1

22- 0

(c + 2)2- 22- 0

 0 ~ 1

- 2 -3,

22- 0 -

- 4

-6,

- 2

-

(c + 4)2- o

  42- 0

42- 0

42- 0

42- 0

42- 0

converges; and shew that the equation

is equivalent to the equation

A(c) =

sin2 Ittc = A (0) sin2 W(9o-- .

(Hill; see\hardsubsectionref{19}{4}{2}.)

