\chapter{The Theory of Residues; Application to the Evaluation of Definite Integrals} 

\Section{6}{1}{Residues}

If the function $f(z)$ has a pole of order $m$ at $z=a$, then, by the
definition of a pole, an equation of the form
$$
f(z)
=
\frac{a_{-m}}{ (z-a)^{m} }
+ \frac{ a_{-m+1} }{ (z-a)^{m-1} }
+ \cdots
+ \frac{a_{-1}}{z-a}
+ \phi(z),
$$
where $\phi(z)$ is analytic near and at $a$, is true near $a$.

The coefficient $a_{-1}$ in this expansion is called the \emph{residue}\index{Residues!defined} of the
function $f(z)$ relative to the pole $a$.

Consider now the value of the integral $\int_{\alpha} f(z) \dmeasure z$, where the path of
integration is a circle\footnote{The existence of such a circle is
  implied in the definition of a pole as an isolated singularity.}
$\alpha$, whose centre is the point $a$ and whose radius $\rho$ is so small
that $\phi(z)$ is analytic inside and on the circle.

We have
$$
\int_{\alpha} f(z) \dmeasure z
=
\sum_{r=1}^{m}
a_{r}
\int_{\alpha} \frac{\dmeasure z}{ (z-a)^{r}}
+
\int_{\alpha} \phi(z) \dmeasure z.
$$

Now $\int_{\alpha} \phi(z) \dmeasure z = 0$ by \hardsectionref{5}{2};
and (putting $z-a = \rho e^{i\theta}$) we have, if $r \neq 1$,
$$
\int_{\alpha} \frac{ \dmeasure z}{ (z-a)^{r} }
=
\int_{0}^{2\pi} \frac{\rho e^{i\theta} i \dmeasure \theta}{ \rho^{r}
  e^{ri\theta}}
=
\rho^{-r+1}
\int_{0}^{2\pi} e^{(1-r)i\theta} i \dmeasure \theta
=
\rho^{-r+1}
\thebracket{ \frac{e^{(1-r) i \theta}}{1-r} }_{0}^{2\pi}
=
0.
$$

But, when $r = 1$, we have
$$
\int_{\alpha}
\frac{\dmeasure z}{z - a}
=
\int_{0}^{2\pi} i \dmeasure \theta
=
2\pi i.
$$

Hence finally
$$
\int_{\alpha} f(z) \dmeasure z = 2 \pi i a_{-1}.
$$

Now let $C$ be any contour, containing in the region interior to it a
number of poles $a,b,c,\ldots$ of a function $f(z)$, with residues
$a_{-1}, b_{-1}, c_{-1},\ldots$
respectively; and suppose that the function $f(z)$ is
analytic throughout $C$ and its interior, except at these poles.

Surround the points $a,b,c,\ldots$ by circles
$\alpha,\beta,\gamma,\ldots$
so small that their respective centres are the only singularities
inside or on each circle; then the function $f(z)$ is analytic in the
closed region bounded by $C$, $\alpha,\beta,\gamma,\ldots$.
%
% 112
%

Hence, by \hardsectionref{5}{2} corollary TODO,
\begin{align*}
  \int_{C} f(z) \dmeasure z
  =&
  \int_{\alpha} f(z) \dmeasure z
  + \int_{\beta} f(z) \dmeasure z
  + \cdots
  \\
  =&
  2\pi i a_{-1} + 2\pi i b_{-1} + \cdots.
\end{align*}

Thus we have the \emph{theorem of residues}, namely that
\emph{if $f(z)$ be analytic throughout a contour $C$ and its interior
  except at a number of poles inside the contour, then
  $$
  \int_{C} f(z) \dmeasure z = 2\pi i \sum R,
  $$
where $\sum R$ denotes the sum of the residues of the function $f(z)$ at
those of its poles which are situated within the contour $C$.}

This is an extension of the theorem of \hardsubsectionref{5}{2}{1}.

%\begin{Remark} TODO:\alpha or a?
Note. If $a$ is a simple pole of $f(z)$ the residue of $f(z)$ at that pole
is $\lim_{z\rightarrow a} (z-a) f(z)$.
%\end{Remark}

\Section{6}{2}{The evaluation of definite integrals.}

We shall now apply the result of \hardsectionref{6}{1} to evaluating
various classes of definite integrals; the methods to be employed in
any particular case may usually be seen from the following typical
examples.

\Subsection{6}{2}{1}{The evaluation of the integrals of certain
  periodic functions taken between the limits $0$ and
  $2\pi$.\index{Integrals!of periodic functions}\index{Periodic
    functions!integrals involving}}
An integral of the type
$$
\int_{0}^{2\pi} R(\cos \theta, \sin \theta) \dmeasure \theta,\index{Trigonometrical integrals}
$$
where the integrand is a rational function of $\cos\theta$ and
$\sin\theta$ finite on the range of integration, can be evaluated by
writing $e^{i\theta}=z$; since
$$
\cos\theta = \half(z+z^{-1}),
\quad
\sin\theta = \frac{1}{2i} (z-z^{-1}),
$$
the integral takes the form $\int_{C} S(z) \dmeasure z$, where $S(z)$
is a rational function of $z$ finite on the path of integration $C$, the
circle of radius unity whose centre is the origin.

Therefore, by \hardsectionref{6}{1}, \emph{the integral is equal to
  $2\pi i$ times the sum of the residues of $S(z)$ at those of its
  poles which are inside that circle.}

\begin{wandwexample}
  If $0 < p < 1$,
  $$
  \int_{0}^{2\pi}
  \frac{\dmeasure \theta}{ 1 - 2p\cos\theta + p^{2}}
  =
  \int_{C} \frac{ \dmeasure z}{ i (1-pz) (z-p) }.
  $$
  The only pole of the integrand inside the circle is a simple pole at
  $p$; and the residue there is
  $$
  \lim_{z \rightarrow p} \frac{z-p}{ i (1-pz)(z-p)}
  =
  \frac{1}{i (1-p^{2})}.
  $$
  % 
  % 113
  % 
  Hence
  $$
  \int_{0}^{2\pi} \frac{\dmeasure\theta}{1 - 2p\cos\theta + p^{2}}
  =
  \frac{2\pi}{1-p^{2}}.
  $$
\end{wandwexample}

\begin{wandwexample}
  If $0 < p < 1$,
  \begin{align*}
    \int_{0}^{2\pi} \frac{ \cos^{2} 3\theta}{1-2p\cos 2\theta + p^{2}}
    \dmeasure \theta
    =&
    \int_{C} \frac{\dmeasure z}{iz}
    \theparen{ \half z^{3} + \half z^{-3}}^{2}
    \frac{1}{ (1-pz^{2})(1-pz^{-2})}
  \end{align*}
  where $\sum R$ denotes the sum of the residues of
  $ \frac{ (z^{6} + 1)^{2} }{ 4z^{5} (1-pz^{2})(z^{2}-p)} $ at its poles
  inside $C$; these poles are $0, -p^{\half}, p^{\half}$;
  and the residues at them are
  $ -\frac{1+p^{2}+p^{4}}{4p^{3}},
  \frac{(p^{3}+1)^{2}}{ 8p^{3}(1-p^{2})},
  \frac{(p^{3}+1)^{2}}{ 8p^{3}(1-p^{2})}$; %TODO:same?
  and hence the integral is equal to
  $$
  \frac{\pi (1-p+p^{2})}{1-p}.
  $$
\end{wandwexample}
\begin{wandwexample}
  If $n$ be a positive integer,
  $$
  \int_{0}^{2\pi} e^{\cos\theta} \cos(n\theta - \sin\theta)
  \dmeasure \theta
  =
  \frac{2\pi}{n!},
  \quad
  \int_{0}^{2\pi}
  e^{\cos\theta} \sin(n\theta - \sin\theta) \dmeasure \theta
  =
  0.
  $$
\end{wandwexample}
\begin{wandwexample}
  If $a>b>0$,
  $$
  \int_{0}^{2\pi}
  \frac{\dmeasure \theta}{ (a+b\cos\theta)^{2}}
  =
  \frac{2\pi a}{ (a^{2}-b^{2})^{3/2} },
  \quad
  \int_{0}^{2\pi}
  \frac{\dmeasure\theta}{ (a+b\cos^{2}\theta)^{2}}
  =
  \frac{\pi(2a+b)}{ a^{3/2} (a+b)^{3/2}}.
  $$
\end{wandwexample}
\Subsection{6}{2}{2}{The evaluation of certain types of integrals
  taken between the limits $-\infty$ and $+\infty$.}
We shall now evaluate $\int_{-\infty}^{\infty} Q(x) \dmeasure x$,
where $Q(z)$ is a function such that (i) it is analytic when the
imaginary part of $z$ is positive or zero (except at a finite number
of poles), (ii) it has no poles on the real axis and (iii) as
$\absval{z}\rightarrow\infty$, $zQ(z) \rightarrow 0$ uniformly for all
values of $\arg z$ such that $0 \leq \arg z \leq \pi$; provided that
(iv) when $x$ is real, $x Q(x) \rightarrow 0$, as
$x \rightarrow \pm\infty$, in such a way\footnote{The condition
  $x Q(x)\rightarrow 0$ is not in itself sufficient to secure the
  convergence of 
  $\int^{\infty} Q(x) \dmeasure x$;
  consider $Q(x) = (x \log x)^{-1}$.}
that $\int_{0}^{\infty} Q(x) \dmeasure x$ and
$\int_{-\infty}^{0} Q(x) \dmeasure x$ both converge.

Given $\eps$, we can choose $\rho_{0}$ (independent of $\arg z$) such
that $\absval{z Q(z)} < \eps/\pi$ whenever
$\absval{z} > \rho_{0}$ and $0 \leq \arg z \leq \pi$.

Consider $\int_{C} Q(z) \dmeasure z$ taken round a contour $C$
consisting of the part of the real axis joining the points $\pm\rho$
(where $\rho > \rho_{0}$) and a semicircle $\Gamma$, of radius $\rho$,
having its centre at the origin, above the real axis.

Then, by \hardsectionref{6}{1}, $\int_{C} Q(z) \dmeasure z = 2 \pi i
\sum R$, where $\sum R$ denotes the sum of the residues of $Q(z)$ at
its poles above the real axis.\footnote{$Q(z)$ has no poles above the
  real axis outside the contour.}

%
% 114
%
Therefore
$$
\absval{
  \int_{-\rho}^{\rho} Q(z) \dmeasure z
  -
  2 \pi i \sum R
}
=
\absval{
  \int_{\Gamma} Q(z) \dmeasure z
}.
$$

In the last integral write $z = \rho e^{i\theta}$, and then
\begin{align*}
  \absval{ \int_{\Gamma} Q(z) \dmeasure z  }
  &=
  \absval{
    \int_{0}^{\pi} Q(\rho e^{i\theta}) \rho e^{i\theta} i
    \dmeasure\theta
  }
  \\
  &<
  \int_{0}^{\pi} (\eps / \pi) \dmeasure \theta
  \\
  &= \eps
\end{align*}
by \hardsubsectionref{4}{6}{2}.

Hence 
$$
\lim_{\rho\rightarrow\infty}
\int_{-\rho}^{\rho} Q(z) \dmeasure z
=
2\pi i \sum R.
$$

But the meaning of $\int_{-\infty}^{\infty} Q(x) \dmeasure x$;
is $\lim_{\rho,\sigma\rightarrow\infty}\int_{-\rho}^{\sigma} Q(x)
\dmeasure x$;
and since
$\lim_{\sigma\rightarrow\infty}\int_{0}^{\sigma} Q(x) \dmeasure x$
and
$\lim_{\rho\rightarrow\infty}\int_{-\rho}^{0} Q(x) \dmeasure x$
both exist, this double limit is the same as
$\lim_{\rho\rightarrow\infty}\int_{-\rho}^{\rho} Q(x) \dmeasure x$.

Hence we have proved that
$$
\int_{-\infty}^{\infty} Q(x) \dmeasure x = 2\pi i \sum R.
$$

This theorem is particularly useful in the special case when $Q(x)$ is a
rational function.

%\begin{smallfont}
[Note. Even if condition (iv) is not satisfied, we still have
$$
\int_{0}^{\infty}
\thebrace{
  Q(x) + Q(-x)
} \dmeasure x
=
\lim_{\rho\rightarrow\infty} \int_{-\rho}^{\rho} Q(x) \dmeasure x
= 2 \pi i \sum R.]
$$
%\end{smallfont}

\begin{wandwexample}
  The only pole of $(z^{2} + 1)^{-3}$ in the upper half plane is a
  pole at $z=i$ with residue there $-\frac{3}{16} i$. Therefore
  $$
  \int_{-\infty}^{\infty} \frac{\dmeasure x}{ (x^{2}+1)^{3} }
  =
  \frac{3}{8} \pi.
  $$
\end{wandwexample}
\begin{wandwexample}
If $a > 0, b > 0$, shew that
$$
\int_{-\infty}^{\infty}
\frac{x^{4} \dmeasure x}{ (a + bx^{2})^{4} }
=
\frac{\pi}{ 16 a^{3/2} b^{5/2}}.
$$
\end{wandwexample}
\begin{wandwexample}
  By integrating $\int e^{-\lambda z^{2}} \dmeasure z$ round a
parallelogram whose corners are $-R, R, R + ai, -R + ai$ and making $R
\rightarrow \infty$, shew that, if $\lambda > 0$, then
  $$
  \int_{-\infty}^{\infty} e^{-\lambda x^{2}} \cos (2\lambda a x)
\dmeasure x = e^{-\lambda a^{2}} \int_{-\infty}^{\infty} e^{-\lambda
x^{2}} \dmeasure x = 2 \lambda^{-\half} e^{-\lambda a^{2}}
\int_{0}^{\infty} e^{-x^{2}} \dmeasure x.
  $$
\end{wandwexample}
\Subsubsection{6}{2}{2}{1}{Certain infinite integrals involving sines
and cosines.}
If $Q(z)$ satisfies the conditions (i), (ii) and (iii) of
\hardsectionref{6}{2}{2}, and $m > 0$,
then $Q(z) e^{miz}$ also satisfies those conditions.

%
% 115
%

Hence
$
\int_{0}^{\infty}
\thebrace{
  Q(x) e^{mix} + Q(-x) e^{-mix}
}
\dmeasure x
$
is equal to $2\pi i \sum R'$, where
$\sum R'$ means the sum of the residues of $Q(z) e^{mix}$ at its poles
in the upper half plane; and so
\begin{itemize}
\item % (i)
  If $Q(x)$ is an even function\index{Even functions}, i.e. if $Q (- x) = Q (x)$,
  $$
  \int_{0}^{\infty} Q(x) \cos(mx) \dmeasure x
  =
  \pi i \sum R'.
  $$
\item %(ii)
  If Q (x) is an odd function\index{Odd functions},
  $$
  \int_{0}^{\infty} Q(x) \sin(mx) \dmeasure x
  =
  \pi \sum R'.
  $$
\end{itemize}
\Subsubsection[Jordan's lemma.]{6}{2}{2}{2}{Jordan's lemma.\footnote{TODO:Jordan, Cours d'Aiiahjse, ii. (1894), pp. 285, 286.}\index{Jordan's lemma}}

The results of \hardsubsubsectionref{6}{2}{2}{1} are true if $Q(z)$ be
subject to the less stringent condition $Q(z) \rightarrow 0$ uniformly
when
$O \leq \arg z \leq \pi$ as $\absval{z} \rightarrow \infty$ in
place of the condition $z Q(z) \rightarrow 0$ uniformly.

To prove this we require a theorem known as Jordan's lemma, viz.

\emph{If $Q(z) \rightarrow 0$ uniformly with regard to $\arg z$ as
  $\absval{z} \rightarrow \infty$ when $0 \leq \arg z \leq \pi$,
  and if $Q(z)$ is analytic when both $\absval{z} > c$ (a constant)
  and $0 \leq \arg z \leq \pi$, then
  $$
  \lim_{\rho\rightarrow\infty}
  \theparen{ \int_{\Gamma} e^{miz} Q(z) \dmeasure z  }
  =
  0,
  $$
  where $\Gamma$ is a semicircle of radius $\rho$ above the real axis with centre at
  the origin.}

Given $\eps$, choose $\rho_{0}$ so that $\absval{Q(z)} < \eps/\pi$
when $\absval{z} > \rho_{0}$ and
$0 \leq \arg z \leq \pi$;
then, if $\rho > \rho_{0}$,
$$
\absval{ \int_{\Gamma} e^{miz} Q(z) \dmeasure z }
=
\absval{
  \int_{0}^{\pi}
  e^{mi (\rho \cos\theta + i\rho \sin\theta)}
  Q(\rho e^{i\theta})
  \rho e^{i \theta}
  i \dmeasure \theta
}.
$$
But $\absval{ e^{mi\rho\cos\theta} } = 1$, and so
\begin{align*}
  \absval{ \int_{\Gamma} e^{miz} Q(z) \dmeasure z }
  <&
  \int_{0}^{\pi} (\eps/\pi) \rho e^{-m\rho\sin\theta} \dmeasure\theta
  \\
  =&
  (2\eps/\pi)
  \int_{0}^{\half\pi} \rho e^{-m\rho\sin\theta} \dmeasure\theta.
\end{align*}

Now $\sin\theta \geq 2\theta/\pi$, when\footnote{This inequality
  appears obvious when we draw the graphs $y = \sin x$, $y = 2x/\pi$;
  it may be proved by shewing that $(\sin\theta)/\theta$ decreases as
  $\theta$ increases from $0$ to $\half\pi$.
} $0 \leq \theta \leq \half\pi$, and so 
\begin{align*}
  \absval{ \int_{\Gamma} e^{miz} Q(z) \dmeasure z }
  <&
  (2\eps/\pi)
  \int_{0}^{\half\pi}
  \rho e^{-2m\rho\theta/\pi} \dmeasure\theta
  \\
  =&
  (2\eps/\pi) \cdot (\pi/2m)
  \thebracket{ -e^{-2m\rho\theta/\pi} }_{0}^{\half\pi}
  \\
  <&
  \eps/m.
\end{align*}

%
% 116
%

Hence
$$
\lim_{\rho\rightarrow\infty} \int_{\Gamma} e^{miz} Q(z) \dmeasure z = 0.
$$

This result is Jordan's lemma.

Now
$$
\int_{0}^{\rho}
\thebrace{
  e^{mix} Q(x) + e^{-mix} Q(-x)
}
\dmeasure x
=
2 \pi i \sum R'
-
\int_{\Gamma} e^{miz} Q(z) \dmeasure z,
$$
and, making $\rho\rightarrow\infty$, we see at once that
$$
\int_{0}^{\infty}
\thebrace{
  e^{mix} Q(x) + e^{-mix} Q(-x)
}
\dmeasure x
=
2 \pi i \sum R',
$$
which is the result corresponding to the result of \hardsubsubsectionref{6}{2}{2}{1}.

\begin{wandwexample}
  Shew that, if $a > 0$, then
  $$
  \int_{0}^{\infty}
  \frac{ \cos x}{x^{2} + a^{2}} \dmeasure x
  =
  \frac{\pi}{2a} e^{-a}.
  $$
\end{wandwexample}
\begin{wandwexample}
  Shew that, if $a \geq 0, b \geq 0$, then
  $$
  \int_{0}^{\infty}
  \frac{ \cos 2ax - \cos 2bx}{x^{2}}
  \dmeasure x
  =
  \pi (b-a)
  $$
  (Take a contour consisting of a large semicircle of radius $\rho$, a
  small semicircle of radius $\delta$, both having their centres at the
  origin, and the parts of the real axis joining their ends; then
  make $\rho \rightarrow \infty, \delta \rightarrow 0$.)
\end{wandwexample}
\begin{wandwexample}
  Shew that, if $b > 0, m \geq 0$, then
  $$
  \int_{0}^{\infty}
  \frac{3x^{2} - a^{2}}{ (x^{2} + b^{2})^{2} }
  \cos mx
  \dmeasure x
  =
  \frac{\pi e^{-mb}}{4b^{3}}
  \thebrace{
    3b^{2} - a^{2} - mb(3b^{2}+a^{2})
  }.
  $$
\end{wandwexample}
\begin{wandwexample}
  Shew that, if $k > 0, a > 0$, then
  $$
  \int_{0}^{\infty}
  \frac{ x \sin ax }{x^{2} + k^{2}}
  \dmeasure x
  =
  \half \pi e^{-ka}.
  $$
\end{wandwexample}
\begin{wandwexample}
  Shew that, if m 0, a > 0, then
  $$
  \int_{0}^{\infty}
  \frac{ \sin mx }{ x (x^{2} + a^{2})^{2} }
  \dmeasure x
  =
  \frac{\pi}{2a^{4}}
  -
  \frac{\pi e^{-ma}}{4a^{3}}
  \theparen{ m + \frac{2}{a} }.
  $$
  (Take the contour of example TODO.)
\end{wandwexample}
\begin{wandwexample}
Shew that, if the real part of $z$ be positive,
$$
\int_{0}^{\infty} (e^{-t} - e^{-tz}) \frac{\dmeasure t}{t} = \log z.
$$
[We have
\begin{align*}
  \int_{0}^{\infty} (e^{-t} - e^{-tz}) \frac{\dmeasure t}{t} = \log z.
  =&
  \lim_{\delta\rightarrow 0, \rho\rightarrow\infty}
  \thebrace{
    \int_{\delta}^{\rho} \frac{e^{-t}}{t} \dmeasure t
    -
    \int_{\delta}^{\rho} \frac{e^{-tz}}{t} \dmeasure t
  }
  \\
  =&
    \thebrace{
    \int_{\delta}^{\rho} \frac{e^{-t}}{t} \dmeasure t
    -
    \int_{\delta z}^{\rho z} \frac{e^{-u}}{u} \dmeasure u
  }
  \\
  =&
    \thebrace{
    \int_{\delta}^{\delta z} \frac{e^{-t}}{t} \dmeasure t
    -
    \int_{\rho}^{\rho z} \frac{e^{-t}}{t} \dmeasure t,
  }
\end{align*}
since $t^{-1} e^{-t}$ is analytic inside the
quadrilateral whose corners are $\delta, \delta z, \rho z, \rho$.

%
% 117
%

Now $\int_{\rho}^{\rho z} t^{-1} e^{-t} \dmeasure t \rightarrow 0$
when $\Re(z) > 0$; and
$$
\int_{\delta}^{\delta z} t^{-1} e^{-t} \dmeasure t
= \log z
-
\int_{\rho}^{\rho z} t^{-1} (1 - e^{-t}) \dmeasure t
\rightarrow
\log z,
$$
since $t^{-1} (1 - e^{-t}) \rightarrow 1$ as $t \rightarrow 0$.]
\end{wandwexample}
\Subsection{6}{2}{3}{Principal values of
  integrals.\index{Integrals!principal values of}\index{Principal part
  of a function!value of an integral}}
%\begin{smallfont}
It was assumed in TODO that the function $Q(x)$ had no
poles on the real axis; if the function has a finite number of simple
poles on the real axis, we can obtain theorems corresponding to those
already obtained, except that the integrals are all principal values
(\hardsectionref{4}{5}) and $\sum R$ has to be replaced by
$\sum R + \half \sum R_{0}$, where $\sum R_{0}$ means the
sum of the residues at the poles on the real axis. To obtain this
result we see that, instead of the former contour, we have to take as
contour a circle of radius $\rho$ and the portions of the real axis joining
the points
$$
-\rho, \ a-\delta_{1};
\quad
a+\delta_{1}, \ b-\delta_{2};
\quad
b+\delta_{2}, \ c-\delta_{3},
\ \ldots
$$
and small semicircles above the real axis of radii
$\delta_{1},\delta_{2},\ldots$ with
centres $a, b, c, \ldots$ where $a, b, c, \ldots$ are the poles of
$Q(z)$ on the real axis; and then we have to make
$\delta_{1}, \delta_{2}, \ldots \rightarrow 0$;
call these semicircles $\gamma_{1},\gamma_{2},\ldots$.
Then instead of the equation
$$
\int_{-\rho}^{\rho} Q(z) \dmeasure z
+
\int_{\Gamma} Q(z) \dmeasure z
=
2 \pi i \sum R,
$$
we get
$$
P \int_{-\rho}^{\rho} Q(z) \dmeasure z
+
\sum_{n}
\lim_{\delta_{n} \rightarrow 0}
\int_{\gamma_{n}} Q(z) \dmeasure z
+
\int_{\Gamma} Q(z) \dmeasure z
=
2 \pi i \sum R.
$$

Let $a'$ be the residue of $Q(z)$ at $a$; then writing
$z = a + \delta_{1} e^{i\theta}$ on $\gamma_{1}$ we get
$$
\int_{\gamma_{1}} Q(z) \dmeasure z
=
\int_{\pi}^{0}
Q(a + \delta_{1} e^{i\theta})
\delta_{1} e^{i\theta} i \dmeasure \theta.
$$
But $Q(a + \delta_{1} e^{i\theta}) \rightarrow a'$ 
uniformly as $\delta_{1} \rightarrow 0$; and therefore
$
\lim_{\delta_{1} \rightarrow 0} \int_{\gamma_{1}} Q(z) \dmeasure z
=
- \pi i a'
$;
we thus get
$$
P \int_{-\rho}^{\rho} Q(z) \dmeasure z
+
\int_{\Gamma} Q(z) \dmeasure z
=
2 \pi i \sum R
+
\pi i \sum R_{0},
$$
and hence, using the arguments of \hardsubsectionref{6}{2}{2}, we get
$$
P \int_{-\infty}^{\infty} Q(x) \dmeasure x
=
2 \pi i \theparen{ \sum R + \half \sum R_{0} }.
$$

The reader will see at once that the theorems of TODO have
precisely similar generalisations.

The process employed above of inserting arcs of small circles so as to
diminish the area of the contour is called \emph{indenting} the contour.
%\end{smallfont}
\Subsection{6}{2}{4}{Evaluation of integrals of the form
  $\int_{0}^{\infty} x^{a-1} Q(x) \dmeasure x.$}
Let $Q(x)$ be a rational function of $x$ such that it has no poles on the
positive part of the real axis and
$x^{a} Q(x) \rightarrow 0$ both when $x \rightarrow 0$ and when
$x \rightarrow \infty$.

%
% 118
%

Consider $\int (-z)^{a-1} Q(z) \dmeasure z$ taken round the contour
$C$ shewn in the figure TODO, consisting of the arcs of circles of
radii $\rho,\delta$ and the straight lines joining their end points;
$(-z)^{a-1}$ is
to be interpreted as
$$
\exp \thebrace{ ( a-1) \log (- z) }
$$
and
$$
\log (-z) = \log \absval{z} + \arg (-z),
$$
where
$$
-\pi \leq \arg (-z) \leq \pi; %TODO:is overlap correct?
$$
with these conventions the integrand is one-valued and analytic on and
within the contour save at the poles of $Q(z)$.

Hence, if $\sum r$ denote the sum of the residues of $(-z)^{a-1} Q(z)$
at all its poles,
$$
\int_{C} (-z)^{a-1} Q(z) \dmeasure z = 2 \pi i \sum r.
$$

On the small circle write $-z = \delta e^{i\theta}$, and the integral
along it becomes
$- \int_{\pi}^{-\pi} (-z)^{a} Q(z) i \dmeasure \theta$,
which tends to zero as $\delta \rightarrow 0$.

On the large semicircle write $-z = \rho e^{i\theta}$, and the
integral along it becomes $- \int_{-\pi}^{\pi} (-z)^{n} Q(z) i
\dmeasure\theta$, which tends to zero as $\rho \rightarrow \infty$. 

On one of the lines we write $-z = x e^{\pi i}$ on the other
$-z = x e^{-\pi i}$ and $(-z)^{a-1}$ becomes $x^{a-1} e^{\pm (a-1 \pi i}$.

Hence
$$
\lim_{\delta\rightarrow 0,\ \rho\rightarrow\infty}
\int_{\delta}^{\rho}
\thebrace{
  x^{a-1} e^{-(a-1)\pi i} Q(x)
  -
  x^{a-1} e^{(a-1)\pi i} Q(x)
}
\dmeasure x
=
2 \pi i \sum r;
$$
and therefore
$$
\int_{0}^{\infty} x^{a-1} Q(x) \dmeasure x
=
\pi \cosec (a\pi) \sum r.
$$

%TODO: change corollary environment
Corollary. If $Q(x)$ have a number of simple poles on the positive
part of the real axis, it may be shewn by indenting the contour that
$$
P \int_{0}^{\infty} x^{a-1} Q(x) \dmeasure x
=
\pi \cosec (a\pi) \sum r
-
\pi \cot (1\pi) \sum r',
$$
where $\sum r'$ is the sum of the residues of $z^{a-1} Q(z)$ at these poles.
\begin{wandwexample}
If $0 < a < 1$,
$$
\int_{0}^{\infty} \frac{x^{a-1}}{1+x} \dmeasure x
=
\pi \cosec a\pi,
\quad
P \int_{0}^{\infty} \frac{x^{a-1}}{1+x} \dmeasure x
=
\pi \cot a\pi
$$
\end{wandwexample}
%
% 119
%
\begin{wandwexample}
  If $0 < z < 1$ and $-\pi < a < \pi$,
  $$
  \int_{0}^{\infty} \frac{t^{z-1}}{ t+e^{ia} } \dmeasure t
  =
  \frac{ \pi e^{i(z-1)a}}{\sin \pi z}.
  $$
\addexamplecitation{Minding.}
\end{wandwexample}
\begin{wandwexample}
  Shew that, if $- 1 < z < 3$, then
  $$
  \int_{0}^{\infty} \frac{ x^{z} }{ (1+x^{2})^{2} } \dmeasure x
  =
  \frac{ \pi (1-z) }{ 4 \cos \half\pi z}.
  $$
\end{wandwexample}
\begin{wandwexample}
  Shew that, if $-1 < p < 1$ and $-\pi < \lambda < \pi$, then
  $$
  \int_{0}^{\infty}
  \frac{ x^{-p} \dmeasure x }{ 1 + 2x \cos\lambda + x^{2} }
  =
  \frac{\pi}{\sin p\pi} \frac{\sin p\lambda}{\sin \lambda}.
  $$
  \addexamplecitation{Euler.}
\end{wandwexample}
\Section{6}{3}{Cauchy's integral.}
We shall next discuss a class of contour-integrals which are
sometimes found useful in analytical investigations.

Let $C$ be a contour in the $z$-plane, and let $f(z)$ be a function
analytic inside and on $C$. Let $\phi(z)$ be another function which is
analytic inside and on $G$ except at a finite number of poles; let the
zeros of $\phi(z)$ in the interior\footnote{$\phi(z)$ must not have
  any zeros or poles on $C$.} of $C$ be $a_{1},a_{2},\ldots$, and let
their degrees of multiplicity be $r_{1},r_{2},\ldots$; and let its
poles in the interior of $C$ be $b_{1},b_{2},\ldots$, and let their
degrees of multiplicity be $s_{1},s_{2},\ldots$.

Then, by the fundamental theorem of residues,
$ \frac{1}{2\pi i} \int_{C} f(z) \frac{\phi'(z)}{\phi(z)} \dmeasure z $
is equal to the sum of the residues of
$f(z) \frac{\phi'(z)}{\phi(z)}$
at its poles inside $C$.

Now
$f(z) \frac{\phi'(z)}{\phi(z)}$
can have singularities only at the poles and zeros of $\phi(z)$.
Near one of the zeros, say $a_{1}$, we have
$$
\phi(z)
=
A (z-a_{1})^{r_{1}}
+ B (z-a_{1})^{r_{1}+1}
+ \cdots.
$$

Therefore
$$
\phi'(z)
=
A r_{1} (z-a_{1})^{r_{1}-1}
+ B (r_{1}+1) (z-a_{1})^{r_{1}}
+ \cdots,
$$
and
$$
f(z) = f(a_{1}) + (z-a_{1}) f'(a_{1}) + \cdots.
$$

Therefore
$
\thebrace{f(z) \frac{\phi'(z)}{\phi(z)}
  -
  \frac{ r_{1} f(a_{1}) }{ z - a_{1} }
}
$ is analytic at $a_{1}$.

Thus the residue of $f(z) \frac{\phi'(z)}{\phi(z)}$, at the point
$z=a_{1}$, is $r_{1} f(a_{1})$.

Similarly the residue at $z=b_{1}$ is
$-s_{1} f(b_{1})$; for near $z=b_{1}$, we have
$$
\phi(z)
=
C (z-b_{1})^{-s_{1}}
+ D (z-b_{1})^{-s_{1} + 1}
+ \cdots,
$$
and
$$
f(z)
=
f(b_{1})
+ (z - b_{1}) f'(b_{1})
+ \cdots,
$$
so
$
f(z) \frac{ \phi'(z) }{\phi(z)}
+
\frac{ s_{1} f(b_{1}) }{ z-b_{1} }
$
is analytic at $b_{1}$.

Hence
$$
\frac{1}{2\pi i}
\int_{C} f(z) \frac{ \phi'(z) }{\phi(z)} \dmeasure z
=
\sum r_{1} f(a_{1})
-
\sum s_{1} f(b_{1}),
$$
the summations being extended over all the zeros and poles of
$\phi(z)$.

\Subsection{6}{3}{1}{The number of roots of an equation contained
  within a contour.\index{Roots of an equation, number of!(inside a contour)}}
The result of the preceding paragraph can be at once applied to find
how many roots of an equation $\phi(z) = 0$ lie within a contour $C$.

For, on putting $f(z) = 1$ in the preceding result, we obtain the
result that
$$
\frac{1}{2\pi i} 
\int_{C} \frac{ \phi'(z) }{\phi(z)} \dmeasure z
$$
is equal to the excess of the number of zeros over the number of poles
of $\phi(z)$ contained in the interior of $C$, each pole and zero
being reckoned according to its degree of multiplicity.

%
% 120
%
\begin{wandwexample}
Shew that a polynomial $\phi(z)$ of degree $m$ has $m$ roots.

Let
$$
\phi(z)
=
a_{0} z^{m} + a_{1} z^{m-1} + \cdots + a_{m},
\quad
(a_{0} \neq 0).
$$
Then
$$
\frac{ \phi'(z) }{ \phi(z) }
=
\frac{ m a_{m}z^{m-1} + \cdots + a_{m-1} }{ a_{0}z^{m} + \cdots a_{m}}.
$$
Consequently, for large values of $\absval{z}$,
$$
\frac{ \phi'(z) }{ \phi(z) }
=
\frac{m}{z} + O\theparen{ \frac{1}{z^{2}} }
$$
Thus, if $C$ be a circle of radius $\rho$ whose centre is at the origin, we
have
$$
\frac{1}{2 \pi i} \int_{C} \frac{\phi'(z)}{\phi(z)} \dmeasure z
=
\frac{m}{2 \pi i} \int_{C} \frac{ \dmeasure z }{z}
+
\frac{1}{2 \pi i} \int_{C} O\theparen{ \frac{1}{z^{2}} } \dmeasure z
=
m + \frac{1}{2 \pi i} \int_{C} O\theparen{ \frac{1}{z^{2}} } \dmeasure z.
$$

But, as in \hardsectionref{6}{2}{2},
$$
\int_{C} O\theparen{ \frac{1}{z^{2}} } \dmeasure z
\rightarrow
0
$$
as $\rho\rightarrow\infty$; and hence as $\phi(z)$ has no poles in the
interior of $C$, the total number of zeros of $\phi(z)$ is
$$
\lim_{\rho\rightarrow\infty}
\frac{1}{2\pi i}
\int_{C} \frac{ \phi'(z) }{\phi(z)} \dmeasure z
=
m.
$$
\end{wandwexample}
If at all points of a contour $C$ the inequality
$$
\absval{ a_{k} z^{k} }
>
\absval{
  a_{0}
  + a_{1} z
  + \cdots
  + a_{k-1} z^{k-1}
  + a_{k+1} z^{k+1}
  + \cdots
  + a_{m} z^{m}
}
$$
is satisfied, then the contour contains $k$ roots of the equation
$$
a_{m} z^{m} + a_{m-1} z^{m-1} + \cdots a_{1} z + a_{0} = 0.
$$

For write
$$
f(z)
=
a_{m} z^{m} + a_{m-1} z^{m-1} + \cdots a_{1} z + a_{0}.
$$

Then
\begin{align*}
  f(z)
  &=
  a_{k} z^{k}
  \theparen{ 1 +
    \frac{ a_{m}z^{m}
      + \cdots
      + a_{k+1}z^{k+1}
      + a_{k-1}z^{k-1}
      + \cdots
      + a_{0}
    }{ a_{k} z^{k} }
  }
  \\
  &= a_{k} z^{k} (1 + U),
\end{align*}
where $\absval{U} \leq a \leq 1$ on the contour, $a$ being
independent\footnote{$\absval{U}$ is a continuous function of $z$ on
  $C$, and so \emph{attains} its upper bound
  (\hardsubsectionref{3}{6}{2}). Hence its upper bound $a$ must be
  less than $1$.} of $z$.

Therefore the number of roots of $f(z)$ contained in $C$
$$
= \frac{1}{2 \pi i} \int_{C} \frac{f'(z)}{f(z)} \dmeasure z
= \frac{1}{2 \pi i}
\int_{C} \theparen{ \frac{k}{z}
  + \frac{1}{1+U} \frac{ \dd U }{ \dd z }
} \dmeasure z
$$

But $\int_{C} \frac{\dd z}{z} = 2 \pi i$; and, since $\absval{U} < 1$,
we can expand $(1 + U)^{-1}$ in the uniformly convergent series
$$
1 - U + U^{2} - U^{3} + \cdots,
$$
so
$$
\int_{C} \frac{1}{1+U} \frac{\dd U}{\dd z} \dmeasure z
=
\thebracket{
  [U - \half U^{2} + \frac{1}{3} U^{3} - \cdots]
}_{C}
=
0.
$$

Therefore the number of roots contained in $C$ is equal to $k$.
\begin{wandwexample}
Find how many roots of the equation
$$
z^{6} + 6z + 10 = 0
$$
lie in each quadrant of the Argand diagram.
\addexamplecitation{Clare, 1900.}
\end{wandwexample}
%
% 121
%

\Section{6}{4}{Connexion between the zeros of a function and the zeros of its
derivative.}
Macdonald\footnote{TODO} has shewn that \emph{if $f(z)$ be a function of $z$
analytic throughout the interior of a single closed contour $C$, defined
by the equation $\absval{f(z)} = M$, where $M$ is a constant, then the
number of zeros of $f(z)$ in this region exceeds the number of zeros of
the derived function $f'(z)$ in the same region by unity.}

On C let $f(z) = M e^{i\theta}$; then at points on $C$
$$
f'(z) = M e^{i\theta} i \frac{\dd \theta}{\dd z},
\quad
f''(z)
=
M e^{i\theta}
\thebrace{
  i \frac{ \dd^{2} \theta }{ \dd z^{2} }
  -
  \theparen{ \frac{\dd \theta}{\dd z} }^{2}
}.
$$
Hence, by \hardsectionref{6}{3}{1}, the excess of the number of zeros
of $f(z)$ over the number of zeros of $f'(z)$
inside\footnote{$f'(z)$ does not vanish on $C$ unless $C$ has a node
  or other singular point; for, if $f = \phi + i\psi$,
  where $\phi$ and $\psi$ are real, since
  $i \frac{\partial f}{\partial x} = \frac{\partial f}{\partial y}$,
  it follows that if $f'(z) = 0$ at any point, then
  $\frac{\partial \phi}{\partial x},
  \frac{\partial \phi}{\partial y},
  \frac{\partial \psi}{\partial x},
  \frac{\partial \psi}{\partial y}$
  all vanish; and these are sufficient conditions for a singular point
  on  $\phi^{2} + \psi^{2} = M^{2}$.
} $C$ is
$$
\frac{1}{2\pi i} 
\int_{C} \frac{ f'(z) }{f(z)} \dmeasure z
-
\frac{1}{2\pi i}
\int_{C} \frac{ f''(z) }{ f'(z) } \dmeasure z
=
-\frac{1}{2 \pi i}
\int_{C} \theparen{
  \left.
    \frac{ \dd^{2} \theta }{ \dd z^{2} }
  \right/
  \frac{\dd\theta}{\dd z}
  } \dmeasure z.
$$
Let $s$ be the arc of $C$ measured from a fixed point and let $\psi$
be the angle the tangent to $C$ makes with $Ox$; then
\begin{align*}
-\frac{1}{2 \pi i}
\int_{C} \theparen{
  \left.
    \frac{ \dd^{2} \theta }{ \dd z^{2} }
  \right/
  \frac{\dd\theta}{\dd z}
  } \dmeasure z
  &=
  - \frac{1}{2 \pi i}
  \thebracket{ \log \frac{\dd\theta}{\dd z} }_{C}
  \\
  &=
  - \frac{1}{2 \pi i}
  \thebracket{
    \log \frac{\dd\theta}{\dd s}
    -
    \log \frac{\dd z}{\dd s}
  }_{C}
  .
\end{align*}
Now $\log \frac{\dd \theta}{\dd s}$ is purely real and its initial
value is the same as its final value; and
$\log \frac{\dd z}{\dd s} = i \psi$; hence the excess of the number of
zeros of $f(z)$ over the number of zeros of
$f'(z)$ is the change in $\psi/2\pi$ in describing the curve $C$; and it is
obvious\footnote{For a formal proof, see TODO} that if $C$ is any
ordinary curve, $\psi$ increases by $2\pi$ as the point of contact of
the tangent describes the curve $C$; this gives the required result.
\begin{wandwexample}
  Deduce from Macdonald's result the theorem that a
  polynomial of degree $n$ has $n$ zeros.
\end{wandwexample}
\begin{wandwexample}
  Deduce from Macdonald's result that if a function $f(z)$,
  analytic for real values of $z$, has all its coefficients real, and all
  its zeros real and different, then between two consecutive zeros of
  $f(z)$ there is one zero and one only of $f'(z)$.
\end{wandwexample}
REFERENCES.

M. C. Jordan, Cours dJ Analyse, 11. (Paris, 1894), Ch. vi.

E. Goursat, Cours d' Analyse (Paris, 1911), Ch. xiv.

E. LiNDELOF, Le Calcid des Residus (Paris, 1905), Ch. il.

%
% 122
%

Miscellaneous Examples.
\begin{enumerate}
\item  A function $\phi(z)$ is zero when $z=0$, and is real when $z$
  is real, and is analytic when $\absval{z} \leq 1$; if $f(x,y)$ is
  the coefficient of $i$ in $\phi(x + iy)$, prove that if $-1 < x < 1$,
  $$
  \int_{0}^{2\pi}
  \frac{x \sin\theta }{ 1 - 2x\cos\theta + x^{2}}
  f(\cos\theta, \sin\theta)
  \dmeasure\theta
  =
  \pi \phi(x).
  $$
  \addexamplecitation{Trinity, 1898.}
\item
  By integrating $\frac{e^{\pm aiz}}{e^{2\pi z}-1}$ round a contour formed by the rectangle whose
corners are $0, R, R+i, i$ (the rectangle being indented at $0$ and
$i$) and making $R\rightarrow\infty$,
shew that
$$
\int_{0}^{\infty}
\frac{ \sin ax }{e^{2 \pi x} - 1}
\dmeasure x
=
\frac{1}{4}
\frac{ e^{a} + 1 }{ e^{a} - 1 }
-
\frac{1}{2a}.
$$
\addexamplecitation{Legendre.}

3. By integrating $\log (-z) Q(z)$ round the contour of
\hardsubsectionref{6}{2}{4}, where $Q(z)$ is a rational function such
that  $z Q(z) \rightarrow 0$ as $\absval{z} \rightarrow 0$ and as
$\absval{z} \rightarrow \infty$, shew that if $Q(z)$ has no poles 
on the positive part of the real axis, I Q (.r) dx is equal to minus
the sum of the

.' residues of log -z)Q (2) at the poles of Q (z); where the
imaginary part of log ( - 2) lies between ± tt.

4. Shew that, if a > 0, 6 > 0,

f dv

I e cos6xgii (a sin hx)- =-|7r(e -l).

5. Shew that

TODO

6. Shew that

\addexamplecitation{Cauchy.}

TODO

if < i) 02)  ni oi) a-i) o-m be real and a be positive and

TODO

\addexamplecitation{Stormer, Acta Math, xix.}

7. If a point 2 describes a circle C of centre a, and if /(2) be
analytic throughout

C and its interior except at a number of poles inside C, then the
point u=f z) will

describe a closed curve y in the w-plane. Shew that if to each element
of y be attributed

a mass proportional to the corresponding element of C, the centre of
gravity of y is the

flz) point r, where r is the sum of the residues of  - -!- at its
poles in the interior of C.

(Amigues, Noiiv. Ann. de Math. (.3), xii. (1893), pp. 142-148.)

8. Shew that

TODO

9. Shew that

TODO

%
% 123
%

10. If Fn z)= n n (1 - 2' P), shew that the series

m = l ]i=l

fiz)=- I "( '' ")

1=2 (2"??~"-l)%""

is an analytic function when z is not a root of any of the equations
2" = %"; and that the sum of the residues of f z) contained in the
ring-shaped space inchided between two circles whose centres are at
the origin, one having a small radius and the other having a radius
between n and n + 1, is equal to the number of prime numbers less than
n + l. (Laurent, A' oiiv. Aim. de Math. (3), xviii. (1899), pp.
234-241.)

11. If -4 and B represent on the Argand diagram two given roots (real
or imaginary) of the equation /(j) = of degree n, with real or
imaginary coefficients, shew that there is at least one root of the
equation/' z) = within a circle whose centre is the middle point

of J and whose radius is lAB cot - . (Grace, Proc. Camb. Phil. Soc.
xi.)

n

12. Shew that, if 0<i'<l,

= s - lim 2

[Consider /

g 2v-l)ziri 2

round a circle of radius + 5; and make /i- -x .]

s\ nirz z - x

\addexamplecitation{Kronecker, Journal fiir Math, cv.} 1 3. Shew that, if m > 0, then

TODO
Discuss the discontinuity of the integral at m = 0.

14. If A + B + C+ ...=0 and a, b, c, ... are positive, shew that

TODO

\addexamplecitation{Wolstenholme.}

/gX(*+ )   1

15. By considering I -j r dt taken round a rectangle indented at the
origin, shew

that, if k > 0,

TODO

and hence deduce, by using the contour of \hardsubsubsectionref{6}{2}{2}{2} example TODO, or its
reflexion in the real axis (according as a; or .v < 0), that

TODO

according as x>0, x = or x < 0. *

[This integral is known as Cauchy's discontinuous factw.'] 1 6. Shew
that, if < a < 2, 6 > 0, / > 0, then

TODO

%
% 124
%

17. Let TODO

TODO

By considering / - - -dz round a rectangle whose corners are ± JV+
)±i, where N is an integer, and making N-*- qo, shew that

By expanding these integrands in powers of e~ ', e-"" respectively
and integrating term-by-term, deduce from\hardsubsectionref{6}{2}{2} example 3 that

TODO

Hence, by putting t = l shew that

TODO

(This result is due to Poisson, Journal de VEcole foly technique, xii.
(cahier xix), (1823), p. 420; see also Jacobi, Journal fiir Math,
xxxvi. (1848), p. 109 [Oes. Werke, ii. (1882), p. 188].) 

18. Shew that, if i;>0,

TODO

(Poisson, Mem. de I'Acad. des Sci. vi. (1827), p. 592; Jacobi,
Journal fur Math. ill. (1828), pp. 403-404 [Ges. Werke, i. (1881), pp.
264-265]; and Landsberg, Journal fur Math. CXI. (1893), pp. 234-253;
see also\hardsubsectionref{21}{5}{1}.)

\end{enumerate}