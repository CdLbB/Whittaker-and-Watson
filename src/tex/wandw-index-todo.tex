% Index entries remaining to be added
\chapter{General Index: Unindexed}

[TVie numbers refer to the pages. References to theorems contained in a few of
the more important examples are given by numbers in italics]

Abel's discovery of elliptic functions, 429, 512; inequality, 16; integral equation, 211, 229, 230;
method of establishing addition theorems, 442, 496, 497, 530, 534; special form,, (2), of
the confluent hypergeometric function, 353; test for convergence, 17; theorem on continuity
of power series, 57; theorem on multiplication of convergent series, 58, 59

Abridged notation for products of Theta-f unctions, 468, 469; for quotients and reciprocals of
elHptic functions, 494, 498

Absolute convergence, 18, 28; Cauchy's test for, 21; D'Alembert's ratio test for, 22; De
 Morgan's test for, 23

Absolute value, see Modulus

Absolutely convergent double series, 28; infinite products, 32; series, 18, (fundamental
property of) 25, (multiplication of) 29

Addition formula for Bessel functions, 357, 380; for Gegenbauer's function, 335; for Legendre
polynomials, 326, 395; for Legendre functions, 328; for the Sigma-function, 451; for
Theta-functions, 467; for the Jacobian Zeta-function and for E[u), 518, 534; for the
third kind of elliptic integral, 523; for the Weierstrassian Zeta-function, 446

Addition formulae, distinguished from addition theorems, 519

Addition theorem for circular functions, 535; for the exponential function, 531; for Jacobian
elliptic functions, 494, 497, 530; for the Weierstrassian elliptic function, 440, 457; proofs
of, by Abel's method, 442, 496, 497, 530, 534

Affix, 9

Air in a sphere, vibrations of, 399

Amplitude, 9

Analytic continuation, 96, (not always possible) 98; and Borel's integi-al, 141; of the hyper-"
geometric function, 288. See also Asymptotic expansions

Analytic functions, 82-110 (Chapter v); defined, 83; derivates of, 89, (inequality satisfied by) 91;
clistinguislied from monogenic functions, 99; represented by integi-als, 92; Eiemann's
equations connected with, 84; values of, at points inside a contour, 88; uniformly convergent
series of, 91

Angle, analytical definition of, 589; and popular conception of an angle, 589, 590

Angle, modular, 492

Area represented by an integi-al, 61, 589

Argand diagram, 9

Argument, 9, 588; principal value of, 9, 588; continuity of, 588

Associated function of Borel, 141; of Eiemann, 183; of Legendre [P "' (z) and Q '" (z)], 323-326

Asymptotic expansions, 150-159 (Chapter viii); differentiation of, 153; integration of, 153;
multiplication of, 152; of Bessel functions, 368, 369, 371, 373, 374; of confluent hyper-
geometric functions, 342, 343; of Gamma-functions, 251, 276; of parabolic cylinder functions,
347, 348; uniqueness of, 153, 154

Asymptotic inequality for parabolic cylinder functions of large order, 354

Asymptotic solutions of Mathieu's equation, 425

Auto -functions, 226 *

Automorphic functions, 455

Axioms of arithmetic and geometry, 579

Barnes' contour integrals for the hypergeometric function, 286, 289; for the confluent hyper-
geometric function, 343-345

Barnes' G-f unction, 264, 278

Barnes' Lemma, 289

Basic numbers, 462

Bernoullian numbers, 125; polynomials, 126, 127

Bertrand's test for convergence of infinite integrals, 71

Bessel coefficients [$\besJ_{n}(z)$], 101; addition formulae for, 357; Bessel's integral for, 362;
differential equation satisfied by, 357; expansion of, as power series, 355; expansion of

?..S- 2

%
% 596
%
functions in series of (by Neumann), 374, 375, 3S4, (by Schlomilch), 377; expansion of
 t-z)~  in series of, 374. 375, 376; expressible as a confluent fomi of Legendre functions,
367; expressible as confluent hypergeometric functions, 358; inequality satisfied by, 879;
Neumann's function 0, (z) connected with, sec Neumann's function; order of, 356; recur-
rence formulae for, 359; special case of confluent hypergeometric functions, 358. See also
Bessel functions
Bessel functions, 355-385 (Chapter xvn), Jn z) defined, 358-360; addition formulae for, 380;
asymptotic expansion of, 368, 369, 371, 373, 374; expansion of, as an ascending series, 358,
371; expansion of functions in series of, 374, 375, 377, 5.S'i; first kind of, 359; Hankel's
integi'al for, 365; integi'al connecting Legendre functions with, 364, 401; integral properties
of, 380, 381, 384, 385; integi-als involving products of, 380, 383, 385; notations for, 356,

372, 373; order of, 356; products of, 379, 360, 383, 385, 428; recurrence formulae for, 359,

373, 374; relations between, 360, 371, 372; relation between Gegenbauer's function and,
378; Schliifli's form of Bessel's integral for, 362, 372; second kind of, Y (2) (Hankel), 370;
yC' (-) (Neumann), 372; 1',, (z) (Weber-Schliifli), 370; second kind of modified, K  z), 373;
solution of Laplace's equation by, 395; solution of the wave-motion equation by, 397;
tabulation of, 378; whose order is large, 368, 383; whose order is half an odd integer, 364;
with imaginary argument. I  z), K  z), 372, 373, 384; zeros of, 361, 367, 378,381. See
aho Bessel coefficients (ind Bessel's equation

Bessel's equation, 204, 357, 373; fundamental system of solutions of (when )i is not an integer),
359. 372; second solution when n is an integer, 370, 373. See also Bessel functions

Binefs integrals for log V (z), 248-251

B6clier's theorem on linear differential equations with five singularities, 203

Bolzano's theorem on limit points, 12

Bonnet's foi-m of the second mean value theorem. 66

Borel's associated function, 141; integral, 140; integral and analytic continuation, 141; method
of ' summing ' series, 154; theorem (the modified Heine-Borel theorem), 58

Boundary, 44
 ' Boundary conditions, 387; and Laplace's equation, 393

Bounds of continuous functions, 55

Branch of a function, 106

Branch -point. 106

Biirmanns theorem, 128; extended by Teixeira, 131

Cantor's Lemma, 183

Cauchy's condition for the existence of a limit, 13; discontinuous factor, 123; formula for the

remainder in Taylor's series, 96; inequality for derivatives of an analytic function, 91;

integral, 119; integi'al representing V [z), 243; numbers, 372; tests for convergence of series

and integrals, 21, 71
Cauchy's theorem, 85; extension to curves on a cone, 87; Morera's converse of, 87, 110
CeU, 430

Cesaro's method of ' summing ' series, 155; generalised, 156

Change of order of terms in a series, 25; in an infinite deteiTninant, 37; in an infinite product, 33
Change of parameter (method of solution of Mathieu's equation), 424
Characteristic functions, 226; numbers, 219; numbers associated with symmetric nuclei are

real. 226
Chartier's test for convergence of infinite integrals, 72
Circle, iirea of sector of, 589; limiting, 98; of convergence, 30
Circular functions, 435, 584; addition theorems for, 585; continuity of, 585; differentiation

of, 585; duplication fonnulae, 585; periodicity of, 587; relation with Gamma-functions,

239
Circular membrane, vibrations of, 356, 396
Class, left (L), 4; right  li), 4
Closed. 44
Cluster point, 13
CoefiBcients, equating, 59; in Fourier series, nature of, 167, 174; in trigonometrical series, values

ot, 163, 165
Coefficients of Bessel. sec Bessel coefficients

Comparison theorem for convergence of integrals, 71; for convergence of series, 20
Complementary moduli, 479, 493; elliptic integrals with, 479, 501, 520

Complete elliptic integrals [K, K, K', 7v"] (first and second kinds), 498, 499, 518; Legendre's re-
lation lietween, 520; properties of  qua functions of the modulus), 484, 498, 499, 501, 521;

 A

 %
 % 597
 %

series for, 299; tables of, 518; the Gaussian transformation, 533; values for small values
of \ k\, 521; values (as Gamma-functions) for special values of k, 524-527; with comple-
mentary moduli, 479, 501, 520
Complex integrals, 77; upper limit to value of, 78
Complex integration, fundamental theorem of, 78

Complex numbers, 3-10 (Chapter i), defined, 6; amplitude of, 9; argument of, 9, 588; dependence
of one on another, 41; imaginary part of (I), 9; logarithm of, 589; modulus of, 8; real part
oi\ R), 9; representative point of, 9
Complex variable, continuous function of a, 44

Computation of elliptic functions, 485; of solutions of integi'al equations, 211
Conditional convergence of series, 18; of infinite determinants, 415. See also Convergence and

Absolute convergence
Condition of integrability (Eiemann's), 63
Conditions, Dirichlet's, 161, 163, 164, 176
Conduction of Heat, equation of, 387
Confluence, 202, 337
Confluent form, 203, 337

Confluent hypergeometric function  Wj  to ( )]' 337-354 (Chapter xv); equation for, 337; general
asymptotic expansion of, 342, 345; integral defining, 339; integrals of Barnes' type for,
343-345; Kummer's formulae for, 338; recurrence formulae for, 352; relations with Bessel
functions, 360; the functions W    (z) and xl/j.\,  (z), 337-339; the relations between functions
of these types, 346; various functions expressed in terms of JFj,,,; (z), 340, 352, 353, 360. See
also Bessel functions mid Parabolic cylinder functions
Confocal coordinates, 405, 547; form a triply orthogonal system, 548; in association with
ellipsoidal harmonics, 552; Laplace's equation referred to, 551; uniformising variables
associated with, 549
Congruence of points in the Argand diagram, 430
Constant, Euler's or Mascheroni's, [7], 235, 246, 248
Constants ci, e., e, 443; E, E', 518, 520; of Fourier, 164; 771, r/.,, 446, (relation between -qi

and 7?,,) 446"; G, 469, 472; A', 484, 498, 499; K', 484, .501, 503 "
Construction of elliptic functions, 438, 478, 492; of Mathieu functions, 409, (second method)

420
'Contiguous hypergeometric functions, 294
Continua, 43
Continuants, 36

Continuation, analytic, 96, (not always possible) 98; and Borel's integral, 141; of the hyper-
geometric function, 288. See also Asymptotic expansions
Continuity, 41; of power series, 57, (Abel's theorem) 57; of the argument of a complex variable,
588; of the circular functions, 585; of the exponential function, 581; of the logarithmic
function, 583, 589; uniformity of, 54
Continuous functions, 41-60 (Chapter in), defined, 41; bounds of, 55; integrability of, 63; of a

complex variable, 44; of two variables, 67
Contour, 85; roots of an equation in the interior of a, 119, 123
Contour integrals, 85; evaluation of definite integrals by, 112-124; the Mellin-Barnes type of,

286, 343; see also under the special function represented by the integral
Convergence, 11-40 (Chapter 11), defined, 13, 15; circle of, 30; conditional, 18; of a double
series, 27; of an infinite determinant, 36; of an infinite product, 32; of an infinite integral,
70, (tests for) 71, 72; of a series 15, (Abel's test for) 17, (Dirichlet's test for) 17; of Fourier
series, 174-179; of the geometric series, 19; of the hypergeometric series, 24; of the series
2n~*', 19; of the series occurring in Mathieu functions, 422; of trigonometrical series, 161;
principle of, 13; radius of, 30; theorem on (Hardy's), 156. See also Absolute convergence.
Non-uniform convergence and Uniformity of convergence
Coordinates, confocal, 405, 547; orthogonal, 401, 548
Cosecant, series for, 135
Cosine, see Circular functions

Cosine- integral [Ci (z)], 352; -series (Fourier series), 165
Cotangents, expansion of a function in series of, 139
Cubic function, integration problem connected with, 452, 512
Cunningham's function [w,  j (z)], 353
Curve, simple, 43; on a cone, extension of Cauchy's theorem to, 87; on a sphere (Seiffert's

spiral), 527
Cut, 281
Cylindrical functions, 355. See Bessel functions

%
% 598
%
D'Alemberfs ratio test for convergence of series, 22

Darboux" formula. 125

Decreasing- sequence, 12

Dedekind's theory of irrational numbers, 4

Deficiency of a plane curve, 455

Definite integrals, evaluation of, 111-124 (Chapter vi)

Degree of Legendre functions, 302, 307, 324

De la Vallee Poussins test for uniformity of convergence of an infinite integral, 72

De Morgan's test for convergence of series, 23

Dependence of one complex number on another, 41

Derangement of convergent series, 25; of double series, 28; of infinite determinants, 37; of

intinite products, 33, 34
Derivates of an analytic function, fi9; Cauchy's inequality for, 91; integrals for, 89
Derivates of elliptic functions, 4 0
Determinant. Hadamard's, 212
Determinants, infinite, 36; convergence of, 36, (conditional) 415; discussed by Hill, 36, 415;

evaluated by Hill in a particular case, 415; rearrangement of, 37
Difiference equation satisfied by the Gamma-function, 237
Diflferential equations satisfied by elliptic functions and quotients of Theta-f unctions, 436, 477,

492; (partial) satisfied by Theta-functions, 470; Weierstrass' theorem on Gamma-functions

and, 236. See also Linear diflferential equations and Partial diflferential equations
DiflFerentiation of an asymptotic expansion, 153; of a Fourier series, 168; of an infinite

integral, 74; of an integral, 67; of a series, 79, 91; of elliptic functions, 480, 493; of the

circular functions, 585; of the exponential function, 582; of the logarithmic function, 583,

589
Dirichlefs conditions, 161, 163, 164, 176; form of Fourier's theorem, 161, 163, 176; formula

connecting repeated integrals, 75, 76, 77; integral, 252; integral for f (z), 247; integral for

Legendre functions, 314; test for convergence, 17
Discontinuities, 42; and non-uniform convergence, 47; of Fourier series, 167, 169; ordinary, 42;

regular distribution of, 212; removable, 42
Discontinuous factor, Cauchy's, 123

Discriminant associated with Weierstrassian elliptic functions, 444, 550
Divergence of a series, 15; of infinite products, 33
Domain, 44

Double circuit integrals, 256, 293
Double integrals, 68, 254
Double series, 26; absolute convergence of, 28; convergence of (Stolz' condition), 27; methods

of summing, 27; a particular form of, 51; rearrangement of, 28
Doubly periodic functions, 429-535. See alt > Jacobian elliptic functions, Theta-functions and

Weierstrassian elliptic functions
Duplication formula for the circular functions, 585; for the Gamma-function, 240; for the

Jacobian ellij)tic functions, 498; for the Sigma-function, 459, 460; for the Theta-functions,

48S; for the Weierstrassian elliptic function, 441; for the Weierstrassian Zeta-function, 459

Electromagnetic waves, equations for, 404

Elementary functions, S2

Elementary transcendental functions, 579-590 (Appendix). Sec (dm Circular functions.
Exponential function and Logarithm

Ellipsoidal harmonics, 536-578 (Chapter xxiii); associated with confocal coordinates, 552
derived from Lame's equation, 538-543, 552-554; external, 576; integi'al equations con
nected with, 567; linear independence of, 560; number of, when the degi-ee is given, 546.
physical applications of, 547; .species of, 537; types of, 537. See also Lamp's equation
and Lame functions

Elliptic cylinder functions, see Mathieu functions

Elliptic functions, 429-535 (Chapters xx-xxii); computation of, 485; construction of, 433, 478
derivate of, 480; discovery of, by Abel, Gauss and Jacobi, 429, 512, 524; expressed by
means of Theta-functions, 473; expressed by means of Weierstrassian functions, 448-451
general addition formula, 457; number of zeros (or poles) in a cell, 431, 432; order of
432; periodicity of, 429, 479, 500, 502, 503; period parallelogram of, 430; relation be
tween zeros and poles of, 433; residues of, 431, 504; transformations of, 508; with no
poles (are constant), 431; with one double pole, 432, 434; with the same periods (relations
between), 452; with two simple poles, 432, 491. See also Jacobian elliptic functions,
Theta-functions and Weierstrassian elliptic functions

%
% 599
%

Elliptic integrals, 429, 512; first kind of, 515; function E (it) and, 517; function Z ( ) and,
518; inversion of, 429, 452, 454, 480, 484, 512, 524; second kind of, 517, (addition formulae
for) 518, 519, 534, (imaginary transformation of) 519; third kind of, 522, 523, (dynamical
application of) 523, (parameter of) 522; three kinds of, 514. See also Complete elliptic
integrals

Elliptic membrane, vibrations of, 404

Equating coefiQcients, 59, 186

Equation of degree m has m roots, 120

Equations, indicial, 198; number of roots inside a contour, 119, 123; of Mathematical Physics,
203, 386-403; with periodic coefficients, 412. See also Difference equation, Integral
equations, Linear differential equations, and under the names of special equations

Equivalence of curvilinear integrals, 83

Error- function [Erf (,() and Erfc (,r)], 341

Essential singularity, 102; at infinity, 104

Eta-function [H ( )], 479, 480

Eulerian integrals, first kind of [B (m, n)], 253; expressed by Gamma-functions, 254; extended
by Pochhammer, 256

Eulerian integrals, second kind of, 241; .see Gamma-function

Euler's constant [7], 235, 246, 24S; expansion (Maclaurin's), 127; method of 'summing' series,
155; product for the Gamma-function, 237; product for the Zeta-function of Riemann, 271

Evaluation of definite integrals and of infinite integrals, 111-124 (Chapter vi)

Evaluation of Hill's infinite determinant, 415

Even functions, 165; of Mathieu [cc,  z, q)], 407

Existence of derivatives of analytic function, 89; -theorems, 888

Expansions of functions, 125-149 (Chapter vii); by Biirmann, 128, 131; by Darboux, 125; by
Euler and Maclaurin, 127; by Fourier, see Fourier Series; by Fourier (the Fourier-Bessel
expansion), 381; by Lagrange, 132, 149; by Laurent, 100; by Maclaurin, 94; by Pincherle,
149; by Plana, 145; by Taylor, 93; by Wronski, 147; in infinite products, 136; in series of
Bessel coefficients or Bessel functions, 374, 375, 381, 384; in series of cotangents, 139; in
series of inverse factorials, 142; in series of Legendre polynomials or Legendre functions,
310, 322, 330, 331, 335; in series of Neumann functions, 374, 375, 384; in series of parabolic
cylinder functions, 351; in series of rational functions, 134. See also Asymptotic expansions.
Series, and under the itaines of special functions

Exponential function, 581; addition theorem for, 581; continuity of, 581; differentiation of,
582; periodicity of, 585

Exponential-integrral [Ei (z)], 352

Exponents at a regular point of a linear differential equation, 198

Exterior, 44

External harmonics, (ellipsoidal) 576, (spheroidal) 403

Factor, Cauchy's discontinuous, 123; periodicity-, 463

Factorials, expansion in a series of inverse, 142

Factor-theorem of Weierstrass, 137

Fej r's theorem on the summability of Fourier series, 169, 178

Ferrers' associated Legendre functions [P '" (z) and Q "  (z)], 323

First kind, Bessel functions of, 359; elliptic integi-als of, 515, (complete) 518, (integration of)
515; Eulerian integral of, 253, (expressed by Gamma-functions) 254; integral equation of,
221; Legendre functions of, 307

First mean-value theorem, 65, 96

First species of ellipsoidal harmonic, 537, (construction of) 538

Floquet's solution of differential equations with periodic coefficients, 412

Fluctuation, 56; total, 57

Foundations of arithmetic and geometry, 579

Fourier-Bessel expansion, 381; integral, 385

Fourier constants, 164

Fourier series, 160-193 (Chapter ix); coefficients in, 167, 174; convergence of, 174-179; differ-
entiation of, 168; discontinuities of, 167, 169; distinction between any trigonometrical
series and, 160, 163; expansions of a function in, 163, 165, 175, 176; expansions of Jacobian
elliptic functions in, 510, 511; expansion of Mathieu functions in, 409, 411, 414, 420; Fejer's
theorem on, 169; Hurwitz-Liapounotf theorem on, 180; Parseval's theorem on, 182; series
of sines and series of cosines, 165; summability of, 169, 178; uniformity of convergence of,
168, 179. See also Trigonometrical series

Fourier's theorem, Dirichlet's statement of, 161, 163, 176

%
% 600
%
Fourier's theorem on integrals. 188, 211

Fourtb species of ellipsoidal harmonic, 537, (construction of) 542

Fredliolm's integral equation, 213-217, 228

Functionality, concept of, 41

Functions, branches of, 106; identity of two, 98; limits of, 42; principal parts of, 102; without
essential singularities, 105; which cannot be continued, 98. See also inidcr the mnnes of
special functions! or special types offitncti(yns, e.g. Legendre functions. Analytic functions

Fundamental formulae of Jacobi connecting Theta-functions, 467, 4S8

Fundamental period parallelogram, 430; polygon (of automorphic functions), 455

Fundamental system of solutions of a linear differential equation, 197, 200, 389, 559. .S (r also
iDulcr the names of special equations

Gamma-function [r( )], 235-264 (Chapter xii); asymptotic expansion of, 251, 276; circular
functions and, 239; complete elliptic integrals and, 524-527, 535; contour integi-al (Hankel's)
for, 244; difference equation satisfied by. 237; differential equations and, 236; duplication
formula, 240; Euler's integral of the first kind and, 254; Euler's integi-al of the second
kind, 241. (modified by Cauchy and Saalschiitz) 243, (modified by Hankel) 244; Euler's
product, 237; incomplete form of, 341; integrals for, (Binet's) 248-251, (Euler's) 241;
minimum value of, 253; multiplication formula, 240; series, (Rummer's) : 50, (Stirling's)
251; tabulation of. 253; trigonometrical integrals and, 256; Weierstrassian product, 235,
236. See also Eulerian integrals and Logaritlimic derivate of the Gamma-function

Gauss' discovery of elliptic functions, 429, 512, 524; integral for r'(~)/r(r), 246; lemniscate
functions, see Lemniscate functions; transformation of elliptic integrals, 533

Gegenbauer's function [C,/ (z)], 329; addition formula, 335; differential equation for, 329;
recuiTence formulae, 330; relation with Legendre functions, 329; relation involving Bessel
functions and, 56-5; Rodrigues' formula (analogue), 329; Schliifii's integral (analogue), 329

Genus of a plane curve, 455

Geometric series, 19

Glaisher's notation for quotients and reciprocals of elliptic functions, 494, 498

Greatest of the limits, 13

Green's functions, 395

Hadamard's lemma, 212

Half-periods of Weierstrassian elliptic functions, 444

Hankel's Bessel function of the second kind, Y,j( ), 370; contour integral for T (z), 244; integral

for J ( ), 365
Hardy's convergence theorem, 156; test for uniform convergence, 50
Harmonics, solid and surface, 392; spheroidal, 403; tesseral, 392, 536; zonal, 302, 392, 536;

Sylvester's theorem concerning integi-als of, 400. See also Ellipsoidal harmonics
Heat, equation of conduction of, 387
Heine-Borel theorem (modified), 53

Heine's expansion of (t - z)-  in series of Legendre polynomials, 321
Hermite's equation, 204, 209, 342, 347. See also Parabolic cylinder functions
Hermite's formula for the generalised Zeta-function f (s,  ), 269
Hermite's solution of Lame's equation, 573-575
Heun's equation, 576, 577

Hill's equation, 406. 413-417; Hill's method of solution, 413
Hill's infinite determinant, 36, 40, 415; evaluation of, 415
Hobson's associated Legendre functions, 325
Holomorphic, 83

Homogeneity of Weierstrassian elliptic functions, 439
Homogeneous harmonics (associated with ellipsoid), 543, 57G\ ellipsoidal harmonics derived

from (Nivcn's formula), 543; linear independence of, 560
Homogeneous integral equations, 217, 219
Hurwitz' definition of the generalised Zeta-function j'(x,  ), 265; formula for f (.<,(/), 268;

theorem concerning Foiu'ier constants, 180
Hypergeometric equation, see Hypergeometric functions
Hypergeometric functions, 281-301 (Chapter xiv); Barnes' integrals, 286, 289; contiguous, 294;

continuation of, 2H,S; contour integrals for, 291; differential equation for, 202, 207, 283;

functions expressed in terms of, 281, 311; of two variables (Appell's), 300; relations between

twenty-four expressions involving, 284, 285, 290; liiemann's P-equation and, 208, 283;

series for (convergence of), 24, 281 squares and products of,  96'; value of F(a, l>; c; 1),

%
% 601
%

281, 393 \ values of special forms of bypergeometric functions, 298, 301. See also Bessel
functions, Confluent hypergeometric functions and Legendre functions

Hypergeometric series, >iee Hypergeometric functions

Hypothesis of Riemann on zeros of f(.s), 272, 280

Identically vanisMng power series, 58

Identity of two functions, 98

Imaginary argument, Bessel functions with [I,j(2) and K  z)'], 372, 373, 384

Imaginary part (/) of a complex number, 9

Imaginary transformation (Jacobi's) of elliptic functions, 505, 506, 5-3.5; of Theta-functions, 124,
474; of E[u) and Z(h), 519

Improper integrals, 75

Incomplete Gamma-functions \ \  y n, c)], 341

Increasing sequence, 12

Indicial equation, 198

Inequality (Abel's), 16; (Hadamard's), 212; satisfied by Bessel coefficients, 379; satisfied by
Legendre polynomials, 303; satisfied by parabolic cylinder functions, 354; satisfied by
j-(.s', a), 274, 275

Infinite determinants, see Determinants

Infinite integrals, 69; convergence of, 70, 71, 72; differentiation of, 74; evaluation of, 111-124;
functions represented by, sec under the names of special functions; representing analytic
functions, 92; theorems concerning, 73; uniform convergence of, 70, 72, 73. See also
Integrals and Integration

Infinite products, 32; absolute convergence of, 32; convergence of, 32; divergence to zero, 33;
expansions of functions as, 136, 137  see also under the names of special functions); expressed
by means of Theta-functions, 473, 488; uniform convergence of, 49

Infinite series, see Series

Infinity, 11, 103; essential singularity at, 104; point at, 103; pole at, 104; zero at, 104

Integers, positive, 3; signless, 3

Integrability of continuous functions, 63; Eiemann's condition of, 63

Integral, Borel's, 140; and analytic continuation, 141

Integral, Cauchy's, 119

Integral, Dirichlet's, 258

Integral equations, 211-231 (Chapter xi); Abel's, 211, 229, 330; Fredholm's, 213-217, 228;
homogeneous, 217, 219; kernel of, 213; Liouville-Neumann method of solution of, 221;
nucleus of, 213; numbers (characteristic) associated with, 219; numerical solutions of, 211;
of the first and second kinds, 213, 221; satisfied by Lame functions, 564-567; satisfied by
Mathieu functions, 407; satisfied by parabolic cylinder functions, 331; Schlomilch's, 229;
solutions in series, 228; Volterra's, 221; with variable upper limit, 213, 221

Integral formulae for ellipsoidal harmonics, 567; for the Jacobian elliptic functions, 492, 494;
for the Weierstrassian elliptic function, 437

Integral functions, 106; and Lame's equation, 571; and Mathieu's equation, 418

Integral properties of Bessel functions, 380, 381, 385; of Legendre functions, 325, 305, 324; of
Mathieu functions, 411; of Neumann's function, 385; of parabolic cylinder functions, 350

Integrals, 61-81 (Chapter iv); along curves (equivalence of), 87; complex, 77, 78; differentiation
of, 67; double, 68, 255; double-circuit, 256, 293; evaluation of, 111-124; for derivates of an
analytic function, 89; functions represented by, see under the juinies of the special functions;
imjjroper, 75; lower, 61; of harmonics (Sylvester's theorem), 400; of irrational functions,
452, 512; principal values of, 75; regular, 201; repeated,
68, 75; representing analytic functions, 92; representing areas, 61, 589; round a contour,
85; upper, 61. See also Elliptic integrals. Infinite integrals, and Integration

Integral theorem, Fourier's, 188, 211; of Fourier-Bessel, 385

Integration, 61; complex, 77; contour-, 77; general theorem on, 63; general theorem on
complex, 78; of asymptotic expansions, 153; of integi'als, 68, 74, 75; of series, 78; pro-
blem connected with cubics or quartics and elliptic functions, 452, 512. See also Infinite
integrals and Integrals

Interior, 44

Internal spheroidal harmonics, 403

Invariants of Weierstrassian elliptic functions, 437

Inverse factorials, expansions in series of, 142

Inversion of elliptic integrals, 429, 452, 454, 480, 484, 512, 524

Irrational functions, integration of, 452, 512

Irrational-real numbers, 5

%
% 602
%

Irreducible set of zeros or poles, 430

Irregular points i singularities) of differential equations, 197, 202

Iterated functions. 222

Jacobian elliptic functions [sn /  en ii, dn;(], 432, 478, 491-535 (Chapter xxii); addition theorems
for, 494, 497, 530, 535; connexion with Weierstrassian functions, 505; definitions of am,
A<p, sn M (sin am  ), en u, dn u, 478, 492, 494; differential equations satisfied by, 477, 492;
differentiation of, 493; duplication formulae for, 498; Fourier series for, 510, 511, 55.5;
geometrical illustration of, 524, 527; general description of, 504; Glaisher's notation for
quotients and reciprocals of, 494; infinite products for, 508, 53:; integral formulae for, 492,
494; Jacobi's imaginary transformation of, 505, 506; Lame functions expressed in terms of,
564, 573; Landen's transfonnation of, 507; modular angle of, 492; modulus of, 479, 492,
(complementary) 479, 493; parametric representation of points on curves by, 524, 537, 527,
555; periodicity of, 479, 500, 502, 503; poles of, 432, 503, 504; quarter periods. A', iK', of,
479, 498, 499, 501; relations between, 492; residues of, 504; Seiffert's spherical spiral and,
527; triplication formulae, 530. 534, 535; values of, when ii is hK, kiK' or h  K riK'), 500,
506. 5(17; values of. when the modulus is small, 533. See ahit Elliptic functions. Elliptic
integrals. Lemniscate functions, Tbeta- functions, and Weierstrassian elliptic functions

Jacobi"s discovery of elliptic functions, 429, 512; earlier notation for Theta-functions, 479;
fundamental Theta-function fomiulae, 467, 4H8; imaginary transformations, 124, 474, 505,
506, 519, 535; Zeta-function, t ee under Zeta-function of Jacobi

Kernel. 213

Klein's theorem on linear differential equations with five singularities, 203

Kummer's formulae for confluent hypergeometric functions, 338; series for logF (z), 250

Lacunary function. 98

Lagrange's expansion, 132, 149; form for the remainder in Taylor's series, 96

Lame functions, defined, 558; expressed as algebraic functions, 556; 577; expressed by Jacobian
elliptic functions, 573-575; expressed by Weierstrassian elliptic functions, 570-572; integi'al
equations satisfied by, 564-567; linear independence of, 559; reality and distinctness of
zeros of, 557, 558, 578; second kind of, 562; values of, 558; zeros of (Stieltjes' theorem),
560. See alsi) Lames equation and Ellipsoidal harmonics

Lame's equation, 204, 536-578 (Chapter xxiii); derived from theory of ellipsoidal harmonics,
538-543, 552-554; different forms of. 554, 573; generalised,' 204, 570, 573, 576, 577;
series solutions of, 556, 577, 578; solutions expressed in finite fomi, 459, 556, 576, 577, 578;
solutions of a generalised equation in finite foim, 570, 573. See alao Lam6 functions and
Ellipsoidal harmonics

Landens transformation of Jacobian elliptic functions, 476, 507, 533

Laplace's equation, 386; its general solution, 388; normal solutions of, 553; solutions involving
functions of Legendre and Bessel, 391, 395; solution with given boundary conditions, 393;
synnnetrical solution of, 399; transformations of, 401, 407, 551, 553

Laplace's integrals for Legendre polynomials and functions, 312, 313, 314, 319, 326, 337

Laurent's expansion, 100

Least of limits. 13

Lebesgue's lemma, 172

Left (L-) class. 4

Legendre's equation, 204, 304; for associated functions, 324; second solution of, 316. See aho
Legendre functions and Legendre polynomials

Legendre functions, 302-336 (Chapter xv); P  z), Q  z), P "Uz), Q '" z) defined, 306, 316, 323,
325; addition formulae for, 328, 395; Bessel functions and, 364, 367, 401; degree of, 307,
324; differential equation for, 204, 306, 324; distinguished from Legendre polynomials,
306; expansions in a.scending series, 311, 326; expansions in descending series, 302, 317,
326, 334; expansion of a function as a series of, 334; expressed by Murphy as hypergeometric
functions, 311, 312; expression of Qn z) in terms of Legendre polynomials. 319, 320, 333;
Ferrers' functions associated with, 323. 324; first kind of, 307; Gegenbauer's function,
Cn" (z), associated with, xee Gegenbauer's function; Heine's expansion of  t - e)~i as a series
of, 321; Hobson's functions associated with, 325; integral connecting Bessel functions with,
364; integi-al properties of, 324; Laplace's integials for, 312, 313, 319, 326, 334; Mehler-
Dirichlet integral for, 314; order of, 326; recuiTence fomiulae for, 307, 318; Schlafli's
integral for, 304, 306; second kind of, 316-320, 325, 320; summation of ::ii"P (:) and
Z/f" Q  (z), 302, 321; zeros of, 303, 316, 335. See also Legendre polynomials and Legendre's
equation

Legendre polynomials [P   z)], 95, 302; addition foraiula for, 326, 387; degi-ee of, 302; differ-
ential equation for, 204, 304; expansion in ascending series, 311; expansion in descending

%
% 603
%

series, 302, 334; expansion of a function as a series of, 310, 322, 330, 331, 332, 335;
expressed by Mui-phy as a hypergeometric function, 311, 312; Heine's expansion of (t - z)~'
as a series of, 321; integi-al connecting Bessel functions with, 364; integi'al properties of,
225, 305; Laplace's equation and, 391; Laplace's integrals for, 312, 314; Mehler-Dirichlet
integi-al for, 314; Neumann's expansion in series of, 322; numerical inequality satisfied by,
303; recurrence fonnulae for, 307, 309; Eodrigues' fonnula for, 225, 303; Schlafli's integral
for, 303, 304; summation of 2/;" P  (z), 302; zeros of, 303, 316. See aim Legendre functions

Legendre's relation between complete elliptic integi-als, 520

Lemniscate functions [sin lemn and cos lemn 0], 524

LiapounofF's theorem concerning Fourier constants, 180

Limit, condition for existence of, 13

Limit of a function, 42; of a sequence, 11, 12; -point (the Bolzano -Weierstrass theorem), 12

Limiting circle, 98

Limits, greatest of and least of, 13

Limit to the value of a complex integi'al, 78

Lindemann's theory of Mathieu's equation, 417; the similar theory of Lame's equation, 570

Linear differential equations, 194-210 (Chapter x), 386-403 (Chapter xviii); exponents of, 198;
fundamental system of solutions of, 197, 200; iixegular singularities of, 197, 202; ordinary
point of, 194; regular integi-al of, 201; regular point of, 197; singular points of, 194, 197,
(confluence of) 202; solution of, 194, 197, (uniqueness of) 196; special types of equations :
- Bessel's for circular cylinder functions, 204. 342, 357, 358, 373; Gauss' for hypergeo-
metric functions, 202, 207, 283; Gegenbauer's, 329; Hermite's, 204, 209, 342, 3 7; Hill's,
406, 413; .Tacobi's for Theta-functions, 463; Lame's, 204, 540-543, .554-558, 570-575;
Laplace's, 386, 388, 536, 551; Legendre's for zonal and surface harmonics, 204, 304, 324;
Mathieu's for elliptic cylinder functions, 204, 406; Neumann's, 3\&5; Riemann's for
P-functions, 206, 283, 291, 294; Stokes', 204; Weber's for parabolic cylinder functions,
204, 209, 342, 347; Whittaker's for confluent hypergeometric functions, 337; equation for
conduction of Heat, 387; equation of Telegraphy, 387; equation of wave motions, 386, 397,
402; equations with five singularities (the Klein-Bocher theorem), 203; equations with three
singularities, 206; equations with two singularities, 208; equations with r singularities,
209; equation of the third order with regular integrals, 210

Liouville's method of solving integral equations, 221

Liouville's theorem, 105, 431

Logarithm, 583; continuity of, 583, 589; differentiation of, 586, 589; expansion of, 584, 589;
of complex numbers, 589

Logarithmic derivate of the Gamma-function [i (z)], 240, 241; Binet's integrals for, 248-251;
circular functions and, 240; Dirichlet's integi-al for, 247; Gauss' integi-al for, 246

Logarithmic derivate of the Riemann Zeta-function, 279

Logarithmic-integral function [Liz], 341

Lower integral, 61

Lunar perigee and node, motions of, 406

Maclaurin's (and Euler's) expansion, 127; test for convergence of infinite integrals, 71; series,
94, (failure of) 104, 110

Many-valued functions, 106

Mascheroni's constant [7], 235, 246, 248

Mathematical Physics, equations of, 203, 386-403 (Chapter x\ an). See aho under Linear dif-
ferential equations and the names of special equations

Mathieu functions [oc (£, q), .sp (z, q), in,  z, q)], 404-428 (Chapter xix); construction of, 409,
420; convergence of series in, 422; even and odd, 407; expansions as Fourier series, 409,
411, 420; integral equations satisfied by, 407, 409; integral formulae, 411; order of, 410;
second kind of, 427

Mathieu's equation, 204, 404-428 (Chapter xix); general form, solutions by Floquet, 412, by
Lindemann and Stieltjes, 417, by the method of change of parameter, 424; second solution
of, 413, 420, 427; solutions in" asymptotic series, 425; solutions which are periodic, .tee
Mathieu functions; the integial function associated with, 418. See also Hill's equation

Mean-value theorems, 65, 66, 96

Mehler's integral for Legendre functions, 314

Mellin's (and Barnes') type of contour integi-al, 286, 343

Membranes, vibrations of, 356, 396, 404, 405

Mesh, 430

Methods of ' summing ' series, 154-156

Minding's formula, 119

Minimum value of T (.r), 253

%
% 604
%

Modified Heine-Borel theorem, -53

Modular aiiplo, 49'i; function, 481, (equation connected with) 482; -surf.ice, 41

Modulus, 430; of a complex number, 8; of Jacobian elliptic functions, 479, 492, (complementary)

479. 493; periods of elliptic functions regarded as functions of the, 484, 498, 499, 501, 521
Monogenic, 83; distinguished from analytic, 99
Monotonic, 57

Morera's theorem (converse of Cauchy's theorem), 87, 110
Motions (if lunar perigee and node, 406
M-test for uniformity of convergence, 49

Multiplication formula for T [z), 240; for the Sigma-function, 460
Multiplication of absolutely convergent series, 29; of asymptotic expansions, 152; of convergent

series (Abel's theorem), 58, 59
Multipliers of Thcta-functions, 463
Murphy's formulae for Legendre functions and polynomials, 311, 312

Neumann's definition of Bessel functions of the second kind, 372; expansions in series of

Legendre and Bessel functions, 322, 374; (F. E. Neumann's) integral for the Legendre

function of the second kind, 320; method of solving integral equations, 221
Neumann's function [0,j (z)], 374; differential equation satisfied by, 385; expansion of, 374;

expansion of functions in series of, 376, 384; integral for, 375; integral properties of,

56.5; recurrence formulae for, 375
Non-uniform convergence. 44; and discontinuity, 47
Normal functions, 224

Normal solutions of Laplace's equation, 553
Notations, for Bessel functions, 356, 372, 373; for Legendre functions, 325, 326; for quotients

and reciprocals of elliptic functions, 494, 498; for Theta-functions, 464, 479, 487
Nucleus of an integi-al equation, 213; symmetric, 223, 228
Numbers, 3-10 (Chapter i); basic, 462; Bernoulli's, 125; Cauchy's, 379; characteristic, 219,

(reality of) 226; complex, 6; irrational, 6; irrational-real, 5; pairs of, 6; rational, 3, 4;

rational-real, 5; real, 5

Odd functions, 166; of Mathieu, [.s(' (, 7)], 407

Open. 44

Order [O and o), 11; of Bernoullian polynomials, 126; of Bessel functions, 356; of elliptic
functions, 432; of Legendre functions, 324; of Mathieu functions, 410; of poles of a
function, 102; of terms in a series, 25; of the factors of a product, 33; of zeros of a
function, 94

Ordinary discontinuity, 42

Ordinary point of a linear differential equation, 194

Orthogonal coordinates, 394; functions, 224

Oscillation, 11

Parabolic cylinder functions [/>  (2:)], 347; contour integi-al for, 349; differential equation for,
204,;i()9, 347; expansion in a power series, 347; expansion of a function as a series of, 351;
general asymptotic expansion of, 348; inequalities satisfied by, 354; integral equation
satisfied by, 231; integral properties, 350; integrals involving, 353; integrals representing,
353; properties when n is an integer, 350, 353, 354; recurrence formulae, 350; relations
between different kinds of [D,  z) and D-n-ii - i )]  348; zeros of, 354. See also Weber's
equation

Parallelogram of periods, 430

Parameter, change of (method of solving Mathieu's equation), 424; connected with Theta-
functions, 463, 464; of a point on a curve, 442, 496, 497, 527, 530, 533; of members of
confocal systems of quadrics, 547; of third kind of elliptic integral, 522; thermometric, 405

Parse val's theorem, 182

Partial differential equations, property of, 390, 391. See (iIko Linear differential equations

Partition function. 462

Parts, real an<l imaginary, 9

Pearson's function [w,, (z)], 353

P-equation, Riemann's, 206, 337; connexion with the hypergeometric equation, 208, 283; solu-
tions of, 2S3, 291, (relations between) 294; transformations of, 207

Periodic coefficients, equations with (Floquet's theory of), 412

Periodic functions, integrals involving, 112, 256. See also Fourier series and Doubly periodic
functions

%
% 605
%
Periodicity factors, 463

Periodicity of circular and exponential functions, 585-587; of elliptic functions, 429, 434, 479,

500, 502, 503; of Theta-funetions, 463
Periodic solutions of Mathieu's equation, 407
Period-parallelogram, 430; fundamental, 430

Periods of elliptic functions, 429; qua functions of the modulus, 484, 498, 499, 501, 521 
Phase, 9

Pincherle's functions (modified Legendre functions), 335
Plana's expansion, 145

Pochhammer's extension of Eulerian integrals, 256

Point, at infinity, 103; limit-, 12; representative, 9; singular, 194, 202
Poles of a function, 102; at infinity, 104; irreducible set of, 430; number in a cell, 431; relations

between zeros of elliptic functions and, 433; residues at, 432, 504; simple, 102
Polygon, (fundamental) of automoi-phic functions, 455
Polynomials, expi'essed as series of Legendre polynomials, 310; of Abel, 333; of Bernoulli, 126,

127; of Legendre, xce Legendre polynomials; of Sonine, 352
Popular conception of an angle, 589; of continuity, 41
Positive integers, 3

Power series, 29; circle of convergence of, 30; continuity of, 57, (Abel's theorem) 57; expan-
sions of functions in, xee under the name  of special functiomi; identically vanishing, 58;
Maclaurin's expansion in, 94; radius of convergence of, 30, 32; series derived from, 31;
Taylor's expansion in, 93; uniformity of convergence of, 57

Principal part of a function, 102; solution of a certain equation,
482; value of an integral, 75;  value of the argument of a complex
number, 9, 588

Principle of convergence, 13

Pringsheim's theorem on summation of double series, 28
Products of Bessel functions, 379, 380, 383, 385, 428; of hypergeometric functions, 298. See

also Infinite products

Quarter periods K, iK', 479, 498, 499, 501. See also Elliptic integrals

Quartic, canonical form of, 513; integi'ation problem connected with, 452, 512

Quasi-periodicity, 445, 447, 463

Quotients of elliptic functions (Glai her's notation), 494, 511; of Theta-f unctions, 477

Radius of convergence of power series, 30, 32

Rational functions, 105; expansions in series of, 134

Rational numbers, 3, 4; -real numbers, 5

Real functions of real variables, 56

Reality of characteristic numbers, 226

Real numbers, rational and irrational, 5

Real part (li) of a complex number, 9

Rearrangement of convergent series, 25; of double series, 28; of infinite determinants, 37; of
infinite products, 33

Reciprocal functions, Volterra's, 218

Reciprocals of elliptic functions (Glaisher's notation), 494, 511

Recurrence formulae, for Bessel functions, 359, 373, 374; for confluent hypergeometric functions,
352; for Gegenbauer's function, 330; for Legendre functions, 307, 309, 318; for Neumann's
function, 375; for parabolic cylinder functions, 350. See also Contiguous hypergeometric
functions

Region, 44

Regvilar, 83; distribution of discontinuities, 212; integrals of linear differential equations, 201,
(of the third order) 210; points (singularities) of linear differential equations, 197

Relations between Bessel functions, 360, 371; between confluent hypergeometric functions
Tr\; . jjj (± ) and M/ .    z, 346; between contiguous hypergeometric functions, 294; be-
tween elliptic functions, 452; between parabolic cylinder functions D,j ( ± z) and D\  \ i ( ± iz),
348; between poles and zeros of elliptic functions, 433; between Riemann Zeta-f unctions
f (s) and f (1 - s), 269. See also Recurrence formulae

Remainder after *; terms of a series, 15; in Taylor's series, 95

Removable discontinuity, 42

Repeated integrals, 68, 75

Representative point, 9

Residues, 111-124 (Chapter vi); of elliptic functions,
425, 497

%
% 606
%
Riemann's associated function, 183, 184, 185; condition of integrability, 63; equations satisfied
by analytic functions, S4; hypothesis concerning f(j,), 272, 280; lemmas, 172, 184, 185;
/'-equation, 206, 283, 291, 294, (transformation of) 207, (and the hypergeometric equation)
208, nee (tho Hypergeometric functions; theory of trigonometiical series, 182-188; Zeta-
function, xer Zeta-function (of Riemann)

Riesz' method of ' summing ' series, 156

Right  U-) class, 4

Rodrigues' formula for Legendre polynomials, 303; modified, for Gegenbauer's function, 329

Roots of an equation, number of, 120, (inside a contour) 123; of Weierstrassian elliptic
functions (<'i . eo, e ), 443

Saalschiitz' integral for the Gamma-function, 243

Schlafli'-s Bessel function of the second kind, [r  (2)], 870

Sclilafli's integral for Bessel functions, 362, 372; for Legendre polynomials and functions, 303,

304, 306; modified, for Gegenbauer's function, 329
Schlomilch's expansion in series of Bessel coefficients, 377; function, 352; integi-al equation, 229
Schmidt's theorem, 223
Schwarz" lemma, 186

Second kind. Bessel function of, (Hankel's) 370, (Neumann's) 372, (Weber-Schliifli), 370,
(modified) 373; elliptic integral of [-E (m), Z ((/)], 517, (complete) 518; Eulerian integral of,
241, (extended) 244; integi-al equation of, 213, 221; Lame functions of, 562; Legendre
functions of, 316-320, 325, 326
Second mean-value theorem, 66

Second solution of Bessel's equation, 370, 372, (modified) 373; of Legendre's equation, 316; of
Mathieu's equation, 413, 427; of the hypergeometric equation, 286, (confluent form) 343; of
Weber's equation, 347
Second species of ellipsoidal harmonics, 537, (construction of) 540
Section, 4

Seififerfs spherical spiral, 527
Sequences, 11; decreasing, 12; increasing, 12

Series (infinite series), 15: absolutely convergent, 18; change of order of terms in, 25; con-
ditionally convergent, 18; convergence of, 15; differentiation of, 31, 79, 92; divergence of,
15  geometric, 19; integration of, 32, 78; methods of summing, 154-156; multiplication
of, 29, 58, 59; of analytic functions, 91; of cosines, 185; of cotangents, 139; of inverse
factorials, 142; of powers, see Power series; of rational functions, 134; of sines, 166; of
variable terms, 44 (see also Uniformity of convergence); order of terms in, 25; remainder of,
15; representing particular functions, see ii) iler tlie name of the fitiietioit; solutions of
differential and integi'al equations in, 194-202, 228; Taylor's, 93. Sec also Asymptotic
expansions. Convergence, Expansions, Foiirier series. Trigonometrical series and Uniformity
of convergence
Set, Irreducible (of zeros or poles), 430

Sigma-functions of Weierstrass [(t(z), <ri z), 0-2(2), 0-3(2)], 447, 448; addition formula for, 451,
458, 460; analogy -with circular functions, 447; duplication formulae, 459, 460; four
types of, 448; expression of elliptic functions by, 450; quasi-periodic properties, 447;
singly infinite product for, 448; three-term equation involving, 451, 461; Theta-functions
connected with, 448, 473, 487; triplication formula, 459
Signless integers, 3
Simple curve, 43; pole, 102; zero, 94
Simply-connected region, 455

Sine, product for, 137. See also Circular functions
Sine-integi-al [Si (2)], 352; -series (Fourier scries), 166 .
Singly-periodic functions, 429. See also Circular functions

Singularities, 83, 84, 102, 194, 197, 202; at infinity, 104; confluence of, 203, 337; equations
with five, 203; equations with three, 206, 210; equations with two, 208; equations with r,
209; essential, 102, 104; irregular, 197, 202; regular, 197
Singular points (singularities) of linear differential equations, 194, 202
Solid harmonics. 392

Solution of Riemann's P-equation by hypergeometric functions, 283, 288
Solutions of differential equations, see Chapters x, xviii, xxiii, and under the names of special

iiiuiitions
Solutions of integral equations, see Chapter xi
Sonine s polynomial ['/'," (2)], 352
Species (various) of ellipsoidal harmonics, 537

%
% 607
%

Spherical harmonics, see Harmonics

Spherical spiral, Seiffert's, 527

Spheroidal harmonics, 403

Squares of Bessel functions, 379, 380; of liypergeometric functions, 298; of Jacobian elliptic

functions (relations between), 492; of Theta-f unctions (relations between), 466
Statement of Fourier's theorem, Dirichlet's, 161, 163, 164, 176
Steadily tending to zero, 17
Stieltjes' theorem on zeros of Lame functions, 560, (generalised) 562; theory of Mathieu's

equation, 417
Stirling's series for the Gamma-function, 251
Stokes' equation, 204

Stolz' condition for convergence of double series, 27
Strings, vibrations of, 160
Successive substitutions, method of, 221
Sum-formula of Euler and Maclaurin, 127

Summability, methods of, 154-156; of Fourier series, 169; uniform, 156
Surface harmonic, 392
Surface, modular, 41
Surfaces, nearly spherical, 332

Sylvesters theorem concerning integi-als of harmonics, 400
Symmetric nucleus, 223, 228

Tabulation of Bessel functions, 378; of complete elliptic integi-als, 518; of Gamma-functions, 253

Taylor's series, 93; remainder in, 95; failure of, 100, 104, 110

Teixeira's extension of Biirmann's theorem, 131

Telegraphy, equation of, 387

Tesseral harmonics, 392; factorisation of, 536

Tests for convergence, see Infinite integrals, Infinite products and Series

Thermometric parameter, 405

Theta-functions [ i (;), S-o (z), \&3  z),  4  z) or   (,-), 9 ( )], 462-490 (Chapter xxi); abridged nota-
tion for products, 468, 469; addition formulae, 467; connexion with Sigma- functions, 448,
473, 487; duplication formulae, 488; expression of elliptic functions by, 473; four types
of, 463; fundamental formulae (Jacobi's), 467, 488; infinite products for, 469, 473, 488;
Jacobi's first notation, G ( ) and H (;/), 479; multipliers, 463; notations, 464, 479, 487;
parameters q, t, 463; jmrtial differential equation satisfied by, 470; periodicity factors,
463; periods, 463; quotients of, 477; quotients yielding Jacobian elliptic functions, 478;
relation S-i' = a-jS 3  4, 470; squares of (relations between), 466; transformation of, (Jacobi's
imaginary) 124, 474, (Landen's) 476; triplication formulae for, 490; with zero argument
( 9,  :j,  4,  1'), 464; zeros of, 465

Third kind of elliptic integral, IT (u, a), 522; a dynamical application of, 523

Third order, linear differential equations of, 210, 298, 418, 428

Third species of ellipsoidal harmonics, 537, (construction of) 541

Three kinds of elliptic integi-als, 514

Three-term equation involving Sigma-f unctions, 451, 461

Total fluctuation, 57

Transcendental functions, see under the names of special functions

Transformations of elliptic functions and Theta-functions, 508; Jacobi's imaginary, 474, 505,
506, 519; Landen's, 476, 507; of Eiemann's P-equation, 207

Trigonometrical equations, 587, 588

Trigonometrical integrals, 263; and Gamma-functions, 256

Trigonometrical series, 160-193 (Chapter ix); convergence of, 161; values of coefficients in, 163;
Eiemann's theory of, 182-188; which are not Fourier series, 160, 163. See also Fourier series

Triplication formulae for Jacobian elliptic functions and E  u), 530,534; for Sigma-functions,
459; for Theta-functions, 490; for Zeta-functions, 459

Twenty-four solutions of the hypergeometric equation, 284; relations between, 285, 288, 290

Two-dimensional continuum, 43

Two variables, continuous functions of, 67; hypergeometric functions (Appell's) of, 300

Types of ellipsoidal hannonics, 587

Unicursal, 455
Uniformisation, 454

%
% 608
%

Uniformising variables, 455; associated with confocal coordinates, 549

Uniformity, concept of, 52

Uniformity of continuity, 54; of sumniability, 156

Uniformity of convergence. 41-60 (Chapter iii), defined, 44; of Fourier series, 172, 179, 180; of

intiuite intei;nils. 70. 72, 73; of infinite products, 49; of power series, 57; of series, 44,

(condition for) 45, (Hardy's test for) 50, (Weierstrass'ili-test for) 49
Uniformly convergent infinite integrals, properties of, 73; series of analytic functions, 91,

(differentiation of) 92
Uniqueness of an asymptotic expansion, 153; of solutions of linear differential equations, 196
Upper bound, 55; integi-al, 61
Upper limit, integral equation with variable, 213, 221; to the value of a complex integi-al, 78, 91

Value, absolute, i ee Modulus; of the argument of a complex number, 9, 588; of the coefficients
in Fourier series and trigonometrical series, 163, 165, 167, 174; of particular hypergeometric
functions, 281, :i93, 298, 301; of Jacobian elliptic functions of  A',  iK', h(K+iK'), 500,
506, 507; of K, K' for special values of A", 521, 524, 525; of  (a) for special values of . *,
267, 269

Vanishing of power series, 58

Variable, uniformising, 455; terms (series of), see Uniformity of convergence; upper limit,
integi-al equation with, 213, 221

Vibrations of air in a sphere, 399; of circular membranes, 396; of elliptic membranes, 404, 405;
of strings, 160

Volterra's integi'al equation, 221; reciprocal functions, 218

Wave motions, equation of, 386; general solution, 397, 402; solution involving Bessel functions,
3S)7

Weber's Bessel function of the second kind [r, ( )], 370

Weber's equation, 204. 209, 342, 347. See also Parabolic cylinder functions

Weierstrass' factor theorem, 137; il/-test for uniform convergence, 49; product for the Gamma-
function, 235; theorem on limit points, 12

Weierstrassian elliptic function [  >( )], 429-461 (Chapter xx), defined and constructed, 432,

433; addition theorem for, 440, (Abel's method) 442; analogy with circular functions,
438; definition of \   [z) - e ], 451; differential equation for, 436; discriminant of, 444;
duplication formula, 441; expression of elliptic functions by, 448; expression of ip (z) -   (y)
by Sigma-functions, 451; half-periods, 444; homogeneity properties, 439; integral formula
for, 437; integi-ation of in-ational functions by, 452; invariants of, 437; inversion problem
for, 484; Jacobian elliptic functions and, 505; periodicity, 434; roots fj, e.,, e., 443. See
also Sigma-fimctions and Zeta-f unction (of Weierstrass)

Whittaker's function TFj., (£), see Confluent hypergeometric functions

Wronskis expansion, 147

Zero argument, Theta-f unctions with, 464; relation between, 470

Zero of a function, 94; at infinity, 104; simple, 94

Zeros of a function and poles (relation between), 438; connected with zeros of its derivate, 121,
123; in-educible set of, 430; number of, in a cell, 431; order of, 94

Zeros of fimctions, (Bessel's) 361, 367, 378, 381, (Lame's) 557, 558, 560, 578, (Legendre's) 303,
316, 335, (parabolic cylinder) 354, (Eiemann's Zeta-) 268, 269, 272, 280, (Theta-) 465

Zeta-function, Z("). (of Jacobi), 518; addition formula for, 518; connexion with E u), 518;
Fourier series for, 520; Jacobi's imaginary transformation of, 519. See also Jacobian
elliptic functions

Zeta-fimction, j'(*), i'(s,a), (of Riemann) 265-280 (Chapter xiii), (generalised by Hurwitz) 265;
Euler's product for, 271; Hermite's integi-al for, 269; Hurwitz' integral for, 268; in-
equalities satisfied by, 274, 275; logarithmic derivate of, 279; Eiemann's hypothesis
concerning, 272, 280; Eiemann's integrals for, 266, 273; Eiemann's relation connecting f (s)
and f (1 - * ), 269; values of, for special values of .s, 267, 269; zeros of, 268, 269, 272, 280

Zeta-function, f(2), (of Weierstrass), 445; addition formula, 446; analogy with circular
functions, 446; constants t/j, t)., connected with, 446; duplication formulae for, 459; ex-
pression of elliptic functions by, 449; quasi-periodicity, 445; triplication formulae, 459.
See also Weierstrassian elliptic functions

Zonal harmonics, 302, 392; factorisation of, 536