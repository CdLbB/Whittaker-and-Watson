\chapter{Appendix}
THE ELEMENTARY TRANSCENDENTAL FUNCTIONS

A*l. On certain results assumed in Chapters I-IV.

It was convenient, in the first four chapters of this work, to assume
some of the properties of the elementary transcendental functions,
namely the exponential, logarithmic and circular functions; it was
also convenient to make use of a number of results which the reader
would be prepared to accept intuitively by reason of his familiarity
with the geometrical representation of complex numbers by means of
points in a plane.

To take two instances, (i) it was assumed \hardsectionref{2}{7}) that lim (exp i) =
exp (lim *, and (ii) the geometrical concept of an angle in the
Argand diagram made it appear plausible that the argument of a complex
number was a many-valued function, possessing the property that any
two of its values diffei-ed by an integer multiple of ir.

The assumption of results of the first type was clearly illogical; it
wjis also illogical to base ai'ithmetical results on geometrical
reasoning. For, in order to put the foundations of geometry on a
satisfactory basis, it is not only desirj ble to employ the axioms of
arithmetic, but it is also necessary to utilise a further set of
axioms of a more definitely geometrical character, concerning
properties of points, straight lines and planes*. And, fm-ther, the
arithmetical theory of the logarithm of a complex nimiber appears to
be a necessary preliminary to the development of a logical theory of
angles.

Apart from this, it seems unsatisfcictory to the c esthetic taste of
the mathematician to employ one branch of mathematics as an essential
constituent in the structui-e of another; particularly when the
former has, to some extent, a material basis whereas the latter is of
a purely abstract nature f.

The reasons for pursuing the somewhat illogical and unaesthetic
procedure, adopted in the earlier part of this work, were, firstly,
that the properties of the elementary transcen- dental functions were
required gradually in the com"se of Chapter n, and it seemed
imdesirable that the coiu e of a general development of the various
infinite processes should be frequently interrupted in oi"der to prove
theoremsf (with which the reader was, in all probabihty, already
famihar concerning a single particuliU* function; and, secondly, that
(in counexioji with the assumption of results based on geometrical
considei-ations) a pui'ely arithmetical mode of development of
Chapters l-iv, deriving no help or illus- trations from geometrical
processes, would have very greatly inci-eased the difliculties of the
reader unacquainted with the methods and the spirit of the analyst,

* It is not our object to give any account of the foundations of
geometry in this work. They are investigated by various writei-s. such
as Tiitehead. Axioms of Projective Geometry (Cambridge Math. Tracts,
no. i. 1906) and Mathews, Projectile Geometry London, 1914). A perusal
of Chapters i, xx, xxii and xxv of the latter work will convince the
reader that it is even more laborious to develop geometiy in a logical
manner, from the minimum number of axioms, than it is to evolve the
theory of the circular functions by purely analytical methods. A
complete account of the elements both of arithmetic and of geometry
has been given by Whitehead and Bussell, Principia Mathematica
(1910-1913).

t Cf. Merz, History of European Thought in the Nineteenth Century, n.
(London, 1903), pp. 631 (note 2) and 707 (note 1), where a letter from
Weierstrass to Schwarz is quoted. See also Sylvester, P/ii7. Mag. (5),
ii. (1876), p. 307 [Math. Papers, lu. (1909). p. -50].

%
% 580
%

A'll. Summary of the Appendix.

The general course of the Appendix is as follows :

In v5§ A*2-A-22, the exponential function is defined by a power
series. From this definition, combined with results contained in
Chapter ll, are derived the elementary properties (apart from the
periodic properties) of this function. It is then easy to deduce
corresponding properties of logarithms of positive numbers (§§
A'3-A'33).

Next, the sine and cosine are defined by power series from which
follows the connexion of these functions with the exponential
function. A brief sketch of the manner in which the formulae of
elementary trigonometry may be derived is then given ( § A'4-A-42).

The results thus obtained render it possible to discuss the
periodicity of the exponential and circular functions by purely
arithmetical methods (§§ A"5, A*51).

In §§ A"52-A'522, we consider, substantially, the continuity of the
inverse circular functions. When these functions have been
investigated, the theory of logarithms of complex numbers (§ A'6)
presents no further difficulty.

Finally, in § A"7, it is shewn that an angle, defined in a purely
analytical manner, possesises properties which are consistent with the
ordinary concept of an angle, based on our experience of the material
world.

It will be obvious to the reader that we do not profess to give a
complete account of the elementary transcendental functions, but we
have confined ourselves to a brief sketch of the logical foundations
of the theory*. The developments have been given by writers- of
various treatises, such as Hobson, Plane Trigonometry; Hardy, A
course of Pure Mathematics; and Bromwich, Theory of Infinite Series.

A'12. -1 logical order of development of the elements of Analysis.

The reader will find it instructive to read Chapters i-iv and the
Appendix a second time in the following order :

Chapter i (omitting + all of § 1 "5 except the first two paragraphs).

Chapter ii to the end of \hardsubsectionref{2}{6}{1} (omitting the examples inf §,
2-31-2-61).

Chapter iii to the end of \hardsubsectionref{3}{3}{4} and §§ 3-5-3-73.

The Appendix, §§ A-2-A-6 (omitting §§ A-32, A-33).-

Chapter ii, the examples of v \hardsubsectionref{2}{3}{1}-2'61.

Chapter in, 5\hardsubsubsectionref{3}{3}{4}{1}-3-4.

Chapter iv, inserting §§ A"32, A-33, Aw after \hardsubsectionref{4}{1}{3}.

Chapter ii, §§ 2-7-2-82.

He should try thus to convince himself that (in that order) it is
possible to elaborate a purely arithmetical development of the
subject, in which the graphic and familiar language of geometry J is
to be regarded as merely conventional.

* In writing the Appendix, frequent reference has been made to the
article on Algebraic Analysis in the Encijklopiidie der Math.
Wissenschaften by Pringsheim and Faber, to the same article translated
and revised by Molk for the Encyclopedic des Sciences Math., and to
Tannery, Introduction h la Theorie dis Fonctions d'uue Variable
(Paris, 1904).

t The properties of the argument (or phase) of a complex number are
not required in the text before Chapter v.

i E.p. 'a point ' for 'an ordered number-pair,' ' the circle of unit
radius with centre at the origin' for 'the set of ordered number-pairs
(.r, y) which satisfy the condition .x'- + y-=l,' 'the points of a
straight line ' for ' the set of ordered number-pairs (x, y) which
satisfy a relation of the type Ax + By + C =0,' and so on.

%
% 581
%

A "2. The exponential function exp z.

The exponential function, of a complex variable 2, is defined bj the
series*

exp.= l + f; + | + | + ... = l+J |;.

This series converges absolutely for all values of z (real and
complex) by D'Alembert's ratio test \hardsubsectionref{2}{3}{6}) since lim | z\ n) | = 0<1
; so the definition is valid for all values of z.

Further, the series converges uniformly throughout any bounded domain
of values of z; for, if the domain be such that j 2 j /iJ when z is
in the domain, then

j(sVm!)I R"/nl,

and the uniformity of the convergence is a consequence of the test of
Weierstrass \hardsubsectionref{3}{3}{4}),

X

by reason of the convergence of the series 1+2 R"i'n !), in which the
terms are indepen-

n = l

dent of z.

Moreover, since, for any fixed value of n, s'V ! is a continuous
function of z, it follows from \hardsubsectionref{3}{3}{2} tha t the exponential function
is continuous for all values of z; and hence (cf. § 3 '2), if z be a
variable which tends to the limit (, we have

lim exp 2= exp f.

A*21. The addition-theorem for the exponential function, and its
consequences. From Cauchy's theorem on multiplication of absolutely
convergent series \hardsubsectionref{2}{5}{3}), it follows thatf

(exp 2i)(exp2,) = (l+- V|', + -)(l + + !;' + ..)

2i + 22 Zi + 2z Z2 + Z-2,

= 1 + -Yj-+ 2", +...

= exp (21 + 22),

so that exp (21 + 22) can be expressed in terms of exponential
functions of 21 and of 22 by the formula

exp (21 + 22) = (exp 2i) (exp z. .

This result is known as the addition-theorem for the exponential
function. From it,

we see by induction that

(exp2i) (exp 22) ... (exp2,i) = exp(2i + 22+...+2 ),

and, in particular,

 exp 2 exp ( - 2) =exp 0=1.

From the last equation, it is apparent that there is no value of 2 for
which exp 2 =; for, if there were such a value of 2, since exp (-2)
would exist for this value of 2, we should have 0=1.

It also follows that, when x is real, exp >0; for, from the series
definition, exp.r l when x O; and, when x 0, exp x= 1/exp ( - .r)>0.

* It was formerly customary to define exp 2 as lim ( 1 + - I, cf.
  Cauchy, Coins cV Analyse, i.

p. 167. Cauchy ibid. pp. 168, 309) also derived the properties of the
function from the series, but his investigation when z is not rational
is incomplete. See also \Schlomilch, Handbuch der alg. Analysis
(1889), pp. 29, 178, 246. Hardy has pointed out (Math. Gazette, in. p.
284) that the limit definition has many disadvantages.

t The reader will at once verify that the general term in the product
series is (2i + Ci2i"- 22 + C22i"-22.\,2 + . . . + z ' n ! = ( i + z
)''ln ! .

%
% 582
%

Further, exp .> is an increasing function of the real variable,v;
for, if X.'>0, exp x + k) - exp x = exp x . exp >l- - 1 > 0, because
exp .r>0 and exp />!.

Also, since expA- 1 /A = 1 +(A/2 :) + (/i2/3 !) + ...,

and the series on the right is seen (by the methods of § A"2) to be
continuous for all

values of h, we have

lim exp/i- 1 //; = 1,

rfexpa,. exp(2+A)-exp2 and so -y = hm - - = exp z.

A'22. Variom properties of the exponential function.

Returning to the formula (exp j) (exp 2.>) ... (exp 2- ) = exp (21+22
+ +2n) we see that,

when n is a positive integer,

(exp 2)" = exp (ns),

and (exp z)-" = 1 /(exp 2)" = 1 /exp nz) = exp ( - ?iz).

In particular, taking 2 = 1 and writing e in place of exp 1 =
2-71828..., we see that, when m is an integer, positive or negative,

e' = exp Hi = l + ( i/i:) + (? -/2 !) + ....

Also, if /Li be any rational number TODO, where// and q are integers,
q being positive)

(exp fi)'i = exp fiq = exp p = e'\

so that the th power of exp is et'; that is to say, expM is a value
of ef'ii=ei, and it is obviously (§ A-21) the real positive value.

If X be an irrational-real number (defined by a section in which Oj
and ao are typical members of the X-class and the ff-class
respectively), the irrational power e is most simply defined as exp x
; we thus have, for all real values of x, rational and irrational,

X x'

e =\ -I 1 1-

1 ! 2! '

an equation first given by Newton*.

It is, therefore, legitimate tg write e for exp x when x is real, and
it is customary to write e for exp 2 when z is complex. The function e
(which, of course, must not be regarded as being a power of e), thus
defined, is subject to the ordinary laws of indices,, viz.

[Note. Tannery, Lecons d'Algehre et d' Analyse (1906), I. p. 45,
practically defines e, when X is irrational, as the only number X
such that e'* X e"-, for every ! and ao. From the definition we have
given it is easily seen that such a unique number exists. For e\ ] )x(
= X) satisfies the inequality, and i( X' X) also did so, then

exj) 2 - exp r i = c'"- - /'' I X' - -V i,

so that, since the exponential function is continuous, a-i - a cannot
be chosen arbitrarily small, and so ( i, a-2) does not define a
section.]

* De Aitdlysi per aequat. nu)n. term. inf. (written before 16G9, but
not published till 1711); it was also given both by Newtou and by
Leibniz in letters to Oldenburg in 1676; it was first published by
Wallis in 1685 in his Treatise on Algebra, p. 343. The equation when x
is irrational was explicitly stated by \Schlomilch, Handbuch der alg.
Analysis (1889), p. 182.

%
% 583
%

A'3. Logarithms of positive numbers*.

It has been seen ( § A"2, A"21) that, when x is real, expA* is a
positive continuous increasing function of x, and obviously exp a- - +
x as x - - + oo, while

exp.'r=l/exp (- ')- >-0 as x- - cc.

If, then, a be any positive number, it follows from \hardsubsectionref{3}{6}{3} that the
equation in x,

exp X = a,

has one real root and only one. This root (which is, of course, a
function of a) will be written t LoggW or simply Log a; it is called
the Logarithm of the positive number a.

Since a one-one correspondence has been established between i- and a,
and since a is an increasing function of x, x must be an increasing
function of a; that is to say, the Logarithm is an increasing
function.

Example. Deduce from § A-21 that Log a + Log 6 = Log a6.

A "31. The continuity of the Logarithm.

It will now be shewn that, when a is positive. Log a is a continuous
function of a.

Let Log a = jp, Log (a + A) = .r + k

so that e =a, e'' =a-Vh, l + (hja) = e''.

First suppose that A>0, so that >0, and then

l + hla) = l+k+U' + ...>l + k, and so 0<k<hja,

that is to say 0<Log( + A) - Loga<A/a.

Hence, h being positive, Log (a + A) - Log can be mjide arbitrarily
small by taking k sufficiently small.

Next, suppose that A<0, so that Z-<0, and then a/ a + h) = e~''. Hence
(taking 0< -h< a, as is obviousl ' permissible) we get

al(a + h) = l + -k) + U' +...>l-k, and so - - < - 1+ a/ (a + A) = - h/
a + h)<- hja.

Therefore, whether h be positive or negative, if e be an arbitrary
positive number and if I A I be taken less than both \ a and |ae, we
have

I Log a + h) - Log a\ <e,

and so the condition for continuity \hardsectionref{3}{2}) is satisfied,

A'32. Diferentiation of the Logarithm.

Retaining the notation of § A-31, we see, from results there proved,
 that, if h- O a being fixed), then also k ~Q. Therefore, when a>0,

(ILosa,. k 11

Since Log 1=0, we have, by \hardsubsectionref{4}{1}{3} example 3,

Loga= I ~i dt.

* Many mathematicians define the Logarithm by the integral formula
given in § A.-32. The reader should consult a memoir by Hurwitz Math.
Ann. lxx. (1911), pp. 33-47) on the founda- tions of the theory of the
logarithm.

t This is in agreement with the notation of most text-books, in which
Log denotes the principal value (see § A-6) of the logarithm of a
complex number.

%
% 584
%

A'33. The expansion of Log (1 + ) in powers of a. From § A "32 we have

Log (l + a) =i" I + t)~idt J >

= r t+f -... + -y>-H'- + -)"(" i+t)- dt

= a - ia- + itf'' -... + (-)"-!- a" + /?,

n

where / = (\ )" /"% (i +<)-i c .

Now, if - l<a<l, we have

i nl i t" l-\ a ) - dt

= |a|" + i (n+l)a-| i) -' -- as ?t - ac .

feence, when - !< <!, Log(l + a) can be expanded into the convergent
series*

X

Log(l+ ) = o--ia2 + ia3-...= 2 (-)"-! "/?;.

n = l If a=+l,

|/?,J=/ i (H-;)-it (;< I i;"rf = (>i + l)-i- Oas n-9-x, ./ ./

so the expansion is valid when a= + 1; it is not valid when = - 1.
Example. Shew that lim (l+-j =e.

[We have Jm. n log (l + 1) = Jiu. ( -, + sTT " -)

= 1,

and the result required follows from the result of § A-2 that lim e- =
e ]

A*4. The definition of the sine and cosine.

The functions t sin z and cos z are defined analytically by means of
power series, thus

23 25 CO /\ .) 22 + l

these series converge absolutely for all values of z (real and
complex) by \hardsubsectionref{2}{3}{6}, and so the definitions are valid for all Vcilues
of z.

On comparing these series with the exponential scries, it is ap )arcnt
that the sine and cosine are not essentially new functions, but they
can be expressed in terms of exponential functions by the equations J

2i sin z = exp iz) - exp ( - iz), 2 cos 2 = exp (iz) + exp ( - iz).

* This method of obtaining the Logarithmic expansion is, in effect,
due to Walhs, P]iil. Trans. 11. (1668), p. 754.

t These series were given by Newtou, Dc Anahjsi... (1711), see § A'22
footnote. The other trigouometrical functions are defined in the
manner with which the reader is famiHar, as quotients and reciprocals
of sines and cosines.

X These equations were derived by Euler [they were given in a letter
to Jobann liernoulH in 1740 and pubhshed in the Hist. Acad. Berlin, v.
(1749), p. 279] from the geometrical definitions of the sine and
cosine, npon which the theory of the circular functions was tlien
universally based.

%
% 585
%

It is obvious that sin z and cos z are odd and even functions of z
respectively; that is to say

sin ( - 2)= -sin 2, cos ( - 2) = cos 2.

A*41. The fundaraental properties of sin z and cos z.

It may be proved, just as in the case of the exponential function (§
A"2), that the series for sin 2 and cos 2 conyerge uniformly in any
bounded domain of values of z, and con- sequently that sin 2 and cos 2
are continuous functions of 2 for all values of 2. Further, it may be
proved in a similar manner that the series

, 22 z* 1-3-1 + 5--- defines a continuous function of 2 for all values
of 2, and, in particular, this function is continuous at 2=0, and so
it follows that

lim (2~ sin2) = L

A "42. The additio7i-theorems for sin 2 and cos 2.

By using Euler's equations ( A*4), it is easy to prove from properties
of the exponential function that

sin (2 + 22) = sin z cos 23 + cos 2 sin z and cos z + 22) - cos 21 cos
22 - sin z sin 22;

these results are known as the additiorit-theorems for sin 2 and cos
2.

It may also be proved, by using Euler's equations, that

sin2 2 + cos-2 = l.

By means of this result, sin(2j + 22) can be expressed as an algebraic
function of sin2i and sin 22, while cos (21 + 22) can similarly be
expressed as an algebraic function of cos 21 and cos 22; so the
addition-formulae may be regarded as addition-theorems in the strict
sense (cf. §§ 20-3, 22-732 note).

By differentiating Euler's equations, it is obvious that

(/sin 2 dcoaz

- J - = cos 2, - 5 - = - sm 2. dz dz

Example. Shew that

sin 2i = 2 sin 2 cos 2, cos 22 = 2 cos- 2 - 1;

these results are known as the duplication-formulae.

A'5. The periodicity of the exponential function.

If 2i and 22 are such that exp2j = exp22, then, multiplying both sides
of the equation by exp( - 22), we get exp (21 - 22) = !; and writing
y for z - Z2, we see that, for all values of 2 and all integral values
of n,

exp (2 -f- ny) = exp 2 . (exp y)'' = exp 2.

The exponential function is then said to have period y, since the
effect of increasing z by y, or by an integral multiple thereof, does
not affect the value of the function.

It will now be shewn that such numbers y (other than zero) actually
exist, and that all the numbers y, possessing the property just
described, are comprised in the expression

2mvi, ( =±1, ±2, ±3, ...)

where it is a certain jjositive number* which happens to be greater
than 2v'2 and less than 4.

* The fact that tt is an irrational number, whose value is 3'141-59...
, is irrelevant to the present investigation. For an account of
attempts at determining the value of tt, concluding with a proof of
the theorem that it satisfies no algebraic equation with rational
coefficients, see Hobson's monograph Squaring the Circle (1913).

%
% 586
%

A*51. The sold tion of the equation e\ \ >y=.

Let y = a + ifi, where a and /3 are real; then the problem of solving
the equation expy=l is identical with that of solving the equation

exp . expi/3=l. Comparing the real and imaginary parts of each side of
this equation, we have

expa.cos =l, expo, sin j3=0. Squaring and adding these equations, and
using the identity cos-/3 + sin2/3= 1, we get

exp2a=l. Xow if a were positive, exp 2a would be greater than 1, and
if a were negative, exp 2a would be less than 1; and so the only
possible value for a is zero. It follows that cos /3 = 1, sin 3 = 0-

Now the equation sin/ii = is a necessary consequence of the equation
cos = l, on account of the identity cos2/3 + sin- i3=l. It is
therefore sufficient to consider solutions (if such solutions exist)
of the equation cos/3 = l.

Instead, however, of considering the equation cos/3 = l, it is more
convenient to consider the equation* cos 07=0.

It will now be shewn that the equation cos.r=0 has one root, and only
one, lying between and 2, and that this root exceeds 2; to prove
these statements, we make use of the following considerations :

(I) The function cos.r is certainly continuous in the range O a' 2.

(II) When A- '2, we have +

1- - >0 -\ \ >0 \ \ \ >o

2! ' 4! 6! ' 8! 10!= ' "'

and so, when .i- y'2, cos x > 0. (Ill) The value of cos 2 is

2<' / 4 \ 210 / 4 \

- + S-72o('-7:Ti)-IO!('-l-Tn-2)--=-i--< -

(IV) WhenO<.r 2,

sin.v .v"\ A''* x \, x',

- = 0-6) + 12oO-6r7) + ->l-6 ' and so, when x 4. 2, sin x \ x.

It follows from (II) and (III) combined with the results of (I) and of
3-63 that the equation cosa'=0 has at least one root in the range v 2
<.i'< 2, and it has no root in the range \$ x J±

Further, there is oiot more than, one root in the range J2<x<2; for,
suppose that there were two, Xi and X2 x.2>Xi); then
0<.r2-A'j<2-y'2<l, and sin (a% - Xij - sin .Cg cos x - sin Xi cos x =
0, and this is incompatible with (IV) which shews that sin (0: 2 - -
'i) M' 2~' 'i)-

The equation cos:r=0 therefore has one and onli/ one root lying
betiveen and 2. This root lies between 2 and 2, and it is called \ n;
and, as stated in the footnote to § A"5, its actual value happens to
be 1 "57079....

* If cos.r = 0, it is an immediate consecjuence of the
duplication-formulae that cos2.r= -1 and tht'uce that co8 4.t = 1, so,
if x is a solution of cos.c = 0, 4.t is a solutiou of cos/3 = l.

t The symbol may be replaced by > except when x =, 2 in the first
place where it occurs, and except when x = in the other places.

%
% 587
%

From the addition-formulae, it may be proved at once by induction that

cos ?;7r = ( - 1)", sin?i7r = 0, where ?i is any integer.

In particular, cos2)i7r = l, where n is any integer.

Moreover, there is no value of /3, other than those values which are
of the form 2?i7r, for which cosj3=l; for if there were such a value,
it must be real*, and so we can choose the integer ra so that

- TT 27mr -I3<7r.

We then have

sin I WITT - /3 I = ±sin(m -l/3)=±sini =±2~2(l\ cos/3)i = 0,

and this is inconsistent t with sin j wtt - /3 | J | j/itt - i/3 [
unless /3 = 2mTr.

Consequently the numbers 2n7r, (/i = 0, ±1, ±2,...), a/id no others,
hare their cosines equal to unity.

It follows that a positive number n exists such that exps has period
iri and that exp2 has no period fundamentally distinct from liri.

The formulae of elementary trigonometry concerning the periodicity of
the circular functions, with which the reader is already acquainted,
can now be proved by analytical methods without any difficulty.

Example . Shew that sini r is equal to 1,-not to - L

Example 2. Shew that tan.i->.z; when 0<x<\ Tr.

[For cos.i'>0 and

sm.r - orcosA':

---A n-'i- 7- 1, l)!i 4>i + lj'

71 = 1 (4

and every term in the series is positive.]

x 77 v 25 x x

Example 3. Shew that 1 - V + ir7 ~; i positive when x =,r, and that
1 - V + Jrr

2 24 / 20 lb z 24

vanishes when .r = (6 - 2 /3)2 = 1-5924...; and deduce that J

3-125 <7r<3-185.

A'52. The solution of a pair of trigonometncal equations. Let X, /x be
a pair of real numbers such that X + jit l. Then, if X=|= - 1, the
equations

cos X = X, sin X = i have an infinity of solutions of which one and
only one lies between § - tt and tt.

First, let X and fi be not negative; then \hardsubsectionref{3}{6}{3}) the equation cos.r
= X has at least one solution Xi such that O .x-j iTr, since cos = 1,
cos 7r=0. The equation has not two solutions in this range, for if Xi
and x.2, were distinct solutions we could prove (cf. § A-51) that sin
(.t'l - .i"2) = 0, and this would contradict § A-51 (IV), since

0<| 2-- i I <i7r<2. Further, sin.ri= + (1 -coh' X])= -1- /(1 -- ) = M5
so x is a solution of both equations.

* The equation cos/3=l implies that exp i(i - l, and we have seen that
this equation has no complex roots.

+ The inequality is true by (IV) since j irnr - |/3 | : i7r<2.

+ See De Morgan, A Budget of Purado.ves (London, 1872), pp. .S16 et.
sfq., for reasons for proving that 7r>3 .

§ If X=: - 1, ±7r are solutions and there are no others in the range (
- tt, tt).

%
% 588
%

The equations have no solutions iu the ranges ( - tt, 0) and ( tt, tt)
since, in these ranges, either sin x or cos x is negative. Thus the
equations have one soUition, and only one, in the range ( - rr, tt).

If X or fi (or both) is negative, we may investigate the equations in
a similar manner; the details are left to the rea<ier.

It is obvious that, if x- is a solution of the equations, so also is
Xy + 'iniY, where n is any integer, and therefore the equations have
an infinity of real solutions.

A'521. The principal solution of the trigonometrical equations.

The unique solution of the equations cos.t' = X, sin,t-=/i. (where X +
fx' = ) which lies between - it and n is called the principal
solution*, and any other solution differs from it by an integer
multiple of Stt.

The principal valuei of the argument of a complex number (=1=0) can
now be defined analytically as the principal solution of the equations

I z \ cos ( = / (2), j 2 1 sin = / z), and then, if = 1 2 1 . (cos + /
sin 6),

we must have d = (P + 2mr, and 6 is called a value of the argument of
j, and is written arg2 (cf i 1-5).

A "522. The continuity of the argument of a complex variable.

It will now be shewn that it is possible to choose such a value of the
argument 6 (z), of a complex variable z, that it is a continuous
function of z, provided that z does not pass through the value zero.

Let 2,1 6 given 'alue of 2 and let 6 be any value of its argument;
then, to prove that 6 (z) is continuous at Zq, it is sufficient to
shew that a number 61 exists such that i=arg2i and that j 1 - 0 1 can
be made less than an arbitrary positive number e by giving 1 21 - 20 1
any value less than some positive number t).

Let fo = '-"o + Vo > 1 = . 'i + >i

Also let j 2i - So 1 l e chosen to be so small that the following
inequalities are satisfied J : (I) I .Ti - 0 i < 2 I - 0 1 5 provided
that Xq =t= 0, (II) I ?/i - 2/0 i < 2 ! J/o I, provided that 3/0 =t=
0, (III) |Xi-.%|<Je|2(,|, i?/i-3/o|<ie|2ol. From (I) and (II) it
follows that x Vi and y f/i are not negative, and

.'ToA-i a'o /o i > i.yo so that xo .vi + ?/o2/i J i 20 j 2.

Now let that value of 61 be taken which differs from 0 by less tlian
it; then, since x, and Xi have not opposite signs and yo f 'id 1/1
have not opposite signs J, it follows from the solution of the
equations of A"52 that and 0 differ by less than n.

Now tan( i-,.) = ':i ",

* If \= - 1, we take +7r as the principal solution; of. p. 9.

t The term principal value was introduced in 1845 by Bjorling; see
the Archiv der Math, und Fhijs. ix. (1847), p. 408.

X (I) or (II) respectively is simply to be suppressed in the case when
(i) .Tq O, or when (ii) 2/0 = 0.

§ The gtometrical interpretation of these conditions is merely that r
and z are not in different quadrants of the plane.

%
% 589
%
aud so (§ A"51
example 2),

  I o(yi-yo)-y o(' i-- o) I

  2 I 2o |~ I 0 I  I yi- o ! + l 'o !  I - 'i - 0 ! . But I A'o 1 ko
! and also | yo j | :o |; therefore

I (9i - 6>o i =\$ 2 I 2o 1~ | yi- ?/o I + 1 1 - . 0 | <e-

Further, if we take | 2i - zq 1 l ss than  - ] J 'o |, (if o + 0)
and i [ yo i ? (if o + 0) and | e j g I, the inequalities (I), (II),
(III) above are satisfied; so that, if ri be the smallest of the
three numbers * i | a-o |, -I ! yo ! > i ko i j by taking j Sj - q 1
< '?, we have | i - o I < <; and this is the condition that 6 z)
should be a continuous function of the complex ariable z.

A '6. Logarithms of comflex numhers. The number f is said to be a
logarithm of z if 2 = e

To solve this equation in (, write = + '7? where and rj are real; and
then we have

z = e (cos Tj + i sin ?;).

Taking the modulus of each side, we see that | 3| = e, so that (§
A-3), = Log \ z\; and

then

z=\ z\ (cos 7/ + 1 sin/;),

so that T] must be a value of argz.

The logarithm of a complex number is consequently a many-valued
function, and it can be expressed in terms of more elementary
functions by the equation

log z = Log j 2 1 4- 1 arg z.

The continuity of logs (when s4=0) follows from § A'31 and § A'522,
since \ z\ is a continuous function of z.

The differential coefficient of any particular branch of logs \hardsectionref{5}{7})
may be determined as in § A*32; and the expansion of § A"33 may be
established for log (1 + ) when | a | < 1.

Corollary. If a be defined to mean e'iog, a is a continuous function
of z aud of a when a=t=0.

A'7. The analytical definition of an angle.

Let 1, 22, 23 be three complex numbers represented by the points Pj,
P., F3 in the Argand diagram. Then the angle between the lines (§
A"12, footnote) PiPo and P1P3 is defined to be any value of arg (23 -
Zj) - arg (22 - Zi).

It will now be shewn f that the area (defined as an integi'al), which
is bounded by two radii of a given circle and the arc of the circle
terminated by the radii, is proportional to one of the values of the
angle between the radii, so that an angle (in the analytical sense)
jjossesses the propei'ty which is given at the beginning of all
text-books on Trigonometry |.

* If any of these numbers is zero, it is to be omitted.

t The proof here given applies only to acute angles; the reader
should have no difficulty in extending the result to angles greater
than lir, and to the case when OX is not one of the bounding radii.

X Euclid's definition of an angle does not, in itself, afford a
measure of an angle; it is shewn in treatises on Trigonometry (cf.
Hobson, Plane Trigonometry (1918), Ch. i) that an angle is measured by
twice the area of the sector which the angle cuts off from a unit
circle whose centre is at the vertex of the angle.

%
% 590
%

Let (o-'i, yi) be any point (both of whose coordinates are positive)
of the circle x' - y- = a? a>0). Let 6 be the principal value of arg(
i + iyi), so that 0<6 <Itt. Then the area bounded by OX and the line
joining (0, 0) to x\, i/y) and the arc of the

circle joining .Vi, y ) to (o, 0) is / f x)dx, where*

./ '

/( )=a;tan (0 .r a cos ),

/ x) = a - x )- (a cos d x a), if an area be defined as meaning a
suitably chosen integral (cf. p. 61).

It remains to be proved that I f(x) dx is proportional to 6.

J "

fa /"a cose fa,

Now I Z f(x) dx = I X tan 6 dx + (a - x-)i dx

J I) Jo J (I cose

 ha- smecose + hr L a --x ')- + x a -x ) \ dx

f" - 1

= a-j a -x-) - dx

J o COS Q

= |a2 1 r (1 - -) 'id(- p' (1 - f') - - dt\

on writing x = at and using the example worked out on p. 64.

That is to say, the area of the sector is proportional to the angle of
the sector. To this extent, we have shewn that the popular conception
of an angle is consistent with the analytical definition.

* The reader will easily see the geometrical interpretation of the
integral by drawing a figure.