\chapter{Continuous Functions and Uniform Convergence} 

3"1. The dependence of one complex number on another. 

The problems with which Analysis is mainly occupied relate to the 
dependence of one complex number on another. If z and   are two complex 
numbers, so connected that, if z is given any one of a certain set of values, 
corresponding values of   can be determined, e.g. if   is the square of z, or if 
  = 1 when z is real and   = for all other values of z, then   is said to be a 
function of z. 

This dependence must not be confused with the most important case of 
it, which will be explained later under the title o  analytic functionality. 

If f is a real function of a real variable z, then the relation between ( and 2, which 
may be written 

can be visualised by a curve in a plane, namely the locus of a point whose coordinates 
referred to rectangular axes in the plane are (s, (). No such simple and convenient 
geometrical method can be found for visualising an equation 

considered as defining the dependence of one complex number f = | + i'7 on another 
complex number z = x + i)/. A representation strictly analogous to the one already given 
for real variables would require four-dimensional space, since the number of variables 
 , fj, X, y is now four. 

One suggestion (made by Lie and Weierstrass) is to use a doubly-manifold system of 
lines in the quadruply-manifold totality of lines in three-dimensional space. 

Another suggestion is to represent   and ;; separately by means of surfaces   

A third suggestion, due to Heffter*, is to write 

then draw the surface r = r x, y) — which may be called the modular-surface of the 
function— and on it to express the values of 6 by surface-markings. It might be 
possible to modify this suggestion in various ways by representing 6 by curves drawn 
on the surface r=r (.r, y). 

3'2. Continuity of functions of real variables. 

The reader will have a general idea (derived from the graphical represen- 
tation of functions of a real variable) as to what is meant by continuity. 

* Zeitschrift fur Math, tind Phys. xlix. (1899), p. 235. 



42 THE PROCESSES OF ANALYSIS [cHAP. Ill 

We now have to give a precise definition which shall embody this vague 
idea. 

Jjetf x) be a function of x defined when a  x  b. 

Let x  be such that a  x b. If there exists a number I such that, 
corresponding to an arbitrary positive number e, we can find a positive 
number rj such that 

\ f a:)-l\ < e, 

whenever \ x — Xi\ <  rj, x   x , and a   x   b, then I is called the limit of /(a;) 

It may happen that we can find a number 1+ (even when I does not exist) 
such that \ f(x) — l+\ < € when x-  < x < Xi + rj. We call Z+ the limit of f x) 
when X approaches x-  from the right and denote it by/(a;i + 0); in a similar 
manner we define f x  — 0) if it exists. 

If f(xi + 0), f xi), f x-  — 0) all exist and are equal, we say that /(a-) is 
continuous at x  ; so that \ if x) is continuous at x-  , then, given e, we can find 
7] such that 

\ f x) -f(x,) I < e, 

whenever \ x — Xi\ <  rj and a  x  h. 

If 1+ and l\  exist but are unequal, f x) is said to have an ordinary 
discontinuity* at Xi] and if 1+ = /\  4=/( i), f  ) is said to have a removable 
discontinuity at x- . 

lif x) is a complex function of a real variable, and \ if x) = g(x) + i h (x) 
where g (x) and h (x) are real, the continuity of f(x) at x  implies the con- 
tinuity of g (x) and of A (x). For when \ f(x) —f x  \ < e, then | g (x) —g(xi) \ < e 
and I h (x) — h xi)\ < e ; and the result stated is obvious. 

Example. From   2*2 examples 1 and 2 deduce that if f x) and cf) (x) are con- 
tinuous at Xi , so are f x) +  x), f x) x    x) and, if   (.rj) =t= 0, f x)/(l) (.*•). 

The popular idea of continuity, so far as it relates to a i-eal variable f x) depending 
on another real variable x, is somewhat different from that just considered, and may 
perhaps best be expressed by the statement "The function f x) is said to depend con- 
tinuously on X if, as x passes through the set of all values intermediate between any 
two adjacent values Xi and X2, f x) passes through the set of all values intermediate 
between the corresponding values /( j) and/(.r2)." 

The question thus arises, how far this popular definition is equivalent to the precise 
definition given above. 

Cauchy shewed that if a real function f x), of a real variable . •, satisfies the precise 
definition, then it also satisfies what we have called the popular definition ; this result 

* If a function is said to have ordinary discontinuities at certain points of an interval it 
is implied that it is continuous at all other points of the interval. 



3*2 1] CONTINUOUS FUNCTIONS AND UNIFORM CONVERGENCE 43 

vill be proved in   3"63. But the converse is not true, as was shewn by Darboux. This 
fact may be iUustrated by the following example*. 

Between x= — I and x= +1 (except at x=0), let f x) = sin — ; and let/(0)=0. 

It can then be proved that/ .r) depends continuously on x near .r=0, in the sense of 
the popular definition, but is not continuous at a; = in the sense of the precise definition. 

Example. If f x) be defined and be an increasing function in the range (a, b), the 
limits /(.r + 0) exist at all points in the interior of the range. 

[If f x) be an increasing function, a section of rational numbers can be found such 
that, if a, A be any members of its Z-cla.ss and its  -clas.s, a<f x + h) for evei-y positive 
value of h and A ' f x- h) for some po.sitive value of h. The number defined by this 
section is/(a; + 0).] 

3"21. Simple curves. Continua. 

Let X and y be two real functions of a real variable t which are continuous 
for every value of t such that a- t %h. We denote the dependence of x and y 
on t by writing 

.'  =  •(0. 3/ = i/(0-  a. t%h) 

The functions x (t), y (t) are supposed to be such that they do not assume the 
same pair of values for any two different values of t in the range a < t < b. 

Then the set of points with coordinates (x, y) corresponding to these values 
of t is called a simple curve. If 

X (a) = X  h), y (a) = y (b), 
the simple curve is said to be closed. 

Example. The circle x  + 7/-= 1 is a simple closed curve ; for we may write t 

x=coiit, y = smt. (0  27r) 

A two-dimensional continuum is a set of points in a plane possessing the 
following two properties : 

(i) If (x, y) be the Cartesian coordinates of any point of it, a positive 
number 8 (depending on x and y) can be found such that every point whose 
distance from  x, y) is less than S belongs to the set. 

(ii) Any two points of the set can be joined by a simple curve consisting 
entirel '' of points of the set. 

Example. The points for which .r-+3/-<l form a continuum. For if P be any 
point inside the unit circle such that OP=r<\, we may take 8=1-?-; and any two 
points inside the circle may be joined by a straight line lying wholly inside the circle. 

The following two theorems | will be assumed in this work ; simple cases 
of them appear ob\ dous from geometrical intuitions and, generally, theorems 
of a similar nature will be taken for granted, as formal proofs are usually 
extremely long and difficult. 

* Due to Mansion, Mathesis, (2) xix. (1899), pp. 129-131. 

t For a proof that the sine and cosine are continuous functions, see the Appendix, § A-41. 
+ Formal proofs will be found in Watson's Complex Integration and Cauchy's Theorem. 
(Cambridge Math. Tracts, No. 15.) 



44 THE PROCESSES OF ANALYSIS [CHAP. Ill 

(I) A simple closed curve divides the plane into two continua (the 
' interior ' and the ' exterior '). ' 

(II) If P be a point on the curve and Q be a point not on the curve, 
the angle between QP and Ox increases by + 27r or by zero, as P describes 
the curve, according as Q is an interior point or an exterior point. If the 
increase is + 27r, P is said to describe the curve ' counterclockwise.' 

A continuum formed by the interior of a simple curve is sometimes called 
an open two-dimensional region, or briefly an open region, and the curve is 
called its boundary; such a continuum with its boundary is then called a 
closed two-dimensional region, or briefly a closed region or domain. 

A simple curve is sometimes called a closed one-dimensional region; a 
simple curve with its end-points omitted is then called an open one-dimensional 
region. 

3"22. Continuous functions of complex variables. 

Lety"( ) be a function of   defined at all points of a closed region (one- or 
two-dimensional) in the Argand diagram, and let z  be a point of the region. 

Then f z) is said to be continuous at z  , if given any positive number e, 
we can find a corresponding positive number 77 such that 

\ f z)-f z,)\ < e, 
whenever \ z — Zx\ < r) and 2- is a point of the region. 

3'3. Series of variable tei-ms. Uniformity of convergence. 
Consider the series 

  1 + a;2 (1+ ieO' (1 +  T 

This series converges absolutely (§ 2*33) for all real values of x. 
If 8n ( ) be the sum of n terms, then ' >• 

' " ">= +" -(TT '  / 

and so lim Sn  x) = \ \  x' ;  xrf 0) 

but Sn 0) = 0, and therefore lim Sn (0) = 0. 

M-*-00 

Consequently, although the series is an absolutely convergent series of 
continuous functions of x, the sum is a discontinuous function of x. We 
naturally enquire the reason of this rather remarkable phenomenon, which 
was investigated in 1841-1848 by Stokes*, Seidelf and Weierstrassj, who 
shewed that it cannot occur except in connexion with another phenomenon, 
that of non-uniform convergence, which will now be explained. 

* Cnmb. Phil. Trans, viii. (1847), pp. 533-583. [Collected Papers, i. pp. 236-313.] 
t Mi'mchener Abhandlungen, v. (1848), p. 381. 
X Ges. Math. U'erke, i. pp. 67, 75. 



3-22-3'3l] CONTINUOUS FUNCTIONS AND UNIFORM CONVERGENCE 45 

Let the functions u  (z), Wg (z), ... be defined at all points of a closed region 
of the Argand diagram. Let 

Sn (z) = U, (z) + M2 (2) + ...+ tin  Z). ' 

oc 

The condition that the series 2 Un  z) should converge for any particular 

n = l 

value of z is that, given e, a number n should exist such that 

I  n+p  z) — Sn (z)\ < € 

for all positive values of jo, the value of 7i of course depending on e. 

Let n have the smallest integer value for which the condition is satisfied. 
This integer will in general depend on the particular value of z which has 
been selected for consideration. We denote this dependence by writing 
n (z) in place of ?i. Now it mag happen that we can find a number N, 

INDEPENDENT OF Z, SUch that 

n z)<N "'  '"  

for all values of z in the region under consideration. 

If this number N exists, the series is said to converge uniformly 
throughout the region. 

If no such number iV exists, the convergence is said to be non-uniform*. 

Uniformity of convergence is thus a property depending on a whole set of 
values of z, whereas previously we have considered the convergence of a series 
for various particular values of z, the convergence for each value being con- 
sidered without reference to the other values. 

We define the phrase ' uniformity of convergence near a point z ' to mean 
that there is a definite positive number 8 such that the series converges 
uniformly in the domain common to the circle \ z — z \ \  h and the region in 
which the series converges. 

3'31. On the condition for uniformity of convergence' . 

If Rn,p  z) = Un+i  z) + iin+2 (z) + ... + Un+p (z), WO have Seen that the 

necessary and sufficient condition that S Un (z) should converge uniformly 

in a region is that, given any positive number e, it should be possible to 
choose N INDEPENDENT OF z (but depending on e) such that 

I Rn, p(z)\ < € 

for ALL positive integral values of p. 

* The reader who is unacquainted with the concept of uniformity of convergence will find it 
made much clearer by consultinf; Bromwich, Iiijinite Series, Ch. vii, where an illuminating 
account of Osgood's graphical investigation is given. 

t This section shews that it is indifferent whether uniformity of convergence is defined by 
means of the partial remainder Rj p(z) or by iJ (2). Writers differ in the definition taken 
as fundamental. 



46 THE PROCESSES OF ANALYSIS [CHAP. Ill 

If the condition is satisfied, by § 2-22, 8n z) tends to a limit, S z), say for 
each value of   under consideration; and then, since e is independent of p, 

and therefore, when n > iV, 

S (2) - Sn (Z) = I lim i v, p  2)  - E v. n-N (z), / 

and so \ S(z)-S iz)\ < 2€. 

Thus (writing  e for e) a necessary condition for uniformity of convergence 
is that \ S z) — Sn (z) \ < e, whenever n>N and N is independent of z ; the 
condition is also sufficient ; for if it is satisfied it follows as in § 2-22 (I) 
that I R rp z) 1 < 2e, which, by definition, is the condition for uniformity. 

Example 1. Shew that, if x be real, the sum of the series 

X X X 

TODO

is discontinuous at .r=0 and the series is non-uniformly convergent near  ' = 0. 

The sum of the first n terms is easily seen to be 1 ; so when x — Q the 

nx+i 

sum is ; when a* 4=0, the sum is 1. 

1 ' 
The value of RnXx) = S x)-Sn x) is — -- if x O; so when x is small, say 

:r=one-hundred-milliouth, the remainder after a million terms is — or l-TTyTj   

100 + 1 
the first million terms of the series do not contribute one per cent, of the sum. And in 

general, to make < e, it is necessary to take 

° nx + 1 

7i>-(--l 

X \ e 

Corresponding to a given e, no number N exists, independent of x, such that n<N for 
all values of x in any interval including x = ; for by taking x sufficiently small we can 
make n greater than any number N which is independent of x. There is therefore non- 
uniform convergence near ,r = 0. 

Example 2. Discuss the series 

x n. n±V)x'''-\ ]  



nZx   + n x   \ +  7i- \ fx y 
in which x is real. 

r. , ,  ,   . nx (n-  )  X  , . X , 

The 7ith term can be written , 5—5 - , — ; , .  — q , so o (x) = n , and 

l+n\ v  l +  n + \ y X'' l+x'' 

  W ] +(,i + 1)2 2 • 

[Note. In this example the sum of the series is not discontinuous at   = 0.] 
But (taking 6<i, and  4=0), \ Rn x)\ < e if e-i(?i + l) \ x\ < l-ir n+\ f x  ; i.e. if 
?i+l>| €-i + Ve-2-4 |.T|-l or ?t + l<J e-i-v'e = 3-4 |:p|- . 



3*32] CONTINUOUS FUNCTIONS AND UNIFORM CONVERGENCE 47 

Now it is not the case that the second inequality is satisfied for all values of n greater 
than a certain value and for all values of x ; and the first inequality gives a value of 
n(x) which tends to infinity as x- 0 ; so that, corresponding to any interval containing the 
point  =0, there is no number N' independent of x. The series, therefore, is non-uniformly 
convergent near :i, = 0. 

The reader will observe that n x) is discontinuous at .r = 0; for n x)- 'x> as .r-  -0, 
but n(0) = 0. 

3'32. Connexion of discontinuity luith non-uniform convergence. 

We shall now shew that if a semes of continuous functions of z is uniformly 
convergent for all values of z in a given closed domain, the sum is a continuous 
function of z at all points of the domain. 

For let the series be f z) = it,  z) + Wo (2 ) + • . . + Un  z)+ ...— Sn (z) + Rn  z), 
where Rn (z) is the remainder after n terms. 

Since the series is uniformly convergent, given any positive number e, we 

can find a corresponding integer n independent of z, such that | R  ( ) | <   e 

for all values of z within the domain. 

Now n and e being thus fixed, we can, on account of the continuity of 
Sn (•2 ), find a positive number rj such that 

\ Sn(z)-S  z')\ < l€, 

whenever \ z — z' \ < r . 
We have then 

l/( ) -/V) I = 1 [Sni ) - Sn z')] \ + lRn(2)- Rn  ) \ 
< ! Sn z) - Sn(z') I + I Rniz) \ + | Rn z') | 

' < 6, 

which is the condition for continuity at z. 
Example 1. Shew that near x = the series 

Ui  x) + U<i  x) + M3 ( 0 + • • • , 

1 1 

where  i(vP)=a;,   (a;) = ;p "~* — :p "~ , 

and real values of x are concerned, is discontinuous and non-uniformly convergent. 

In this example it is convenient to take a slightly different form of the test ; we shall 
shew that, given an arbitrarily small number f, it is possible to choose values of x, as 
small as we please, depending on n in such a way that | R  [x) \ is not less than e for any 
value of n, no matter how large. The reader will easily see that the existence of such 
values of x is inconsistent with the condition for uniformity of convergence. 
i\    

The value of S ix) is .r" "-i ; as n tends to infinity, S   x) tends to 1, 0, or - 1, accord- 
ing as x is positive, zero, or negative. The series is therefore absolutely convergent for all 
values of or, and has a discontinuity at .t' = 0. 



48 THE PROCESSES OF ANALYSIS [cHAP. Ill 

1 

In this series R   x) = l-x ''-\  x > 0) ; however great n may be, by taking* x = e- (2" - 1) 
we can cause this remainder to take the value l-e'  which is not arbitrarily small. The 
series is therefore non-uniformly convergent near .r = 0. 

Example 2. Shew that near z = the series 

 il  H-(l+2) -l  l+(l + 2)"  

is non-uniformly convergent and its sum is discontinuous. 
The nth term can be written 

l-(l+3)" 1-(1+Z)"-  

l + (H-3)  H-(l+z) -i' 
so the sum of the first n terms is  , — — . Thus, considering real values of z greater 

than - 1, it is seen that the sum to infinity is 1, 0, or — 1, according as z is negative, zero, 

or positive. There is thus a discontinuity at 2 = 0. This discontinuity is explained by the 

fact that the series is non-uniformly convergent near 2=0 ; for the remainder after n terms 

in the series when z is positive is 

-2 

r+rr+2)"' 

and, however great n may be, by taking z = ~, this can be made numerically greater 

2 

than -  , which is not arbitrarilv small. The series is therefore non-uniformlv con- 
1-1-e' 

vergent near 2 = 0. 

3'33. The distinction between absolute and uyiiform convergence. 

The uniform convergence of a series in a domain does not necessitate 
its absolute convergence at any points of the domain, nor conversely. Thus 

the series S vz  r converges absolutely, but (near z = 0) not uniformly ; 

(1 +  ")" 

while in the case of the series 

 =i z  + n ' 
the series of moduli is 

1 

5", = 



 =i \ n + Z'\ \   
which is divergent, so the series is only conditionally convergent; but for all 
real values of z, the terms of the series are alternately positive and negative 
and numerically decreasing, so the sum of the series lies between the sum of 
its first n terms and of its first (n -f 1) terms, and so the remainder after 
n terms is numerically less than the nth. term. Thus we only need take a 
finite number (independent of z) of terms in order to ensure that for all real 
values of z the remainder is less than any assigned number e, and so the 
series is uniformly convergent. 

Absolutely convergent series behave like series with a finite number of 
terms in that we can multiply them together and transpose their terms. 

* This value of x satisfies the condition i a; | < 5 whenever 2rt - 1 > log 5~i. 



3'33-3"341] CONTINUOUS functions and uniform convergence 49 

Uniformly convergent series behave like series with a finite number of 
terms in that they are continuous if each term in the series is continuous 
and (as we shall see) the series can then be integrated term by term. 

334. A condition, due to Weierstrass* , for uniform convergence. 

A sufficient, though not necessary, condition for the uniform convergence 
of a series may be enunciated as follows : — 

If, for all values of z within a domain, the moduli of the terms of a series 
*S' = Ui (z) + i<2 ( ) + W3 ( ) + • • • 
are respectively less than the corresponding terms in a convergent series 
of positive terms 

where M  is independent of z, then the series S is uniformly convergent in 
this region. This follows from the fact that, the series T being convergent, 
it is always possible to choose n so that the remainder after the first n terms 
of T, and therefore the modulus of the remainder after the first n terms 
of *Si, is less than an assigned positive number e ; and since the value of n 
thus found is independent of z, it follows (§ 3-31) that the series S is uni- 
formly convergent ; by § 234, the series S also converges absolutely. 

Example. The .scries 

1 ., 1 , 

cos Z +  , CO.S- 2+57, CO.S-* Z-  ... 

is uniformly convergent for all re<il values of r, because the moduli of its terms are not 
greater than the corresi)onding terms of the convergent series 

I i 
whose terms are positive constants. 

3 •341 . Uniformity of convergence of infinite products t. 

A convergent product n  1 + ?<  (z)  is said to converge uniformly in a domain of values 

M=l 

of z if, given e, we can tind m independent of z such that 

n  1 +   (z)  - n  i+u  z)  \ < € 

for all positive integral values of p. 

The only condition for imiformity of convergence which will be used in this work 
is that the product converges uniformly if | m (s) | < J/  where J/  is independent of 2 and 

2 J/  convei-ges. 

n = l 

* Abhandlungen aus der Funktionenlehre, p. 70. The test given by this condition is usually 
described (e.g. by Osgood, Annals of Mathematics, iii. j[1889), p. 130) as the M-test. 

t The definition is, effectively, that given by Osgood, Funktionentheorie, p. 462. The 
condition here given for uuiformity of convergence is also established in that work. 

W. M. A. 4 



50 THE PROCESSES OF ANALYSIS [CHAP. Ill 

To prove the validity of the condition we observe that n (l + J/ ) converges (§ 2-7), 

M = l 

and so we can choose m such that 

Vl+p VI 

n  l + M,, - n  l + M  <€; 

n = l )( = 1 

and then we have 

m+p m I I m p ni+p ~] I 

n  l+un (.-)  - n  1 +t<  (s)  =1 n  1 +   (z)  n  i +u  (z)  - 1 

11=1 M=l I I n=l \ \  n=m+i J I 

m r ni+2> ~] 

 n(i + 14) n  i+i/  -i 

m=l L n = m+l J 

and the choice of ?h is independent of z. 

3 "35. Hardy's tests for uniform convergence*. 



The reader will see, from § 2-31, that if, in a given domain. 



p 



2 a   z)   k where a  (2) is 



real and k is finite and independent of   and 2, and if / (2)   t + i (s) and fn  )-  

uniformly as w - - oo , then 2 a   z) f   £) converges uniformly. 
 t=i 

Also that if 
where k is independent of 2 and 2 a  (2) converges uniformly, then 2 a,j  £) w  (s) con- 

Ii = l M = l 

verges uniformly. [To prove the latter, observe that m can be found such that 

 7n + l(2),  m + l(2)+ 7  + 2(2), •••,  m + 1 ( ) +  m + 2 ( )+ • •   +  m + p (2) 

are numerically less than e\ k ; and therefore (§ 2-301) 

2 a  (2) M  (s) < e <,n+i (2)//(-< e, 
n=jn+l I 

and the choice of e and hi is independent of 2.] 



°  cos nQ "" sin %  
2 , 2 

7t=l '* n = l 'i 



Example 1. Shew that, if S>0, the series 

converge uniformly in the range 

S   (9   27r - S. 

Obtain the corresponding result for the series 

  (-)"cos?i<9 '  ( - )  sin n6 

2i , 2 , 

n=l n n=\ n 

by writing O + n for  . 

Example 2. If, when a a,' 6, | co,i (.1;) | <  -j and 2 | <   + i (.r) — co  (.r) | <j('2, where 

)(=i 

 •1, k.> are independent of n and .r, and if 2 a  is a convergent series independent of x, 

n=i 

then 2 a,tC > (:*;) converges uniformly when a  .r  / . (Hardy.) 

n = l 

* Proc. London Math. Soc. (2) iv. (1907), pp. 247-265. These results, which are generalisa- 
tions of Abel's theorem (§ 3-71, below), though well known, do not appear to have been published 
before 1907. From their resemblance to the tests of Dirichlet and Abel for convergence, 
Bromwich proposes to call them Dirichlet's and Abel's tests respectively. 



3"35, 3"4] CONTINUOUS FUNCTIONS AND UNIFORM CONVERGENCE 51 

[For we can choose m, independent of .v, such that 
corollary, we have 



m+p I 

2 a  < e, and then, l)y § 2-301 

n=m+l I 



m+p I 

2 ttnOin (  •) I < ( 'l + ' '2) f •] 



n=m+l I 

3 "4. Discussion of a particular double senes. 

Let (1)1 and w.  be any constants whose ratio is not purely real ; and let 
a be positive. 

The series 2 ;   r- , in which the summation extends over 

all positive and negative integral and zero values of in and n, is of great 
importance in the theory of Elliptic Functions. At each of the points 
z = — 2mcoi — 2/10)2 the series does not exist. It can be shewn that the series 
converges absolutely for all other values of   if a > 2, and the convergence is 
uniform for those values of z such that ; 2 + 2niQ)i + 2nco2     8 for all integral 
values of m and n, where 8 is an arbitrary positive number. 

Let S' denote a summation for all integral values of m and n, the term for 
which 7n = n = being omitted. 

Now, if ni and n are not both zero, and if \ z + 2m(o  + 2nco.2\ "  8 > for 
all integral values of m and n, then we can find a positive number C. de- 
pending on B but not on 2, such that 

I 1 ! 



(z + 2w\&)i + 2 a)a)'' 



 2m(Oi + 2710)2)" 



Consequently, by § 3"34, the given series is absolutely and uniformly* 
convergent in the domain considered if 

2' 1 

I mcoi + no)2 1 " 
converges. 

To discuss the convergence of the latter series, let 

0), = CTj 4- 1/3, , 0)0 = Qfo + z'/So , 

where a , a.,, /3i, /Sa are real. Since co../ coi is not real, a /S. — ou/3i 4= 0. Then 
the series is 

2'   

 (a,m + aojiy- +  /3,m + /SoTi)   * 

This converges (§ 2"5 corollary) if the series 

S =  ' i — - 

(m-+n2)   

converges ; for the quotient of corresponding terms is 

The reader will easily define uniformity of convergence of double series (see § 3-5). 

4—2 



52 THE PROCESSES OF ANALYSIS [cHAP. Ill 

where /z = njm. This expression, qua function of a continuous real variable jx, 
can be proved to have a positive minimum* (not zero) since ofi/3o — ao/3i =|= ; 
and so the quotient is always greater than a positive number K (independent 
of/x). 

We have therefore only to study the convergence of the series S. Let 

PI 1 

V   V S' 



 'p,q 



00 00 1 

 4 S S' - , . 

m = n=0 (m  + 71 )2 * 

Separating Sp g into the terms for which m = n, m > n, and m < n, re- 
spectively, we have 

pi p m-l 1 q n-1 1 

IS.   = S - I- s S - + s s . 

'%  1 ml 

But S r- < 



n=o  ni" + n ) "- (m ) " 



m° 



Therefore IS   t -J— + S — + i  . 

But these last series are known to be convergent if a — 1 > 1. So the series S 
is convergent if a > 2. The original series is therefore absolutely and uni- 
formly convergent, when a > 2, for the specified range of values of z. 

Example. Prove that the series 

1 

2 



(mj  + 7112  + . . . + my?y  

in which the summation extends over all positive and negative integral values and zero 
values of mj, m2, ... wi , except the set of simultaneous zero values, is absolutely convergent 
if fi>ir. (Eisenstein, Journal fur Math, xxxv.) 

3'5. The concept of uniformity. 

There are processes other than that of summing a series in which the idea 
of uniformity is of importance. 

Let e be an arbitrary positive number; and let f z,  ) be a function of 
two variables z and  , which, for each point z oi a, closed region, satisfies the 
inequality \ f z,  ) | < 6 when t, is given any one of a certain set of values 
which will be denoted by (  z) ; the particular set of values of course depends 
on the particular value of z under consideration. If a set ( )o can be found 
such that every member of the set ( )o is a member of all the sets ( j), the 
function f z,  ) is said to satisfy the inequality uniformly for all points z of 

* The reader will find no difficulty in verifying this statement ; the minimum value in 
question is given by 

K "' = h W + a  +  .  +  r-  (a,-/3,r-+(a2 + ,3i)=l   (ai + /32)2+ (a  - iP  ]- 



3"5, 3'6] CONTINUOUS FUNCTIONS AND UNIFORM CONVERGENCE 53 

the region. And if a function (f>  z) possesses some property, for every positive 
value of e, in virtue of the inequality \ f z, ):<e,(f) (z) is then said to possess 
the property uniformly. 

In addition to the uniformity of convergence of series and products, we shall have 
to consider uniformity of convergence of integrals and also uniformity of continuity ; thus 
a series is uniformly convergent when \ R,  z)\ <e, t( =  0 assuming integer values in- 
dependent of z only. 

Further, a function f z) is continuous in a closed region if, given e, we can find a 
I ositive number r/  such that 1/(2 +  2) —fi ) \ <   whenever 

0<\ C \ < r   
and 2 + f is a point of the region. 

The function will be uniformly continuous if we can find a positive number  ; inde- 
pendent of z, such that rjKr   and \ f z + C)~f  ) i <  whenever 

0<UI<'7 
and 2 + f is a point of the region, (in this case the set (f)o is the set of points whose 
moduli are less than r)). 

We shall find later (§ 3-61) that continuity involves uniformity of continuity; this is 
in marked contradistinction to the fact that convergence does not involve uniformity 
of convergence. 

36. The modified Heine-Borel theorem. 

The following theorem is of great importance in connexion with properties 
of uniformity ; we give a proof for a one-dimensional closed region*. 

Given (i) a straight line CD and (ii) a latv by which, corresponding to 
each point f P of CD, we can determine a closed interval I P) of CD, P being 
an interior  point of I (P). 

Tlien the line CD can be divided into a finite number of closed intervals 
Ji, Jo, ... Jk, such that each interval Jr contains at least one point  not an end 
point) Pr, such that no point of Jr lies outside the interval I (Pr) associated 
(by means of the given law) luitli that point Pr§. 

A closed interval of the nature just described will be called a suitable 
interval, and will be said to satisfy condition  A). 

If CD satisfies condition  A ), what is required is proved. If not, bisect CD ; 
if either or both of the intervals into which CD is divided is not- suitable, 
bisect it or them||. 

* A formal proof of the tlieorem for a two-dimensional region will be found in Watson's 
Complex Integration and Cauchy s Theorem (Camb. Math. Tracts, No. 15). 

t Examples of such laws associating intervals with points will be found in §§ 3'61, 5'13. 

t Except when P is at C or D, when it is an end point. 

§ This statement of the Heine-Borel theorem (which is sometimes called the Borel-Lebesgue 
theorem) is due to Baker, Proc. London Math. Soc. (2) i. (1904), p. 24. Hobson, The Theonj of 
Functions of a Real Variable (1907), p. 87, points out that the theorem is practically given in 
Goursat's proof of Cauchy's theorem  Trans. American Math. Soc. i. (1900), j). 14) ; the ordinary 
form of the Heine-Borel theorem will be found in the treatise cited. 

II A suitable interval is not to be bisected ; for one of the parts into which it is divided 
might not be suitable. 



54 THE PROCESSES OF ANALYSIS [CHAP. Ill 

This process of bisecting intervals which are not suitable either will 
terminate or it will not. If it does terminate, the theorem is proved, for CD 
will have been divided into suitable intervals. 

Suppose that the process does not terminate ; and let an interval, which 
can be divided into suitable intervals by the process of bisection just described, 
be said to satisfy condition (B). 

Then, by hypothesis, CD does not satisfy condition  B) ; therefore at least 
one of the bisected portions of CD does not satisfy condition  B). Take that 
one which does not (if neither satisfies condition  B) take the left-hand one) ; 
bisect it and select that bisected part which does not satisfy condition  B). 
This process of bisection and selection gives an unending sequence of intervals 
5o, Si, S2, ... such that : 

(i) The length of s  is 2-" Ci). 

(ii) No point of s,i+i is outside Sn- 

(iii) The interval s  does not satisfy condition (-4). 

Let the distances of the end points of s  from G be Xn, yn\ then 
Xn < a? +i < 2 i+i   Hn- Therefore, by § 2*2, x  and yn have limits ; and, by the 
condition (i) above, these limits are the same, say   ; let Q be the point whose 
distance from C is  . But, by hypothesis, there is a number hq such that 
every point of CD, whose distance from Q is less than 5q, is a point of the 
associated interval /(Q). Choose n so large that  t CDk 8q ; then Q is an 
internal point or end point of Sn and the distance of every point of Sn from 
Q is less than Sq. And therefore the interval 5  satisfies condition (A), which 
is contrary to condition (iii) above. The hypothesis that the process of 
bisecting intervals does not terminate therefore involves a contradiction ; 
therefore the process does terminate and the theorem is proved. 

In the two-dimensional form of the theorem* the interval CD is replaced by a closed 
two-dimensional region, the interval I P) by a circlet with centre P, and the interval 
Jj. by a square with sides parallel to the axes. 

3'61. Uniformity of continuity. 

From the theorem just proved, it follows without difficulty that if a 
function f(x) of a real variable x is continuous when a x b, then f(x) 
is unifurmly continuous  throughout the range a x  b. 

For let e be an arbitrary positive number ; then, in virtue of the con- 
tinuity of f x), corresponding to any value of x, we can find a positive 
number S , depending on x, such that 

1/( 0 -/( O I < e 

for all values of x' such that \ x' — x\ < Sx- 

* The reader will see that a proof may be constructed on similar lines by drawing a square 
circumscribing the region and carrying out a process of dividing squares into four equal squares. 

t Or the portion of the circle which lies inside the region. 

:J: This result is due to Heine; see Journal fiir Math. lxxi. (1870), p. 361, and lxxiv. (1872), 
p. 188. 



3'61, 3'62] CONTINUOUS functions and uniform convergence 55 

Then by § 3"G we can divide the range (a, b) into a finite number of closed 
intervals with the property that in each interval there is a number Xi such 

that \ fioc') — f xx) \ < -€, whenever x lies in the interval in which a;, lies. 

Let So be the length of the smallest of these intervals ; and let f , |' be 
any two numbers in the closed range (a, 6) such that |   —  ' | <  o- Then 
 , f ' lie in the same or in adjacent intervals ; if they lie in adjacent intervals 
let  , be the common end point. Then we can find numbers x , Xo, one in 
each interval, such that 

\ f  )-f  .)\ < \ e, ,/(?o)-/(- 0 < f> 

' /(r ) -/( e) \ < \ \  , fit) -fu  i < 5 6, 

so that 

i/(i) -/(r) = ,  /(B -fM] - [fit) -fM] 

- /(r)-/( -.)  +  /( o)-/( .) i 

< 6. 

If  ,  ' lie in the same interval, we can prove similarly that 

i/( )-/(r)i<2 - 

In either case we have shewn that, for (iny number   in the range, 
we have 

\ f( )-f  +0 <e 
whenever  +   is in the range and —Bo<  < Bq, where So is independent of  . 
The uniformity of the continuity is therefore established. 

Corollary (i). From the two-dimensional form of the theorem of § 3"6 we can prove 
that a function of a complex variable, continuous at all jwints of a closed region of the 
Argand diagram, is uniformly continuous throughout that region. 

Corollary (ii). A function f  x) which is continuous throughout the range a x b is 
hounded in the range ; that is to say we can find a number k independent of x such that 
\ f x) ! <K for all points x in the range. 

[Let n be the number of parts into which the range is divided. 

Let  ,  i,  2> ••• In-ij   be their end points ] then if x be any point of the rth interval 
we can find numbers Xi, x-i, ... Xn such that 

l/( )-/(- 'i)|< , l/(.i-i)-/'(li)|<i , l/( i)-/(' 2)|<ie, l/(. 2)-/( 2)|<if,... 

••• \ f  ryi)-f x)\ < h. 
Therefore \ f a)-f x) |< ire, and so 

which is the required result, since the right-hand side is independent of x."] 

The corresponding theorem for functions of complex variables is left to the reader. 

3'62. A real function, of a real variable, continuous in a closed interval, 
attains its upper bound. 

Let f x) be a real continuous function of x when a x b. Form a 
section in which the i?-class consists of those numbers r such that r >f x) 



56 THE PROCESSES OF ANALYSIS [cHAP. Ill 

for all values of x in the range (a, h), and the X-class of all other numbers. 
This section defines a number a such that f x) a., but, if h be any positive 
number, values of x in the range exist such that f(x)>a — 8. Then a is 
called the upper bound oi f x); and the theorem states that a number x' 
in the range can be found such thai f x) = a. 

For, no matter how small h may be, we can find values of x for which 
|/( ) — aj"  >'8~ ; therefore |  /(. ) - a| |~  is not bounded in the range; 
therefore (§ 3'61 cor. (ii)) it is not continuous at some point or points of the 
range ; but since | f x) — a | is continuous at all points of the range, its re- 
ciprocal is continuous at all points of the range (§ 3*2 example) except 
those points at which f(x) = a', therefore f x) = a at some point of the 
range ; the theorem is therefore proved. 

Corollary (i). The lower bound of a continuous function may be defined 
in a similar manner; and a continuous function attains its lower bound. 

Corollary (ii). If /( ) be a function of a complex variable continuous in 
a closed region, | f z) \ attains its upper bound. 

3'63. A real function, of a. real variable, continuous in a closed interval, 
attains all values between its upper and loiuer bounds. 

Let 31, m be the upper and lower bounds off x) ; then we can find numbers 
X, -v, by § 362, such that/( ) = M,f x) = m; let //. be any number such that 
m< fjb< M. Given any positive number e, we can (by § 3-61) divide the range 
(x, x) into 'A finite number, r, of closed intervals such that 

l/(.r,'-')-/(*2' ')|<6, 

where a i""*, iCjC* are any points of the rth interval; take  -i**"', x./'' to be 
the end points of the interval ; then there is at least one of the intervals 
for which /(*'!<' ') - f ,f(x.J ) — /x have opposite signs ; and since 

|[/( ,<'-')-/z - /( ,"-))- j|<6, 

it follows that j /(aa""') — /ii\ <  e. 

Since we can find a number  i'''* to satisfy this inequality for all values 
of 6, no matter how small, the lower bound of the function \ f(x)-fA,\ is 
zero ; since this is a continuous function of x, it follows from § 3-62 cor. (i) 
that/( ) — yLt vanishes for some value of . . 

3'64. The fluctuation of a function of a real variable*. 

Let/( ;) be a real bounded function, defined when a x b. Let 

a Xi X2  ... : Xn b. 
Then I /(a) -f(x,) \ + \ f(x,) -f(x,) | + ... + l/C J -f(b) | is called the 
fluctuation oi f x) in the range (a, b) for the set of subdivisions x , X2, ... Xn. 

The terminology of tliis section is partly that of Hobson, The Theory of Functions of a Real 
Variable (1907) and partly that of Young, The Theory of Sets of Points (190(5). 



3'63-37l] CONTINUOUS FUNCTIONS AND UNIFORM CONVERGENCE 57 

If the fluctuation have an upper bound FJ*, independent of n, for all choices of 
iCi, j-o, ... Xn, then f x) is said to have limited total fluctuation in the range 
(a, h). Fa!' is called the total fluctuation in the range. 

Example 1. If f x) be monotonic* in the range (a, b), its total fluctnation in the range 
is|/(a)-/(6)|. '>i    

Example 2. A fimction with limited total fluctuation can be expressed as the differ- 
ence of two positive increasing monotonic functions. 

[These function.s may be taken to be |  Fa' +fix) , h  Fa'-f x) .] 

Example 3. If f x) have limited total fluctuation in the range  a, b), then the limits 
f x±0) exist at all points in the interior of the range. [See § 3*2 example.] 

Example 4. li f x\ g x) have limited total fluctuation in the range (a, b) so has 
f x)g x). 

[For \ f x')g x')-f x)g x)\ \  \ f :>f). \ g x')-g x)\ + \ g  ) \ f  ')-fi. )l 
and so the total fluctuation of f x) g  x) cannot exceed g . FJ'+f. G K where   g are the 
upper bounds of |/( ) |, \ g (x) \ .] 

- 3'7. Uniformity of convergence of power series. 
Let the power .series 

ao + tti ;-!- ... +an2"+ ••• ' 

converge absolutely when z = Zo. 

Then, if |   |   |  'o | , , a " |   | a Zo" | . 

00 CO 

But since S | anZj" ' converges, it follows, by § 3-34, that S ttn " converges 

w = ' 71 = 

uniformly with regard to the variable z when \ z  \ Zq.  

Hence, by § 3"32, a power series is a continuous function of the variable 
throughout the closed region formed by the interior and boundary of any 
circle concentric with the circle of convergence and of smaller radius (§ 2"6). 

  3'71. Abel's theoreni-'r on continuity up to the circle of convergence. 

00 

Let S a,j " be a po ver series, whose radius of convergence is unity, and 

M = 

cc 

let it be such that S a  converges ; and let   a;   1 ; then Abel's theorem 

>i = 

(OC \ 00 

S    '*] = S an. 
. n=0 J M=0 

For, with the notation of § 3-35, the function x  satisfies the conditions 

00 

laid on u,  x), when 0 a; l; consequently /(a:;) = 2 On " converges uni- 

M = 

* The function is monotonic if  f x)-f x')\ \  x-x') is one-signed or zero for all pairs of 
different values of .r and x' . 

t Journal fiir Math. i. (1826), pp. 311-339, Theorem iv. Abel's proof employs directlj' the 
arguments by which the theorems of § 3-32 and § 3-35 are proved. In the case when S | a  I 
converges, the theorem is obvious from § 3-7. 



58 THE PROCESSES OF ANALYSIS [CHAP. Ill 

formly throughout the range  a;:  1 ; it is therefore, by § 3"32, a continuous 
function of x throughout the range, and so lim f x)=f l), which is the 

a;-*l-0 

theorem stated. 

3'72. Abel's theorem* on multiplication of series. 

This is a modification of the theorem of § 2"53 for absolutely convergent 
series. 

Let Cn = ttohn +  i \& \ i + . . . + an  0 • 

Then the convergence of "  an, S bn and 2 c  is a sufficient condition that 

M=0 n=0 w=0 

 00 \ / 00 \ 00 

For, let , 

 (a;)= 2 an   B(x)= 2 6na ", C(a.')-= i CnX''. 
n=0 n=0 M=0 

Then the series for A (x), B x), C(x) are absolutely convergent when 
I   I < 1, (§ 2-6) ; and consequently, by § 2-53, 

A(x)B(x)=C x) 
when <  < 1 ; therefore, by § 2-2 example 2, 

  lim A x)]  lim B(x)  =   lim C(x)\ 

a; .l-0 a;- -l-0 x -l-O 

provided that these three limits exist; but, by § 3-71, these three limits are 

00 00 30 

2 a , 'Z bn, % Cn', and the theorem is proved. 

n=0 w = w=0 

3'73. Power series which vanish identically. 

If a convergent poiver seynes vanishes for all values of z such that \ z\  r- , 
where r  > 0, then all the coefficients in the power series vanish. 

For, if not, let a  be the first coefficient which does not vanish. 

Then am +  '7/t+i  +  m+2' "+ ... vanishes for all values of z (zero excepted) 
and converges absolutely when \ z\ \  r<r ; hence, if s = a +i + a,n+2Z + . . ., we 
have 

00 

I S I   Z, I (ljn-|-7i I T , 
>j = l 

and so we can findf a positive number S r such that, whenever |   |   S, 

I , 2 \ L I 1 ' I 

I ( m+i   + ( f"m+2  + • • • 1 2'  '" I ' 

and then | a  + s |   |  m | - ! •' ' >   i a.m\, and so   a,  + s =f when \ z \ < S. 

* Journal filr Math. i. (1826), pp. ;:ili-339, Theorem vi. This is Abel's original proof. In 
some text-books a more elaborate proof, by the use of Cesaro's sums (§ 8-43), is given. 



372,373] CONTINUOUS FUNCTIONS AND UNIFORM CONVERGENCE 59 

We have therefore arrived at a contradiction by supposing that some 
coefficient does not vanish. Therefore all the coefficients vanish. 

Corollary 1. "We may 'equate corresponding coefficients' in two power series whose 
sums are equal throughout the region 1  |<S, where 8>0. 

Corollary 2. We may also equate coefficients in two power series which are proved 
equal only when z is real. 



REFERENCES. 

T. J. 1'a. Bromwich, Theory of Infinite Series (1908), Ch. vii. 

E. GouRSAT, Cours d'Analyse (Paris, 1910, 1911), Chs. i, xiv. 

C J. DE LA Vall e Poussin (Louvain and Paris, 1914), Cours d  Analyse Infinitesimale, 
Introduction and Ch. viii. 

G. H. Hardy, A course of Pure Mathematics (1914), Ch. v. 

VV. F. Osgood, Lehrbiich der Funktionentheorie (Leipzig, 1912), Chs. ii, iii. 

G. N. Watson, Complex Integration and Cauchy's Theorem (Camb. Math. Tracts, 
No. 15), (1914), Chs. I, II. 



Miscellaneous Examples. 

1. Shew that the series 



 =l(l-2 )(l-2  l) 

is equal to ,-, —. when 1 2 I < 1 and is equal to ,, -, when 1 2 1 > 1. 

  zy   z  z)- 

Is this fact connected with the theory of uniform convergence ? 

2. Shew that the series 

2sini + 4sinl + ... + 2"sin  + ... 

converges absolutely for all values of z  z = excepted), but does not converge uniformly 
near 2=0. 

3. If Un  x)=-2 n-\ f .re-'"-''' ' +  n xe''' '''', 

shew that 2 ?<   x) does not converge uniformly near x=0. (Math. Trip., 1907.) 

n = \ 

4. Shew that the series —pr j- + —r —... is convergent, but that its square (formed 

by Abel's rule) 

T  2+1,73 + 2; U'4' vW - 

is divergent. 

5. If the convergent series 5=— — — +—-—+... (r>0) be multiplied by itsel, 

the teiTus of the product being arranged as in Abel's result, shew that the resulting series 
diverges if r  | but converges to the sum s  \ i r>.  (Cauchy and Cajori.) 



60 THE PROCESSES OF ANALYSIS [CHAP. Ill 

6. If the two conditionally convergent series 

2   -   and 2     



n=i W   =i n' 

where r and s lie between and 1, be multiplied together, and the product an-anged as in 
Abel's result, shew that the necessary and sufficient condition for the convergence of the 
resulting series is r + s >1. (Cajori.) 

7. Shew that if the series 1 - 3 + 5 - t + • • • 

be multiplied by itself any number of times, the terms of the product being arranged as 
in Abel's result, the resulting series converges. (Cajori.) 

8. Shew that the qth. power of the series 

 ! sin d+a  sin 2  + ... +a,i sin n6 + ... 
is convergent whenever 5- (1 - r)< 1, r being the greatest number satisfying the relation 

for all values of n. 

9. Shew that if 6 is not equal to or a multiple of 2n, and if %0) Ui, u , ...he a 
sequence such that u - O steadily, then the series 2i/  cos  nd + a) is convergent. 

Shew also that, if the limit of t<  is not zero, but ?i  is still monotonic, the sum of the 

S 

series is oscillatory if - is rational, but that, if - is irrational, the sum may have any value 

TT TT 

between certain bounds whose difference is a cosec  6, where a= lim u - 

(Math. Trip., 1896.) 

