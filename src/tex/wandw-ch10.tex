%
% 194
%
\chapter{Linear Differential Equations}

\Section{10}{1}{Linear Differential Equations\footnote{The analysis contained in this chapter is mainly theoretical; it
    consists, for the most part, of existence theorems. It is assumed that
    the reader has some knowledge of practical methods of solving
    differential equations; these methods are given in works exclusively
    devoted to the subject, such as
    Forsyth, A Treatise on Differential Equations (1914). %TODO:formatting
  }.
  Ordinary points and singular points.}

In some of the later chapters of this work, we shall be concerned with
the investigation of extensive and important classes of functions
which satisfy linear differential equations of the second order.
Accordingly, it is desirable that we should now establish some general
results concerning solutions of such differential equations.

The standard form of the linear differential equation of the second
order will be taken to be
\begin{equation}
  \frac{\dmeasure^{2} u}{\dmeasure z^{2}}
  + p(z) \frac{\dmeasure u}{\dmeasure z}
  + q(z) u
  = 0
\end{equation}
and it will be assumed that there is a domain $S$ in which both
$p(z), q(z)$ are analytic except at a finite number of poles.

Any point of $S$ at which $p(z), q(z)$ are both analytic will be called
an \emph{ordinary point} of the equation; other points of $S$ will be called
\emph{singular points}.

\Section{10}{2}{Solution\footnote{This method is applicable only to equations of the second order.
    For a method applicable to equations of any order, see
    Forsyth, Theory of Differential Equations, iv. (1902), Ch. I. %TODO:formatting
    } of a differential equation valid in the vicinity of an ordinary point.}

Let $b$ be an ordinary point of the differential equation, and let $S_{b}$ be
the domain formed by a circle of radius $r_{b}$, whose centre is $b$, and its
interior, the radius of the circle being such that every point of $S_{b}$ is
a point of $S$, and is an ordinary point of the equation.

Let $z$ he a variable point of $S_{b}$.

In the equation write
$u = v \exp \thebrace{
  - \half \int_{b}^{z} p(\zeta) \dmeasure \zeta
}$,
and it becomes
\begin{equation}
  \frac{\dmeasure^{2} v}{\dmeasure z^{2}}
  + J(z) v
  = 0
\end{equation}
where
$$
J(z)
=
q(z)
- \half \frac{\dmeasure p}{\dmeasure z}
- \frac{1}{4} \thebrace{p(z)}^{2}.
$$
%
% 195
%

It is easily seen (\hardsubsectionref{5}{2}{2}) that an ordinary point of equation (A) %TODO:ref
is also an ordinary point of equation (B). %TODO:ref

Now consider the sequence of functions $v_{n}(z)$, analytic in $S_{b}$,
defined by the equations
\begin{align*}
  v_{0}(z)
  =& a_{0} + a_{1} (z-b),
  \\
  v_{n}(z)
  =&
  \int_{b}^{z} (\zeta - z) J(\zeta) v_{n-1}(\zeta) \dmeasure \zeta,
  \quad
  (n=1,2,3,\ldots)
\end{align*}
where $a_{0}, a_{1}$ are arbitrary constants.

Let $M, \mu$ be the upper bounds of $\absval{J(z)}$ and $\absval{v_{0}(z)}$ in the domain
$S_{b}$. \emph{Then at all points of this domain}
$$
\absval{v_{n}(z)}
\leq
\mu M^{n} \absval{z-b}^{2n} / (n!).
$$

For this inequality is true when $n=0$; if it is true when $n=0,1,\ldots,m-1$,
we have, by taking the path of integration to be a straight
line,
\begin{align*}
  \absval{v_{m}(z)}
  &=
  \absval{
    \int_{b}^{z} (\zeta-z) J(\zeta) v_{m-1}(\zeta) \dmeasure \zeta
  }
  \\
  &\leq
  \frac{1}{(m-1)!}
  \int_{b}^{z}
  \absval{\zeta-z} \cdot \absval{J(\zeta)}
  \mu M^{m-1}
  \absval{\zeta-b}^{2m-2} \cdot \absval{\dmeasure \zeta}
  \\
  &\leq
  \frac{1}{(m-1)!}
  \mu M^{m} \absval{z-b}
  \int_{0}^{\absval{z-b}} t^{2m-2} \dmeasure t
  \\
  &<
  \frac{1}{m!} \mu M^{m} \absval{z - b}^{2m}.
\end{align*}

Therefore, by induction, the inequality holds for all values of $n$.

Also, since $\absval{v_{n}(z)} \leq \frac{\mu M^{n} r_{b}^{2n}}{n!}$ when $z$ is in $S_{b}$, and
$\sum_{n=0}^{\infty} \mu M^{n} r_{b}^{2n}/(n!)$ converges,
it follows (\hardsubsectionref{3}{3}{4}) that
$v(z) = sum_{n=0}^{\infty} v_{n}(z)$ is a series of analytic functions
uniformly convergent in $S_{b}$; while, from the definition of $v_{n}(z)$,
\begin{align*}
  \frac{\dmeasure}{\dmeasure z} v_{n}(z)
  &= -\int_{b}^{z} J(\zeta) v_{n-1}(\zeta) \dmeasure \zeta,
  \quad (n = 1,2,3,\ldots)
  \\
  \frac{\dmeasure^{2}}{\dmeasure z^{2}} v_{n}(z)
  &= -J(z) v_{n-1}(z);
\end{align*}
hence it follows (\hardsectionref{5}{3}) that
\begin{align*}
  \frac{\dmeasure^{2} v(z)}{\dmeasure z^{2}}
  =&
  \frac{\dmeasure^{2} v_{0}(z)}{\dmeasure z^{2}}
  +
  \sum_{n=1}^{\infty} \frac{\dmeasure^{2} v_{n}(z)}{\dmeasure z^{2}}
  \\
  =&
  - J(z) v(z).
\end{align*}

\emph{Therefore $v(z)$ is a function of $z$, analytic in $S_{b}$, which satisfies
the differential equation
$$
\frac{\dmeasure^{2} v(z)}{\dmeasure z^{2}} + J(z) v(z) = 0,
$$
%
% 196
%
and, from the value obtained for
$\frac{\dmeasure}{\dmeasure z} v_{n}(z)$, it is evident that
$$
v(b) = a_{0},
\quad
v'(b) = \thebrace{\frac{\dmeasure}{\dmeasure z}v(z)}_{z=b} = a_{1},
$$
where $a_{0}, a_{1}$ are arbitrary.}

\Subsection{10}{2}{1}{Uniqueness of the solution.}

If there were two analytic solutions of the equation for $v$,
say $v_{1}(z)$ and $v_{2}(z)$ such that
$v_{1}(b) = v_{2}(b) = a_{0}, v_{1}'(b) = v_{2}'(b) = a_{1}$, then,
writing $w(z) = v_{1}(z) - v_{2}(z)$, we should have
$$
\frac{\dmeasure^{2} w(z)}{\dmeasure z^{2}} + J(z) w(z) = 0.
$$

Differentiating this equation $n-2$ times and putting $z = b$, we get
$$
w^{(n)}(b) + J(b) w^{(n-2)}(b) + TODO
+ \cdots + J^{(n-2)}(b) w(b) = 0.
$$
Putting $n = 2, 3, 4, \ldots$ in succession, we see that all the
differential coefficients of $w(z)$ vanish at $b$; and so, by Taylor's
theorem, $w(z) = 0$; that is to say the two solutions
$v_{1}(z), v_{2}(z)$ are identical.

Writing
$$
u(z) = v(z) \exp \thebrace{
  - \half \int_{b}^{z} p(\zeta) \dmeasure \zeta
},
$$
we infer without difficulty that $u(z)$ is the only analytic solution
of (A) %TODO:ref
such that $u(b) = A_{0}, u'(b) = A_{1}$, where
$$
A_{0} = a_{0},
\quad
A_{1} = a_{1} - \half p(b) a_{0}.
$$

Now that we know that a solution of (A) %TODO:ref
exists which is analytic in $S_{b}$,
and such that $u(b), u'(b)$ have the arbitrary values
$A_{0}, A_{1}$, the simplest method of obtaining the
solution in the form of a Taylor's series is to assume
$u(z) = \sum_{n=0}^{\infty} A_{n} (z-b)^{n}$, substitute this
series in the differential equation and
equate coefficients of successive powers of $z-b$ to zero
(\hardsubsectionref{3}{7}{3}) to
determine in order the values of
$A_{2}, A_{3}, \ldots$ in terms of $A_{0}, A_{1}$.

%\begin{smallfont}
[Note. In practice, in carrying out this process of substitution, the
re;vder will find it much more simple to have the equation 'cleared of
fractions' rather than in the canonical form (A) %TODO:ref
of \hardsectionref{10}{1}. Thus the
equations in examples 1 and 2 %TODO:ref
below should be treated in the form in
which they stand; the factors
$1 - z^{2}, (z-2)(z-3)$
should \emph{not} be
divided out. The same remark applies to the examples of
\hardsectionref{10}{3}, \hardsubsectionref{10}{3}{2}.] %TODO:multiref
%\end{smallfont}

From the general theory of analytic continuation (\hardsectionref{5}{5}) it follows
that the solution obtained is analytic at all points of $S$ except at
singularities of the differential equation. The solution however is
\emph{not}, in general, 'analytic throughout $S$'
(\hardsectionref{5}{2} cor. 2, footnote), %TODO:refs
except at these points, as it may not be one-valued; i.e. it may not
return to the same value when $z$ describes a circuit surrounding one or
more singularities of the equation.
%
% 197
%

[The property that the solution of a linear differential equation is
analytic except at singularities of the coefficients of the equation
is common to linear equations of all orders.]

When two particular solutions of an equation of the second order are
not constant multiples of each other, they are said to form a
\emph{fundamental system}.

Example . Shew that the equation

(l-s2)w"-2.-w' + |i = has the fundamental system of solutions

i/ - 1 \ U ?2 \ JUL 4 \

Determine the general coefficient in each series, and shew that the
radius of con- vergence of each series is 1.

Example 2. Discuss the equation

 z-2) z-S)u"- 2z-5)u' + 2u = in a manner similar to that of example 1.

\Section{10}{3}{Points which are regular for a differential equation.}

Suppose that a point c of S is such that, although p z) or q z) or
both have poles at c, the poles are of such orders that (z - c)p(z), z
- c)'-q z) are analytic at c. Such a point is called a regular point*
for the differential equation. Any poles of p z) or of q (z) which are
not of this nature are called irregular points. The reason for making
the distinction will become apparent in the course of this section.

If c be a regular point, the equation may be written f

(z-cY' '', + (z-c).P(z-c)' +Q z-c)u O,

where P (z - c), Q z - c) are analytic at c; hence, by Taylor's
theorem,
TODO
where Pq, pi, ..., g, qi,  are constants; and these series
converge in the domain S,. formed by a circle of radius r (centre c)
and its interior, where r is so small that c is the only singular
point of the equation which is in Sc-

Let us assume as - formal solution of the equation

u = z- cY

1 + S an (z - cr

where a, a, a.,, ... are constants to be determined.

* The name 'regular point' is due to Thome, Journal fiir Math. lxxv.
(1873), p. 266. Fucbs had previously used the phrase ' point of
determinateness.' t Frobenius calls this the normal form of the
equation.

%
% 198
%

Substituting in the differential equation (assuming that the
term-by-term differentiations and multiplications of series are
legitimate) we get

(z-cY

a(a-l)-l- S an a + 7i)(a + n-l)(z-cY

+ z-cYP z-c).

a+ 1 an(a + n)(z-cy'

+ z-cYQ z-c)

1+ 1 ttniz-cT

= 0.

Substituting the series for P(z - c), Q z - c), multiplying out and
equating to zero the coefficients of successive powers of z - c, we
obtain the following sequence of equations :

oi- + h-l)a + qo = 0,

, (a + iy + (po-'i-)(o( + l) + qo + ap, + q, = 0, a, (a + 2y + (po -
1) ( + 2 ) + o] + a, (a + l)p, + q + ap, + q, = 0,

an (a -1- nf + (p,-l) ci-]- n) + qo]

+ 2 Qn-m ( + n - m) p,n + ?, + dpn + qn = 0. m=l

The first of these equations, called the indicial equation*,
determines two values (which may, however, be equal) for a. The reader
will easily convince himself that if c had been an irregular point,
the indicial equation would have been (at most) of the first degi'ee;
and he will now appreciate the distinction made between regular and
irregular singular points.

Let a = pi, a = p.2 be the roots i* of the indicial equation F(a) = a?
+ po-l)oi + q, = 0; then the succeeding equations (when a has been
chosen) determine a, a, ..., in order, uniquely, provided that F a +
n) does not vanish when ?i = 1, 2, 3, ...; that is to say, if a = pi,
that p is not one of the numbers TODO; and, if a =
p.., that p is not one of the numbers p - - 1, p., + 2,

Hence, if the difference of the exponents is not zero, or an integer,
it is always possible to obtain two distinct series which formally
satisfy the equation.

Example. Shew that, if ra is not zero or an integer, the equation

is formally satisfied by two series whose leading terms are

.*-' jl+.

-+

.:,

16(H-to) 7' " i ' 16(l-m)

determine the coefficient of the general term in each series, and shew
that the series converge for all values of z.

* The name is due to Cayley, Quarterly Journal, xxi. (1886), p. 326.

t The roots pi, po of the indicial equation are called the expotients
of the differential equation at the point c.

%
% 199
%

\Subsection{10}{3}{1}{Convergence of the expansion o/§ lO'S.}

If the exponents TODO, p are not equal, let p- be that one whose real
part is not inferior to the real part of the other, and let
TODO
then
TODO.
Now, by \hardsubsectionref{5}{2}{3}, we can find a positive number M
such that \ pn\ < Mr-\ \ qn\ < Mr-'\ p.j n + qn i < Mr-'\ where M is
independent of n; it is convenient to take if 1. Taking a = pi, we
see that

aAJMl±,<

M

<

M

\ F p, + ) \ r s+1 since |5 + 1 1 >1.

If now we assume 1 a,,, I < i/"r-" when n=\,2, ...m- 1, we get

ttm. -

2 a -t (pi + vi-t)pi-it qt] + piPm + qr.

F(p, + m)

t I a, t \ piPt + qt] + \ piPm + q,a ! + s (m - 1 - I \ vt

m- 1

m\ 8 +in\

m,M' r-"'+ \ 2 m-t)\ M' r

']

m" 1 1 + smr |

Since 1 1 + snr j 1, because R (s) is not negative, we get

m + 1

a, <

2m

jpn,.-m < j,/' r-'",

and so, by induction, | a i < M' r'" for all values of ?2.

If the values of the coefficients corresponding to the exponent po be
a (iz, ... we should obtain, by a similar induction,

\ an\ < M'Wr-'\

where k is the upper bound of |1- s|~\ \ l - s\~\ 1 1 - |-s |~S ...;
this bound exists when s is not a positive integer.

We have thus obtained two formal series

Wi (z) = z - C)P'

 2 (2 ) = - CY

1+ S aniz-cr

M = 1

1+ ian(z-cY

The first, however, is a uniformly convergent series of analytic
functions when \ z-c\ < rM~\ as is also the second when \ z-c\ < rM-'
K-\ provided

%
% 200
%

in each case that arg (z - c) is restricted in such a way that the
series are one-valued; consequently, the formal substitution of these
series into the left-hand side of the differential equation is
justified, and each of the series is a solution of the equation;
provided always that pi - p., is not a positive integer or zero*.

With this exception, we have therefore obtained a fundamental S3 stem
of solutions valid in the vicinity of a regular singular point. And by
the theory of analytic continuation, we see that if all the
singularities in S of the equation are regular points, each member of
a pair of fundamental solutions is analytic at all points of S except
at the singularities of the equation, which are branch- points of the
solution.

\Subsection{10}{3}{2}{Derivation of a second solution in the case when
  the difference of the exponents is an integer or zero.}

In the case when p - po, = s is a positive integer or zero, the
solution Wo z) found in § lO'Sl may break downf or coincide with Wi
z).

If we write u = w z), the equation to determine f is

(z - cy -r + \ 2 z - cf '-- - + (z-c)Piz-c) = 0, . dz- [ u ( z) ) dz

of which the general solution is

 = A-rB\ ~ - - ry- exp - / - - dz. dz

J W, Z)Y i J 2-C ]

= A+B

(z-c)-P°

= A+Bj (z- c)-P - p g (z) dz,

where A, B are arbitrary constants and g z) is analytic throughout the
interior of any circle whose centre is c, which does not contain any
singu- larities of P ( - c) or singularities or zeros of z - c)"''i w
z); also g (c) = 1.

00

Let g z)= + S gn.(z-cy\

n = l

Then, Hsi O,

 =A+Br\ l+i(z-cf ( z-c)-'-'dz

J [ w = l J

= A+B --(z- c)- - 'i - (z - cy-' + gs log (z - c)

\ S n=l n

+ % - (z - cY-' n = s+l n -s

* If Pi -p2 is a positive integer, k does not exist; if p = p2, the
two solutions are the same.

t The coefficient a/ may be indeterminate or it may be infinite; in
the former case to-, (z) will be a solution containing two arbitrary
constants qq' and aj; the series of which Ug is a factor will be a
constant multiple of wi (z).

%
% 201
%

Therefore the general solution of the differential equation, which is
analytic at all points of C (c excepted), is

Awi z) + B [gsti\ (z) log (z-c) + w (z)], where, by \hardsubsectionref{2}{5}{3}, w (z) = z
- c) p- \ - - + S h,, z - c)"l,

the coefficients hn being constants.

When s = 0, the corresponding form of the solution is

Aw,(z) + B w, (z) log (z-c) + (z- cY X hn (z - c)"

L n = l \

The statement made at the end of \hardsubsectionref{10}{3}{1} is now seen to hold in the
exceptional case Avhen s is zenj or a positive integer.

In the special case when gg = 0, the second solution does not involve
a logarithm.

The solutions obtained, which are valid in the vicinity of a regular
point of the equation, are called regular integrals.

Integrals of an equation valid near a regular point c may be obtained
practically by first obtaining w- z), and then determining the
coefficients in

30

a function u\ (z) = S hn z - c)''-+'*, by substituting lu (z) log (z -
c) + n\ z) in

the left-hand side of the equation and equating to zero the
coefficients of the various powers of z - c in the resulting
expression. An alternative method due to Frobenius* is given by
Forsyth, Treatise on Differential Equations, pp. 243-258.

Example L Shew that integrals of the equation

<Pu \ du -

regular near z - are

Wi z) = l+ 2

and ..,(.) log.- 2 . ( \ + \ +...+-j.

Verify that these series converge for all vahies of z. Example 2. Shew
that integrals of the equation

,d u, -.sdu 1

regular near 2 = are

'°'W = + .!.( 2.4...2,. )-

and .,(.)log. + 4 j( \ - - j ( \ \ + \ \ ...\ \ J..

Verify that these series converge when | . i < 1 and obtain integrals
regular near 3= L * Journal fur Math, lxxvi. (1874), pp. 214-224.

%
% 202
%

Example 3. Shew that the hypergeometric equation

z z) :i + c-(a->rh + ) z -ahu = Q dz dz

is satisfied by the hypergeometric series of \hardsubsectionref{2}{3}{8}.

Obtain the complete sohition of the equation when c = .

\Section{10}{4}{Solutions valid for large values of $z$.}

Let = l/5i; then a solution of the differential equation is said to be
valid for ' large values of | | ' if it is valid for sufficiently
small values of Ui |; and it is said that ' the point at infinity is
an brdinary (or regular or irregular) point of the equation ' when the
point j = is an ordinary (or regular or irregular) point of the
equation when it has been transformed so that Zi is the independent
variable.

Since

we see that the conditions that the point 2= oc should be (i) an
ordinary point, (ii) a regular point, are (i) that 2z - z~p 2), z*q z)
should be analytic at infinity \hardsubsectionref{5}{6}{2}) and (ii) that zp z), z-q (z)
should be analytic at infinity.

Example 1. Shew that every point (including infinity) is either an
ordinary point or a regular point for each of the equations

z l-z)-jj ->r c- a-vh + ) z - -ahu = 0,

(1- 2), - -2s +?i(?i + l) = 0,

where a, \&, c, n are constants.

Exam-pie 2. Shew that every point except infinity is either an
ordinary point or a regular point for the equation

 S-"+4"- ( '- ') =°'

where n is a constant.

Example 3. Shew that the equation

has the two solutions

,, .,, d' u du .

2\ 1 i i 3.4.5.6 1

 " 3' 7 ' 277 7 ' 2. 4. 7. 9 z

the latter converging when | s | > 1.

\Section{10}{5}{Irregular singularities and confluence.}

Near a point which is not a regular point, an equation of the second
order cannot have two regular integrals, for the indicial equation is
at most of the first degree; there may be one regular integral or
there may be none. We shall see later (e.g.\hardsectionref{16}{3}) what is the nature
of the solution near

%
% 203
%

such points in some simple cases. A general investigation of such
solutions* is beyond the scope of this book.

It frequently happens that a differential equation may be derived from
another differential equation by making two or more singularities of
the latter tend to coincidence. Such a limiting process is called
confluence; and the former equation is called a confluent form of the
latter. It will be seen in § lO'G that the singularities of the former
equation may be of a more complicated nature than those of the latter.

\Section{10}{6}{The differential equations of mathematical physics.}

The most general differential equation of the second order which has
every point except a, a, as, a and oo as an ordinary point, these
five points being regular points with exponents a, ySr at ar r = 1,
2, 3, 4) and exponents fjbj, /j,2 at X, may be verified f to be

dz'' Vti z-ar )dz \ rZi z-ayf T (z-ar)

r = \

where A is such that;): / i and i., are the roots of

and B, C are constants.

The remarkable theorem has been proved by Klein§ and B6cher|| that all
the linear differential equations which occur in certain branches of
Mathematical Physics are confluent forms of the special equation of
this type in which the difference of the two exponents at each
singularity is |; a brief investigation of these forms will now be
given.

If we put y9,. = a,. +, (r = l, 2, 8, 4) and write in place of z, the
last written equation becomes

d i U h- Adu (4 ar(ar + h) 1 ? + 2 r+C ] \ Q

r=l

* Some elementary investigations are given in Forsyth's Differential
Equations (1914). Complete investigations are given in his Theory of
Differential Equations, iv. (1902).

t The coefficients of - and u must be rational or they would have an
essential singularity dz

4 4

at some point; the denominators of p z), q z) must be H z-a ), n z-a )
respectively;

r=l J-=l

putting p(z) and q (z) into partial fractions and remembering that p
z) = 0(z- ), q z) - 0(z~-) as 1 2 - 00, we obtain the required result
without difficulty.

4

X It will be observed that mi, fJ-i are connected by the relation
M1+M2+ 2 (a,. + /3 ) = 3.

j=i

§ Ueher lineare Differentialgleichungen der zweiter Ordnung (1894), p.
40; see also Vorlesung

Uber Lamd schen Funktionen.

II Ueber die Reihenentwickelungen der Potentialtheorie (1894), p. 193.

%
% 204
%

where (on account of the condition yu.., - f i = 2)

\ r=l

This differential equation is called the generalised \\Lame\\ equation.

It is evident, on writing i = ao throughout the equation, that the
confluence of the two singularities a, a., yields a singularity at
which the exponents a, /3 are given by the equations

a + /3 = 2 ( ! + oo), a = a, (oc, + h) + a, (a, + |) + D,

where D = (Aa - + 2Ba + C)/ (ai - a. ) (a - a ) .

Therefore the exponent-difference at the confluent singularity is not
, but it may have any assigned value by suitable choice of B and C. In
like manner, by the confluence of three or more singularities, we can
obtain one irregular singularity.

By suitable confluences of the five singularities at our disposal, we
can obtain six types of equations, which may be classified according
to (a) the number of their singularities with exponent-difference |,
(6) the number of their other regular singularities, (c) the number of
their irregular singu- larities, by means of the following scheme,
which is easily seen to be exhaustive*:

(a)

(b)

(c)

(I)

3

1

Lame

(H)

2

1

Mathieu

(III)

1

2

Les:eudre

(IV)

1

1

Bessel

(V)

1

1

Weber, Hermite

(VI)

1

Stokes t

These equations are usually known by the names of the mathematicians
in the last column. Speaking generally, the later an equation comes in
this scheme, the more simple are the properties of its solution. The
solutions of (II)-(YI) are discussed in Chapters xv-xix of this work,
and J of (I) in Chapter xxiii. The derivation of the standard forms of
the equations from the generalised \\Lame\\ equation is indicated by the
following examples :

* For instance the arrangement (a) 3, h) 0, (c) 1 is inadmissible as
it would necessitate six initial singularities.

t The equation of this type was considered by Stokes in his researches
on Diffraction, Camb. Phil. Trans, ix. (1856), pp. 168-182; it is,
however, easily transformed into a particular case of Bessel's
equation (example 6, below).

t For properties of equations of type (I), see the works of Klein and
Forsyth cited at the end of this chapter; also Todhunter, The
Functions of Laplace, \\Lame\\ and Bessel (1875).

%
% 205
%

Example . Obtain \Lame's equation

r=l

(where h and n are constants) by taking

n, =0 = 03 = 04 = 0, 8B = n('/i- l) a, 4(7=/i 4, and making 04 - x .

Example 2. Obtain the equation

dc' \ rc- y dc 4c (c- 1) ~ '

(where a and j are constants) by taking ai = 0, 0-2 = 1, and making 3
= c/4 -x. Derive Mathieu's equation \hardsectionref{19}{1})

-7 + (a + 16jcos 2i) =

by the substitution f = cos 2.

Example 3. Obtain the equation

 *4./K\ JLW, 1 f!L( L)\ \ !!!!\ l \ \ n

c r " (r c- 1/ c/c * I f f - ij c (c- 1) ~ '

by taking

 i = 2=l, 3 = a4 = 0, ai = a.j = a3 = 0, a4 = |.

Derive Legendre's equation (§§ 15-13, 15'5)

by the substitution ( = z~\

Example 4. By taking, = (/.2 = 0, 01 = 02 = 03 = 04 = 0, and making
a3 = a4-*-x, obtain the equation

Derive Bessel's equation (\hardsubsectionref{17}{1}{1})

d u

dz " dz

gd u du,,

by the substitution C=z' .

Example 5. By taking i=0, 01 = 02 = 03 = 04 = 0, and making (/o = 3 =
a4-*-x, obtain the equation

. dhi du,,,,,

f +i5 +i( +*-if) =o.

Derive Weber's equation \hardsectionref{16}{5})

d u,,, x

  + ( +*- -') =o

by the substitution f= 2.

Example 6. By taking 0 =0, and making 0 -9- x ( = 1, 2, 3, 4), obtain
the equation

  + ( if+Ci) = 0. By taking

u = B,C+C\ r-v, B,C+C\ = IB,z), shew that

 d v dv,,,.

%
% 206
%

Example 7. Shew that the general form of the generalised \\Lame\\ equation
is un- altered (i) by any homographic change of independent variable
such that qo is a singular point of the transformed equation, (ii) by
any change of dependent variable of the type

'll = z-ar) V.

Example 8. Deduce from example 7 that the various confluent forms of
the generalised \Lame\ equation may always be reduced to the forms given
in examples 1-6.

[Note that a suitable homographic change of variable will transform
any three distinct points into the points 0, 1, qo .]

\Section{10}{7}{Linear differential equations ivith three singularities.}

d?u, du, . \ Let d ' dz' '

have three, and only three singularities*, a, h, c; let these points
be regular points, the exponents thereat being a, a!; /3, /3'; 7,
7'.

Then p z) is a rational function with simple poles at a, b, c, its
residues at these poles being 1 - a - a', 1 - /3 - /S', 1 - 7 - 7';
and as 2 co, p z)- 2z- is z- ). Therefore

  z - a z - z - c

andf a + a' + + y8' + 7 + 7' = 1.

In a similar manner

,, [aoL a-h) a-c) B 'ih-c) h-a) yy' (c - a) c - b) l z - a z - b z - c

1

X

 z - a) z- b) z - g) and hence the differential equation is

d-'u U- o- ' 1-/3-/3 1-7-7 ) du

dz' [ z - a z - b z - c \ dz

  oia' a-b ) (a-c) /3 ' (b -c)(b-a) 77 (c -a)(c- b) \ z - a z - b z~c
I

= 0.

 z- a) z - b) z- c) This equation was first given by Papperitzj,

To express the fact that u satisfies an equation of this type (which
will be called Riemann's P-equation), Riemann§ wrote

fa b c \ u = p a /3 7 z[. W /3' 7 J

* The point at infinity is to be an ordinary point.

t This relation must be satisfied by the exponents.

+ Math. Ann. xxv. (1885), p. 21B.

§ Abh. d. k. Ges. d. Wiss. zu Gottingen, vii. (1857). It wiU be seen
from this memoir that, although Kiemann did not apparently construct
the equation, he must have inferred its existence from the
bypergeometric equation.

%
% 207
%

The singular points of the equation are placed in the first row with
the corresponding exponents directly beneath them, and the independent
variable is placed in the fourth column.

Example. Shew that the hypergeometric equation

d-u,, .,, . c?

,,, d-u,, 7,, . du,

is defined by the scheme

j' 00 1 \

pJ a 2-.

[l-o h c - a-h j

10 71. Transformations of Riemanns P-equation. The two transformations
which are typified by the equations

a b

z-h

(I) iv l) [v k) z\ = P a -k -k-l y- l

; (

b

c

7 W

y3

7

a'

/3'

i

a

b

c

(II) P<a

a'



7

IS'

7'

c

a' + k '-k-l y'+l

J (a /5' 7' J

(where z, a, 61, Cj are derived from z, a, b, c by the same
homographic transformation) are of great importance. They may be
derived by direct transformation of the differential equation of
Papperitz and Riemann by suitable changes in the dependent and
independent variables respectively; but the truth of the results of
the transformations may be seen intuitively when we consider that
Riemann's P-equation is determined uniquely by a knowledge of the
three singularities and their exponents, and (I) that if
$$
TODO
$$
IS'
$$
TODO
$$
then Ui = i- -jj (-37) satisfies a differential equation of the second

order with the same three singular points and exponents a + k, a +k;
l3 - k - l,j3' - k - l; y+ I, y' + I; and that the sum of the
exponents is 1,

Az + B Also (II) if we write z = j -, the equation in z- is a linear
equation

of the second order with singularities at the points derived from a,
b, c by this homographic transformation, and exponents a, a; fS, ';
y, y thereat.

%
% 208
%

\Subsection{10}{7}{2}{The connexion of Riemanns P-equation with the hypergeometric equation.}

B) means of the results of § lO'Tl it follows that

 a - a /3 + a + y 7' - 7 J

where x =

a' - a /3' + a + 7 7' - 7 (z - a)(c- b)

(z - h) c - a)'

Hence, by \hardsectionref{10}{7} example, the solution of Riemann's P-equation can

always be obtained in terms of the solution of the hypergeometric
equation

1 1,7 r n,' -1, z - a) c- ))

whose elements a, 0, c, x are a+ + 7, a-t-p+7, i+a - a, j - 7 -

 z-b) c - a)

respectively.

\Section{10}{8}{Linear differential equations tuith two singularities.}

If, in \hardsectionref{10}{7}, we make the point c a regular point, we must have

, ',, aa(a-b)(a-c) /3/3' (b - c) (b - a) l\ \ y=0, 77' = and - 4. 1
\ Lv 1 j g

divisible by - c, in order that p (z) and q (z) may be analytic at c.
Hence a + a + + jS' = 0, aa = yS/S', and the equation is d-ii j 1 - a
- a' 1 + a + a'] du atx a - by u dz- \ z - a z - b ) dz z- of z - b)-
'

of which the solution is

. fz - aY -D fz-( u=A J + B

\ z -b) \ z - bj

that is to say, the solution involves elementary functions only. When
a - a, the solution is

. fz-aY fz - aY fz-a

bJ'

REFEREXCES.

L. FuCHS, Journal fur Math. Lxvi. (1866), pp. 121-160.

L. W. Thomis, Journal fur Math. Lxxv. (1873), pp. 265-291, Lxxxvii.
(1879), pp. 222-349.

L. ScHLESiNGER, Handbuch der linearen Differentialgleichungen.
(Leipzig, 1895-1898.)

G. Frobenius, Journal fur Math. Lxxvi. (1874), pp. 214-235.

G. F. B. Riemann, Ges. Math. Werke, pp. 67-87.

F. C. Kleix, Ueber Uneai-e Differentialgleichungen der zineiter
Ordnung. (Gottingen, 1894.)

A. R. Forsyth, Theory of Differential Equations, iv. (1902).

T. Craig, Differential Equations. (New York, 1889.)

E. Goursat, Coxirs d\ inalyse, 11. (Paris, 1911.)

%
% 209
%

Miscellaneous Examples.

1. Shew that two solutions of the equation

are 2 - 3 7 2* + ..., 1- 2 + ..., and investigate the region of
convergence of these series.

2. Obtain integrals of the equation

d" 1,, ON

regular near 2 = 0, in the form

. i = "r + r6 + 1024 + -/'

3

 2 = Wll0g2-jg+....

3. Shew that the equation

has the solutions

TODO

and that these series converge for all values of z.

4. Shew that the equation

dZ tr=l Z-ttr ) dz \ r=i z - a,.Y r=l*- rj

where

2 (a, + 3,) = -2, 2 Z), = 0, 2 (a, A-+ari3r) = 0, 2 (a,2/) + 2a,a,i3,)
= 0,

r=I r=I r=l r=l

is the most general equation for which all points (including x ),
except aj, a-.,, ... a, are ordinary points, and the points a are
regular points with exponents a, r respectively.

\addexamplecitation{Klein.}

5. Shew that, if /iJ + y + /3' + y' =, then

(0 X 1 I i-1 X 1 ~|

P\ 0 y 22' p' 2/3 y 2-. \addexamplecitation{Riemann.}

h \& y J i y' 23' y J

[The dififerential equation in each case is

c/22 + 2- 1 dz r 22 . IJ,2\ l ""J

6. Shew that, if y + y' = and if co, or are the complex cube roots of
unity, tfeen rO X 1 \ rl CO 0)2 j

pJo y 23- = p'y y y 2-. \addexamplecitation{Riemann.}

U i y J ly y 7' >'

[The differential equation in each case is

d' u 222 du yy zu,

C£22 23\ l dz z -\ f -

W. M. A. 1

%
% 210
%

7. Shew that the equation

(l\ .2)g\ (2o + i)2 Vn( + 2a) =

is defined b\ \ the scheme

pj -n z

\ jf-a n + 2a - a j and that the equation

may be obtained from it by taking a = l and changing the independent
variable.

8. Discuss the solutions of the equation

dhi,,, dii (,1

\addexamplecitation{Halm.}

'-U,, -.du (, . 1 \ + ( + l+m) + (/i + l+ mj w =

valid near s = and those valid near s = oo . \addexamplecitation{Cunningham.}

 22

9. Discuss the solutions of the equation

valid near 2=0 and those valid near z=<x, . Consider the following
special cases :

(i) pi= -% (ii) /i - 5, (iii) /x + i/ = 3.
\addexamplecitation{Curzon.}

10. Prove that the equation '

s(l-2) +-(l-22) + (a2 + fe)w =

has two particular integrals the product of which is a single-valued
transcendental function. Under what circumstances are these two
pax'ticular integrals coincident ? If their product be F z), prove
that the particular integrals are

where C is a determinate constant.
\addexamplecitation{Lindemann; see \hardsectionref{19}{5}.}

l. Prove that the general linear differential equation of the third
order, whose singularities are 0, 1, oc, which has all its integrals
regular near each singularity (the exponents at each singularity being
1, 1, - 1), is

dhi (2 2 \ d \ \ \ 3 1 \ "1 du

d W ) dz V z z-\ y z- If] dz \ \ 3cos2a 3sin2a 1 1 \ n

+ p - w ) " ¥ ' " m '

where a n y ha 'e any constant value. \addexamplecitation{Math. Trip. 1912.}
