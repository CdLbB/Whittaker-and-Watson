\chapter{The Jacobian Elliptic Functions} 

22"1. Elliptic functions with two simple poles.

In the course of pr6ving general theorems concerning elliptic
functions at the beginning of Chapter xx, it was shewn that two
classes of elliptic functions were simpler than any others so far as
their singularities were concerned, namely the elliptic functions of
order 2. The first class consists of those with a single double pole
(with zero residue) in each cell, the second consists of those with
two simple poles in each cell, the sum of the residues at these poles
being zero.

An example of the first class, namely z), was discussed at length in
Chapter xx ; in the present chapter we shall discuss various examples
of the second class, known as Jacobian elliptic functions*.

It will be seen (§ 22"122, note) that, in certain circumstances, the
Jacobian functions degenerate into the ordinary circular functions ;
accordingly, a notation (invented by Jacobi and modified by Gudermann
and Glaisher) will be employed which emphasizes an analogy between the
Jacobian functions and the circular functions.

From the theoretical aspect, it is most simple to regard the Jacobian
functions as quotients of Theta-functions (§ 21"61). But as many of
their fundamental properties can be obtained by quite elementary
methods, without appealing to the theory of Theta-functions, we shall
discuss the functions without making use of Chapter xxi except when it
is desirable to do so for the sake of brevity or simplicity.

22'11. The Jacobian elliptic functions, snu, cnu, dnu. It was shewn in
§ 21 "61 that if

y '\$,,% ul%'y the Theta-functions being formed with parameter t, then

where k = o (0 ! t)/ 3 (0 | r). Conversely, if the constant h (called
the modulus-f) be given, then, unless k- 1 or /c- O, a value of r can
be found

* These functions were introduced by Jacobi, but many of their
properties were obtained independently by Abel, who used a different
notation. See the note on p. 512.

t If 0</c<l, and is the acute angle such that sin d = k, 6 is called
the modular angle.



492 THE TRANSCENDENTAL FUNCTIONS [CHAP. XXII

(§§ 21-7-21-712) for which V (0 t) V (0 t) = A--, so that the sokition
of the differential equation

;|y=(i-yn(i-%')

subject to the condition ( -y ) = 1 is

the Theta-functions being formed with the parameter t which has been
determined.

The differential equation may be written

to=i\ l-t-)- - l-k-H')- -dt, Jo

and, by the methods of § 21-73, it may be shewn that, if y and a are
con- nected by this integral formula, y may be expressed in terms of u
as the quotient of two Theta-functions, in the form already given.

Thus, if

Jo y may be regarded as the function of u defined by the quotient of
the Theta- functions, so that y is an analytic function of u except at
its singularities, which are all simple poles ; to denote this
functional dependence, we write

y = sn(? k), or simply y= sn u, when it is unnecessary to emphasize
the modulus*. The function sn u is known as a Jacobian elliptic
function of ii, and

 °"=a;§:w5?) -

[Unless the theory of the Theta-functions is assumed, it is
exceedingly difficult to shew that the integral formula defines y as a
function of u which is analytic except at simple poles. Cf. Hancock,
Elliptic Functions, i. (New York, 1910).]

 ow write cnOa-) = ;( ;y (B),



''"("• > = a, aTWV) * >-

Then, from the relation of § 21"6, we have

 — sn i( = en M dn (/ (I),

du

* The modulus will alwaj's be iuserted when it is not k.



2'2'12, 22-121] THE JACOBIAN ELLIPTIC FUNCTIONS 493

and from the relations of § 21 '2, we have

sn- u + cn u = 1 (II),

k' sn- u + dn- u = l (Ill),

and, obviously, en -= dn = 1 (I )-

"We shall now discuss the properties of the functions sn u, en u, dn u
as defined bj' the equations (A), (B), (C) by using the four relations
(I), (II), (III), (IV) ; these four relations are sufficient to make
sn u, on u, dn u determinate functions of u. It will be assumed, when
necessary, that sn u, en m, dn u are one-valued functions of ti,
analytic except at their poles ; it will also be assumed that they are
one-valued analytic functions of k' when cuts are made in the plane of
the complex variable k from 1 to -f oo and from to — xi .

22'12. Simple pi'operties of sn u, en u, dn u.

From the integral u= j (1 — t-) (1 — k'-t-) 2 df it is evident, on
writing

Jo

— t for t, that, if the sign of y be changed, the sign of u is also
changed.

Hence sn u is an odd function of u.

Since sn (— ii) = — sn u, it follows from (II) that en (— u) = + en u;
on account of the one-valuedness of en u, by the theory of analytic
continuation it follows that either the upper sign, or else the lower
sign, must always be taken. In the special case u = 0, the upper sign
has to be taken, and so it has to be taken always; hence cn -u) = cnu,
and cnu is cm even function of u. In like manner, dn u is an even
function of u.

These results are also obvious from the definitions (A), (B) and (C)
of §2211.

Next, let us differentiate the equation sn- u + en- u = l ; on using
equation (I), we get

d en a T

— z — = — sn u dn u ; du

in like manner, from equations (III) and (I) we have

d dn u , — -, — = — A;- sn u en u. du

22"121. Tlie compleme)dary modulus.

If k- + k'- = 1 and •' -f 1 as k 0, k' is known as the complementary
modulus. On account of the cut in the --plane from 1 to -f x , k' is a
one- valued function of k.

[With the aid of the Theta-functions, we can make k' one-valued, by
defining it to be

Example. Shew that, if

J V then ?/ = en u, k).



494 THE TRANSCENDENTAL FUNCTIONS [CHAP. XXII

Also, shew that, if = [ \ l - -') - J ( 2 \ /.'2) - I dt

then 3/ = dn ( , /(•).

[These results are sometimes written in the form

J en ti J An It

22122. Glaiskers notation* for quotients.

A short and convenient notation has been invented by Glaisher to
express

reciprocals and quotients of the Jacobian elliptic functions ; the
reciprocals

are denoted by reversing the order of the letters which express the
function,

thus

ns u = 1/sn u, nc u = l /en u, nd = 1 /dn u ;

while quotients are denoted by writing in order the first letters of
the

numerator and denominator functions, thus

sc u = sn u/cn u, sd u — sn ? /dn u, cd u = en w/dn w, cs w = en w/sn
w, ds u = dn ujsn u, dc u = dn u/cn u.

[Note. Jacobi's notation for the functions sn rt, en u, dn u was sinam
u, cosam m, Aamw, the abbreviations now in use being due to
Gudermannt, who also wrote tuu, as an abbreviation for tanam u, in
place of what is now written sc u.

The reason for Jacobi's notation was that he regarded the inverse of
the integral

u= |*(l-Fsin2 )- rf

as fundamental, and wrote J ( = am w; he also wrote A = (l-/?-- sin c
)- for -j- .] Example. Obtain the following results :

J ./ cs u

= r' '\ i-k'-'fi)- ii+kH )-- dt= r f'-f )- t''+k r dt

Jo 7 ds H

= /' t )- l-kH )- dt =[' t''-l)- t -k )- dt

J cd i( J dc u

= / t- -l)- t -k )~ dt = ["'" t''- ) - - FH'-+k- )~ -dt

r Ud M 1 1

= 1 fi-i)-i i-r-f )- dt.

22'2. The addition-theorem for the function Qnu.

We shall now shew how to express sn (u + v) in terms of the Jacobian
elliptic functions of u and v ; the result will be the
addition-theorem for the function sn u ; it will be an
addition-theorem in the strict sense, as it can be written in the form
of an algebraic relation connecting sn u, sn v, sn u + v).

* Messenger of Mathematics, xi. (1882), p. 86. t Journal fiir Math,
xviii. (1838), pp. 12, 20. J Fundamenta Nova, p. 30. As k—*-Q, am )(-
-(/.



22'122, 22*2] the jacobian elliptic functions 495

[There are numerous methods of establishing the result ; the one given
is essentially due to Euler*, who was the first to obtain (in 1756,
1757) the integral of

in the form of an algebraic relation between x and , when X denotes a
quartic function of X and Y is the same quartic function of y.

Three t other methods are given as examples, at the end of this
section.]

Suppose that xi and v vary while u- v remains constant and equal to a,
say, so that

dv \ die

Now introduce, as new variables, Sj and Sg defined by the equations

Si — sn %i, Sn = sn v,

so thatj .V = (1 - s ) (1 -]es \

and 4- = (1 — s.f) (1 — I:? si), since xr = .

Differentiating with regard to u and dividing by 2si and 24
respectively, we find that, for general values § of u and v,

Si = - ( 1 + A;-) Si + IhH , if. = - (1 + A; 52 + Ih s-f.

Hence, by some easy algebra,

S\ Sn — S Sx Zk S1S2 \ Si — 2 )

Si's.:r-S./S,' ~ si-Sx ) l-k'Sx si) '

and so

(s,s, - s,s,)-' ~ (s,s, - s,s;) = (1 - k's.W)-' (1 - k's,%') ;

on integrating this equation we have

S1S2 s Si

1 — k 81 82'

where C is the constant of integration.

Replacing the expressions on the left by their values in terms of u
and v we get \

en ?i dn M sn y + en w dn w sn u 1 —k sn u sn v

* Acta Petropolitana, vi. (1761), pp. 35-57. Euler had obtaiued some
special cases of this result a few years earlier.

t Another method is given by Legendre, Fonctions Elliptiques, i.
(Paris, 1825), p. 20, and an interesting geometrical proof was given
bj- Jacobi, Journal fur Math. iii. (1828), p. 376. X For brevity, we
shall denote differential coefficients with regard to u by dots, thus

dv .. d v du dii

8 I.e. those values for which en u dn u and cnv nv do not vanish.



496 THE TRANSCENDENTAL FUNCTIONS [CHAP. XXII

That is to say, we have two integrals of the equation du + dv = 0,
namely ) u + v= a and (ii)

sn u en V dn V + sn ?; en M dn u



1 — k sn" u sn v



= G,



each integral involving an arbitrary constant. By the general theory
of differential equations of the first order, these integrals cannot
be functionally independent, and so

sn u en v dn v + sn v en u dn u 1 — A;- sn ii sn"-* v

is expressible as a function oi u + v ; call this function f(u + v).

On putting i; = 0, we see that /(u)=snw; and so the function / is the
sn function.

We have thus demonstrated the result that

sn tt en V dn V 4- sn v en u dn u

sn u + v) = j- — ,

1 — '- sn- u sn- V

which is the addition-theorem.

Using an obvious notation*, wc may write

. s.cdo + SoCidi

Example 1. Obtain the addition-theorem for sin ic by using the results

Example 2. Prove from first principles that

3 d\ SiC2d2 + s.2Cidi\



\ cv cuj



ycv cuJ 1 — k'-s -s-r

and deduce the addition-theorem for sn u.

(Abel, Journal fUr Math. n. (1827), p. 105.) Example 3. Shew that

s —8 \ S iCid + s Cidi SidiC +SjdiCi s c d - s Cidi CiC2 + Sidj Sod2
did2+khiS2CiC2'

(Cayley, Elliptic Functions (1876), p. 63.)

Example 4. Oljtain the addition-theorem for sn u from the results

 i(y-H ) 4(y- ) 253=-9,(y)54(y) 2( ) 3( )+- 2(y) 3(3/)5i( )54(4 4 y+z)
Si(2/-2) V= V (y) i' (z) - 1- ii/) l' (z),

given in Chapter xxi, Miscellaneous Examples 1 and 3 (pp. 487, 488).
(Jacobi.)

Example 5. Assuming that the coordinates of any point on the curve

?/- = (l-;f2)(l-y5;2a;2)

can be expressed in the form (sn u, en u dn u), obtain the
addition-theorem for sn u by Abel's method (§ 20-312).

* This uotation is due to Glaisher, Messenger, x. (1881), pj). 9'2,
124.



I



22*21] THE JACOBIAN ELLIPTIC FUNCTIONS 497

[Consider the intersections of the given curve with the variable curve
y = I + mx + n.v' ; one is (0, 1) ; let the others have parameters u ,
Uz, M3, of which u . xii may be chosen arbitrarily Vjy suitable choice
of m and n. Shew that U1 + U2 + U3 is constant, by the method of §
20-312, and deduce that this constant is zero by taking

 i = 0, = -|(1+F).

Observe also that, by reason of the relations

we have

 3(1 -k' Xi x. ) = X3- + ,2\ 2 ) mx x x - mxix.,- nxxX2 ( 1 + 2+ 3)

= xi+x-2- -x — nxi x-2 X3) — Xi + X2) — 2mxi Xo - nx Xo xi + 2)

22-21. The addition-theorems for en u and dn u. We shall now establish
the results

en M en w — sn w sn V dn w dn w

en (z< + v) = = Y, — 2 2 .

- . , dn dn V — k- sn w sn v en u en v dn (u + v) = rw- - . ;

the most simple method of obtaining them is from the formula for sn (u
+ v).

Using the notation introdueed at the end of § 22*2, we have (1 - k's
's.-y en- (u + v) = (l - k s s Y 1 - sn- (u + v)]

= (1 - k' S ' S y - (SjCaC o + SoCidiY

= 1 - 'Ik'-si's + k*s,*s.2' - sf (1 - si) (1 - k si)

— si (1 — si) (1 — k-si) — 2sxS.2CiC.2did2 = (1 - si) (1 - si) + sisi
(1 - k'si) (1 - k'si)

— 2siS2CiC.,dxd.2 = (ciCo - SySod d y

and so en (u + v) = + - — r ~ — •

• 1 — k-sisi

But both of these expressions are one- valued funetions of u, analytic
exeept at isolated poles and zeros, and it is ineonsistent with the
theory of analytic continuation that their ratio should be + 1 for
some values of u, and — 1 for other values, so the ambiguous sign is
really definite ; putting li = 0, we see that the plus sign has to be
taken. The first formula is consequently proved.

The formula for dn u + v) follows in like manner from the identity (1
- k' sisiy - k (s CoJz + s.Cxdn y

= (1 - k'-si) (1 - fi'si) + k*sisi (1 - si) (1 - si) -
2k's,SoC,C2d,d2,

the proof of which is left to the reader.

w. ii. A. 32



498 THE TRANSCENDENTAL FUNCTIONS [CHAP. XXII

Example 1. Shew that

du u + v) dn u-v) = -j j i ' •

(Jacobi.) [A set of 33 formulae of this nature connecting functions of
u + v and of - v is given in the Fundamenta Nova, pp. 32-34.]

Example 2. Shew that

8 cnu + cnv \ 9 cnw+cnv

cu sn M dn V + sn V dn ?< ov sn i dn v + sn i; dn u '

so that (en r(+cn v)/(snMdn?' + sn vdnii) is a function of i + v only;
and deduce that it is

equal to 1 +cn u + v) lsvi u + v).

Obtain a corresponding result for the function ( iC9 +
S2Ci)/(c?j+(a?2).

(Cayley, Messenger, xiv. (1885), pp. 56-61.) Example 3. Shew that

1 - Fsn2 u + v) sn2 u-v) = Fsn* u) l- F sn* v) (1 - Fsn2 ?< sn2 y) -2,
'- + >fc2cn2(?< + v) (in u~v) = k" + k-cn*u) (/ •'2+Fcn' r) (1
-Fsn2%sn2w)-2.

(Jacobi and Glaisher.)

Example 4. Obtain the addition-theorems for cn(-w + r), dn(w-|-'i?) by
the method of § 22-2 example 4.

Example 5. Using Glaisher's abridged notation Messenger, x. (1881), p.
105), namely

s, c, d=s\ \ u, en u, dn u, and >S', C, Z) = sn 2i , en 2i(, dn 2 ,
prove that

2gCt l-2g2 .2g4 1 - 2/t2s2 + y[;y

l-> ;254' - l\ y[,.254 ' - 1 L y[ s4- '

(i+ )4\ (i\ i



(l + -?:,S') + (l-Z-,S') ' Example 6. With the notation of example 5,
shew that

1-C D D-BC-k- D-G



   =



l+Z>~F(l + (7) k D-G) k' '+D-BC \ D + C \ D + k- C-k ' \ /;'2(i-Z)) \
/j:'2(i + (7) ~l-|-i>~ k: l + G)' ~ k- D -C)~ k"- + D- PC

k' +D + k' G \ D+C \ k' C) k' \ + D ) \ + D "l + C" D~C ~k' + D-k- G'



(Glaisher.)



22'3. The constant K. We have seen that, if

u=\ \ l- )- ~ l-kH'')- ~dt,

J then y = sn u, k).

If we take the upper limit to be unity (the path of integration being
a straight line), it is customary to denote the value of the integral
by the symbol K, so that sn K, k)= 1.

[It will be seen in § 22*302 that this detinition of A' is equivalent
to the definition as TT a in §21-61.]



22 3-22 '302] the jacobian elliptic functions 499

It is obvious that en = and dn K= ± k' : to fix the ambiguity in sign,
suppose < A' < 1, and trace the change in (1 — k'H-) as t increases
from to 1 ; since this expression is initially unity and as neither of
its branch points (at t = ± k~ ) is encountered, the final value of
the expression is positive, and so it is + A ' ; and therefore, since
dn is a continuous function of k, its value is always + k'.

The elliptic functions of K are thus given by the formulae sniT-l, cn
= 0, dn K = k'.

22'301. The expression of K in terms of k.

In the integral defining K, write t = sin < , and we have at once

K \ (l-A-2sin-(/))- cZ(/>.

J

When \ k\ < \, the integi-and may be expanded in a series of powers of
k, the series converging uniformly with regard to (by § 3'34, since
sin-" ( 1 ) ; integrating term-by-term (§ 4*7), we at once get

K=\ \ F \, \; 1; k =\ \ Fl\, 1;



where c = k" . By the theory of analytic continuation, this result
holds for all values of c when a cut is made from 1 to -l- oo in the
c-plane, since both the integrand and the hypergeometric function are
one-valued and analytic in the cut plane. Example. Shew that

(Legendre, Fonctions Elliptiques, i. (1825), p. 62.)

22'302. The eqvAvalence of the definitions of K-

Taking = 3- in § 21-61, we see at once that \&n hTrd ) = \ and so
cn(|7r53-) = 0. Consequently, 1 — sn m has a double zero at \ 7r i .
Therefore, since the number of poles of sn w in the cell with corners
0, nS , n t- ) B , tt (r - 1) .93'- is two, it follows from § 20'13
that the only zeros of 1 — sum are at the points = Jn-
(4?n-l-l-|-2?ir) 3 , where m and n are integers. Therefore, with the
definition of § 22'3,

A'=|7r(4m-l-l+2 r)532.

Now take t to be a pure imaginary, so that < /(• < 1, and K is real ;
and we have ?i = 0, so that

\ TT Am + l)B = j ''(l->t2sin2 0)~2(/ ,

where m is a positive integer or zero ; it is obviously not a negative
integer.

If m is a positive integer, since / (1 - k' sin 4>)~ i * continuous
function of a and

J so passes through all values between and K as a increases from to
tt, we can find a value of a less than tt, such that

A7(4?ra -I- 1 ) = hrrS = f " ( 1 - P sin2 cj))- dc ;

and so sn ( TrSs ) = sin a < 1 ,

which is untrue, since sn (-177 32) = 1.

32—2



500 THE TRANSCENDENTAL FUNCTIONS [CHAP. XXIT

Therefore m must he zero, that is to say we have

But both K and n - are analytic functions of /• when tlie o-plane is
cut from 1 to + 00, and so, by the theory of analytic continuation,
this result, proved when 0< -<l, persists throughout the cut plane.

The equivalence of the definitions of K has therefore been
established. Example 1. By considering the integral

J shew that sn 2K= 0.

Example 2. Prove that

sn A'=(l +/ ;')" S cnhK=M \ + k')' , dn|A' = /CA

[Notice that when u = \ K, cn2M = 0. The simplest way of determining
the signs to be attached to the various radicals is to make --a-O,
X'-a-l, and then sn ?<, en w, dn u degenerate into sin u, cos u, 1.]

Example 3. Prove, by means of the theory of Theta-functions, that

cs iA'=dn \ K=k' .

22*31. The periodic proper ties (associated with K) of the Jacobian
elliptic functions.

The intimate connexion of K with periodic properties of the functions
snu, en 11, dnu, which may be anticipated from the periodic properties
of

Theta-functions associated with - tt, will now be demonstrated
directly from

the addition-theorem.

By § 22"2, Ave have

 , snucnK dnK + iiKcn.udnu ,

sn(u + K)— ; — y ., — , -—77 = cd u.

  \ - k' sn- a sn K

In like manner, from § 22*2],

en (it + K) = — •' sd u, dn (a + ii ) = ' nd u.

TT / , Tj x cn((t + / ) •'sdw

Hence sn ( (i 4- 2it ) = , - 7 ; = — .-> — ~ = — sn u,

  dn(u-l-ir) A;ndM

and, similarly, en u -f IK) = — en u, dn it + 'IK) = dn u.

Finally, sn u -h K) = — sn u -f 2K) = sn u, en (?( -I- 4 A") = en u.

Thus 4iK is a period of each of the functions sn u, en ( while dn u
has the smaller period 2K.

Example 1. Obtain the results

sn ( u + A') = cd u, en (m + K) = - k' sd u, dn ( u + K) = k' nd m,
directly from the definitions of sn ?t, en ?<, dn u as quotients of
Theta-functions. Example 2. Shew that cs u cs ( K — u) = k'.



22*31, 2'2-3'2] the jagobiax elliptic functions 501

22-32. The constant K'. We shall denote the integi'al

Jo

by the symbol K', so that K' is the same function of k'- (= c) as K is
of k (= c) ; and so

K'= rrFi -, \; 1; k" ,

when the c'-plane is cut from 1 to + oo , i.e. when the c-plane is cut
from

to - X .

To shew that this definition of K' is equivalent to the definition of
§ 21 "61, we observe that if T7-'= — 1, is the one-valued function of
P, in the cut plane, defined by the equations

K= M - (0 1 r), F = 5, (0 ! r) 3* (0 I r),

while, with the definition of § 21'51,

 ' = 1 32(010, ; :'2 = 52*(01r') 3'*(0|r'),

so that K' must be the same function of kf as K is of k" ; and this is
consistent with the integral definition of K' as

Jo It will now be shewn that, if the c-plane be cut from to — 00 and
from

1 to + X , then, in the cut plane, K' may be defined by the equation

K' =r\ s'-l)-- l- k's') - ds.

J 1

First suppose that 0< '<1, so that < ' < 1, and then the integrals
concerned are real. In the integral

f\ l -t )-iO--k'H')- dt .'0

make the substitution

s = l-k'H-)-' ,

which gives

(s' - 1)4 = k't (1 - I -f) -i, (1 - k's'-)i = k' (1 - i?f (1 - k'H-) '
*, ds \ k'H dt Xl-k'H-'f

it being understood that the positive value of each radical is to be
taken. On substitution, we at once get the result stated, namely that

K' = I ' (6- - 1) - i (1 - k's') - i ds,

provided that < k <\; the result has next to be extended to complex
values of .



502 THE TRANSCENDENTAL FUNCTIONS [CHAP. XXII



Consider T' ' l-t )~ l -kH"-) " dt.



the path of integration passing above the point 1, and not crossing
the imaginary axis*. The path may be taken to be the straight lines
joining to 1 - 8 and 1 + 8 to k~ together with a semicircle of (small)
radius S above the real axis. If (l—t-y and ( k' t-)

reduce to + 1 at = the value of the former at 1 + S is e" "" S (2 + 8)
= - i (t - Vf, where each radical is positive ; while the value of the
latter at <=1 is +/•' when k is real, and hence by the theory of
analytic continuation it is always +/;'.

Make 8-*-0, and the integral round the semicircle tends to zero like
8- ; and so

Now f-)- - k'H y -clt=\ (F-?('-i)"2(i\ ,,2)-2o;?(,

which t is analytic throughout the cut plane, while K is analytic
throughout the cut plane.

Hence \ \ ' i: -\ y hH y dt

is analytic thi'oughout the cut plane, and as it is equal to the
analytic function K' when < X-'< 1, the equality persists throughout
the cut plane ; that is to say



/ l/A; 1 1



when the c-plane is cut from to — qo and from 1 to + oo ,



Since



K + iK'=\ l-t-)-- il-hH')- ~dt,

Jo



we have sn (K + iK') = i/k, dn (K + iK') = ;

while the value of en (K + iK') is the value of (1 — P) when t has
followed the prescribed path to the point 1/A-, and so its value is
—ik'/k, not +ik'lk.

Example 1. Shew that

I f \ t l-t) l-Pt) ~- dt = l r t t-' ) kH- ) ~ dt=K,

- / -t l-t) kH) - dt = ( - 1) (1 - FOl- ~ 'dt = A".

Example 2. Shew that K' satisfies the same linear differential
equation as K (§ 22'301 example).

22*33. The periodic propertiesX associated with K + iK') of the
Jacohian elliptic functions.

If we make use of the three equations

sn K + iK') = k-\ en (K + iK') = - ik'jk, dn K + iK') = 0,

* II (A:) > because | arg c I < tt.

t The path of integration passes above the point u = k.

J The double periodicity of snw may be inferred from dynamical
considerations. See Whittaker, Analytical Dynamics (1917), § 44.



22 33-22-341] the jacobian elliptic functions 503

we get at once, from the addition-theorems for sn u, en u, dn u, the
following results :

, . , sn u en K + iK') dn (K + iK') + sn (Z H- iK') cnudnw

sn ( 4- ii + *A ) = z Tz — 2/ ir , ir'\ '

  1 — k sn u sn K +%K )

= k~ do (t, and similarly en ' u + K + iK') = — ik'k~ nc u,

dn (zt + if + iK') = ik' sc w. By repeated applications of these
formulae we have

f sn (u + 2K+ 2iK') = - sn u, ( sn (a + 4>K + UK') = sn u, cn(u+2K-
2iK') en t<, -. en (u + 4 " + 4tX') =cn m, [dn (m + 2Z + 2iK') = - dn
w, [dn u 4<K + UK') = dn u. Hence the functions sn u and dn u have
period 4iK + UK', ivhile en u has the smaller period 2K + 2iK'.

22-34. The periodic proper-ties (associated with iK') of the Jacobian
elliptic functions.

By the addition-theorem we have

sn (u + iK') = Sn(u-K + K+ iK') = k-' dc (u - K) = k~ ns u. Similarly
we find the equations

en (u + iK') — — ik~ ds u, dn u + iK') = — ics u. By repeated
applications of these formulae we have

' Bn u- 2iK') = sn u, ( sn (ii + UK') = sn u, - en (u + 2iK') = - en
u, - en u + UK) = en u, ,dn (u + 2iK') = — dn u, [dn u + UK') = dn u.
Hence the functions en u and dn u have period UK', luhile sn u has the
smaller period 2iK' .

Example. Obtain the formulae

sn iu + 2mK+ 2mK') = ( - )'" sn ti, en (u + 27nK+ 2niK') = ( - )™ + "
en i , dn ( + 2mK+2niK') = ( - )" dn u.

22'341. The behaviour of the Jacobian elliptic functions near the
origin and near iK'.

We have

d d

- sn u = en M dn u, -i— sn u = 4 gn u en u dn m - en m dn u (dn u + k
- cn u). du du



504 THE TRANSCENDENTAL FUNCTIONS [CHAP. XXII

Hence, by Maclaurin's theorem, we have, for small values of | w|, sn u
=u-~(l + k-) u' + (u'), on using the fact that sn u is an odd
function.



In like manner



dn u = l-l khr + (u'). It follows that

sn (u + iK') = k~ ns u

ku [ b )

1 1 + '- , 3,

= ,- + -7rr- u+0(u'); ku bk

— i 2k- — 1 and similarly en (u + iK') = tt; + , iu + (u'),

dn (u + iK') = - - + / ' ill + (u').

u b

It follows that at the point iK' the functions sn r, cnv, dnv have
simple poles with residues k~ , — ik~ , — i ixspectively.

Example. Obtain the residues of snw, cnw, dnw at iK' hy the theory of
Theta- fiinctions.

2235. General description of the functiotis sn u, en u, dn u.

The foregoing investigations of the functions sn u, en u and dn u may
be summarised in the following terms :

(I) The function sn m is a doubly-periodic function of u with period:?
K, 2iK'. It is analytic except at the points congruent to iK' or to 2K
+ iK' (mod. 4jfir, 2iK'); these points are simple poles, the residues
at the first set all being k~ and the residues at the second set all
being - k~ ; and the function has a simple zero at all points
congruent to (mod. 2K, 2iK').

It may be observed that sn u is the only function of u satisfying this
description ; for if (m) were another such function, sn m — (m) would
have no singularities and would be a doubly-periodic function; hence
(§ 20-12) it would be a constant, and this constant vanishes, as may
be seen by putting u = 0; so that (tt) = sn u.

When 0< A; < 1, it is obvious that K and K' are real, and sn u is real
for real values of u and is a pure imaginary when a is a pure
imaginary.

(II) The function cm* is a doubly-periodic function of u with periods
K and 2K -+ 2iK'. It is analytic except at points congruent to iK' or
to 2ir-f- iK' (mod. 4/1", 2K - 2iK'); these points are simple poles,
the residues



22*35-22"4] the jacobian elliptic functions 505

at the first set being — ik~\ and the residues at the second set being
ik~ ; and the function has a simple zero at all points congruent to K
(mod. 2K, 2iK'). (Ill) The function dn u is a doubly-periodic function
of u with periods 2,K and 4iiK'. It is analytic except at points
congruent to iK' or to ZiK' (mod. ''2K, 4tiK') ; these points are
simple poles, the residues at the first set being — i, and the
residues at the second set being i ; and the function has a simple
zero at all points congruent to K + iK' (mod. 2K, 2iK').

[To see that the fvinctions have no zeros or poles other than those
just specified, recourse must be had to their definitions in terms of
Theta-functions.]

22*351. The connexion between Weierstrassian and Jacobian elliptic
functions. If ej, 62) 3 be any three distinct numbers whose sum is
zero, and if we write

61-63



  = 3 +



sn2 (Xm, k) '



we have ( ;7 ) (' 'i ~" z)' " "' ** '

= 4 (ei ~e'i)-\ \ ns Xzi (ns Xw — 1) (ns Xw- ) = 4X2 (ci - 63) ~ V / -
3) ( / - ei) y - F e - 63) - 63 . Hence, if X2 = 6i — 63 and >?'2 •=
(e-. - 63)/(6i — 63), then y satisfies the equation*

and so e3 + (ei-63) ns Jm (61-63)2, A/ zyr = § (' + ; Qi, 9z\

where a is a constant. Making u - 0, we see that a is a period, and so

  (u; g2, 93) = 63 + ei - 63) ns2 u (ci - 63)% the Jacobian elhptic
function having its modulus given by the eqviation

1-63



61-63 22'4. Jacobi's imaginary transformation'f .

The result of | 21-51, which gave a transformation from
Theta-functions with parameter t to Theta-functions with parameter t'
= — l/r, naturally produces a transformation of Jacobian elliptic
functions ; this transformation is expressed by the equations

sn (iu, k) = i sc (u, k'), en iu, k) nc (u, k'), dn in, k) = dc (a,
¥). Suppose, for simplicity, that < c < 1 and y > ; let

' 'l-t')' l- kH-) ' dt = iu,



f

Jo



so that iy = sn (iu, k) ;

take the path of integration to be a straight line, and we have en
(iu, k) = (1 + y' ) , dn iu, A;) = (1 -1- k-y-) .

* The values of 2 '' 93 , usual, - J2e2 3 and eie e-s.

f Fundamenta Nova, pp. 34, 35. Abel Journal fUr Math. 11. (1827), p.
104) derives the double periodicity of elliptic functions from this
result. Cf. a letter of Jan. 12, 1828, from Jacobi to Legendre
[Jacobi, Ges. Werke, i. (1881), p. 402].



506 THE TRANSCENDENTAL FUNCTIONS [CHAP. XXII

Now put V = 77/(1 — ?;-)-, where < 77 < 1, so that the range of values
of t is from to 177/(1—77-)-, and hence, if t = iUl \ — t ) , the
range of values of t is from to 77.

Then dt = i t )- -idU, (1 - r-) = (1 - ,-)-*,

1 - kH' = (1 - kH ) ~ - l- /r) - i

and we have in=\ (1 — fj-) " (1 — h'-ti-) ' idti,

Jo

so that 77 = sn (u, k')

and therefore tj = sc u, k').

We have thus obtained the result that sn hi, k) = i sc u, k').

Also en in, k) = (1 -f- y'-)- -- (1 — 77-) ~ = nc u, k'),

and dn iu, k) = l- ¥if) = ( 1 - k'-'n'') ( 1 - 77-) ~ * = dc a, k').

Now sn iu, k) and isc (w, k') are one-valued functions of u and A; (in
the cut c-plane) with isolated poles. Hence by the theory of analytic
continuation the results proved for real values of u and k hold for
general complex values of u and k.

22*41. Proof of Jacohi's imagimiry transformation by the aid of Theta-
functions.

The results just obtained may be proved very simply by the aid of
Theta-functions. Thus, from § 21 61,

sndu M\ 3(0|T) ,(i iT) '' '''' -% Oit)% J t)'

where = m/ 3- (0 | t),

and so, by § 21-51, sn tu, k) = |4 |4> . "f '"'

= — isc (v, k'),

where v = izr" (0 j t) = izr' . (- ir) 3- (0 j t) = - w,

so that, finally, sn iu, k) = i sc u, k').

Example 1. Prove that en iu, l-) = nc(u, k' , dn (in, i-) = do ri, k')
by the aid of Theta- functions.

Example . Shew that

sn hiK', k) = iiic K', k') = ik~-,

en i iK', k) = l+k) k- , dn iiK', k) = l+k)K

[There is great difficulty in determining the signs of sn iK', ci\ j
iE', dnit'A'', if any method other than Jaeobi's transformation is
used.]



2i -41, 22-42] the jacobian elliptic functions 507

Example 3. Shew that

Example 4. If < Z- < 1 and if be the modular angle, shew that

sn \ K + ?:/ :') = e'' *' " *' V(cosec ), en h K + iK') = e " i' '
/(cot 6),

(Glaisher.) 22"42. Landens transformation* . We shall now obtain the
formula

f ' (1 - l- sin- d,) ~ dd, = l- k') f (1 - -' sin- 6) " *c , Jo Jo

where sin j = (1 + k') sin ( cos < (1 — kr sin- ) ~

and k, = l-k')l l + k').

This formula, of which Landen was the discoverer, may be expressed by-
means of Jacobian elliptic functions in the form

sn (1 + k') u, k \ = \ ->r k') sn u, k) cd u, k), on writing (f) = am
u, (f) — am u .

To obtain this result, we make use of the equations of | 21-52, namely
% z\ r) , z\ r ) % z\ t) ; z\ t ) 3 (0 | t) 4 (0 I t) 4(2 12t) -
(22I2t) 4(0|2t)

Writef Ti = 2t, and let k- , A, A' be the modulus and quarter-periods
formed with parameter Tj ; then the equation

% z\ r)X z\ r) \ ' 'lz\ n;) % z\ r) , z\ r) ,(20iTO

may obviously be written

k sn 2Kz/7r, k) cd (2 A /tt, k) = k, sn (4 A tt, k,) (A).

To determine k in terms of k, put z = jTt, and we immediately get

  /(l + k') = k , which gives, on squaring, k = (1 — '')/(l + k'), as
stated above.

To determine A, divide equation (A) by z, and then make — >0; and we
get

2Kk = 4 'l* A,

so that A=~ [+k')K.

* Phil. Trans, of the Royal Sac. lxv. (1775), p. 285.

t It will be supposed that \ R(t)\ < : , to avoid difficulties of sign
which arise if R (tj) does not lie between ±1. This condition is
satisfied when 0<k<l, for r is then a pure imaginary.



508 THE TRANSCENDENTAL FUNCTIONS [CHAP. XXII

Hence, writing ii in place of Kzjir, we at once get from (A) (1 + A;')
sn u, k) cd u, k) = sn (1 + k') u, ki], since 4<Az/7r = 2Au/K =(1 +
k')ii;

so that Landen's result has been completely proved.

Example 1. Shew that JA = 2K'jK, and thence that A' = (l +1-') K'.
Example 2. Shew that

en (!+/•')", /•i = l -(!+>(•') sn2( , H') nd n, k), dn (l+ ')w, i] = W
+ (l - k') cn (u, X-) nd(M, k). Example 3. Shew that

dn u, k) = l-k')cn l+k')u, ki] + I + i-') dn I + k') u, l\ \ },

where X-=2 -ii/(l+ 'i).

22*421. Transformations of elliptic functions.

The formula of Landen is a particular case of what is known as a
transformation of elliptic functions ; a transformation consists in
the expression of elliptic functions with parameter t in terms of
those with parameter a + bT)j c + dT , where a, b, c, d are integers.
"We have had another transformation in which = - 1, 6 = 0, c = 0,
c?=l, namely Jacobi's imaginary transformation. For the general theory
of transformations, which is out- side the range of this book, the
reader is referred to Jacobi, Fundamenta JYova, to Klein, Vorlesungen
iiher die Theorie der elliptischen Modulfunktionen (edited by Fricke),
and to Cayley, Elliptic Functions (London, 1895).

Example. By considering the transformation ro = r+l, shew, by the
method of § 22-42, that

sn k'u, k2)=k' sd u, k),

where -0= ± ik/k', and the upper or lower sign is taken according as R
t)<0 or R (r) > ; and obtain formulae for en k'u, 2) and dn k' l, k' .

22'5. I nfhdte products for the Jacohian elliptic functions* .

The products for the Theta-functions, obtained in § 21-3, at once
yield products for the Jacobian elliptic functions ; writing a =
KxJtt, we obviously have, from § 22-11, formulae (A), (B) and (C),

  i , - It • ( 1 - 2m cos 2a; + o " ]

   =i 1 1 - 25-'*-! cos 1x + (7 "-- J

g i;47 -4 n 1 1 + 29 " COS 2 + " [

en II = 2q*k -k - cos x H \ ?r- r~~, t;;— i .

  ,j i (1 - 2 2"-> cos 2x + 5'*"-2)

- lA u (l + 2g - cos2a; + g -'



''1 [1 — 2cf' ''' COS 2x + q

From these results the products for the nine reciprocals and quotients
can

be written down.

There are twenty-four other formulae which may be obtained in the
following manner :

From the duplication-formulae (§ 22-21 example 5) we have

1-cntt 1 ,1 l-HduM ,1 1 dn?i-|-cnM 1 ,1

= sn - udc - u, =as -u uc - , • = en ? as - u.

Huu 2 2 sn 2 2 sn m 2 2

* Fundamenta Nova, pp. 84-115.



22-421, 22*5] the jacobian elliptic functions 509

Take the first of these, and use the products for sn u, en |i(, dn u ;
we get

7+q j '



l — cnu l-cos. ' ° fl -2 ( -o)" cos j;-



snu ~ sin =i (1 + 2 (-g )"cosa;+g'2 on combining the various products.

Write u + K for u, x+h-rr for x, and ve have

dnw + snw l + sin.-?; * fl + 2 (-g)" sin.- g + g " ! cnu " cos j; =i
[1 - 2 ( - g)" sin a; + ( - j '

Writing u + iK' for in these formulae we have

. - ri + 2 • ( - ) o" - sin .v - <f'' " ] k sn if + dn ?< = I n - =
rr-. — r ; — -. - — t ,

and the expression for cd ?i + ik' nd u is obtained by writing cos x
for sin .t? in this product.

From the identities I - cmi) (I + cmi) = an ti, (ksmi + idnu)
l-sm(-idmi) l, etc., we at once get four other formuhxe, making eight
in all ; the other sixteen follow in the same way from the expressions
for ds-|Mnc M and cn tids ic. The reader may obtain these as an
example, noting specially the following :

Example 1. Shew that

  >-i ((i-ij '-'Xi+t '""')!






Example 2. Deduce fron\ example 1 and from § 22 '41 example 4, that,
if 6 be the modular angle, then

" tl+(-)" . +*/'

and thence, by taking logarithms, obtain Jacobi's result

J = 2 ( - )" arc tan q"'~ - = arc tan s,U] - arc tan s,'q + arc tan Jq
- • •,

H =

' quae inter formulas elegantissimas censeri debet.' Fund. Nova, p.
108.) Example 3. By expanding each term in the equation log sn M = log
(2 J') - i log /t + log sin x + 2 log ( 1 - j " e * )

+ log (1 - j2ne-2ia;)\ log(l-j2n-l e' ) -\ og (1 - J n-l g -2ia:-)j

in powers of e ' , and rearranging the resulting double series, shew
that

, K 1 , 7 1 • ** 2o'" cos 2mA'

logsnw=log(2g')-|log/ - + logsm.r+ 2 ; , , ,

when l/(2)l<-|7r/(r).

Obtain similar series for log en u, log dn i.

(Jacobi, Fundamenta Nova, j). 99.)

Example 4. Deduce from example 3 that

 K

log sn udu= - irK' — \ K log k.

(Glaisher, Proc. Royal Soc. xxix.)



/:



510 THE TRANSCENDENTAL FUNCTIONS [cHAP. XXII

226. Fourier series for the Jacobian elliptic /mictions* .

If u = Rxjir, sn u is an odd periodic function of x (with period 27r),
which obviously satisfies Dirichlet's conditions (§ 9"2) for real
values of x ; and therefore (§ 9*22) we may expand sn w as a Fourier
sine-series in sines of multiples of x, thus

sn a — hn sin nx,

n = \

the expansion being valid for all real values of x. It is easily seen
that the coefficients 6 are given by the formula



TTibn — I sn u . exp nix) dx.

J —77

To evaluate this integi'al, consider I snu. exp nix) dx taken round
the parallelogram whose corners are — tt, tt, ttt, — 27r + ttt.

Tttt

From the periodic properties of sn w and exp nix), we see that 1
cancels

r -It ' IT

1 ; and so, since —tt + ttt and ttt are the only poles of the
integrand

 qua function of x) inside the contour, with residuesf

— •~ ( 2 tt/K ] exp ( — niir + - niriT j

and k- Q Tr/iTJ exp Q nirir)

respectively, we have

\ \ — [ sn li . exp nix)dx = - g-" 1 — (—)" .

(J -TT J -2;7 + 7rTj -"- "

Writing a; — tt + ttt for x in the second integral, we get

[1 + (-)"?" j sn I* . exp (mic) rfa; = -| 3 " 1 - (-)~ .



Hence, when /i is even, hn = ; but when n is odd Consequently






sn ti =



27r J5'- sin a; g sin Sx q sin oa; ]



when X is real ; but the right-hand side of this equation is analytic
when q " exTp nix) and q exp —nix) both tend to zero as w— >x, and the
left- hand side is analytic except at the poles of sn u.

* These results are substantially due to Jacobi, Fundamenta Nova, p.
101. t The factor irlK has to be inserted because we are dealing with
sn (2KxJTr).



22*6, 22'6l] THE JACOBIAX ELLIPTIC FUNCTIONS 511

Hence both sides are analytic in the strip (in the plane of the
complex variable x) which is defined by the inequality [ I(oc) : < irl
r).



And so, by the theory of analj tic continuation, we have the result sn



\ 27r I g" sin(2;? + ) x (where u = KxJtt), valid throughout the strip
[ / (a;) | < - tt/ (r)



Example 1. Shew that, if ?i = 2A'.iY7r, then

CU M =



27r 5 cos(2?;.+ l) A- \ tt Stt °° §'" cos 2??.r

A n=o l + ? -i ' '''''*"2Z + T !i l+j- '

I " J . 7. , 29" sin 2?i ./o ,1=1 %(l+j2 )

these results being valid when | /( ) | <c\ tt I r).

Example 2. By writing x-k-\ iT for x in results ah'eady obtained, shew
that, if u= Kxl-K and \ I x)\ < \ ivI t\

then cd - 1 (-rg--+ cos(2 + l ) sd..-- 1 (- ?"+ sin (2n + l).: then
cdw- , 2 l\ j2 + i ' '" '-A'M'io T+ i '

, TT 27r "" ( — )" o" COS 2 a;

nd =-.>,, + -FT, 2 — - — 5 .

2AX-' Kk' =i l+j2n

22"61. Fourier series for reciprocals of Jacobian elliptic functions.

In the result of § 22"6, write u + iK' for u and consequentl a; + ttt
for a; ;

then we see that, ifO>/(a7)> — tt/ (t),

and so (§ 22-:34)

ns li = (- iirjK) S " + * [5" + ie(2 +i) \ - - ie-(2n+i) ix|/( i \ 2
+i

w =

X

= (- iV/iT) S [2t5-"+i sin (2n + 1) a; + (1 - q--' - ) e-(- +i '
a'J/(i \ 2n+i =o

  27r - r+' sin (27 + 1) a : \ iV ,, , .

That is to say

TT 27r o-'*+i sin (2/i 4- 1) a; ns =j eosee + yJ\ L\ \ .

But, apart from isolated poles at the points x = wrr, each side of
this equation is an analytic function of x \ r\ the strip in which

IT I (r) >I x)>-'TrI (r) : — a strip double the width of that in which
the equation has been proved to be true ; and so, by the theory of
analytic continuation, this expansion for ns u is valid throughout the
wider strip, except at the points x = iiir.



512 THE TRANSCENDENTAL FUNCTIONS [cHAP. XXII

Example. Obtain the following expansions, valid throughout the strip '
I x)\ < itI t) except at the poles of the first term on the right-hand
sides of the respective expansions :

TT 27r == a2n + isin(2?i + l) ds u = cosec..- 2 i: .T- >

TT , 27r " o- " sin 2nx

-' cot y;r 2



2K K =i 1+?= " '

TT 27r " (-) g2n + lcos(2?l + l).r

 . secA- + 2 l-g- -i '



nc = 2ZX ,sec.r - , 2 - ,, ,

TT , 27r " ( — )" o " sin 2?U7

2 A A: Kk =i l+y n

227. Elliptic integrals.

An integi-al of the form IR(w, x)dx, where R denotes a rational
function

of w and x, and w is a QUARTIG, or CUBIC function of x (without
repeated factors), is called an elliptic integral*.

[Note. Elliptic integrals are of considerable historical importance,
owing to the fact that a very large number of important properties of
such integrals were discovered by Euler and Legendre before it was
realised that the inverses of certain standard types of such
integrals, rather than the integrals themselves, should be regarded as
fundamental functions of analjsis.

The first mathematician to deal with elliptic functions as opposed to
elliptic integrals was Gauss (§ 22 "S), but the first results
published were by Abelt and Jacobil.

The results obtained by Abel were brought to the notice of Legendre by
Jacobi immediately after the publication by Legendre of the Traite des
fonciions elliptiques. In the supplement (tome in. (1828), p. 1),
Legendre comments on their discoveries in the following terms : "A
peine mon ouvrage avait-il vu le jour, a peine son titre pouvait-il
§tre connu des .savans etrangers, que j'appris, avec autant
d'etonnement que de satisfaction, que deux jeunes geometres, IM.
Jacobi (C.-G.-J.) de Koeuigsberg et Abel de Christiania, avaient
reussi, par leurs travaux particuliers, a perfectionner
considerablement la theorie des fonctions elliptiques dans ses points
les plus eleves."

An interesting correspondence between Legendre and Jacobi was printed
in Journal fur Math. Lxxx. (1875), pp. 20.5-279; in one of the letters
Legendre refers to the claim of Gauss to have made in 1809 many of the
discoveries published by Jacobi and Abel. The validity of this claim
was established by Schering (see Gauss, Werke, in. (1876), pp. 493,
494), though the researches of Gauss ( Werke, in. pp. 404-460)
remained unpubli-shed until after his death.]

We shall now give a brief outline of the important theorem that every
elliptic integral can be evaluated by the aid of Theta-functions,
combined

* Strictly speaking, it is only called an elliptic integral when it
cannot be integrated by means of the elementary functions, and
consequently involves one of the three kinds of elliptic integrals
introduced in § 22 "72.

t Journal fur Math. ii. (1827), pp. 101-196.

i Jacobi announced his discovery in two letters (dated June 13, 1827
and August 2, 1827) to Schumacher, who published extracts from them in
Astr. Nach. vi. (No. 123) in September 1827 — the month in which
Abel's memoir appeared. .



22 '7, 2271] THE JACOBIAN ELLIPTIC FUNCTIONS 513

with the elementary functions of analysis ; it has already been seen
(§ 20'6) that this process can be carried out in the special case of
jiv~ dx, since

the Weierstrassian elliptic functions can easily be expressed in terms
of Theta-functions and their derivates (§ 21'73).

[The most important case practically is that in which R is a real
function of x and w, which are themselves real on the path of
integration ; it will be shewn how, in such circumstances, the
integral may be expressed in a real form.]

Since R (lu, x) is a rational function of w and x we may write

R (w, x) = P (w, x)IQ w, x),

where P and Q denote polynomials in w and x ; then we have

R(w x)= ' ~ ' ~ wQ (w, x) Q (— w, x) '

Now Q (w, x) Q (— w, x) is a rational function of w- and x, since it
is unaffected by changing the sign of lu ; it is therefore expressible
as a rational function of x.

If now we multiply out wP w, x) Q (— w;, x) and substitute for w- in
terms of X wherever it occurs in the expression, we ultimately reduce
it to a poly- nomial in X and w, the polynomial being linear in iv. We
thus have an identity of the form

R (w, x) = Ri (x) + tvRz x)]lw,

by reason of the expression for w- as a quartic in x ; where jRj and
R2 denote

rational functions of x.



Now \ Ro x) dx can be evaluated by means of elementary functions
only*;

so the problem is reduced to that of evaluating jw~ Ri (x) dx. To
carry out

this process it is necessary to obtain a canonical expression for w'-,
which we now proceed to do.

22'71. The expression of a quartic as the product of sums of squares.

It will now be shewn that any quartic (or cubicf) in x (with no
repeated factors) can be expressed in the form

 A,(x- a) + BJx- Y] A, (x - af + B, (x - f] ,

where, if the coefficients in the quartic are real, A , B , A.., B ,
a, /3 are all real.

* The integratiou of rational functions of one variable is discussed
in text-books on Integral Calculus.

t In the following analysis, a cubic may be regarded as a quartic in
which the eoefiScient of X* vanishes.

W. M. A. 33



514 THE TRANSCENDENTAL FUNCTIONS [CHAP. XXII

To obtain this result, we observe that any quartic can be expressed yi
the form S S.. where Si, S2 are quadratic in x, say*

<S'i = a x + 2biX + Ci, S.2 = aox- + 2h..x + c.,.

Now, X being a constant, 1 — \ S. will be a perfect square in x if

(tti - Xftj) (ci - Xco) - (61 - \ h - = 0.

Let the roots of this equation be X , X ', then, by hypothesis,
numbers a, yQ exist such that

Si — \ 1S.2 = (o-i — Xitto) (x — a)-. Si — X S.. (a I — Xstu) x — f ;

on solving these as equations in j, S.,, we obviously get results of
the form

Si = Ai x- a)- + Bi x- /3)-, S. s A (x - a)- + Bo x - /S)

and the required reduction of the quartic has been effected.

[Note. If the quartic is real and has two or four complex factors, let
Si have com- plex factors ; then Xi and Xi are real and distinct since

( ! — Xa2) (c'l — XC2)— (61— X62)

is positive when X = and negative! when X = ai/ 2.

When Si and S-i have real factors, say x - i) -v — i), - 2) — 2), the
condition that Xi and X2 should be real is easily found to be

( 1 - 2) (6' - 2) ( 1 - 2') ( 1' - f/) > 0,

a condition which is satisfied when the zeros of Si and those of \& do
not interlace ; this was, of course, the reason for choosing the
factors S"! and So of the quartic in such a way that their zeros do
not interlace.]

22'72. The three kinds of elliptic integrals.

Let a, /3 be determined by the rule just obtained in §22'71, and, in
the integral w~ Ri x) dx, take a new variable t defined by the
equation;!:

t=(x-a)l x- );

 , , dx (a—B)~ dt we then have — = + .

    Aif' + Bi)(Ad' + B.2)\ i

* If the coefficients in the quartic are real, the factorisation can
be carried out so that the coefficients in .Sj and So are real. In the
special case of the quartic having four real linear factors, these
factors should be associated in pairs (to give Sj and S2) in such a
waj* that the roots of one pair do not interlace the roots of the
other pair ; tlie reason for this will be seen in the note at the end
of the section.

t Unless ttj ; 02 = 1 : \&2i i" which case

Si ai x-a) + Bi, S., = ao x - a) + Bo.

t It is rather remarkable that Jacobi did not realise the existence of
this homographic substitution ; in his reduction he employed a
quadratic substitution, equivalent to the result of applying a Landen
transformation to the elliptic functions which we shall introduce.



22 72] THE JACOBIAN ELLIPTIC FUNCTIONS 515

If we write R x) in the form ± (a — /3) R (t), where R is rational, we
get

fRi (x) dx \ r R., (t) dt

-' ~f (A,t' + B,)(A,t'+B,)]i' Now R, (t) + R, (- t) ~ 2R, tr), R, t) -
R, (- t) = 2tR, t% where R and R are rational functions of t-, and so

Ro (t) = R, (f ) + tR, (t ).

But l (A,t;' + B,)(A.J' + Bo) tR, t")dt

can be evaluated in terms of elementary functions by taking t" as a
new variable*; so that, if we put Ri t') into partial fractions, the
problem of



integrating I R (tv, x) dx has been reduced to the integration of
integrals of the following types :

 f A,P + B,) A,t' + B,)] - dt,

[(1 + m y A,t' + B,) A,t' + B,)] - dt ;

in the former of these m is an integer, in the latter m is a positive
integer andi\ \ 0.

By differentiating expressions of the form

t? -' [ A,t + B,) A.J- + i?,)]*, t (1 + m y- [ A,i:' + B,) A,t' +
5,);i,

it is easy to obtain reduction formulae by means of which the above
integrals can be expressed in terms of one of the three canonical
forms :

(i) [ A,t + B,) A,P + B.;)]-idt, (ii) 1 [(.4, + B,) A,t- + B,)] -Ut,

(iii) [(1 + m )-' (A,t' + B,) A,r- + 5,) -idt.

These integrals were called by Legendref elliptic integrals of the
first, second and thii'd kinds, respectively.

The elliptic integral of the first kind presents no difficulty, as it
can be integrated at once by a substitution based on the integral
formulae of §§22121, 22-122; thus, if A B A., B., are all positive and
A B,>A,B., we write

A, t = Bi cs (u, k). [k'-' = (A,B,)/(A,B,).]

* See, e.g., Hardy, Integration of Functions of a single Variable
(Camb. Math. Tracts, No. 2). t Exercices de Calcul Integral, i.
(Paris, 1811), p. 19.

. 33—2



516



THE TRANSCENDENTAL FUNCTIONS [cHAP. XXII



Example 1. Verify that, in the case of real integi'als, the following
scheme gives all possible essentially diflerent arrangements of sign,
and determine the appropriate substitutions necessary to evaluate the
corresponding integrals.



 1


+


+


-


+


+


-


A


+


-


+


-


-


+


A,


+


+


+


+


-


-


1


+


+


+


-


+


+ :



1- cd u



Example 2. Shew that

I sn \ i du = -rr log r- , , ,

I dmidu = a.m u,



I en u du = k~ arc tan k sd u),



1 , dn?i + X-'



/



, , 1 , 1 — en ?

as udu = - log — ,

2 °l+cnM'



I sc udu=z , log '-. J-, ,

I A 7 1 1 1 + sn ?i

I dc 2( du — ~ log , ,

] 2 ° 1 - sn M '

and obtain six similar formulae by writing u- K for u.

(Glaisher.)

Example 3. Prove, by differentiation, the equivalence of the following
twelve expressions :

n — Ifi \ \&vr n du, J dn- M du,

// tf + dn u%cu — l<!''-\ vi.o''udu, dn tt sc — 2 J gc2 j (j 2(

M + P sn M cd — F Jcd u du,

k"- It — dn ?; cs M — Jds- u du,

Example 4. Shew that

'!" = n (n - 1) sn --' u - n- (\ + k-) sn" u + n (n + 1) F- sn + 2 du-

and obtain eleven similar formulae for the second differential
coefficients of en" ?i,

dn" ? ... nd"M. What is the connexion between these formulae and the
reduction

formula for \ t'' Ait' - Bi) A2fi + Bo)]~ dt]

(Jacobi ; and Glaisher, Messenger, xi.)

Example 5. By means of § 206 shew that, if a and ;3 are positive.



k'- u + k- I en- u du,

u — dmi cs u — j ns- u du,

k sn % cd it + kf Jnd xi dti,

k" u + k SQU cdu + P k" j sd - z< du,

- dn u cs ? — J cs'' u du,

u + dn w sc M — I" dc u du.



i:



 i



-a ./ <>,

where e is the real root of the cubic and

92 = .2 a - 'f-a' ', 93= - a'- ') a• - 2)2 -36a2/32 /216 ; and prove
that, if 2 = 0, then a and are given by the equations

a- - - = - 3 (2 3)4, d' + /a2 = 2 /3 . I 2g3 |* .



2273] THE JACOBIAN ELLIPTIC FUNCTIONS 517

Example 6. Deduce from example 5, combined with the integral formula
for en m, that, if g-i is positive,

where a2 = (V3-f) (2 3)*, /32 = (v/3 + f) (2 3)*, and the modulus is
a(aH )" -

22"73. The elliptic integral of the second kind. The function* E(u).
To reduce an integral of the type

jf' (A,t' + B,) (A,t + B,)l - idt,

we employ the same elliptic function substitution as in the case of
that elliptic integral of the first kind which has the same expression
under the radical. We are thus led to one of the twelve integrals



jsn udu, icn udu, ... jnd' udu.



By § 22'72 example 3, these are all expressible in terms of u,
elliptic functions of u and Jdn-udu ; it is convenient to regard

fu

E(ii) = dn- udu



Jo



as the fundamental elliptic integral of the second kind, in terms of
which all others can be expressed ; when the modulus has to be
emphasized, we write E u, k) in place oi E u).

We observe that

dE(it)



du



= dn-u, E 0) = 0.



Further, since dn u is an even function with double poles at the
points 2mK + (2n + l)iK, the residue at each pole being zero, it is
easy to see that E(u) is an odd one-valuedf function of u with simple
poles at the poles of dn u.

It will now be shewn that E () may be expressed in terms of Theta-
functions ; the most convenient type to employ is the function (u).



 °'''''' '' m %U\ '



it is a doubly-periodic function of u with double poles at the zeros
of © (u), i.e. at the poles of dn u, and so, if A be a suitably chosen
constant,

dn-u - A - - , ; ' du (©(z<.)

* This notation was introduced by Jacobi, Journal fiir Math. iv.
(1829), p. 373 [Ges. Werke, I. (1881), p. 299J. In the Fundamenta
Nova, he wrote E (am u) where we write E (u).

+ Since the residues of dn m are zero, the integral defining E u) is
independent of the path chosen (§ 6-1).



518 THE TRANSCENDENTAL FUNCTIONS [CHAP. XXII

is a doubly-periodic function of u, with periods 2K, 2 K', with only a
single

simple pole in any cell. It is therefore a constant ; this constant is
usually

written in the form JE/K. To determine the constant A, we observe that

the principal part of dn- u at iK' is — (u — iK')~ , by § 22"341 ; and
the

residue of \&' (u)IS (u) at this pole is unity, so the principal part
of

d 10'(u)) . \ .,

-y- < , , ' y IS — n — iK ) -. Hence yi = 1, so

du [yd u))

Integrating and observing that H' (0) = 0, we get E (w) = ©' (;f )/0
(w) + tiEjK. Since ©' K) = 0, we have E (K) = E ; hence

E=( dn- ucZw=r (l-k'sm"-0dcf> = l7rF(-l, ~; 1; kA .

It is usual (cf. § 22"3) to call K and E the complete elliptic
integrals of the first and second kinds. Tables of them qua functions
of the modular angle are given by Legendre, Fonctions Elliptiques, ii.

Example 1. Shew that E u + 2nK)= E ti) + 2nE, where n is any integer.

Example 2. By expressing e u) in terms of the function 9 ( irulK), and
expanding about the point u = iK', shew that

 = (2-P-V7W5i') ir.

22-731. The Zeta-f unction Z (w).

The function E (u) is not periodic in either 2K or in 2iK', but,
associated with these periods, it has additive constants 2E, 2iK'E —
7ri]/K ; it is convenient to have a function of the same general type
as E u) which is singly-periodic, and such a function is

Z (u) = ©' (m)/© (u) ; from this definition, we have*

Z (u) = E (ti) - uE/K, © (m) = © (0) exp I j" Z t) dt\ .

22*732. The addition-formulae for E u) and Z u).

Consider the expression

e'(u + v) ©'(m) W(v) , , ,

-— ' / - T TT x - /-v / + f ' sn u sn V sn u + v)

@(u + v) ©(m) S v)

* The integral in the expression for (u) is not one-valued as Z (f)
has residue 1 at its poles; bat the difference of the integrals taken
along any two paths with the same end points is 2/i7r/ where n is the
number of poles enclosed, and the exponential of the integral is
therefore one- valued, as it should be, since 0(m) is one-valued.



22 •731-22-734] the jacobian elliptic functions 519

qua function of m. It is doubly-periodic* (periods 2if and 2iK') with
simple poles congruent to iK' and to iK' — v ; the residue of the
first two terms at iK' is — 1, and the residue of sn u sn v sn (u + v)
is k~ sn v sn (iK' + v) = k~ .

Hence the function is doubly-periodic and has no poles at points
congruent to iK' or (similarly) at points congruent to iK' — v. By
Liouville's theorem, it is therefore a constant, and, putting u = 0,
we see that the constant is zero.

Hence we have the addition-formulae

Z (w) -f- Z (v) — Z(u+v) = k sn M sn V sn (?< + v),

E(u) + E (v) — Eiu 4- v) = k sn u sn v sn u + v).

[Note. Since Z u) and E (u) are not doubly- periodic, it is possible
to prove that no algebraic relation can exist connecting them with sn
u, en u and dn u, so these are not addition-theoreras in the strict
sense t.]

22'733. Jacobi's imaginary transformation\ of7i u).

From § 21'51 it is fairly evident that there must be a transformation
of Jacobi's type for the function Z ii). To obtain it, we translate
the formula

 2 ix i t) = (— ir) exp (— iTX /ir) . 4 (ixr \ r) into Jacobi's
earlier notation, when it becomes



H (iu + K, k) = (- ir)i exp (\ ( , k'),



and hence



/ Tru-" \ 4 (0 I t) © (u, k')



en (in, k) = (- V)* exp ( j



Taking the logarithmic ditferential of each side, we get, on making
use of § 22-4,

Z iu, k) = i dn (u, k') sc u, k') — iTj u, k') — 7riu/(2KK').

22 734. Jacobi's imaginary transformation of E(u).

It is convenient to obtain the transformation of E (u) directly from
the integral definition ; we have

E (in, k) = I "dn- t, k) dt=\ dn- (it', k) idt'

Jo Jo

= i dc- (t', k') dt', ]

on writing t — it' and making use of § 22 "4.

* 2iK' is a j eriod since the additive constants for the first two
terms cancel.

t A theorem due to Weierstrass states that an analytic function,/ (2),
possessing an addition- theorem in the strict sense must be either

(i) an algebraic function of z, or (ii) an algebraic function of exp
(Trizjw),

or (iiij an algebraic function of (2 | wi , W2) J

•where w, wj, W2 are suitably chosen constants. See Forsyth, Theory of
Functions (1918), Ch. xiii.

J Fundamenta Nova, p. 161.



520 THE TRANSCENDENTAL FUNCTIONS [CHAP. XXII

Hence, from § 22*72 example 3, we have

E hi, k) = i \ u + dn u, k') sc ( , k') - \ dn= t', k') dt' [ ,

and so E iu, k) = iu + i dn u, k') sc u, k') — iE u, k').

This is the transformation stated.

It is convenient to write E' to denote the same function of k' as E is
of k, i.e. E' = E(K', k'X so that

E 2 K',k) = 2i(K' -E').

22'735. Legendres relation*.

From the transformations of E u) and Z u) just obtained, it is
possible to derive a remarkable relation connecting the two kinds of
complete elliptic integrals, namely

EK' + E'K-EK' = \ ir.

For we have, by the transformations of §§ 22"733, 22*734,

E iu, k) - Z iu, k) = iu - i [E u, k') - Z u, k')\ + '7riu/ 2KK'),

and on making use of the connexion between the functions E u, k) and Z
u, k), this gives

iuE/K = iu - i [uE'jK'] + 'Triu/ 2KK'). Since we may take m =| 0, the
result stated follows at once ft-om this equation ; it is the analogue
of the relation rj coo — Vzf i = 9 tJ"* which arose in the
Weierstrassian theory (§ 2041 1). Example 1. Shew that

E u + K)-F ii) = E-k-mucdu. Example 2. Shew that

E(2u + 2iK') E (2u) + 2i K' - E'). Example 3. Deduce from example 2
that

E ti + iK') = E 2u + 2iK') + W sn ( u + iK') sn (2m + 2iK') = E (u)
-fen M ds M + i (A" - £"). Example 4. Shew that

E u + K+ iK') = E (u) - sn u dc u + E+i E' - E'). Example 5. Obtain
the expansions, vaHd when | I x) l< 7r/(T),

/7 r'NO o tt; ,'T-, o ' iw" COS 2ii.r . -, °° Q' s\ Vi2nx n=i q " n=\
1-?''"

(Jacobi.)

* Exercices de Calcul Integral, i. (1811), p. 01. For a geometrical
proof see Glaisher, Messenger, iv. (1874), pp. 95-96,



22-735-22737] the jacobian elliptic functions 521

22'736. Properties of the complete elliptic integrals, regarded as
functions of the modulus.



If, in the formulae B=\ (1 — k sin (f)y d<f>, we differentiate under
the

Jo

L (§ 4"2), we have

fh . . \ i E-

= — I k sm- (f) (1 — k" sm (f)) -d(J3 = — j-

J K

Drmula for K in the same manner, we ha'

= ' sin- < ( 1 — • ' sm" (j>) d(f) = k \ sd u di

J a Jo



sign of integration (§ 4*2 ), we have

dE fh , . . . .. , . ..\ ! .. E-K

dk

Treating the formula for K in the same manner, we have

dJ dk

K



jLj/ dn'u*,.-



k!''-u



by § 22-72 example 3 ; so that

dk kk'- k If we write k- = c, k'-= c', these results assume the forms
dE E-K dK E-Kc' dc c dc cc

Example 1. Shew that

 dE K'~E' 2 = cK'-E'

dc c' ' dc cc'

Example 2. Shew, by difterentiation with regard to c, that EK' + E'K —
KK' is constant.



Example 3. Shew that K and K' are solutions of

d dk

and that E and E' — K' are sohitions of



|m' | |= ..,



 '" ~M- \ \ 'M ' ~ ' (Legendre. )



22'737. The values of the complete elliptic integrals for small values
of k. From the integral definitions of E and K it is easy to see, by
expanding in powers of k, that

lim K = lim E l-TT, lim (K - E)lk' = tt.

In like manner, lim E' = I cos (f)d(j) = 1.

k- -o J

It is not possible to determine lim/i' in the same way because

fc O

(1 — •'-sin ( )" 2 is discontinuous at </> = 0, k = ; but it follows
from example 21 of Chapter xiv (p. 299) that, when | argA- 1 < tt,

lim i '-log(4/ •)l=0.



522 THE TRANSCENDENTAL FUNCTIONS [CHAP. XXII

This result is also deducible from the formulae 2iK' = -!TT i , k .
jBi , by making q-*-0; or it may be proved for real values of k by the
following elementary method :

By § 22-32. A" = / f- - k-) ' l-t')' dt; now, when /• < < Jk, l-t-)
lies between

1 and 1-/-; and, when s,'k<t<\ \ fi-k )/t lies between 1 and 1 ~k.
Therefore A'' lies between

Jk i /A-

and (i-/.)-*]/" ( 2\ .2)-ij;+ / t-- l-f- )-hdt\;

and therefore

A =(1 - 6k) ijlog log 1-- /(i l.

= (1 - Ok) - h [2 log 1 + v'(l - k) - log /•], where 0 1.

Xow lira [2 (1 - Ok) ' i log 1 + v/(l - k) - log 4] = 0,

\ \ m I - I - 6k) - log k = 0,

and therefore lim A' - log (4/) ,-) = 0,

which is the required result.

Example. Deduce Legendre's relation from § 22-736 example 2, by making
k- 0.

22"74. The elliptic integral of the third kind*. To evaluate an
integral of the type

f(l + Nt')-' A,P + B,) A,t'+B. ] - dt



/<



in terms of known functions, we make the substitution made in the
corre- sponding integrals of the first and second kinds (§§ 22*72,
22"73). The integral is thereby reduced to

I -— du = ait + (p — av) — du,

where or, yS, v are constants ; if i = 0, — 1 , x or — the integral
can be expressed in terais of integrals of the first and second kinds
; for other values of V we determine the parameter a by the equation v
= — k sn- a, and then it is evidentl ' permissible to take as the
fundamental integral of the third kind

  , [ k-snacn a dn a sn' u ,

n (u, a) = — j~. — du.

  Jo 1 — k sn a sn u

To express this in terms of Theta-functions, we observe that the inte-
grand may be written in the form

I k- sn u sn a sn (u + a) -1- sn (u — ) = ( (" - ) - (" + a) + 2Z (a)
,

* Legendre, Exercices de Calcul Integral, i. (1811), p. 17; Fonctions
EUiptiques, i. (1825), pp. 14-18, 74, 75; Jacobi, Fundamenta Nova
(1829), pp. 137-172; we employ Jacobi's notation, not Legendre's.



2274, 22741] the jacobian elliptic functions 523

by the addition-theorem for the Zeta-fimction ; making use of the
formula Z (i() = B'(m)/ (u), we at once get

a result which shews that U(u, a) is a many- valued function of u with
logarithmic singularities at the zeros of @ (u ± a).

Example 1. Obtain the addition-formula*

e (u+v + a) e ( tt-g) e (v -a)

n u, a)+n v, a)-niu+v, ) = iloge( +.,-a)e( -ha)e( + )

j 1 - Psnasn - sn-?;sn (tc + v — a) ~2 °"l + Psn asnwsnvsn (w + v +
a)'

(Legendre.)

(Take a- : y : s : tf = m : w : ± a : i( -h r ± a in Jacobi's
fundamental formula

[4] + [l] = [4]'-r[l]'.)

-Example 2. Shew that

n u, a) - n (a, m) = uZ (a) - aZ (tt).

(Legendre and Jacobi.)

[This is known as the formula for interchange of argument and
parameter.]

Example 3. Shew that

l—PsnaHnbsnuHn a + b-u) n (iL a) + n hi, h) — n (u, a + b) = i log \ ,
— j r -. — , i , ,,\

+ uk sn a sn 6 sn (a 4- 6).

(Jacobi.) [This is known as the formula for addition of parameters.]

Example 4. Shew that

IT iio, ia + /i, k) = Il u,a + K\ k'). (Jacobi.)

Example 5. Shew that

n (m + v, a-irb) + n (u-v, a-b)-2n u, a)-2U v, b)

, , , , l-Fsn2(2<-a)sn2(i;-6) = - F sn a sn 6 . (u + v) sn (a -f b) -
(w - y) sn ( - 6) -H i log i + 2 . 2 ( ;, ,) s ( + 6) '

and obtain special forms of this result by putting v or h equal to
zero. (Jacobi.)

22 •741. A dynamical application of the elliptic integral of the third
kind. It is evident from the expression for n (m, a) in terms of
Theta- functions that if u, a, k are real, the average rate of
increase of n (m, a) as u increases is Z (a), since 9 ii±a) is
periodic with resjject to the real period 2K.

This result determines the mean precession about the invariable line
in the motion of a rigid body relative to its centre of gravity under
forces whose resultant passes through its centre of gravity. It is
evident that, for purposes of computation, a result of this nature is
preferable to the corresponding result in terms of Sigma-functions and
Weierstrassian Zeta-functions, for the reasons that the
Theta-functious have a specially simple behaviour with respect to
their real period — the period which is of importance in Applied
Mathe- matics — and that the -series are much better adapted for
computation than the product by which the Sigma-function is most
simply defined.

* No fewer than 96 forms have been obtained for the expression on the
right. See Glaisher, Messenger, x. (1881), p. 124.



524 THE TRANSCENDENTAL FUNCTIONS [CHAP. XXII

22'8. The lemniscate functions.

The integral (1 — ) dt occurs in the problem of rectifying the arc of

Jo

the lemniscate*; if the integral be denoted by , we shall express the
relation between and x by writingi* x — sin lemn < .

In like manner, if

J X - .'

we write

x = cos lemn j, and we have the relation



sin lemn < = cos lemn ( tn- — ( j .



These lemniscate functions, which were the first functions:]: defined
by the inversion of an integral, can easily be expressed in terms of
elliptic functions with modulus l/VS; for, from the formula (§ 22-122
example)

r sd u u=\ [ l-k' f) l+l: f)]-Uy,

.

it is easy to see (on writing y = t \/2) that

sin lemn = 2 " sd (< \/2, l/V ) ; similarly, cos lemn = en (</> V2, 1/
2).

Further, is the smallest positive value of for which

cn(< V2, 1/V2) = 0, so that OT=V2 o,

the suffix attached to the complete elliptic integral denoting that it
is formed with the particular modulus l/\/2.

This result renders it possible to express Kq in terms of
Gamma-functions,



thus



K,= 2 [ P)- -dt='2---\ \ u - l-u)- -di Jo Jo



a result first obtained by Legendre§.

Since k = // when k = l/\/2, it follows that 7 = 7iV. and so o = e ".

* The equation of the lemniscate being r- = a' cos 26, it is easj' to
derive the equation

 %)' = ar~* '•' '" ' ' ' " " tj= 1 + (1-)' •

t Gauss wrote si and cl for siu lemn and cos lemn, Werke, iii. (1876),
p. 493.

t Gauss, Werke, iii. (187(5), p. 40i. The idea of investigating the
functions occurred to Gauss on January 8, 1797.

§ Exercices de Calcul Integral, i. (Paris, 1811), p. 209. The value of
Kq is 1-85407468... while 07 = 2-62205756....



22-8, 22*81] THE JACOBIAN ELLIPTIC FUNCTIONS 525

Example . Express A'o in terms of Gamma-fuuctions by usiug Kummer's
formula (see Chapter xiv, example 12, p. 298).

Example 2. By writing t = \ -m ) j the formula

shew that 2 o = / ( 1 - ?**) ' du+T u l- ic*) ~ i du,

and deduce that 2 o - J o = Stt r ( j) ~ 2,

Example 3. Deduce Legendre's relation (§ 22-735) from example 2
combined with § 22-736 example 2.

Example 4. Shew that

. , , 1 — cos lemn (b

sm Iemn2d) = - --t- .

H-coslemn''9

22-81. The values of K and K' for special values of k.

It has been seen that, when k=ljJ2, K can be evaluated in terms of
Gamma-functions, and K=E' ; this is a special case of a general
theorem* that, whenever

E' \ a + hjn E c + djn'' where , b, c, d, n are integers, k is a root
of an algebraic equation with integral coefficients.

This theorem is based on the theory of the transformation of elliptic
functions and is beyond the scope of this book ; but there are three
distinct cases in which k, 7i, E' all have fairly simj le values,
namely

(I) >(- = V2-l, E' = EJ2,

(II) k = Hm 7r, E' = E %

(III) k=tsin 7r, E' 2E. Of these we shall give a brief investigation
t.

(I) The quarter-periods with the modulus /2 — 1.

Landen's transformation gives a relation between elliptic functions
with any modulus k and those with modulus ki = l — k')/ l+k'); and the
quarter-periods A, A' associated with the modulus k satisfy the
relation A'/a = 2E'/E.

If we choose k so that ky = k', then A = /i'' and k = k so that i. ' =
E; and the relation A7A=2A'7A'gives A'2=2a2.

Therefore the quarter-periods A, A' associated with the modulus k-
given by the equation •i = (l — i)/(l-|-X-i) are such that A'=±Av/2;
i.e. \ i ki = l2-\, then A' = Av/2 (since A, A' obviously are both
positive).

(II) The quarter-periods associated ivith the modidus sin y tt.

The case of k = s,\ n -rrr was discussed by Legendre|; he obtained the
remarkable result that, with this value of k,

E' = Ey/3.

* Abel, Journal fiir Math. in. p. 184 [Oeuvres, i. (1881), p. 377].

t For some similar formulae of a less simple nature, see Kronecker,
Berliner Sitzungsherichte, 1857, 1862.

X Exercices de Calcul Integral, i. (1811), pp. 59, 210; Fonctions
Elliptiques, i. (1825), pp. 59, 60.



526 THE TRANSCENDENTAL FUNCTIONS [CHAP. XXII

This result follows from the relation between detinite integi-als

To obtain this relation, consider l l-z )~ dz taken round the contour
formed by the part of the real axis (indented at s=l by an arc of
radius R' ) joining the points and B the line joining e*"' to and the
arc of radius B joining the points R and Re "" ; as R a: , the
integral round the arc tends to zero, as does the integral round the
indentation, and so, by Cauchy's theorem,

on writing .v and e*'"* respectively for z on the two straight lines.
Writing [\ l-x )- -dx = Ii, j\ x- -l)- dx = L, j\ l+.f )-- ds=j\ l-x
)- dx I

we have /j + iV. = 4 (1 + is/S) h ;

so, equating real and imaginary parts,

A = 5- 3 5 - 2 = 5- 3X 3,

and therefore h-V I - Lis'' = \ h + h-¥i=

which is the relation stated*. Now, by § 22-72 example 6,

/2 = 4(a2 + 2)-*£', / + /3 = 4(a2 + /32)-4 ',

where the modulus is a d' + ) ~ and

a2=2v3-3, /32 = 2V3 + 3, so that , > :2 = i(2-V3) = sin2Js7r.

We therefore have

3-i. 2/1 = 3-2. 2 ' = /2 = 3 /i

= 3- |V"s(i- )-ic =i,r r(J)/r(|),

when the modulus k is sin jV -

(III) The quarter-periods iiith the modtdus tan -i-Tr.

If, in Landen's transformation (§ 22-42), we take l::=i; '2, we have
A'/A=2K'/K=2;



now this value of k gives



7 V -1 . "1



and the corresponding quarter-periods \, A' are |(1 + 2 ) A'o and (1 +
2 ) Kq.

Example 1. Discuss the quarter-i)eriods when k has the values
(2;,y2-2) , sin f 7r, and 2 ( 2-1).

* Another method of obtaining the relation is to express Ij , I , h in
terms of Gamma- functions by writing t , t~ , ( -' - 1)* respectively
for x in the integrals by which Ii, I2, I3 are defined.



22'82] THE JACOBIAN ELLIPTIC FUNCTIONS 527

Example 2. Shew that

?l = )i = l

(Glaisher, Messenger, v.) Example 3. Express the coordinates of any
point on the curve y =ofi— 1 in the form

1 3

3 (1— en u) \ 2.3 snMdnM

 ~ l+cn?< ' ' ~ (l + cnM)2 '

where the moduhis of the eUiptic functions is sin jV"", and shew that
-=- = ~ iy.

By considering / y~ dx = 3~* I die, evahiate K in terms of
Gamma-functions when

Example 4. Shew that, when y' =x —l,

r y-Hx-lYdx=[- y- l-x-' fl +' T x~ y- -x-h/- )dx;

7 1 L ' Ji ' y 1

and thence, by using example 3 and expressing the last integral in
terms of Gamma- functions by the substitution x = t~ , obtain the
formula of Legendre Calcul Integral, p. 60) connecting the first and
second complete elliptic integrals with modulus sin jJjtt:

Example 5. By expressing the coordinates of any point on the curve Y'
= X in the form

 \ 32(l-cnv) \ 2. 3*sn vdn??

1+cni? ' (1-l-cn ') '

in which the modulus of the elliptic functions is sin y tt, and
evaluating



 />/



Y- XfdX



in terms of Gamma-functions, obtain Legendre's result that*, when k sm
w,

22*82. A geometrical illustration of tlie functions sn u, en ii, dn u.

A geometrical representation of Jacobian elliptic functions with
k=ljJ2 is afforded by the arc of the lemniscate, as has been seen in §
22"8 ; to represent the Jacobian functions with any modulus k 0 <k < )
, we may make use of a curve described on a sphere, known as Seifferfs
spherical spiral .

Take a sphere of radius unity with centre at the origin, and let the
cylindrical polar coordinates of any point on it l e (p, , z), so that
the arc of a curve traced on the sphere is given by the formula \

 dsr =pHdcf>y+ i-p )- dpy.

* It is interesting to observe that, when Legendre had proved by
differentiation that EK' + E'K- KK' is constant, he used the results
of examples 4 and 5 to determine the constant, before using the
methods of § 22-8 example 3 and of § 22-737.

t Seiffert, Ueher eine neue geometrische Eiiifilhrung in die Theorie
der elliptisclien Funktiunen (Charlottenburg, 1896).

X This is an obvious transformation of the formula ds)"= dp)' + p'
(d(p)~ + dz) when p and z are connected by the relation p' + z'- — l.



528 THE TRANSCENDENTAL FUNCTIONS [CHAP. XXII

Seiffert's spiral is defined by the equation

(f) = ks,

where s is the arc measured from the pole of the sphere (i.e. the
point where the axis of s meets the sphere) and k is a positive
constant, less than unity*.

For this curve we have

and so, since s and p vanish together,

p = sn (s, k).

The cylindrical polar coordinates of any point on the curve expressed
in terms of the arc measured from the pole are therefore

 p, cf), z) = (sn s, /ts, en s) ;

and dn s is easily seen to be the cosine of the angle at which the
curve cuts the meridian. Hence it may be seen that, if K be the arc of
the curve from the pole to the equator, then sn s and cu s have period
AK, while dn 6* has period 2K.



REFERENCES.

A. M. Legendre, Traite des Fonctions Elliptiques (Paris, 1825-1828).

C. G. J. Jacobi, F\ hiidamenta Nova Theoriae Functionv.m Ellipticanim
(Konigsberg, 1829).

J. Tannery et J. Molk, Fonctions Elliptiques (Paris, 1893-1902).

A. Cayley, Elliptic Functions (London, 1895).

P. F. Verhulst, Traite eUmentaire des fonctions elliptiques (Brussels,
1841).

A. Enneper, Elliptische Funktionen, Zweite Auflage von F. Miiller
(Halle, 1890).

Miscellaneous Examples.

1. Shew that one of the values of



11 1 1

dnM + cnwV- /dn w-cn •wV'j U l-sn% / l + sni* \ \ '\

l+cn?t / \ - cnn J ) \ \ dinu — k'su.u) \ 6.nu + k' snu) J



is 2 (1 +k'). (Math. Trip. 1904.)

2. li x + iy = sn" ( u + iv) and x — iy — sn (% — iv), shew that

 ( -l)2+2/- - = ( -+y ) dn22( + cn2w.

(Math. Trip. 1911.)

3. Shew that

 l±cn( + .) l±cn( -i;) = /'-'?/= . 1 — '= sn w sn- V

4. Shew that

, cn u + cn V

1 +cn (w + v) en u-v) = - — -p, — 5— .

  1 — A: sn-M sn V

(Jacobi.) * If fc>l, the curve is imaginary.



THE JACOB IAN ELLIPTIC FUNCTIONS



529



p \ + cn(ti + v)cn(u-v) r . c o , i

5. Express — \ (— ; — 7 ( as a function or sn-'w + sn v.

   + an u + v)axi(u — v)



6. Shew that



7. Shew that



sn u — v) dn u + v) =



(Math. Trip. 1909. sn M dn -M en V — sn v dn i? en u



1 — 2 sn" u su" r



(Jacobi.)



    + k')s,r).us,n u- E)) l~ k') sn m sn u + K)] = su ( . + ) - sn uf.

(Math. Trip. 1914.)



8. Shew that



9. Shew that



sn (i< + |iA''') =



(1 -/(:')

\ 1 (1 + ) sn M + en w dni< 1 + sn M

. , , , , ,, 2sn?icnMdnv

sm am i ,, + v) + - m u-v) j-j ,~,

, , . , ., cn V — sn V dn m

cos am u + v)- m ( - )H 2 3 2,,s 2 •



(Jacobi.)



10. Shew that and hence express



dn u+v) dn u — v) =



ds2?ids- v + F '2



ns ?i ns V — -2 '

(u + v) — €2 iP u — vj — e-



as a rational function of ip u) and v). (Trinity, 1903.)

11. From the formulae for cn(2 — ) and diW 2K-u) combined with the
formulae for 1 +cn 2u and l + dnSw, shew that

(l-cn|ir)(H-dn§ ) = l. (Trinity, 1906.)

12. With notation similar to that of § 22*2, shew that

Ci d-i — c<idx en ( i + u-i) — dn ( Mj + 2) Sj-sa sn( i4- 2)

and deduce that, if Mi + (/o f-% + '4 = 2A', then

(Ci 0?9 - C2 0?i) (C3 c/4 - C4 ( 3) = 2 (sj \ s.,) (S3 - S4).

(Trinity, 1906.)

13. Shew that, if u + i + ?r = 0, then

1 - dn XL — dn v - dn- • + 2 dn m dn v dn u'=k* sn ?< sn v sn m;.

(Math. Trip. 1907.)

14. By Liouville's theorem or otherwise, shew that

dn?< dn u + w)-dn <;dn v + w) = k sn yen u sn (v + w) en (u + w)

— snMcn vsn u + xo) en (y + ;) .

(Math. Trip. 1910.)

15. Shew that

2 en Uo en u sn (?<2 - %) dn Mj + sn ( 2 - M3) sn (W3 - ?<i) sn (i<i -
Mo) dn ?ii dn Mo dn % = 0, the summation applying to the suffices 1,
2, 3. (Math. Trip. 1894.)

W. M. A. 34



530 THE TRANSCENDENTAL FUNCTIONS [CHAP. XXII

16. Obtain the formulae

sn 3 = AID, cii 3?< = BjD, dn 'Mi = CLD, where = 3s - 4 ( 1 + /f- ) s
+ Qk' s - k* s%

B = c l-4:S + 6k s*-4kU' + k s ,

C=d l-4k s + 6/5-2 4 \ 4X.2 c + ia gsx

D = l- 6 -2 S-* + 4:k- ( 1 + /?;2) §0 - Zk s% and s = sn , c' = cni(,
c/=dn;<.

17. Shew that

1 - dn 3u \ / 1 - dn ?t\ /I + a dn u + 02 dn u + c'3 dn ?< + a dn 1 +
dn 'iu \ l + dn tcj \ l - aj dn ?< + 2 dn 14 - ag dn % + a dn* uj
where Ui, ao, 3, a are constants to be determined. (Trinity, 1912.)



18. If



shew that



  . . , 1 + dn 3i( W + dn ?<



P u) + F tt+2iK')



sn 2u en u en 2m sn u '



F u)-F u + 2iK')

Determine the poles and zeros of P (u) and the fii'st term in the
expansion of the function about each pole and zero.

(Math. Trip. 1908.)

19. Shew that

sn ui + U-2 + U3) = A/D, en ui + U2 + ii ) = BjD, dn ui + M2 + %) =
C'/i), where

A=SiSoSi - 1 --?-2 + 2F2V-('<'-' + -'') 252 532+ 2ytisi2s22 532)

+ 2 s c.c dod i l + 2k' s.isi-B s. si)],

B =CiC2Cs 1 - k 2s.2 S3 +2k*S S2 S-/

+ 2 CiS2.?3O?20?3(- 1 +2Ps2 S32 + 2Fsi2-F25225g2))

C = d,d2d3 l-P2s./ss' + 2F-s, s.2H:i

+ F2 diS SsCoC-i ( - 1 +2/ -2s22s32 + 2Si2-F2S22532) , D=l - 2P2S2'-
S32 + 4 (F + F) Si2 2 3- " 2k*Si S2-S3~2Si- + B2s.*S3*,

and the summations refer to the suffices 1, 2, 3. (Glaisher,
Messenger, xi.)

20. Shew that

sn (ui + W2 + '-s) = - '/D', en ui + U2 + its) = '/- 'j dn (u + U2 +
u. ) = G'jD', where J' = 2siC2C3C 2 3--5iS2S3(l+ — 2si2+/
;4si2s22s32),

5' = Ci C2C3 (1 - * Si2s22 32) — di d d 'Ss s Ci di ,

C" = 0?i C?2 0?3 ( 1 — F Si2 §22 S32) - F Cj C2 C3 2 2 S3 <?! C l ,

Z)' = l - F 25-22 S32 + (F + *) Si2s22s32-FsjS2S32SiC2C3C 2' 3-

((/'ayley, Journal fur Matli. XLi.)

21. By applying Abel's method (§ 20*3 12) to the intersections of the
twisted curve ;j72\ |.y2\ i 22 + F\ .2\ .i itii the variable plane Ix
+ my + nz—l, shew that, if

'Mi + M2 + % + 4 = 0j

then Si Ci d 1 =0.

52 C2 2 1

53 C3 0?3 1

54 C4 0 4 1

OVjtain this result also from the equation

( 2 - i) ( 3 4 - 40 3) + ( 4 - s which may be proved by the method of
example 12.



 cid2~C2di) = 0,

(Cay ley, Messenger, xiv.)



THE JACOBIAN ELLIPTIC FUNCTIONS 531

22. Shew that

by expressing each side in terms of Si, §2, s , s ; and deduce from
example 21 that, if

iii + U2 + U3 + Ui = 0, then S4 Ci o?2 + S3 C2C/1 + 52 63( 4 + 51 040
3 = 0,

S4 C2 c i + S3 Ci o?2 + ''2 <?4 3 + 1 C3 c 4 = 0.

(Forsyth, Messenger, xiv.)

23. Deduce from Jacobi's fundamental Theta-function formulae that, if

Ui + U.2 + M3 + M4 = 0,

then k'- — k k"'si So S3 S4 4- k Ci c, c c — di d.2 d d = 0.

(Gudermann, Journal fur Math, xviii.)

24. Deduce from Jacobi's fundamental Theta-function formulae that, if

Ul + U.2 + lt3 + Ui = 0,

then F (si S2 C3 C4 - Ci 6'2 S3 S4) — c ia?2 + < 3 0 4= 0,

k' S1S2- 838 ) +did2C3Ci — CiC2d3di = 0,

.SiS2C 3( 4 — did2S Si + C3C4 — Ci C2 = 0.

(H. J. S. Smith, Proc. London Math. Soc. (1), x.)

25. If Ui + Uo + n3-)rUi = 0, shew that the cross-ratio of sn Ui, sn
Uo, sn M3, sn ?<4 is equal to the cross-ratio of sn (u + K)., sn
u-i+K), sn 11,3 + K\ sn Ui + K).

(Math. Trip. 1905.)

26. Shew that

iin ii,+v) sn (ji-f-y) sn (w- <;) fiTi? u-v) , Sk"' SiS2 CiC did

cn2 (w -I- v) en (m -f v) en ( — v) en- (w - v) (, - 'Si S2 J

dn' ( -|-i ) dn (m -t- 2 ) dn (?( - y) dn u — v)

(Math. Trip. 1913.)

27. Find all systems of values of u and f for which iivi u + iv) is
real when u and y are real and 0<k-<. (Math. Trip. 1901.)

28. If k' = J a ~ — a)'- , where < a < 1, shew that

•' 1 r- 4a3

 "(l-f-a2)(l + 2a-a- )'

and that sn'- fiT is obtained by writing —a~ for a in this expression.

(Math. Trip. 1902.)

29. If the values of en z, which are such that en 3z = a, are Cj, C2,
... Cg, shew that

9 9

3k* n Cr + k'* 2 c,. = 0.

(Math. Trip. 1899.)

y,. g-l-sn (M-l-y) \ 6-|-cn( f-f y) \ c + dn u + v)

a + su u — v) b + cn u — v) c + dii it — v)'

and if none of snv, cnu, dnu, 1 — -' sn wsn y vanishes, shew that ti
is given by the equation

X-2 (t' d' + b''- c ) sn- u = k'-+k' b'' - c\

(King's, 1900.)

34— :i



532 THE TRANSCENDENTAL FUNCTIONS [CHAP. XXII

31. Shew that

(Math. Trip. 1912.)

32. Shew that

l-sn(2 .r/7r)



"Jl-2j2n-lgin-j; + j4n-2



 dn 2KxlTr) — /;' sn (2A'j

(Math. Trip. 1904.)

33. Shew that if k be so small that k may be neglected, then

sn M = sin u — k-cosic. (t< — sin iccos u), for small values of ? .
(Trinity, 1904.)

34. Shew that, if | / x) \ < nl (r), then

4j" sin nx



log en (2 A'.r/Tr) = log cos .r — 2 -



 =iw l + (-?)"

(Math. Trip. 1907.) [Integrate the Fourier series for sn (2A'' /7r)dc
(2A'r/7r).]

35. Shew that

cn 71 dn'= m

(Math. Trip. 1906.) [Express the integrand in terms of functions of
2u.]

36. Shew that

/cnvdu \ . 5i (|a' + -| — i7r) i(| .r + 5y-|7r — -iTrr) i iZ + hirr)
snv-snT<~ Ml- - y) h il/ - i r) '* Ij y+i" ) '

where 2Kx = Tru, 2Ky=irv. (Math. Trip. 1912.)

37. Shew that

  jo l+cuu)dn u

(Math. Trip. 1903.)



, ,' + , , l+X-snasn/3

k I sn ?(a; = io2;



38. Shew that

/ I v'r -i/ /-/)/ — Irvnr

1 — /;sn asn j3'

(St John's, 1914.)

39. By integrating je ' dmicsudz round a rectangle whose corners are
±i7r, ± ir + cci (where 2Kz = itii) and then integrating by parts,
shew that, if </:- < 1, then

\ cos (iru/K) logsn t(,dii = K timh. ( irir). Jo'

(Math. Trip. 1902.)

40. Shew that K and K' satisfy the equation

c(l- ) + (l-2c) -i = 0,

where c = 2. j j-kJ deduce that they satisfy Legendre's equation for
functions of degree — with argument 1 — 2k' .



THE JACOBIAN ELLIPTIC FUNCTIONS 533

41. Express the coordinates of any point on the curve x +y =l in the
form 2.3 snMdnM-(l-cn u) \ 2 cos 3 tt (1 - on w) 1 + tan jLtt en u)

2.3isn%dntt + (l-cn w)2 2. 3 snwdn m + (1 -cnw)'

the modulus of the elliptic functions being sin yV 't ; and shew that

J X Jo

Shew further that the sum of the parameters of three coUinear points
on the cubic is a period.

[See Richelot, Journal fiir Math. ix. (1832), pp. 407-408 and Cayley,
Proc. Camb. Phil. Soc. IV. (1883), pp. 106-109. A uniformising
variable for the general cubic in the canonical form X + F + Z +
6rnXYZ=0 has been obtained by Bobek, Einleitung in die Theorie dex
elliptischen Funktionen (Leipzig, 1884), p. 251. Dixon Quarterly
Journal, xxiv. (1890), pp. 167-233) has developed the theory of
elliptic functions by taking the equivalent curve a +y' — cuicy=\ as
fundamental, instead of the curve

y2=(l-. 2)(1\ .2 .2).]

42. Express I 2x-x-) (4 -2 + 9) ~ " dx in terms of a complete elliptic
integral of the first kind with a real modulus. (Math. Trip. 1911.)

43. If u=l t + ) t: + t + ) ]- dt,

express x in terms of Jacobian elliptic functions of u with a real
modulus.

(Math. Trip. 1899.)



44. If i= P (1+ 2 \ 2 4)- i ;



express x in terms of by means of either Jacobian or Weierstrassian
elliptic functions.

(Math. Trip. 1914.)

45. Shew that

2' 7r

(Trinity, 1881.)

46. When a>x> >y, reduce the integrals

\ \ a-t) t- ) t-y)]- dt, j-" (a-t) t- )(t-y) -idt

by the substitutions

x-y = a- y) dn w, x — y = (p-y) nd- ??

respectively, where k' = a — )/ a — y).

Deduce that, ii u + v = K, then

1-sn u — sn v + k'sn uiin v = 0.

By the substitution y = a — t) t — j3)l t - y) applied to the above
integral taken between the limits j3 and a, obtain the Gaussian form
of Landen's transformation,

I a cos d + bi' Hm-d)~ de= I a- eos 6 + b' sin (9) ~ a dd,

where ai, bi are the arithmetic and geometric means between a and b.

(Gauss, Werke, ill. p. 352; Math. Trip. 1895.)



534 THE TRANSCENDENTAL FUNCTIONS [cHAP. XXII

47. Shew that

sc ?f = - k' - 1 C u - A') - C (u - K - 2iK') - C 2iK% where the
Zeta-functions are formed with periods 2a)i, 2co2 = 2K, AiE'.

(Math. Trip. 1903.)

48. Shew that E — k" K sati.sfies the equation

where c=lfi, and obtain the primitive of this equation. (Math. Trip.
1911.)

49. Shew that ni k K' dk= n- ) ( /(•'' - E' dk,

(71 + 2) j t'E'dk= n + l) I k' K'dk. (Trinity, 1906.)

50. If u W"" t t) ct) ~ dt,

  J

shew that o(c-l) +(2c-l) + 4. = | 3

51. SHew that the primitive of

du u k \

dk' J' T '

A E-K) + A'E'



(Trinity, 1896.)



'""'AE+A'iE'-K'y

where J, A' are constants. (Math. Trip. 1906.)

52. Deduce from the addition-formula for E u) that, if

Ui + U2 + U + Ui = 0,

then (sn ?<! sn l<l — sn u sn Ui) sn (uy + %i

is unaltered by any permutation of suffices. (Math. Trip. 1910.)

53. Shew that

(Math. Trip. 1913.)

54. Shew that

U- i' u cd udu = 2K (2 + P) A'- 2 (1 -l-F) E). [Write = A''+r.] '
(Math. Trip. 1904.)

55. By considering the curves ?/2 = (i \ .) (1 \ .2 ,) y = l- mx +
n.v' , shew that, if u 1 + Uo + Us + Ui = 0, then

E %lx)+E U2) + E ll3) + E Ui) = k\ 2 .V + 2CiC2C3C4-2SiS2S3S4-2r.

(Math. Trip. 1908.)

56. By the method of example 21, obtain the following seven
expressions for

E ui) + E u2) + E(u3) + E tii) when Ui + U2 + U3 + Ui = 0:

l + k s s Si r=i ' ' " k'-' + d d dsdi r=i '' '' ' k cc c c, - X-'2
,=1 ' ' '' ' k SiS2S3Sidid 2d3di * c Ks d ) - k CiC2 C3Cidi d2d3di * g
u d)

Pkf SiS2S3Si-did2d3di r=l d d2d3di-lrk' CiC2C3Ci r=l

k'SiS2S3Si + CiC2C3Ci *

ClC2C3C4 + A:2siS253S4 ,.=i

4

- 2;(gj52S3S4)-l + (CiC2C3C4)-l + X' (a?lC/2C 3<: 4)~' ~ 2 ll s,Crdr).

r=l

(Forsyth, Messenger, xv.)



THE JACOBIAN ELLIPTIC FUNCTIONS 535

57. Shew that

when I I x) | <7r/(r) ; and, by differentiation, deduce that

6 f y ns* [— ") = 6 cosec* x + i (1 + F) ( y - 1 cosec x

TT \ TT /

 -, * f 79S /2/r\ 2 "] o2 COS 2 A'

Shew also that, when | /( ) | < |7r/(r),



n.M + 2



2Kx\ \ - fl-hf-2 \ 2n + lf fjn yi 27rg"- - sin (2ot + 1) j - ,!o 1 2P
' 2F - \ 2k) j (1 - ? - 1) •

(Jacobi.)

58. Shew that, if a be the semi-major axis of an ellipse whose
eccentricity is sin j-Vtt, the perimeter of the ellipse is

(Ramanujan, Quarterly Journal, XLV.)

59. Deduce from examjjle 19 of Chapter xxi that

TO o —k" + dn udn3u , ,\ k' + k cn u en Su

F cn3 2u = - — T-, — 5 r— , dn 2u = r— 71 — 5 i~ •

l+k sn usn3u l + Fsn wsnSM

(Trinity, 1882.)

60. From the formula sd i.u, k) = i sd (n, k') deduce that

rq '+i . f n + i)7ru\ \ 1 - (-)'*?i' "' ;. f 7i + i)nu

K'



1 1 ( rv i.br ' i V- i - g ""'



where q = exp ( — ttK'/K), q = exp ( — irKjK'),

and u lies inside the parallelogram whose vertices are

±iK±K'. By integrating from u to K\ from to u and again from ti to K',
prove that

[A formula which may be derived from this by writing 2i = \$ + iri,
where | and rj are real, and equating imaginary parts on either side
of the equation was obtained by Thomson and Tait, Natural Philosopki
ii. (1883), p. 249, but they failed to observe that their formula was
nothing but a consequence of Jacobi's imaginary transformation. The
formula was suggested to Thomson and Tait by the solution of a problem
in the theory of Elasticity.]

