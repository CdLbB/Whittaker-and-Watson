\chapter{The Confluent Hypergeometric Function}

\Section{16}{1}{The confluence of two singularities of Riemanns equation}

We have seen \hardsectionref{10}{8}) that the linear differential equation with two
regular singularities only can be integrated in terms of elementary
functions; while the solution of the linear differential equation
with three regular singularities is substantially the topic of Chapter
xiv. As the next type in order of complexity, we shall consider a
modified form of the differential equation which is obtained from
Riemann's equation by the confluence of two of the singularities. This
confluence gives an equation with an irregular singularity
(corresponding to the confluent singularities of Riemann's equation)
and a regular singularity corresponding to the third singularity of
Riemann's equation.

The confluent equation is obtained by making c -* oo in the equation
defined by the scheme

X c

-.--m -c c - k

P

1;

2

.7 - ni k

The equation in question is readily found to be

We modify this equation by writing TODO and obtain as the equation*
for Wkmi)

The reader will verify that the singularities of this equation are at
6 and X, the former being regular and the latter irregular; and when
2m is not an integer, two integrals of equation (B) which are regular
near and valid for all finite values of z are given by the series

* This equation was given by Whittaker, Bulletin American Math. Soc.
x. (1904), pp. 125-134. W. M. A. 22

%
% 338
%

These series obviously form a fundamental system of solutions.

[Note. Series of the type in \{ \} have been considered by Kummer* and
more recently by Jticobsthalt and Barnes J; the special series in
which k = had been investigated by Lagrange in 1762-1765 (Oeuvres, I.
p. 480). In the notation of Kummer, modified by Barnes, they would be
written iF-i \{h±ni - k; ± 2m + 1; 2\}; the reason for discussing

solutions of equation (B) rather than those of the equation 2

which iFi (a; p; z) is a solution, is the greater appearance of
symmetry in the formulae, together with a simplicity in the equations
giving various functions of Applied Mathe- matics (see\hardsectionref{16}{2}) in
terms "of solutions of equation (B).]

16"11. Kummer s formulae.

(I) We shall now shew that, if '2'm is not a negative integer, then

---il/,,() = (-)-i\~''W,,,(-), that is to say,

., (I + m - k) (f + m - k)

 "" -" 2 ! (2w + 1) (271 + 2)

h

+ m -

k

1!

\{2m +

1)

i

+ m +

k

  . ( + w + k) (f + m + k) ., 

l!(2m + l) 2 ! (2m + 1) (2m +2) ' '■**

For, replacing TODO by its expansion in powers of z, the coefficient
of TODO in the product of absolutely convergent series on the left is

by\hardsubsectionref{14}{1}{1}, and this is the coefficient of " on the right§; we have
thus obtained the required result.

This will be called Kummej-'s first formula.

(II) The equation

valid when 2m is not a negative integer, will be called Kummer s
second formula.

To prove it we observe that the coefficient of £"+"'+? in the product
TODO

* Journal fur Math. xv. (1836), p. 139. t Math. Ann. lvi. (1903), pp.
129-154. X Trans. Camb. Phil. Soc. xx. (1908), pp. 253-279.

§ The result is still true when ?h +  + k is a negative integer, by
a slight modification of the analysis of\hardsubsectionref{14}{1}{1}.

%
% 339
%

of which the second and third factors possess absolutely convergent
expansions, is \hardsubsectionref{3}{7}{3})

nl\{2m+l)\{2m + 2) ...\{27n+7i); - -

by Kummer's relation*

TODO

valid when a' i; and so the coefficient of 2"+'""'" (by 
\hardsubsectionref{14}{1}{1}) is

(| + w)(§ + m) ... \{n-m-- ) T \{-n + h-m)r \{ ) n (2i+l)(2m +
2)... (2m + 7i) T \{-m-hi)T\{-n)

n (2m + 1 ) \{2m -- 2) . . . \{2m + n) r ( - w - hi) r ( - 1?0 '
and when n is odd this vanishes; for even values of n\{ = 2p) it is

r(i-/)(-)(-|)...(i-p)

l.S...\{2p-l) 1

2pl 23p(m + l)(m + 2)...(m+p) 2*P.p \{m + 1) \{m + 2) ... \{m+p)'

16'12. Definition f of the function Wic,m (2) 

The solutions -il/t.imC') of equation (B) of\hardsectionref{16}{1} are not,
however, the most convenient to take as the standard solutions, on
account of the disappearance of one of them when 2m is an integer.

The integral obtained by confluence from that of\hardsectionref{14}{6}, when
multiplied by a constant multiple of TODO is TODO:

It is supposed that arg has its principal value and that the contour
is so chosen that the point t = - z is outside it. The integrand is
rendered one- valued by taking; arg (- ) |  tt and taking that
value of arg (1 + t/z) which tends to zero as i -*- by a path lying
inside the contour.

Under these circumstances it follows from\hardsubsectionref{5}{3}{2} that the integral is
an analytic function of z. To shew that it satisfies equation (B),
write

* See Chapter xiv, examples 12 and 13, p. 298.

+ The function TODO was defined by means of an integral in this manner
by Whittaker, ioc. cit. p. 125.

* A suitable contour has been chosen and the variable t of\hardsectionref{14}{6}
replaced by - 1.

22-2

%
% 340
%

and we have without difficulty*

TODO

since the expression in \{ \} tends to zero as i -* + oo; and this is
the condition that TODO should satisfy (B).

Accordingly the function Wk, m \{z) defined by the integral

is a solution of the differential equation (B).

The formula for TfA;,m () becomes nugatory when ' - 2 - w is a
negative integer. To overcome this difficulty, we observe that
whenever

and k - - m is not an integer, we may transform the contour integral
into an infinite integral, after the manner of\hardsubsectionref{12}{2}{2}; and so, when

Rik -i-:- 7n 0,

This formula suffices to define Wimiz) in the critical cases when m
+ 2->t is a positive integer, and so Wk,m\{z) is defined for all
values of k and m and all values of z except negative real values f.
Example. Solve the equation

du ( b c 

in terms of functions of the type Tft, (2), where a, 6, c are any
constants.

\Section{16}{2}{Expression of various functions by functions of the
  type Wk,n(z)}

It has been shewn:!: that various functions employed in Applied Mathe-
matics are expressible by means of the function Wk,m (2); the
following are a few examples :

* The differentiations under the sign of integration are legitimate by\hardsubsectionref{4}{4}{4} corollary.

t When z is real and negative, TODO may be defined to be either TODO
or TODO whichever is more convenient.

+ Whittaker, Bulletin American Math. Soc. x.; this paper contains a
more complete account than is given here.

%
% 341
%
(I) The Error function* which occurs in connexion with the theories of
Probability, Errors of Observation, Refraction and Conduction of Heat
is defined by the equation

Erfc()=[ e-''dt, where a; is real.

Writing t = x\{w - l) and then iu = s/x in the integral for TT.j.
.(a), we get

Ji = TODO

J X

and so the error function is given by the formula

Erfc (a;) = TODO

Other integrals which occur in connexion with the theory of Conduction
rb of Heat, e.g. TODO, can be expressed in terms of error functions,
and

J a

SO in terms of TODO functions.

Exaviple. Shew that the formula for the eri'or function is true for
complex values of x.

(II) The Incomplete Gamma function, studied by Legendre and othersf,
is defined by the equation

y\{n, x)= j f'-'e-dt. Jo

By writing t = s - x in the integral for Wi.,,i in(x), the reader
will

verify that

TODO

(III) The Logarithmic-integral function, which has been discussed by
Euler and others :|:, is defined, when | arg \{- log ■j < tt, by the
equation

* This name is also applied to the function

Erf(x)= I ''e-«'d( = 7r-Erfc(j-).

\{ Legendre, Exercices, i. p. 339; Hocevar, Zeitschrift fiir Math,
und Phys. xxi. (1876), p. 449; Schlomilch; Zeitschrift fur Math, und
Phys. xvi. (1871), p. 261; Prym, Journal filr Math, lxxxii. (1877),
p. 165.

X Euler, Inst. Calc. Int. i.; Soldner, Monatliche- Correspondenz, von
Zach (1811), p. 182; Briefwechsel zwischen Gauss und Bessel (1880),
pp. 114-120; Bessel, Konigsherger Archiv, i. (1812), pp. 369-405;
Laguerre, Bulletin de la Soc.Math.de France, vii. (1879), p. 72;
Stieltjes, Ann. de VEcole norm. sup. (3), iii. (1886). The
logarithmic-integral function is of considerable importance in the
higher parts of the Theory of Prime Numbers. See Landau, Primzahlen,
p. 11.

%
% 342
%

On writing 5 - log z = u and then u = - log t in the integral for

it may be verified that

TODO

It will appear later that Weber's Parabolic Cylinder functions
(\hardsectionref{16}{5})
and Bessel's Circular Cylinder functions (Chapter xvii) are
particular cases of the Wkrn function. Other functions of like
nature are given in the Miscellaneous Examples at the end of this
chapter.

[Note. The error function has been tabulated by Encke, Berliner ast.
Jahrbuch, 1834, pp. 248-304, and Burgess, Tram. Roy. Soc. Edin. xxxix.
(1900), p. 257. The logarithmic- integral function has been tabulated
by Bessel and by Soldner. Jahnke und Emde, Fiinktionentafeln (Leipzig,
1909), and Glaisher, Factor Tables (London, 1883), should also be
consulted.]

\Section{16}{3}{The asymptotic expansion of TT'a;,w \{z), when z is large}
From the contour integral by which Wk,rn,\{z) was defined, it is
possible to obtain an asymptotic expansion for W]cm\{z) valid when
|arg| < tt.

For this purpose, we employ the result given in Chap, v, example 6,
that

1+-) =1+- + ...+ - - + Rn (t, Z),

z) TODO

Substituting this in the formula of\hardsubsectionref{16}{1}{2}, and integrating
term-by-term, it follows from the result of\hardsubsectionref{12}{2}{2} that

yyjc,m\{2) =  -z ji + YY +- 212 + 

[m - (k - y-] \{m' -(k- f] . . . (m -\{k-n-if f]

where

n ! z

I ««

provided that n be taken so large that R \{n - k---- m j > 0.

Now, if I arg 2 |  tt - a and z > 1, then

11(1 + /0)11+ R\{z)Q I (1 + tjz) 1  sin a R (z)  O] '

and so*

TODO

* It is supposed that is real; the inequality has to be slightly
modified for complex values of

%
% 343
%

Therefore Rn\{t,z)

TODO

since 1 +;< < 1 + t

Therefore, when j  | > 1,

= TODO

= TODO

since the integral converges. The constant implied in the symbol is
independent of arg2, but depends on a, and tends to infinity as a-
0.

That is to say, the asymptotic expansion of TODO is given by tlie
formula

TODO

for large values ofz when arg 2; |  tt - a < tt.

16*31. The second solution of the equation for TODO

The differential equation (B) of\hardsectionref{16}{1} satisfied by Wk,m\{z) is
unaltered if the signs of z and k are changed throughout.

Hence, if TODO is a solution of the equation.

Since, when $absval{\arg z} < Tr$,

TODO

whereas, when | arg ( - 2) | < tt,

W.k,.n\{-z) = e'H-z)-[l + 0\{z-%

the ratio W]cm\{z)jW-]cjn\{- z) cannot be a constant, and so
Wkm\{z) and -k,m\{- z) form a fundamental system of solutions of
the differential equation.

\Section{16}{4}{Contour integrals of Barnes type for Wc,n(z)}

Consider now

TODO

where | arg; ] < - tt, and neither of the numbers k + m + 5 is a
positive integer

%
% 344
%

or zero*; the contour has loops if necessary so that the poles of T
(s) and those of r ( - s - k-m + 5 ) F (- s - k + m + j are on
opposite sides of it.

It is easily verified, by\hardsectionref{13}{6}, that, as s->cc on the contour,

TODO

and so the integi-al represents a function of 2 which is analytic at
all points f in the domain j arg z TODO

Now choose N so that the poles oi T (- s - k - m + j T (- s - k+ m +
j

are on the right of the line R(s) = - N-; and consider the integral
taken

round the rectangle whose corners are TODO, where  is
positive J and large.

The reader will verify that, when 1 arg z <x, the integrals TODO

tend to zero as - > 00; and so, by Cauchy's theorem,

TODO

where Rn is the residue of the integrand at s = - n.

Write s = - N - I + it, and the modulus of the last integrand is

where the constant implied in the symbol is independent of z.

Since TODO converges, we find that

TODO

* In these cases the series of\hardsectionref{16}{3} terminates and TODO is a
combination of elementary' functions.

t The integral is rendered one- valued when J? (z) <0 by specifying
arg z.

X The line joining ±j may have loops to avoid poles of the integrand
as explained above.

%
% 345
%
But, on calculating the residue Rn, we get

TODO

and so TODO has the same asymptotic expansion as TODO.

Further / satisfies the differential equation for Wk, m () ! for, on

substituting 1 r\{s) F (- s - k - m + j F (- s - k + m + j zds
for v in

the expression (given in\hardsubsectionref{16}{1}{2})

TODO

we get

TODO

Since there are no poles of the last integrand between the contours,
and since the integrand tends to zero as | * j - > oo, s being
between the contours, the expression under consideration vanishes, by
Cauchy's theorem; and so / satisfies the equation for TTjt, m \{2).

Therefore TODO,

where A and B are constants. Making $\absval{TODO} \rightarrow \infty$
when R(z)>0 we see, from the asymptotic expansions obtained for / and
W±k,m\{± ). that

 = 1, 5 = 0.

Accordingly, by the theory of analytic continuation, the equality

I=W,,,n(z) persists for all values of z such that argr|<7r; and, for
values* of arg' such that TT < I arg  I < I TT, Wk,m (z) may be
defined to be the expression /.

Example 1. Shew that

taken along a suitable contour.

* It would have been possible, by modifying the path of integration in\hardsectionref{16}{3}, to have shewn that that integral could be made to define an
analytic function when $\arg z < TODO$. But the reader will see that
it is unnecessary to do so, as Barnes' integral affords a simpler
definition of the function.

%
% 346
%

Example 2. Obtain Barnes' integral for TODO by writing for TODO in the
integral of TODO and changing the order of integration.

TODO

16"41. Relations betiveen Wk,m\{z) '-nd Mk,±m\{z)- If we take the
expression

TODO

which occurs in Barnes' integral for TODO. and write it in the form

TODO

r(s + k + m + )r\{s + k - m + )cos\{s + k + m) ir cos \{s--k - m)
it ' we see, by\hardsectionref{13}{6}, that, when B (s)  0, we have, as, 5 | - > x
,

F\{s) = exp|C-s--2'jlog5 +

sec \{s + k -- m) tt sec \{s + k - m) ir.

Hence, if | arg 2 | <  tt, jF\{s)z--ds, taken round a semicircle
on the

right of the imaginary axis, tends to zero as the radius of the
semicircle tends to infinity, provided the lower bound of the distance
of the serai- circle from the poles of the integrand is positive (not
zero).

Therefore Tf,,..(.) = - r(,,, +)r(-yfc + m + f) '

where SR' denotes the sum of the residues of F(s) at its poles on the
right of the contour (cf.\hardsectionref{14}{5}) which occurs in equation (C) of §
16*4.

Evaluating these residues we find without difficulty that, when

I arg 2 : < I TT, and 2m is not an integer*,

,,, T(-2m),.,, r(2m),,,

Example 1. Shew that, when | arg ( -s) | <|7r and 2m is not an
integer,

\addexamplecitation{Earnest.} Example 2. AVhen - -stt < arg s < f tt and - f tt < arg ( -
) < tt, shew that

* When 1m is an integer some of the poles are generally double poles,
and their residues involve logarithms of z. The result has not been
proved when fe- |i 7h. is a positive integer or zero, but may be
obtained for such values of k and m by comparing the terminating
series for ')t,m (2) with the series for Mj.±, (2).

t Barnes' results are given in the notation explained in\hardsectionref{16}{1}.

%
% 347
%

Example 3. Obtain Kummer's first formula \hardsubsectionref{16}{1}{1}) from the result

TODO \addexamplecitation{Barnes.}

Inl J -xi

\Section{16}{5}{The parabolic cylinder functions. Weber's equation}

Consider the differential equation satisfied by TODO it is

TODO

this reduces to TODO

Therefore the function satisfies the differential equation

Accordingly Dn(z) is one of the functions associated with the
parabolic cylinder in harmonic analysis*; the equation satisfied by it
will be called Weber's equation.

From\hardsubsectionref{16}{4}{1}, it follows that

Q

when I arg z < -tt.

But

z

4 -i

and these are one-valued analytic functions of z throughout the
TODO-plane. Accordingly Dn (z) is a one-valued function of z
throughout the TODO-plane; and,

by\hardsectionref{16}{4}, its asymptotic expansion when arg  < - tt is

TODO

16'51. The second solution of Weber's equation.

Since Weber's equation is unaltered if we simultaneously replace n and
z by - n - 1 and + iz respectively, it follows that Dn-i (iz) and
D-n-i (- iz) are solutions of Weber's equation, as is also Dn (- z).

* Weber, Math. Ann. i. (1869), pp. 1-36; Whittaker, Proc, London Math.
Soc. xxxv. (1903), pp. 417-427.

%
% 348
%

It is obvious from the asymptotic expansions of Dn\{z) and
Z)i(2e'*), valid in the range -  tt < arg z < -ir, that the
ratio of these two solutions is not a constant.

16'511. The relation between the functions Dn\{z), Dn (+ iz).

From the theory of linear diiSerential equations, a relation of the
form Dn \{z) = aDn-i \{iz) + h Di (- iz) must hold when the
ratio of the functions on the right is not a constant.

To obtain this relation, we observe that if the functions involved be
expanded in ascending powers of z, the expansions are

H i=nr - 1 -  - z + ... ■

Comparing the first two terms we get

a = (27r) - * r (w + 1) e'''' h = (27r) "  T (w + 1) e " and so

r(n + l)

I>n\{z) =

TODO

16'52. The general asymptotic expansion of Dn \{z). So far the
asymptotic expansion of D (z) for large values of z has only

been given (§ ] 6*5) in the sector arg z < jTt. To obtain its form for
values

of arg z not comprised in this range we write - iz for z and -n - 1
for 7i in the formula of the preceding section, and get

TODO

Now, if TODO, we can assign to TODO and TODO arguments between

TODO + J TT; and arg (- z) = arg z - ir, arg (- iz) = arg z - tt;
and then, applying the asymptotic expansion of\hardsectionref{16}{5} to Dn\{- z) and
Dn-i\{-iz), we see that, TODO

%
% 349
%

This formula is not inconsistent with that of\hardsectionref{16}{5} since in their
common range of validity, viz. TODO for all positive values of w.

To obtain a formula valid in the range TODO, we use the formula

and we get an asymptotic expansion which differs from that which has
just been obtained only in containing e""'' in place of e'"'.

Since Dn(z) is one-valued and one or other of the expansions obtained
is valid for all values of arg z in the range - tt  arg z tt, the
complete asymptotic expansion of i) (z) has been obtained.

\Section{16}{6}{A contour integral for Dn(z)}

TODO Consider TODO, where TODO; it represents a one-valued

analytic function of z throughout the -plane \hardsubsectionref{5}{3}{2}) and further

the differentiations under the sign of integration being easily
justified; accordingly the integral satisfies the differential
equation satisfied by e \~ i Z) (2); and therefore

e-'- e-'f-i-t)-'-'dt = aB\{z) + bDi\{iz),

where a and b are constants.

Now, if the expression on the right be called E (2)5 we have

En\{0)= e-¥\{-t)--dt, En'\{0)= e-¥-t)-dt.

To evaluate these integrals, which are analytic functions of n, we
suppose first that R \{n) <0; then, deforming the paths of
integration, we get

TODO

Similarly TODO.

Both sides of these equations being analytic functions of TODO, the
equations are true for all values of n; and therefore

TODO

Therefore TODO.

%
% 350
%

16*61. Recurrence formulae for Dn \{z). From the equation

after using\hardsectionref{16}{6}, we see that

Dn+, (z) - z Dn (z) + n Dn-, \{z) = 0. Further, by differentiating the
integral of\hardsectionref{16}{6}, it follows that

D \{z) + zDn \{z) - nDn-, \{z) = 0. Example. Obtain these results
from the ascendiug power series of\hardsectionref{16}{5}.

\Section{16}{7}{Properties of Dn (z) when n is an integer}

When n is an integer, we may write the integral of\hardsectionref{16}{6} in the form

TODO

If now we write t = v - z, we get

TODO

a result due to Hermite*.

Also, if m and n be unequal integers, we see from the differential
equations that

Dn \{z) Dm" (z) - D, \{z) Dn" (z) + (m - n) Dm (z) Dn (z) = 0, and so

Dn\{z)DJ\{z)-Dm(z)Dn'\{z)

\{m - 71) I Di (z) Dn (z) dz =

J -ex

= 0,

by the expansion of\hardsectionref{16}{5} in descending powers of z (which terminates
and is valid for all values of arg z when n is a positive integer).

Therefore if m and n are unequal positive integers

D,iz)Dn\{z)dz=0.

■>

Comptes Rendus, lviii. (1864), pp. 266-273.

%
% 351
%

On the other hand, when 7n = n, we have

J - cc

= D, (Z) Dn+, (Z)] + I \l zDn iz) D,,, \{Z) - Dn+ (z) D' (z) dz

=r [D,,\{z)Ydz,

J -X

on using the recurrence formula, integrating by parts and then using
the recurrence formula again.

It follows by induction that

f \{Dr,(z)Y-dz = nir [D,\{z)Ydz

J -CO

= (27r)n!, by\hardsubsectionref{12}{1}{4} corollary 1 and\hardsectionref{12}{2}.

It follows at once that if, for a function /(), an expansion of the
form

/(z) = aoDo (z) + a, D,\{z)+... + aDn \{z)+ ...

exists, and if it is legitimate to integrate term-by-term between the
limits - oc and oo, then

TODO

REFERENCES.

W. Jacobsthal, Math. Ann. LVi. (1903), pp. 129-154.

E. W. Barnes, Trans. Camb. Phil. Soe. xx. (1908), pp. 253-279.

E. T. Whittaker, Bulletin American Math. Soc. x. (1904), pp. 125-134.

H. Weber, Math. Ann. i. (1869), pp. 1-36.

A. Adamoff, Ann. de VInstitut Polytechnique de St Petershourg, v.
(1906), pp. 127-143.

E. T. Whittaker, Proc. London Math. Soc. xxxv. (1903), pp. 417-427.

G. N. Watson, Proc. London Math. Soc. (2), viii. (1910), pp. 393-421;
xvn. (1919), pp. 116-148.

H. E. J. CuRZON, Proc. London Math. Soc. (2), xii. (1913), pp.
236-259.

A. Milne, Proc. Edinburgh Math. Soc. xxxii. (1914), pp. 2-14; xxxill.
(1915), pp. 48-64.

N. Nielsen, Meddelelser K. Danske Videnskabernes Selskab, i. (1918),
no. 6.

352 the transcendental functions [chap. xvi

Miscellaneous Examples.

1. Shew that, if the integral is convergent, then

TODO

2. Shew that TODO

3. Obtain the recurrence formulae

4. Prove that Ht,,1(2) is the integral of an elementary function when
either of the numbers k-h + m is a negative integer,

5. Shew that, by a suitable change of variables, the equation can be
brought to the form

\{a.2 + b2.v) + \{ai + biX)-£ + \{ao + box)i/ =

derive this equation from the equation for F\{a, b; c; x) by writing x
= lb and making 6-*-« .

6. Shew that the cosine integral of Schlomilch and Besso \{Oiornale di
Matematiche, VI.), defined by the equation

TODO

is equal to TODO

Shew also that Schlomilch's function, defined \{Zeitschrift filr Math,
und Physik, iv. (1859), p. 390) by the equations

S\{v,z)=j \{l + t)-''e-'dt = z''-e' I du,

is equal to i" - 1 i W  1  1  1  (2).

7. Express in terms of W,ci functions the two functions

TODO

Jot J z t

8. Shew that Sonine's polynomial, defined \{Math. Ann. xvi. p. 41) by
the equation TODO

is equal to TODO

%
% 353
%

9. Shew that the function TODO defined by Lagrange in 1762-1765
\{Oeuv)-es, i. p. 520) and by Abel (Oeuvres, 1881, p. 284) as the
coefficient of /('*' in the expansion of (1 - A)-i e-''Mi-ft) is eqnal
to

10*. Shew that the Pearson-Cunningham function \{Proc. Royal Soc.
Lxxxi. p. 310), <>>n,m (■')) defined as

T\{n

n- hm

is equal to  TODO

11. Shew that, if | arg z < n, and | arg (1 + ) | < tt,

\addexamplecitation{Whittaker.}

12. Shew that, if n be not a positive integer and if ! arg z < frr,
then

TODO

and that this result holds for all values of args if the integral be /
, the contours enclosing the poles of r ( - but not those of r \{t
-n).

13. Shew that, if j arg a | < it,

J tc

= - r, n(-*". hn-¥i<;hn-hn + - l-ia-).

r(-w)r(m-i + l)aH'+i)

14. Deduce from example 13 that, if the integral is convergent, then

|J e - i'' z""' A«+i (2) c/2 = (v/2)-i-'" r (i + 1) sin ( - \{m) n.

\addexamplecitation{Watson.}

15. Shew that, if n be a positive integer, and if

E, \{x) = T J- '' \{z - .v) - 1 n,, (z) dz,

then TODO.

the upper or lower signs being taken according as the imaginary part
of x is positive or negative. \addexamplecitation{Watson.}

16. Shew that, if n be a positive integer,

TODO

Jo sm

where fi is n or J(n- 1), whichever is an integer, and the cosine or
sine is taken as n is even or odd. \addexamplecitation{AdamoflF.}

* The results of examples 8, 9, 10 were communicated to us by Mr
Bateman. W. M. A. 23

%
% 354
%

17. Shew that, if n be a positive integer,

where TODO

TODO \addexamplecitation{Adamoff}

18. With the notation of the preceding examples, shew that, when x is
real,

TODO

while Ja satisfies both the inequalities

TODO

Shew also that as v increases from to 1, o- (/') decreases from to a
minimum at- TODO and then increases to at i'=l; and as v increases
from 1 to $\infty$, o-(v) increases to a maximum at l + /?2 and then
decreases, its limit being zero; where

TODO \addexamplecitation{Adamoff.}

19. By employing the second mean value theorem when necessary, shew
that

A.(.') = V2.(v'0 e

cos(.r«2 -:-)iTr) + '

s'n J'

where co,j(.r) satisfies both the inequalities

I X I 'TT D

when X is real and n is an integer greater than 2. \addexamplecitation{AdamofF.}

3-35... ia;2,,,, .1 i

20. Shew that, if n be positive but otherwise unrestricted, and if m
be a positive integer (or zero), then the equation in z

has m positive roots when TODO. \addexamplecitation{Milne.}