\chapter{The Hypergeometric Function} 

141. The hypergeometric series. 

We have already (§ 2*38) considered the hypergeometric series* 

1 +  - . ,  (  + l) (  + l)   a(a + l)(a+2)6(6+l)(6 + 2) 3 
l.c 1.2.c(c + l)   "*  1.2.3.c(c + l)(c + 2)  " ••  

from the point of view of its convergence. It follows from § 2-38 and § 5-3 
that the series defines a function which is analytic when i 2  j < 1. 

It will appear later (§ 14-53) that this function has a branch point at 2 = 1 
and that if a cutf (i.e. an impassable barrier) is made from + 1 to +   along 
the real axis, the function is analytic and one-valued throughout the cut 
plane. The function will be denoted by F  a, b;c;z). 

Many important functions employed in Analysis can be expressed by 
means of hypergeometric functions. Thus| 

(1 +2)" = F(-n,  : ;-z), 

log l + z) = zF(l, l;2;-z), 

e'=\ im F l,0;l;zl0). 



Example. Shew that 
d 



F a, b; c; z) = ~F a + \, 6 + 1; c + 1; 2). 



1411. The valuel of F(a, b; c; I) when E(c-a-b)>0. 

The reader will easily verify, by considering the coefficients of .2;'* in the 

* The name was given bj- Wallis in 1655 to the series whose Kth term is 
a   + b   a + 2b  ...  a +  n-l)b . 
Euler used the term hypergeometric in this sense, the modern use of the term being apparently 
due to Kummer, Journal fiir Math. xv. (1S36). 

t The plane of the variable z is said to be cut along a curve when it is convenient to consider 
only such variations in z which do not involve a passage across the curve in question ; so that 
the cut may be regarded as an impassable barrier. 

X It will be a good exercise for the reader to construct a rigorous proof of the third of these 
results. 

§ This analysis is due to Gauss. A method more easy to i-emember but more difficult 
to justify is given in § 14-6 example 2. 



282 THE TRANSCENDENTAL FUNCTIONS [CHAP. XIV 

various series, that if \$ ir < 1, then 
c c-l- 2c-a-b-l)x]F (a, b : c ; x) + (c - a) c - b) xF (a, b ; c + I ; x) 

= c(c-l) l-x)F(a,b;c-l;x) 

= C(C -l)jl+   (Un-Un-i)x'>'l, 

where Un is the coefficient of  " in i''(a, 6 ; c — 1 ; x). 

Now make x—>l. By § 3*71, the right-hand side tends to zero if 

1+2 (un — tin-i) converges to zero, i.e. if ?i,j— >0, which is the case when 
 =i 

R c- a - 6) > 0. 

Also, by § 2*38 and § 3'71, the left-hand side tends to 

c(a+b-c)F a,b;c;l) +  c-a)(c- b) F(a, b;c+l;l) 
under the same condition ; and therefore 

Repeating this process, we see that 

F(a, b: c;l) = i Yl ) ——  -, ( F(a, b:c + 7n; 1) 

  ' '   \ n=o c + n) c-a-b + n)\  ' ' '   

= ] lira n ) . ,   f  [ lim F(a, b;c + m; 1), 

if these two limits exist. 

But (  12'13) the former limit is  tt r-Fr, rz, if c is not a negative 

•  T  c — a) 1 (c— 6) ° 

integer; and, if Un(ci, b, c) be the coefficient oi x  in F (a, b; c; x), and 

m > I c I , we have 

00 

\ F a, b] c + m; l) — l\ \   2 \ un(a, b, c + m)\ 

n = l 

00 

  X Un (ja,, ib\, m — \ c )  

n=l 

' Q,b I " 

<  , 2 Mn (I a i -h 1, I 6 I + 1, 7?l -f- 1 - I C I). 

m-\ c\ n=o 

Now the last series converges, when ??i > ic| 4- a -f j l — 1, and is a positive 
decreasing function of m; therefore, since [m — '\ c\ \ }~ —>0, we have 

lim F(a, b: c + m ; 1)= 1 ; 

and therefore, finally, 

   ,  , r(c)r(c-a-6) 



14*2, 14-3] THE HYPERGEOMETRIC FUNCTION 283 

14"2. The differential equation satisfied by F (a, b; c: z). 

The reader will verify without difficulty, by the methods of § lO'S, that 
the hypergeometric series is an integral valid near   = of the hypergeometric 

equation* 

   ~   d? + (c - (a +   + 1)     -  -  \&'  = ; 

from § 10"3, it is apparent that every point is an 'ordinary point' of this 
equation, with the exception of 0, 1, x , and that these are ' regular points.' 

Example. Shew that an integral of the equation 

i  

z< F a- a, 6 + a; a-  + 1; z). 

14'3. Solutions of Riemann's P-equation by hypergeometric functions. 

In § 10"72 it was observed that Riemann's differential equation f 

dru j] -a -a' 1- -/3' l-y-y\ du 
dz- \ z— a z—b z — c ) dz 

\    aa' (a-b)(a-c)  jB'  h - c) b - a ) \   77 (c-a)(c-6) 
z— a z — b z — c 

u 



 z — a)  z — b) z — c) 



= 0. 



by a suitable change of variables, could be reduced to a hypergeometric 
equation ; and, carrying out the change, we see that a solution of Riemann's 
equation is 

fz-ay /z-c\ y    3 r./ 1 ' ( -a)(c-\&)] 

provided that a — a' is not a negative integer ; for simplicity, we shall, 
throughout thi  section, suppose that no one of the exponent differences 
a — a', /3 —  , 7 — 7' is zero or an integer, as (§ 10"32) in this exceptional 
case the general solution of the differential equation may involve logarithmic 
terms ; the formulae in the exceptional case will be found in a memoir :J: by 
Lindelof, to which the reader is referred. 

Now if a be interchanged with a', or 7 with 7', in this expression, it must 
still satisfy Riemann's equation, since the latter is unaffected by this change. 

* This equation was given by Gauss. 

t The constants are subject to the condition a + a' +   +  ' + y + y' = l. 

J Acta Soc. Sclent. Fennicae, xix. (1893). See also Klein's lithographed Lectures, Veher die 
hypergeometrische Funktion (Leipzig, 1894). 



284 THE TRANSCENDENTAL FUNCTIONS [CHAP. XIV 

We thus obtain altogether four expressions, namely, 

'z-aX /z-c\ y  f .  , ,  - . , , ,\  c-h) z-a)\ 



M, = 



(z-ay /z-c\ y p ,   ry, -, >  c-b) z-a)] 

 ==(.— i) U— ft) J |  + /3 + 7', o +  +t; l+°-'''; (o-a)(.-6) j- 

which are all solutions of the differential equation. 

Moreover, the differential equation is unaltered if the triads (a, a', a), 
(yS, y8', h), (y, y', c) are interchanged in any manner. If therefore we make 
such changes in the above solutions, they will still be solutions of the 
differential equation. 

There are five such changes possible, for we may write 

 b, c, a\,  c, a, b],  a, c, b], [c, b, a], [b, a, c] 

in turn in place of [a, b, c], with corresponding changes of a, a, /S, (3', 7, 7'. 

We thus obtain 4 x = 20 new expressions, which with the original four 
make altogether twenty-four particular solutions of Riemann's equation, in 
terms of hypergeometric series. 

The twenty new solutions may be written down as follows : 



u. = 



z-by fz-aY  (  , ,  ,  a-c)(z-b)\ 
iP'j/8 + 7 + a, /3 + 7'+a; l + yS-/3'; )  , ( 

z-b\ \  ' [z-ay t  q,  ,   o' , ',  . 1 , /D' o. (a-c)( -\&)] 



Ua = 



 s=('— ) ('- Y Fy+y + oc, '+y'+a-l + /3'-/3- ,.  . 

\ z - cj \ z - cj [ "   a-b) z- c) 

(z-b fz — a-y \  (  , \  , ,    , (a — c)(z — b) 

\ z — cl \ z — c \ (a —b) z — c) 

i /3+7+a, /3+7+a ; l + /3'-/3; ~ '-\ 

\ Z-C' \ Z-C' y  I ' r- I ) r- r- ' (   \  J   - c)\ 

(z-c\ y (z-b\ \        , o -. ,  b-a) z- c)] 

" "=U- ) U- J  |7   +  ,7+a-+ ;l + y-7;  \  ;  \   | 

U,,= l - )'( - r F\ y a +  ',y c  '; I + 7-7';  j  H 
\ z-aj \ z-al ['   ' "    ' ' ' (l,-c)(z-a)\ 

   = (lllfy7l Yf f/'+o + ZS'. 7'+a+/3'; 1 + 7-7; P -  
\ z-aj \ z-a)  ' "  '   ' ' > (i,-c) z- a) 



14-4] THE HYPERGEOMETRIC FUNCTION 285 



Z ~ Cj 



a,.= 



MlQ — 



FL + y +  ',a + y' +  '; I + a - a ;  l' '\ \ ' ' ''\ \ , 
[  b-a) z-c)  

/ -cy'/2-a\  j ' , /o' , 1, '  a-b)(z-c)) 

.T b) U— 6J  |t+/3 +  ,7+ +< : 1+7-7;   \  ;( \   |, 

\ z - a J \ z-a) [ ' (c-b)(z- a)  ' 

—  / i  c — o)(z — a)) 

1  V - / \ z-aj ( ' "  I   r ' (c-6)( -a)J ' 

By writing 0, 1 - C,  , B, 0, 6'-  -5, a; for a, a', yg,  S', y, y, 

/ — AU —   respectively, we obtain 24 solutions of the hypergeoraetric 

equation satisfied hy F  A, B; G ; x). 

The existence of these 24 solutions was first shewn by Kuinmer*. 

14"4. Relations betiueen particular solutions of the hyper geometric equation. 

It has just been shewn that 24 expressions involving hypergeometric 
series are solutions of the hypergeometric equation ; and, from the general 
theory of linear differential equations of the second order, it follows that, if 
any three have a common domain of existence, there must be a linear relation 
with constant coefficients connecting those three solutions. 

If Ave simplify n- , u.,, ih, ih] u , u g-, u i, W22 in the manner indicated at 

* Journal fUr Math. xv. (1836), pp. 39-83, 127-172. They are obtained in a different manner 
in Forsyth's Treatise on Differential Equations, Chap. vi. 



' •22 = , 

\ z — aj \ z 



286 THE TRANSCENDENTAL FUNCTIONS [CHAP. XIV 

the end of § 14-3, we obtain the following solutions of the hypergeometric 
equation with elements A, B, G, x: 

y, =F A,B;C-x), 

y  = (- xy- 'FiA -C+\,B-G+1;2-C;x), 

y,=(l-xr-''- F C-B,C-A;G;xl 

y  = (\  ccY-'' l - xf- - 'Fil -B,l-A;2-G;x), 

y,r = F A,B; A + B - G + 1 ; 1 - x), 

y  =   xf- -''F G-B, G-A; G-A-B-hl; l-x), 

y  = (\  xY 'FiA, A-G+\;A-B + \; x-% 

y  = (- x)-  F B,B-G + l]B-A + l; x-'). 
If 1 arg(l — a;) I < TT, it is easy to see from § 2-53 that, when j .r | < 1, the 
relations connecting y , y.2, y  y  must be y y-i, y2= rj , by considering the 
form of the expansions near c ; = of the series involved. 

In this manner we can group the functions Wj, ... z/24 into six sets of four*, 
viz. u-i, v-i, Mi3, Wis; Un, Hi, u , iiis; ii , u , M21, u ', Wg, u , U22, M24 j '"9? Wji, t<i7> ' 19 j 
Wjo, U12, u-18, '' ho, such that members of the same set are constant multiples of 
one another throughout a suitably chosen domain. 

In particular, we observe that Ui, u , u s, u  are constant multiples of a 
function which (by §§ 5*4, 2"o3) can be expanded in the form 



 z-aY\ l+ i en(2-a)4 



when \ z — a\ is sufficiently small ; when arg  z — a) is so restricted that 
(z — a)* is one-valued, this solution of Riemann's equation is usually written 
P'"*. And P'"'; P' ', P' '; P* ', P* * are defined in a similar manner when 



z — a 



\ z — b\, \ z — c\ respectively are sufficiently small. 



To obtain the relations which connect three members of separate sets 
of solutions is much more difficult. The relations have been obtained by 
elaborate transformations of the double circuit integrals which will be obtained 
later in § 14-61 ; but a more simple and singularly elegant method has recently 
been discovered by Barnes ; of his investigation we shall give a brief account. 

14"5. Barnes contour integrals for the hypergeometric function . 

Consider  r—. ~ -- p/ , . — — '(-zyds, 

273-1 j\ xj r(c + s) 

where |arg(— -z)] <7r, and the path of integration is curved (if necessary) to 
ensure that the poles of r(a + 5)r(6 + 6'), viz. s = — a — n, —h—n  n = 0, 1, 2, . . .), 

* The special formula 

FiA,l;C;.)  F( C-A,l;C; ), 

which is derivable from the relation counectiug Mj with  i3, was discovered in 1730 by Stirling, 
Methodus Differential is, prop. vii. 

t Piuc. London Math. Soc. (2), vi. (1908), pp. 141-177. References to previous work on similar 
topics by Pincherle, Mellin and Barnes are there given. 



14-5] 



THE HYPERGEOMETRIC FUNCTION 



287 



lie on the left of the path and the poles of r(— 5), viz. s = 0, 1, 2, ..., lie on 
the right of the path *. 

From § 13'6 it follows that the integrand is 

Oils j +''-c-i exp  - arg  - z) . I (s) - 7r\ I  s)\ \ ] 

as s— ►CO on the contour, and hence it is easily seen (| 5"32) that the integrand 
is an analytic function of z throughout the domain defined by the inequality 
I arg 2\  TT — S, where B is any positive number. 

Now, taking note of the relation F (— 5) F (1 + s) = — tt cosec s-rr, consider 

T(a+s)r(b+s) iri-zy 



If 
IttiJ c 



ds, 



27rij(7 F(c + s) F(l +s) sinsTT 

where C is the semicircle of radius iV + - on the right of the imaginary axis 

with centre at the origin, and N is an integer. 

Now, by § 13"6, we have 

r(a+ )F(6 +  ) -n-i-zy   . a+,\ ,\ ,) (- y 
T(c + s)r l+s) sinsTT   ' sin stt 

as N—> 00 , the constant implied in the symbol being independent of arg s 
when s is on the semicircle ; and, if 6  = f j\ \   +   j e''* and j   | < 1, we have 

(— zy cosec sir = exp j ( iV + .j j cos 6 log |   | — ( lY +   j sin   arg (— z) 

-(iV +  )7r|sm ||] 

exp I   i\ \   + . ) cos 6' log I   I - (iV +  ] 8 1 sin (  1 11 

exp|2- iV +. )logj  

exp|-2- 8(i\ r + i)| 



= 



  I 6* k 7 TT, 







1 I zii 1 



Hence if log |   | is negative (i.e. \ z\ <  1), the integrand tends to zero 
sufficiently rapidly (for all values of 6 under consideration) to ensure that 



/. 



'0 as iV—  oo 



Now 



J -ooi [J ~xi J C J (iV+i) i) 



by Cauchy's theorem, is equal to minus 2'Tri times the sum of the residues 
of the integrand at the points s = 0, 1,2, ... N. Make N—>cc , and the last 

* It is assumed that a and b are such that the contour can be drawn, i.e. that a and b 
are not negative integers (in which case the hypergeometric series is merely a polynomial). 



288 THE TRANSCENDENTAL FUNCTIONS [CHAB. XIV 

three integrals tend to zero when | arg (—  ) |   tt — S, and \ z \ < 1, and so, in 
these circumstances, 

the general term in this summation being the residue of the integrand at 

S = 71. 

Thus, an analytic function  namely the integral under consideration) exists 
throughout the domain defined by the inequality j arg   | < tt, and, when \ z\ <  1, 
this analytic function may he represented by the sei'ies 



V 



r(a + n)r(b + n) 
r  c + n) .n ! 



The symbol F(a, h; c; z) will, in future, be used to denote this function 
divided by r( .)r(6)/r(c). 

14'51. The continuation of the hyper geometric series. 

To obtain a representation of the function F (a, b; c; z) in the form of 
series convergent when j 2  | > 1, we shall employ the integral obtained in 
§ 14'5. If -D be the semicircle of radius p on the left of the imaginary axis 
with centre at the origin, it may be shewn* by the methods of § 14'o that 

V a + s)V h + s)T -s) 



Jd r c + s)   



as p—  x , provided that j arg  — z) \ < tt, \ z \ > 1 and p—><xi in such a way 
that the lower bound of the distance of D from poles of the integi and is 
a positive number (not zero). 

Hence it can be proved (as in the corresponding work of § 14"5) that, when 
I arg  —z)\ < 7r and \ z', > 1, 

1 r r a + s)r(b + s)T -s) 

\  V r(a+7i)r(l -c + a + n) sin ( c- a - n) tt /\  x- -h 
~ rZo r (1 + w) r (1 - 6 + ft + n) cos n-rr sin (b - ft - n) it 

  T(b + n)r l-c + b + n) sin (c - 6 - ?i) tt f \   

n=o r (1 + n) r (1 — ft + 6 + n) cos nir sin (ft — 6 — w) tt 

the expressions in these summations being the residues of the integrand at 
the points s = - a- n, s = - b — n respectively. 

It then follows at once on simplifying these series that the analytic 

* In considering the asymptotic expansion of the integrand when | s [ is large on the contour 
or on D, it is simplest to transform T (a + s), V (b + s), T c + s) by the relation of § 12-14. 



14-51, 14*52] THE HYPERGEOMETRIC FUNCTION 289 

continuation of the series, by which the hypergeometric function was originally 
defined, is given by the equation 

r(c)  ""' '- '   r(a-c) -(- r' - ( >i-  +  ;i-  +    ) 

1  0 — c) 
where arg (— z)\ <  tt. 

It is readily seen that each of the three terms in this equation is a solution 
of the hypergeometric equation (see § 14"4). 

This result has to be modified when a — 6 is an integer or zero, as some of the poles of 
r a + s)r h + s) are double poles, and the right-hand side then may involve logarithmic 
terms, in accordance with   1 43. 

Corollary. Putting h = c, we see that, if | arg  - z)\ <  n, 

r (a)   z)-" =  -. f "' r  a+s) r ( -s) ( - zyds, 
'ZniJ \ xt 

where (1 - )~<'- -l as 2-*.0, and so the value of [ arg(l - z) \ which is less than tt always 
has to be taken in this equation, in virtue of the cut (sec i:; 14'1) from to -f-oo caused 
by the inequality | arg( - z) \ <  n. 

14*52. Barnes' lemma that, if the 'path of integration is curved so that the poles of 
T  y — s)T b — s) lie on the right uf the path and the poles of T (a + s) r (/3 + s) lie on the left*, 
then 

Write / for the expression on the left. 

If C be defined to be the semicircle of radius p on the right of the imaginary axis with 
centre at the origin, and if p-s-oo in such a way that the lower bound of the distance of 
C from the poles of r (y - s) r (5 - s) is positive (not zero), it is readily seen that 

T a + s)T \& + s)T y-s)T b-s) =  -  ''' '      

= 0[..' - +>+ - exp -2 1/( )l J, 
as ] s |- -oo on the imaginary axis or on C. 

Hence the original integral converges ; and / - -Oasp-a-oc , when (a + /3-|-y-|-S- 1)<0. 

Thus, as in § 14'5, the integral involved in 7 is - 2ni times the sum of the residues of the 
integrand at the poles oiT y — s)T b — s) ; evaluating these residues we gett 

/= i r( a + y+%)rO + y + ? ) TT I V a + b + n)T   + b + n) 



a=oV n +  ) T  + y-8 + 7l) sin(S-y)7r ,,=o T (?i+l) r (1 -f-S-y + w) sin(y-S)7r* 

* It is supposed that a, /3, y, 5 are such that no pole of the first set coincides with any pole 
of the second set. 

t These two series converge (§ 2-38). 

W. M. A. 19 



290 THE TRANSCENDENTAL FUNCTIONS [cHAP. XIV 

And so, using the result of   12"14 freely, by § 14-11 : 

\  7rr(l-a-/3-y -5) r T (a + d) T (3 + 5) r(a + y ) rO + y) ) 

sin(y-8)7r |r(l-a-y)r(l-/3-y) T (1 - a -8)~r (1 -  -8)J 

r(a+y)r(  + y)r(a + g)r(  + a) f   r   ,   ro  , 

- r(a+/J + y + 8)sm(a +   + y + 5).sin(y-8).  "  (  +  )   ' "  ('    

-sin (a + S) 77 sin 04-S)7r|. 

But 2sin(fi+y)7rsiu(/3 + y)7r-2sin(a + S)7rsin (3 + S)7r 

= cos (a — /3) TT - cos (a + /3 + 2y) tt - cos (a — jS) tt + cos (a +   + 28) tt 

= 2sin(y-8)7r sin(a+/3 + y + S)7r. 

Therefore  Vja + y)n ym-\ + ) V     ) 

 (a +   + y + S) 

which is the required result ; it has, however, only been prt)ved when 

 (o + /3 + y + 8-l)<0; 

hut,  )y the theory of analytic continuation, it is true throughout the domain through 
which both sides of the equation are analytic functions of, say, a: and hence it is true for 
all values of a, /ii, y, 8 for which none of the poles of r (a + s) r (/3 + s), qua function of s, 
coincide with any of the poles of r (y - s) r (8 - s). 

Corollary. Writing s + k., a — k, \$-k, y + k, 8 + k in place of s, a, /3, y, 8, we see that 
the result is still true when the limits of integration are —; ,• + cx) i, where k is any real 
constant. 

14 "53. The connexion between hypergeometrie functions of z and of \ —z. 
We hav§ seen that, if | arg ( - s) | < tt, 

r(c)   ' ' '   2ni j \ x,- T c + s) "• ' 

=   . \ \  r-.\ T a t)Y h-irt)T s-t)V c-a-h-t)dt\ 

'' TTi J \ x i l TTl J ~k—x i ) 

.T c-a)T c-b)  ' 
by Barnes' leinma. 

If k be so chosen that the lower bound of the distance between the s contour and the 
t contour is positive (not zero), it may be shewn that the order of the integrations* 
may be interchanged. 

Carrying out the interchange, we see that if arg (1 -z) be given its principal value, 

T(C'-a)T  c-h)T  a)r b)F a, b; c; 2)/r (c) 

= s—  / r a + t)r b+t)r c-a-b~t) —. r s-t)ri-s)(~zYds  dt 

 T lJ-k-Jii  'ZtvI J -jci   

1 f-k+oci 

=  r—. I r  a + t)r  b+t)r  c~a-b-t)r  -t)  i-zy dt. 

2ni J -A-K/ 

* Methods similar to those of § 4-51 may be used, or it may be proved without much difficulty 
that conditions established by Bromwich, Infinite Series, % 177, are satisfied. 



14'53, 14 6] THE HYPERGEOMETRIC FUNCTION 291 

Now, when I arg (1 -k) \ < Stt and | I -z\ < 1, this last integral may be evaluated by the 
methods of Barnes' lemma (§ 14-52) ; and so we deduce that 

r c-a)r c-b)T a)rib)F a,b;c;z) 

= r  c)r  a)r  b)T  c-a-b) F  a, b ; a + b-c + l ; l-z) 

+ r  c)r  c - a) r (c-b) r  a + b - c)  I -zy-'"-" F c- a, c-b; c-a-b + l;  z), 

a result which shews the nature of the singularity of F  a, b ; c ; z) at z=.  

This result has to be modified if c — a — 6 is an integer or zero, as then 

Y  a-- t) V  b->r t) T  c - a -b - t)T  -t) 

has double poles, and logarithmic terms may appear. With this exception, the result is 
valid when | arg  —z)\ < tt, | arg (1 - 2) ; < tt. 

Taking | s | < 1, we may make 2 tend to a real value, and we see that the result still 
holds for real values of 2 such that < 2 < 1. 

14"6. Solution of Riemann's equation hy a contour integral. 

We next proceed to establish a result relating to the expression of the 
liypergeometric function by means of contour integrals. 

Let the dependent variable u in Riemann's equation (§ 10"7) be replaced 
by a new dependent variable   defined by the relation 

u = (z - aY  z - bf (z - c)y I. 

The differential equation satisfied by / is easily found to be 

d  f 1 + a a' l+ -§' 1-f - 7) dl 
dz- \ z — a z — h z — c ] dz 

( g + /3 + 7)  (g +   + ry + 1)   + Sa (g + /?' + 7' - 1)1 
 z — a) z — 0)  z — c) 

which can be written in the form 

Q(z)  -[ X-2)Q' z)+R z)]f  

+ li (  - 2) (X - 1 ) (/' (z) - (X-DR (z)] 1 = 0, 

where / \=l-ot- -j = a' +  ' + y', 

iQ(z)  z-a)(z-h) z-c), 

(R(z) = X a' +/3 + ry) z-b) z-c). 

It must be observed that the function / is not analytic at x , and consequently the 
above differential equation in / is not a case of the generalised hypergeometric equation. 

We shall noiu sheiu that this differential equation can be satisfied by an 
integral of the form 

1=1  t- ay'+ +y-  (t - 6)-+ '+y-i (t - c)-+ -y -1  z -  )- - -y dt, 
J c 

jirovlded that C, tJie contour of integration, is suitably cJiosen. 

19—2 



292 THE TRANSCENDENTAL FUNCTIONS [CHAP. XIV 

For, if we substitute this value of / in the differential equation, the con- 
dition* that the equation should be satisfied becomes 

j (t- ay+ +y-'  t - by+ '+y-'  t - cY+p+y'-'  z - t)-''- -y--Kdt = 0, 

J c 
where 

ir = (\ - 2) |q (z) +  t- z) Q' (z) + l(t- zj Q"  z) 



+ (t-z) R(z) + (t-z)R'(z)] 

= ( 2)  Q (t) -it- zy]  -it-z) [R (t) -a- zy s ( ' + /3 + 7)1 

= - (1 + a + /3 + 7) (  - a) (  - 6) (i - c) 

+ S (a + /9 + 7) (  -  ) (  - c)  t - z). 

f dV 
It follows that the condition to be satisfied reduces to -7- dt = 0, where 

J c dt 

v=(t- ay+ +y (t - by+ '+y (t - cy+ +y (t -  )-(i+-+p+v . 

The integral / is therefore a solution of the differential equation, when 
C is such that V resumes its initial value after t has described C. 

Now 

V= t- ay'+ +y-' (t - by+ '+y-' (t - cy+ +y-' [z - t)-''- -y U, 

where U = (t- a)  t -b) t- c) (z - ty\ 

Now Z7 is a one-valued function of t ; hence, if C be a closed contour, it 
must be such that the integrand in the integral / resumes its original value 
after t has described the contour. 

Hence finally any integral of the type 

(z-ay(z-by(z - c)y \ (t-ay+y+''-\ t-b)y+''+ '-' t-cy+ y'-' z-ty''~ -y dt, 

JC 

•where C is either a closed contour in the t-plane such that the integrand 
resumes its initial value after t has described it, or else is a simple curve such 
that V lias the same value at its termini, is a solution of the differential equation 
of the general hypergeometric function. 

The reader is referred to the memoirs of Pochhammer, Math. An)i. xxxv. (1890), 
pp. 495-526, and Hob.son, Phil. Trans. 187 a (1896), pp. 443-531, for an account of the 
methods by which integrals of this type are transformed so as to give rise to the relations 
of  § 14-51 and 14-53. 

Example 1. To deduce a real definite integral which, in certain circumstances, 
represents the hypergeometric series. 

• The iliEferentiations imder the sign of integration are legitimate (§ 4-2) if the path C does 
not depend on z a3d does not pass through the points a, b, c, z ; if C be an infinite contour or if 
C passes through the points a, b, c or z, further conditions are necessary. 



14-61] THE HYPERGEOMETRIC FUNCTION 293 

The hypergeometric series F a, b; c; z) is, as already shewn, a solution of the differential 
equation defined by the scheme 

r 00 1 \ 

P   a zy 

\ \ —c b c- a- b ] 
If in the integral 

which is a constant multiple of that just obtained, we make 6  -x (without paying 
attention to the validity of this process), we are led to consider 

/ t''''  t-\ y- -  t-zydt. 

Now the limiting form of V in question is 

and this tends to zero at t= 1 and  = cc , provided R (c) > R(b)>0. 

We accordingly consider / t"~'=  t — Vf-''-'   t-z)~" dt, where z is not* positive and 
greater tlum 1. 

In this integral, write t = u~  ; the integral becomes 

/ ? ''-' (1-jt)' -''-' (l-M2)- 0??<. 
.' 

We are therefore led to expect that this integral may be a solution of the differential equation 
for the hypergeometric series. 

The reader will easily see that if R (c) > R b)> 0, and if arg u = arg (1 - m) =0, while the 
branch of l-uz is specified by the fact that (1-m2)-  1 as m- 0, the integral just 
found is 

T b)r c-b) 

  ( , \&; c;z . 

This can be proved by expandingf (1 —  s)-a in ascending powers of z when | 3 | <1 and 
using § 12 "41. 

Example 2. Deduce the result of § 14-11 from the preceding example. 

14'61. Determination of an integral which represents i "\ 

We shall now shew how an integral which represents the particular solution i '  
(§ 14-3) of the hypergeometric difterential equation can be found. 

We have seen (§ 14-6) that the integral 

l=  z- df [z~bf  z - cy (  t-a)P+y+- '-\ t-b)y+<'+P'- (t-c)'' +  +y'-Ht-z)-<'~ -ydt 

satisfies the difl:erential equation of the hypergeometric function, provided C is a closed 
contour such that the integrand resumes its initial value after t has described C. Now the 
singularities of this integrand in the  plane are the points a, b, c, z; and after describing 
the double circuit contour (§ 12'43) symbolised by (6 + , c-h, 6 — , c — ) the integi-and returns 
to its original value. 

* This ensures that the point t — ljz is not on the path of integration, 
t The justification of this process by § 4'7 is left to the reader. 



294 THE TRANSCENDENTAL FUNCTIONS [CHAP. XIV 

Now, if z lie iu a circle whose centre is  , the circle not containing either of the points 
h and c, we can choose the path of integration so that t is outside this circle, and so 
\ z — a\ < i\ t — a\ for all points t on the path. 

Now choose arg(2 — a) to be numericall ' less than it and arg(i — 6), ai*g(2-c) so that 
they reduce to* arg(a — 6), arg(a-c) when z- a\ fix arg(  — a), arg (< — 6), arg (< — c) at 
the point iV at which the path of integration starts and ends ; also choose arg  t - z) to 
reduce to arg  t — a) when z- a. 

Then (z-bf = (a-bf |l +/3 (j ' ) + ...j- , 

(.-c)V=( -o)' l+ (i ?) + ...l, 
and since we can expand (t-z)~' ~°~'  into an absolutely and uuifornily convergent series 

we may expand the integral into a series which converges absolutely. 

Multiplying up the absolutely convergent series, we get a series of integer powers of 
z — a multiplied by (s — a) . Consequently we must have 

We can define P '''\ P \ P '\ F y\ P '  by double circuit integrals in a similar 
manner. 

14'7. Relations hetiueen contiguous hypergeometric functions. 

Let P z) be a solution of Riemann's equation with argument z, singularities 
a, h, c, and exponents a, a', /3, j3', y, 7'. Further let P(z) be a constant 
multiple of one of the six functions P * , P'"'', P' ', P' \ P<y , P't*. Let 
Pi+i,m-i(z) denote the function which is obtained by replacing two of the 
exponents, I and m, in P 2) hy I + 1 and m — 1 respectively. Such functions 
P;+i, i\ i (z) are said to be contiguous to P (z). There are 6 x 5 = -30 contiguous 
functions, since I and m may be any two of the six exponents. 

It was first shewn by Riemannf that the function P(z) and any tivo of 
its contiguous functions are connected hy a linear relation, the coeffi,cients in 
which are polynomials in z. 

There will clearly be 5 x 30 x 29 = 435 of these relations. To shew how 
to obtain them, we shall take P  z  in the form 

F z) =  z- ay (z - hY (z - c)y f  t- ay+y+' '-  (t - 6)v+' +  -  

J c 

(t - cY+ +y'-' (z - ty- -y dt, 

where C is a double circuit contour of the type considered in § 14*61. 

* The values of arg (a-b), arg (a - c) being fixed. 

t Abh. der k. Ges. der Wiss. zu Gottingen, 1857; Gauss had previously obtained 15 rehitions 
between contiguous hypergeometric functions. 



147] THE HYPERGEOMETRIC FUNCTION 295 

First, since the integral round G of the differential of any function which 
resumes its initial value after i has described C is zero, we have 

= [ I-  (t - ay+ +y (t - by ' y-Ht - cY+ +y'-'   t - zy- - -y] dt. 
J c dt 

On performing the differentiation by differentiating each factor in turn, 

we get 

(a' + + y)P + (a +  ' + y-l) Pa-+,, '-, + (a + (3 + y'-l) Pa'+i,y-i 

\  (a + /3 + 7) p 

Considerations of symmetry shew that the right-hand side of this 
equation can be replaced by 

These, together with the analogous formulae obtained by cyclical inter- 
change* of (a, a, a) with (6, /3,  ') and (c, 7, 7'), are six linear relations 
connecting the hypergeometric function P with the twelve contiguous 
functions 

Pa+], '-l, -I p+i y'—i, -/y+i,a\ l, -ta-],v'-l5 -Lfi+i a'-l,  y~\, '~\ i 
-I a.'+l, '— 1) -I a'4-],Y'\ i, i  '4.),-y'\ i, -t j3'+l, a'-l >  '' y'+I, a'- 1 ) -*7'+l, '— 1- 

Next, writing t — a = (t- h) + (6 —  ), and usingf Pa'\ i to denote the result 
of writing a' — 1 for a' in P, we have 

P = P..\ :,,.+:- (6- )P.-:. 

S-imilarly P = P \ i,y'+i + (c - a) Pa'-i- 

Eliminating Pa-i from these equations, we have 

 c-h)P + (a - c) Pa-\ :, vi + (b- a) Pa'-.,y-,, = 0. 
This and the analogous formulae are three more linear relations con- 
necting P with the last six of the twelve contiguous functions written above. 

Next, writing (  —  ) = (  — a) — (  — a), we readily find the relation 

P =   P +:, v'-i - (  -  )" ' (  - f y (  - 0  

Jc 
which gives the equations 

(  \  a)-i  P - (  - 6)- P +,, Y--:  = (  - h)-  [P- z- c)-  Py+, a'-i  

= Kz - c)-' [P- z- a)-' Pa+i,  \ i . 

* The interchange is to be made only in the integrands ; the contour C is to remain 
unaltered. 

t Pa'-l is not a function of Riemann's type since the sum of its exponents at a, h, c is not 
unity. 



296 THE TRANSCENDENTAL FUNCTIONS [CHAP. XIV 

These are two more linear equations between P and the above twelve 
contiguous functions. * 

We have therefore now altogether found eleven linear relations between 
P and these twelve functions, the coefficients in these relations being rational 
functions of 2. Hence each of these functions can be expressed linearly in 
terms of P and some selected one of them ; that is, between P and any two of 
the above functions there exists a linear relation. The coefficients in this 
relation will be rational functions of z, and therefore will become polynomials 
in z when the relation is multiplied throughout by the least common multiple 
of their denominators. 

The theorem is therefore proved, so far as the above twelve contiguous 
functions are concerned. It can, without difficulty, be extended so as to be 
established for the rest of the thirty contiguous functions. 

Corollary. If functions be derived from P by replacing the exponents a, a, /3, /3', y, y 

by a+p, a' + g, /3 + r, jS'+s, y + t, y' + x, where p, q, r, s, f, u are integers satisfying the 

relation 

p- -q + r + s + t + u=0, 

then between P and any two such functions there exists a linear relation, the coefficients 
in which are polynomials in z. 

This result can be obtained by connecting P with the two functions by a chain of 
intermediate contiguous functions, writing down the linear relations which connect them 
with P and the two functions, and from these relations eliminating the intermediate 
contiguous functions. 

Many theorems which will be established subsequently, e.g. the recurrence-formulae 
for the Legendre functions (§ 15-21), are really cases of the theorem of this article. 

REFERENCES. 
C. F. Gauss, Ges. Werke, in. pp. 12.3-163, 207-229. 
E. E. KuMMER, Journal fur Math. xv. (1836), pp. 39-83, 127-172. 
G. F. B. RiEMANN, Ges. Math. Werke, pp. 67-84. 
E. Papperitz, Math. Ann. xxv. (1885), pp. 212-221. 
S. PiNCHERLE, Rend. Accad. Lincei (4), iv. (1888), pp. 694-700, 792-799. 
E.  Y. Barxes, Proc. London Math. Soc. (2), vi. (1908), pp. 141-177. 
Hj. Mellin, Acta Soc. Fennicae, xx. (1895), No. 12. 

Miscellaneous Examples. 

1. Shew that 

F a, h + \; c; z)-F a, b; c; z) = — F a + l, h + l; c + l; z). 

2. Shew that if o is a negative integer while /3 and y are not integers, then the ratio 
F(a, IB; a + /3 + l -y; \ - .v)-7-F a,   ; y ; A') is independent of x, and find its value. 



THE HYPERGEOMETRIC FUNCTION 297 

dP d P 

3. If P iz) be a hypergeometrie function, express its derivates -,- and --  linearly in 

dP 

terms of P and contiguous functions, and hence find the linear relation between P, -j- , 

d'-P 
and -TV , i.e. verify that P satisfies the hypergeometrie difterential equation. 

4. Shew that i  j, j ; 1 ; 4 (1 -z)) satisfies the hypergeometrie equation satisfied by 
jP(|, I ; 1 ; z). Shew that, in the left-hand half of the lemniscate \ z  z) | = j, these two 
functions are equal ; and in the right-hand half of the lemniscate, the former function is 
equal to F \, | ; 1 ; 1 —2). 

5. \ i Fu, =F  a +  ,h; c; .r), /  \  =i (a - 1, h; c; x), determine the 15 linear relations 
with polynomial coefficients which connect F a, h; c; x) with pairs of the six functions 
Fa , Fa-, F,,, F,\ , F, , F,\ . (Gauss.) 

6. Shew that the hypergeometrie equation 

x(x-l/J - y-(a+  + l)x '  + a i/ = 

is satisfied by the two integrals (suppo.sed convergent) 

[\ \  -\ l-z)y- -'  l-xz)-''dz 
J 

and ['/-i(l-2)"-i' l-(l- )2 ~"rf2. 

.' " 

7. Shew that, for values of x between and 1, the solution of the equation 

is AF  a,i ; i;  \ --Ixy-l + B  l-2x) F  h a + l), h fi + l); |;  l-2xy- , 

where A, B are arbitrary constants and F a,  •,y;x) represents the hypergeometrie series. 

(Math. Trip. 1896.) 

8. Shew that 

Jim|F( ,/3,y,..)- J (-) ,;r(y-a)r(y- )r( )r( )  '   J 

\  r(y-a- )r(y ) 
 (y-a) (y- ) 

where h is the integer such that k   R (a +   — y)<k- .  

(This specifies the manner in which the hj'pergeometric function becomes infinite when 
.r- -l -0 provided that o-f |3-y is not an integer.) (Hardy.) 



9. Shew that, when i? (y - a - /S) < 0, then 

-/3-y 

- -l 



T y)n'' -y 



 "•( 4- -y)r(a)r( ) 

as '/i -x ; where S  denotes the sum of the first n terms of the series for F a, /3 ; y ; 1). 

(M. J. M. Hill, Proc. London Math. Soe. (2), v.) 



298 THE TRANSCENDENTAL FUNCTIONS [CHAP. XIV 

10. Shew that, ii i/i, y-i be indeijeiident sokitions of 



• 2- 1- -. 



then the oreneral sohition of 



is z = Ay - Byiy.i- Cyi', where J, B, C are constants. 

(Appell, Comptes Rendus, xci.) 

11. Deduce from example 10 that, if   + ft + | = c, 

 F(a h- r- r)\ \  =  \ rW\ r(2c-l) - T  2a + 7>) F ia + b+n)T  2b+7i)  ,, 

'  ' ' '  ' r 2a)T 2b)T a + b)n=o nl T  c + n)T  2c- 1+n) *' ' 

(Clausen, Journal fiir Math, iii.) 

12. Shew that, if |   | <i and j x \ —s) j <|, 

F 2a, 2/3; a+/3 + *; x  = F a, /3 ; a+  + l ; 4.r(l-.r) . (Kummer.) 

13. Deduce from example 12 that 

14. Shew that, if co = e ' ' and i  (a) < 1, 

i (a,3a-l; 2a; " - ) = 3 '' '* exp [K Sa - 1)   |  | , 

i ( ,3a-l; 2a; -co) =3  -   exp   W (1 -3a)   M | . 

(Watson, Quarterly Journal, xhi.) 

15. Shew that 

V 2 ) 2 -rj> -r\ j ..y V9; r(|)r('/i + f) 

(Heymann, Zeitschrift fur Math, und Phys. XLiv.) 

16. If il-xY+ -y F 2a,2 ;2y; x) =  + Bx + C:'f- + Dx +..., 
shew that 

i (a, /3; y+ ;  ) i (y-a, y-/3 ; y+l; x) 

y+l (7+l)(7 + i-) (y+*)(y + f)(7+|) 

(Cayley, PAz7. Mag. (4), xvi. (1858), pp. 356-357. See also Orr, Camb. Phil. 
Trans, xvii. (1899), pp. 1-15.) 

17. If the function F  a, ji, /3', y ; .*;, y) be defined by the equation 

i (a, ,/3', y; x,y) = \ -  \ \  i' -\ \ ~xc)y-''-\ \ - ux)-\ l-uy)- ' du, 

r (a) r (y - a) j 

then shew that between F and any three of its eight contiguous functions 

i (a±l), F ii± ) , F ii'±l), F y±\ \  

there exists a homogeneous linear equation, whose coefficients are polynomials in x and y. 

(Le Vavasseur.) 



THE HYPERGEOMETRIC FUNCTION 299 

18. If y - a - iS < 0, shew that, a.s x - 1-0, 

and that, if  -0-/3 = 0, the corresponding approximate formula is 

(Math. Trip. 1893.) 

19. Shew that, when \ x\ <  1, 



/•(2 + ,0 + ,a:-,0-) 1 1 fl 



= \   ,r/' sin OTT sin (y - a) TT .  ' — " - "V (a, /3 ; y ;  ), 

where c denotes a point on the straight line joining the points 0, x, the initial arguments 
of v — x and of v are the same as that of .>;, and arg (1 — j/) -*-0 as v- 0. 

(Pochhammer.) 

20. If, when \ arg (1 -x)\ <  2n, 

K x)=—r  T -s)Ti ; + s)\ \  '  xYds, 

LTTl J \  y, , 

and, when | arg.*; | < 27r, 

l-lZl J \ K,- 

by changing the variable s in the integral or otherwise, obtain the following relations : 
K (.r) = A" (1 -  ), if I arg (1 - :f) I <    

K l-x) = K'(x), if|argj;|<7r. 

K x) =  l - .r)-4 A' ( -  , if i arg (1 - .r) \ < n. 

K l-x) x-h K f"-—   , if I arg.r | <7r. 



K'  x) = x 2 / ' (Ijx), if I arg.r |'< tt. 

A" (1 - a;) = (1 - X) - i A" (  \ J\ - \ ") , if j arg (1 - .x-) I <  - (Barnes.) 

21. With the notation of the preceding example, obtain the following results : 

11=0 [ 'I- ) 

2 K'  x)= - l  f lt— j'- -  log -4log 2 + 4 (1 - 1 + --   . 

when \ x\ < l, | arg .i? | < tt ; and 

iT ( ) = + i ( -  ) - 5 a: (l/.r) + ( -  ) - 2  ' (1/:??), 

when j arg   — x)\ < ir, the ambiguous sign being the same as the .sign of / (x). 

(Barnes.) 



300 THE TRANSCENDENTAL FUNCTIONS [CHAP. XIV 

22. Hypergeometric series in two variables are defined by the equations 
F, a; , ';y; .r, y) = 2 "Xn f" '"J/'S 

F, (a ; /3,  ' ; y, y ; ,r, y) = 2   '-, x y . 



-ts a,a, , ; y; x, y)= 2   , --,-\  x 'y" 



 n( n Pm Pn 
,( ??l . n i y i   ,1 



FUa, ;y,y';.,y)  .XpV ' '"'" '' 
where a,  = a(a + l). ..(a + m — 1), and 2 means 2 2 . 

?n, n m=0  =0 

Obtain the differential equations 

c"F. d'F  ?Ft ?F  

-(l-)8 +y(l-)5j;5 + r-( +e+l)- 5;j'-* -. /-,=0, 

d F  d F  dF; 

.il-o:)- +y  Hy- a +   + l)x]j -a F, = 0, 

and four similar equations, derived from these by interchanging x with y and a, /3, y with 
a',  3', y when a',  ', y occur in the corresponding series. 

(Appell, Comptes Rendus, xc.) 
23. If a is negative, and if 

a= —V + a, 

where v is an integer and a is positive, shew that 



r (x) r (a) 






u D (-)"( -l)(a-2) ... (a-%)  , 
where    =   —  - - — ' — —   "  ' G( -n), 



" i (•*/ = —    . (Hermite, Joiirnal mr Math, xcii.) 

x+n \ 1 . / 



24. When a < 1, shew that 



T x)T a-x)  I   \ l Rn 



T  a) n=\ X + n n=\ X — a — n'' 

where (-) a(a + l) ... (a + .-l)   

7i ! 

25. When a > 1, and v and a are respectively the integral and fractional parts of 
a, shew that 

T x)T a-x)    G x)p.,\  - (?(  ')p,   
r(a)  =i .r + ?i  =i . r-a — n 

-G x)  +   + ...+ f'-'X 
  '\ \  X — a X-a-\ .r-a-i  + lj' 



THE HYPERGEOMETRIC FUNCTION 301 



where (?( )=(l-?  (l --  ) ... (l \  . \  ) 

V a/ V a+l/ \ a + v — lj 

ajjj \ (-)"a(a+l)... (a +  -l) 



It, ; 

(Hermite, Journal fiir Math, xcil.) 
26. If 

/• (T   , \ r - '(jZ +   + n-l) ., x x+l)(7/ + v + n-l)(  + v + n) 

./  (.r, y, .)- 1 -  C, - - - ,) + C.3 3,(  + 1) (  + ,)( . , + i) - ... , 

where n is a positive integer and  C'i,  C2, ... are binomial coefficients, shew that 

f (T y ,  r(y)rCy- -H/ )r(.r-n;)r(y+?0 

./H .y. ) Y y-x)T y + n)V :6)V x-irV n)' 

(Saalschiitz, Zeitschrift fur Math. xxxv. ; a number of similar results are given 
by Dougall, Proc. Edinburgh Math. Soc. xxv.) 

shew that, when liib + e -  ,a —  ) > 0, then 

i (a,a-5+l, -c + l'; S, e ; l)=2-  ..A  P W " " — 
 ' '   ' ' '   r(8- a)r(f-ia)r(  + ia)r(S + e-a-l) 

(A. C. Dixon, P/'oc. London Math. Soc. xxxv.) 

28. Shew that, if R (a) < .if, then 

; f a(a+l)... (a-l-n- l)l  ,. ,,, . r(l-3a) 

(Morley, Proc. London Math. Soc. xxxiv.) 
2!). If 

[' / '....-I (1 \ .,.)>-iy-i (1 \  )A-. (1 -a,yy--J->'dxdi/ = B(i,J, X-,   m), 
J i> J i> 

shew, by integrating with respect to jc, and also with respect to i  that B  i,j, /•,   m) is a 
symmetric function of i+J,j + i\ /• +     + in, /n+i. 

Deduce that 

F a,ld,y; fi, f ; l) r (8) F ( ) T (6 + 6 - a -/3-y) 

is a symmetric function of 8, f , S + e — a - /i, 8 + e —   - y, 8 + e - y - a. 

(A. ('. Dixon, Proc. London Math. Soc. (2), II. (1905), jjp. 8-16. For a proof of 
a special case by Barnes' methods, see Barnes, Quarterly Journal, XLI. 
(1910), pp. 136-140.) 

30. If 

x ~  il-x f'"- d" I 

/  = i (- , a + ;.; y, .r) = - - y—  -        W ( x) \ 

shew that, when n is a large positive integer, and 0< .r < 1, 

  = , -  (sin, )i-V(cos0)V- -icos (2u + a)( -i,r(2y-l)  + o( 1 ), 

where .r = sin2 . 

(This result is contained in the great memoir by Darboux, "Sur I'approxi- 
mation des fonctions de trfes grands nombres," Journal de Math. (3), iv. 
(1878), pp. 5-56, 377-416. For a systematic development of hyper- 
geometric functions in which one (or more) of the constants is large, see 
Camh. Phil. Trans, xxii. (1918), pp. 277-308.) 

