\chapter{The Theta Functions} 

211. The definition of a Tli eta-function. 

When it is desired to obtain definite numerical results in problems 
involving Elliptic functions, the calculations are most simply performed 
with the aid of certain auxiliary functions known as Thetafunctions. These 
functions are of considerable intrinsic interest, apart from their connexion 
with Elliptic functions, and we shall now give an account of their funda- 
mental properties. 

The Theta-functions were first systematically studied by Jacobi*, who 
obtained their properties by purely algebraical methods ; and his analysis 
was so complete that practically all the results contained in this chapter 
(with the exception of the discussion of the problem of inversion in §§ 21-7 
et seq.) are to be found in his works. In accordance with the general scheme 
of this book, we shall not employ the methods of Jacobi, but the more 
powerful methods based on the use of Cauchy's theorem. These methods 
Avere first employed in the theory of Elliptic and allied functions by Liouville 
in his lectures and have since been given in several treatises on Elliptic 
functions, the earliest of these works being that by Briot and Bouquet. 

[Note. The first function of the Theta-function type to appear in Analysis was the 

Partition function  U (l-.*'" )"  of Euler, Introductio in Anaiysin Infinitorum, i. 

(Lausanne, 1748), § 304; by means of the results given in 5  21-3, it is easy to express 
Theta-functions in terms of Partition functions. Euler also obtained properties of products 
of the type 

n (i±A-"), n (i±a;2 ), n (i± -2 -i). 

n=l n = l n=\ 

The associated series 2 ?n "('*+ \ 2 m-" ""*   and 2 m ' had previously occurred in the 

>i=0 n=0 M=0 

posthumous work of Jakob Bei'nouUi, Ars Conjectandi (1713), p. 55. 

* Fundamenta Nova Theoriae Fimctionum Ellipticarum (Konigsberg, 1829), and Ges. Werke, 
I. pp. 497-538. 

t The Partition function and as-sociated functions have been studied by Gauss, Comm. Soc. 
reg. sci. Gottinnensis rec. i. (1811), pp. 7-l'2 [Werke, ii. pp. 16-21] and Werke, iii. pp. 433-480 and 
Cauchy, Coviptes liendus, x. (1840), pp. 178-181. For a discussion of properties of various functions 
involving what are known as Basic  lumbers (which are closely connected with Partition functions) 
see Jackson, Froc. Roijal Soc. Lxxiv. (1905), pp. 64-72, Froc. London Math. Soc. (1) xxviii. (1897), 
pp. 475-486 and (2) i. (1904), pp. 63-88, ii. (1904), pp. 192-220; and Watson, Camb. Fhil. Trails. 
XXI. (1912), pp. 281-299. A fundamental formula in the tlieory of Basic numbers was given by 
Heine, Kugelfunktionen (Berlin, 1878), i. p. 107. 



2ri, 2ril] THE THETA FUNCTIONS 463 

Theta-functions also occur in Fourier's La Theorie Analytique de la Chaleur (Paris, 
1822), cf. p. 265 of Freeman's translation (Cambridge, 1878). 

The theory of Theta-functions was developed from the theory of elliptic functions 
by Jacobi in his Fundameata Nova Theoriae Functionum Ellipticarum (1829), reprinted 
in his Ges. Werke, i. pp. 49-239; the notation there employed is explained in § 21-62. 
In his subsequent lectures, he introduced the functions discussed in this chapter ; an 
account of these lectures (1838) is given by Borchardt in Jacobi's Ges. Werl-e, i. pp. 497-538. 
The most important results contained in them seem to have been discovered in 1835, 
cf. Kronecker, Sitzungsherichte der ALad. zu Berlin (1891), pp. 653-659.] 

Let T be a (constant) complex number whose imaginary part is positive ; 
and write q = e' '", so that q\ < l. 

Consider the function   z, q), defined by the series 

00 

qua function of the variable z. 

If A be any positive constant, then, when \ 2\ \  A, we have 

I qII' Q-k2.niz i < I (7 i '*'" QinA 

n being a positive integer. 

00 

Now d'Alembert's ratio (§ 2'36) for the series S | q ["'e ji  is j q |2n+ig2  

  = — 00 

which tends to zero as ?i   x . The series for    z, q) is therefore a series of 
analytic functions, uniformly convergent (§ 334) in any bounded domain of 
values of vS", and so it is an integral function (§§ 5"3, 5*64). 

It is evident that 

 70 

  ( , 5) - 1 + 2 S  -y'q" ' COS 2nz, 





 =i 


and that 


 (  + 7r,g) =  (ir, g); 


further 


'  z + 'TTT,q)= S (\ ) 5"- 2ng2nw 

 - - 00 



-- \  Q—1 Q—2iz y (\ \ n+iQ(n+i)-Q'i n+ ) iz 

and SO    z + ttt, q) = ~ q~  e~ '  '  (z, q). 

In consequence of these results,   z, q) is called a quasi doubly -periodic 
function of z. The effect of increasing 2  by tt or ttt is the same as the effect 
of multiplying    z, q) by 1 or — q~ e~' , and accordingly 1 and — q~ e~'   are 
called the multipliers or periodicity factors associated with the periods ir and 
TTT respectively. 

21'11. The four types of Theta functions. 

It is customary to write  4  z, q) in place of    z, q) ; the other three 
types of Theta-functions are then defined as follows : 



464 THE TRANSCENDENTAL FUNCTIONS [CHAP. XXI 

The function  3(2',  ) is defined by the equation 

/ 1 \ "  , 

 3 ( , g) =  4 (   + 2 TT, 9 1 = 1 + 2 1 7"- COS 2)12. 

Next, %(z, q) is defined in terms of  4(2 , q) by the equation 

 i = — 00 

and hence* % 2, q) = 2 t (-)"(/(" + *)' sin (2 n + l)z. 

71 = 

Lastly,  2 ( . q) is defined by the equation 

' ., z,q  = %(z + l7r,q) = 2 t q'<" +  - ' cos 2n+  ) z. 

V   / H=0 

Writing down the series at length, we have 

 1 ( ) q) =  q* sin z - 2q* sin 3  + 25"*"" sin bz — ..., 
 2  z, q) = 2q* cos z + 2g' cos Sz + 2q' '' cos 5  + . . . , 
 3 (z, fy) = 1 + 2 ' cos 2z + 2q* cos 4  + 29  cos 62 + ... , 
 4 ( >  ) = 1 ~ 2g cos 22 + 2q* cos 4  — S *" cos Qz + — 

It is obvious that  j (2, q) is an odd function of 2 and that the other 
Theta-functions are even functions of z. 

The notation which has now been introduced is a modified form of 
that employed in the treatise of Tannery and Molk ; the only difference 
between it and Jacobi's notation is that  4 (2, q) is written where Jacobi 
would have written   (2, q). There are, unfortunately, several notations in 
use ; a scheme, giving the connexions between them, will be found in § 2r9. 

For brevity, the parameter q will usually not be specified, so that  1 (2), ... 
will be written for  1 (2, q), .... When it is desired to exhibit the dependence 
of a Theta-function on the parameter r, it will be written   (2 j t). Also 
 2(0),  3(0), ' 4(0) will be replaced by  2>  3, ' 4 respectively; and  / will 
denote the result of making 2 equal to zero in the derivate of  1(2). 
Example 1. Shew that 

3i z, q) = 9s 2z, q )-h 2z, q*). 
Example 2. 01)tain the results 

Bi z)= -B.2 z + i ) =-iMB-i z +  n + UT)=-iMSi z +  !TT\ 

\$2(2)= MS3 z + Ut)= mi z +  \ n +  nr)= S Z + hTr), 

 3(2)= 3i z+U) = .l/ ,(3 + W+i7rr)= m, z + UT), 

3i (2) = - iJfSi (2 +  777") = im.2  z+U+ i rr) =  3 (2 + U), 
where M=q* e". 

* Throughout the chapter, the many-valued function q  is to be interpreted to mean 
exp (Xttit). 



21-12] 



THE THETA FUNCTIONS 



465 



Example 3. Shew that the multipliers of the Theta-functions associated with the 
periods tt, ttt are given by the scheme 





S,iz) 


 2 Z) 


h z) 


3, z) 


n 


-1 


- 1 

N 


1 


1 


TTT 


-N- 


iV" . 


-N 



where iV = q~ e '  . 

Example 4. If -9 (2) be any one of the four Theta-functions and d' (2) its derivate with 
respect to z, shew that 



 9'(2+7r) \ .9'(2) 
 (0 + ,r) ~T £)' 






21'12. TAe  reros of the Theta-functions. 

From the quasi-periodic properties of the Theta-functions it is obvious 
that if    z) be any one of them, and if  o be any zero of    z), then 

z  - - riiTT + niTT 

is also a zero of    z), for all integral values of in and n. 

It will now be shewn that if G be a cell with corners i, t -f- vr,   -I- vr + ttt, 
t -  TTT, then    z) has one and only one zero inside G. 

Since    z) is analytic throughout the finite part of the  -plane, it follows, 
from § 6'31, that the number of its zeros inside G is 



 ' z) 



dz. 



27riJc  ( ) 
Treating the contour after the manner of § 2012, we see that 
1 f " '(z) 



Inriir   ( ) 



dz 



1 /• +-| ( )  '(  + 7 rT)) \ J\  
27rtj, l ( )  (s-f-TTT)  " "  27n 






dz 



t + T 



2idz, 



27rij t 
by § 21*11, example 4. Therefore 

1 r y( ) 



27ri J c   ( ) 



C?  = 1, 



that is to say,    z) has one simple zero only inside G ; this is the theorem 
stated. 



W. M. A. 



30 



466 THE TRANSCENDENTAL FUNCTIONS [cHAP. XXI 

Since one zero of  i  z) is obviously z = 0, it follows that the zeros of 
 i( ), %(z), %, z),  4( ) are the points congruent respectively to 0,  , 

Itt-I-Ittt, Ittt. The reader will observe that these four points form the 
corners of a parallelogram described counter-clockwise. 

21-2. The relations between the squares uf the Theta-f unctions. 

It is evident that, if the Thcsta-functions be regarded as functions of a 
single variable z, this variable can be eliminated from the equations defining 
any pair of Theta-functions, the result being a relation* between the functions 
which might be expected, on general grounds, to be non-algebraic; there 
are, however, extremely simple relations connecting any three of the Theta- 
functions ; these relations will now be obtained. 

Each of the four functions  i"  z), "   z),  3-  z), V  z) is analytic for all 
values of z and has periodicity factors 1, q-" e-*  associated with the periods 
IT, TTT ; and each has a double zero (and no other zeros) in any cell. 

From these considerations it is obvious that, if a, b, a' and b' are suitably 
chosen constants, each of the functions 

aV(g)-h6V ( ) a'% '(z)+b'X'(z) 
V( ) ' V( ) 

is a doubly-periodic function (with periods tt, ttt) having at most only a 
simple pole in each cell. By § 20-13, such a function is merely a constant; 
and obviously we can adjust a, b, a, b' so as to make the constants, in each 
of the cases under consideration, equal to unity. 

There exist, therefore, relations of the form 

     z) = a%' (z) + 6V (z), %' (z) = a%' (z) + b'X' (z). 

To determine a, b, a, b' , give z the special values   ttt and ; since 

we have ' 3- = — (( 4-,  2" =  4"; %- = — a" , ' 3- = 6" 4-. 

Consequently, we have obtained the relations 

X'  z)  4' =  4   z) X' - X' (z) %', %  (z) %' = X'  z) %' - %'  z)  ,1 
If we write z + - ir for z, we get the additional relations 

%' (z) V = %' (z) %' -  .  (z)  3% V (z) V = %' (z) V - X' (z) V. 
By means of these results it is possible to express any Theta-function in 
terms of any other pair of Theta-functions. 

* The analoo;ou.s relation for the fuuctions sinz and cos 2 is, of course, (sin2)"''+(cos,:)2= 1. 



21-2 — 21-22] THE THETA FUNCTIONS 467 

Corollary. Writing z=0 in the last relation, we have 
that is to say 

21*21. The addition-formulae for the Theta functions. 

The results just obtained are particular cases of formulae containing two 
variables ; these formulae are not addition-theorems in the strict sense, as 
they do not express Theta-functions of 2 + y algebraically in terms of Theta- 
functions of z and y, but all involve Theta-functions of z — y as well as of 
z - -y, z and y. 

To obtain one of these formulae, consider  3  z -h y) ' 3  z — y) qua function 
of z. The periodicity factors of this function associated with the periods tt 
and TTT are 1 and (f  e-2''2+!/) . q-  Q-iiKz-y) = q-2  -Hz  

But the function a j-  z) + 6 1-  z) has the same periodicity factors, and 
we can obviously choose the ratio a:b so that the doubly -periodic function 

a%' z) +  fh Hz) 
% z + y)% z-y) 
has no poles at the zeros of  3  z — y) ; it then has, at most, a single simple 
pole in any cell, namely the zero of  3(2:4- y) in that cell, and consequently 
(§ 20'13) it is a constant, i.e. independent of z ; and, as only the ratio a : 6 is 
so far fixed, we may choose a and b so that the constant is unity. 
We then have to determine a and b from the identity in z, 

a%' (z) + b -' (z) =%(z-h y) X (z - y). 
To do this, put z in turn equal to and - ir + -  ttt, and we get 

aX' =  Hy\ h  '( 7r +  rTT  = %  ir + \ 7rr + y)%  'rr +  rrT-y 

and so a =  3-  y)l . , b = " i   y)l 3\ 

We have therefore obtained an addition-formula, namely 

 3 (  + y) %  z - y)  / = V  y) V  z) +  i  iy) X'  z). 

The set of formulae, of which this is typical, will be found in examples 1 
and 2 at the end of this chapter. 

21 '22. Jacobi's fundameMal formulae *. 

The addition-formulae just obtained are particular cases of a set of identities first given 
by Jacobi, who obtained them by purely algebraical methods ; each identity involves as 
many as four independent variables, w, x, y, z. 

Let iv', x\ y\ z' be defined in terms of lu, x, y, z by the set of equations 

2w' = —w- x- y- z  
2x' = iv — x+y + z, 
2?/' = w+x — y + z, 
22' = tv- x- y — z. 

* Ges. Werke, i. p. 505. 

30—2 



463 



THE TRANSCENDENTAL FUNCTIONS 



[chap. XXI 



The re<\ der will easily verify that the connexion between w, x, y, z and iv\ x\ y\ z' is a 
reciprocal one*. 

For brevity t, write [? ] for 5, (w) 5  (.r) 5   y) 3   z) and [r]' for S, (to') \$, (x') \$, (if) S, (z'). 

Consider [3], [1]', [2]', [3]', [4J qua functions of z. The effect of increasing   by tt or nr 
is to transform the functions in the first row of the following table into those in the second 
or third row respectively. 





[3] 


[1]' 


[2]' 


[3]' 


[4]' 


in) 


[3] 


-[ J 


-[I]' 


[4]' 


[3]' 


(rrr) 


 [3J 


-iy[4]' 


iV[3]' 


iV[2]' 


-.V[l]' 



For brevity, iV has been written in place of q~  e~- . 

Hence both -[l]' + [2]' + [3]'-|-[4]' and [3] have periodicity factors 1 and N, and so 
their quotient is a doubly-periodic function with, at most, a single simple pole in any cell, 
namely the zero of  3 (z) in that cell. 

By § 20-13, this quotient is merely a constant, i.e. independent of 2; and considerations 
of symmetry shew that it is also independent of w, x and y. 

We have thus obtained the result 

J[3]=-[l]' + [2]' + [3]' + [4]', 

where A is independent of w, x, y, z; to determine A put w=x=y=z=  and we get 

J g - .-' + V +  i  
and so, by § 21-2 corollary, we see that  4 = 2. 

Therefore 2 [3]= -[l]' + [2]' + [3]' + [4j' (i). 

This is one of Jacobi's formulae ; to obtain another, increase if, .r, y, z (and therefore 
also tp', x\ y', z) by  w ; and we get 

2[4] = [lJ-[2]' + :3]' + [4]' (ii). 

Increasing all the variables in (i) and (ii) by htrr, we obtain the further results 

2[2] = [l]' + [2]' + [3]'-[4]' (iii), 

2[l] = [l]' + [2]'-[3]' + [4j (iv). 

[Note. There are 256 expressions of the form dp (ic) 3g (x) S  (y) 3  (2) which can be 
obtained from  3 (w)  3 (x)  3 (y)  3 (2) by incre;ising w, x, y, z by suitable half-period.s, but 
only those in which the suffixes p, q, r, s are either equal in pairs or all different give rise 
to formulae not containing quarter-periods on the right-hand side.] 

Example 1. Shew that 

[1] + [2] = [!]' + [2]', [2]-h[3] = [2]' + [.3]', [l]-h[4]=.[l]'-f [4]', [3] -h [4] = [3]' -h [4]', 

[l] + [3] = [2]' + [4]', [2]-f[4] = [l]'-h[3]'. 

In Jacobi's work the signs of u-, .r', y', z' are changed throughout so that the complete 
symmetry of the relations is destroyed ; tlie symmetrical forms just given are due to H. J. S. Smith, 
Proc. London Math. Soc. 1. (May 21, 1860, pjx 1-12). 

t The idea of this abridged notation is to be traced in H. J. S. Smith's memoir. It seems, 
however, not to have been used before Kronecker, Journal j'ilr Math. cii. (1887), pp. 260-272. 



21-3] THE THETA FUNCTIONS 469 

Example 2. By writing tv +  n, x + \ tv for iv, x (and consequently  /- \ i , s' +  tt 
for y\ /), shew that 

[3344] + [2211] = [4433]' + [1122j, 

where [3344] means  3  w) S3 (x) S   y)  4 (z), etc. 
Example 3. Shew that 

2[1234] = [3412]'+[2143]'-[1234]' + [4321]'. 
Example 4. Shew that 

21"3. Jacobi's expressions for the Theta-f unctions as infinite products*. 
We shall now establish the result 

M = l 

(where G is independent of z), and three similar formulae. 
Let fi2)= n (1 -  n-i e-- '>) n  l-q->'- e--''); 

each of the two products converges absolutely and uniformly in any bounded 

domain of values of z, by § 3'341, on account of the absolute convergence of 
00 
V  2n-i. hence / (2:) is analytic throughout the finite part of the 2 -plane, 

and so it is an integral function. 

The zeros of/(2') are simple zeros at the points where 

g2iz = g(2H+l) T  (w=..., -2, - 1,0, 1,2, ...) 

i.e. where 2iz = (2?i + 1) ttit + 2ni7ri; so that f(z) and  4 (z) have the same 
zeros; consequently the quotient ' i z)/f z) has neither zeros nor poles in 
the finite part of the plane. 

Now, obviously / (2  + tt) =f z) ; 

GO QO 

and f z+'77r)= IT (1 - 5- +ie-'' ) IT (1 - 52 -3 g-2fe) 

  = 1 n = \ 

=f z) l-q-U- )!  qe  ) 

= -q- e'- f z). 
That is to say f z) arid  i z) have the same periodicity factors (§ 2 I'll 
example 3). Therefore  i z)/f(z) is a doubly-periodic function with no 
zeros or poles, and so (| 20*12) it is a constant G, say; consequently 

 4 (z)=G U  1- 2q"'"-' cos 22 + 5 "--). 
11=1 

00 
[It will appear in § 2142 that G= U (1 - q-'').] 

n = l 

Write z +  TT for 2  in this result, and we get 

%,(z)=G n (1 + 25 ' -! cos 2z + q' '-'). 

n=l 
* Cf. Fundamenta Nova, p. 145. 



470 THE TRANSCENDENTAL FUNCTIONS [CHAP. XXI 

Also  1 (z) = - iq  e"  ,   +   ttt) 

00 3C 

= \  igi e'z G IJ (1 - 9=" e-'') TI (1 - q'''~"- e -'''') 

n=l   = 1 

= 26 5* sin   11 (1 - f"e-") U  I -  e'-'O, 

w = 1   = 1 

and so  , (z) = 2Gq  sin   ft ( I - 25-" cos 2z + q' ) 

n = \ 

while ' , z)=' Jz + l7r] 

= 2Gq  cosz U  1 + 2q-'' cos 2z +  "). 

n = l 
Example. Shew that* 

( cc 18 (-00 ISfoc 18 

J n (i-?2 -i)i +16?- n (i+?2 )l = ' n (i+92 -i)  . 

bi = l J 'n=l J ln = l J 

(Jacobi.) 

21"4. TZ/e differential equation satisfied hy the Theta-functions. 

We may regard  s(z t) as a function of two independent variables z 
and t; and it is permissible to differentiate the series for  3(2 |t) any 
number of times with regard to z or r, on account of the uniformity of 
convergence of the resulting series (§ 4*7 corollary) ; in particular 

—   '   = — 4 2 n- exip n-7nT + 2mz) 

OZ ft = — 00 

Consequently, the function ' 3 (2 \ t) satisfies the partial differential equation 

1 .d y dy   

The reader will readily prove that the other three Theta-functions also 
satisfy this equation. 

21*41. A relation between Theta-functions of zero argument. 

The remarkable result that 

V(0) =  2 (0) 3 (0) 4(0) 

will now be established f. It is first necessary to obtain some formulae for 
differential coefficients of all the Theta-functions. 

* Jacobi describes this result (Fund. Nova, p. 90) as 'aequatio identica satis abstrusa.' 

t Several proofs of this important proposition have been given, but none are simple. 

Jacobi's original proof (Gfis. Werke, i. pp. 515-517), though somewhat more difficult than the 

proof given here, is well worth study. 



21-4, 21-41] 



THE THETA FUNCTIONS 



471 



Since the resulting series converge uniformly, except near the zeros of 
the respective Theta-functions, we may differentiate the formulae for the 
logarithms of Theta-functions, obtainable from § 21"3, as many times as we 
please. 

Denoting differentiations with regard to z by primes, we thus get 



%' z) = X z) 



1—1 a—'iiZ 






L =i (1 + q e ) n=\ 

Making 2r   0, we get 



a%  (>2n— 1 g-2i3 



=1 (1 + g "-i e-- )- 



V (0) = 0,  3" (0) = - 8 3 (0) J  (3 - .  



In like manner. 



V(0)-0, V(0) = 8 4(0) 2 



 2' (0) = 0,  2" (0) =  2 (0) 



..=i(] -cf- r 



-1-8 S 



 =i(l + <? '?J 



and, if we write  1  z) = sin z .   (z), we get 

</)'(0) = 0, f (0) = 8(/>(0) i  '" 

If, however, we differentiate the equation  1 (z) = sin z . cj) (z) three times, 
we get 



 / (0) = (/> (0),  /" (0) = S<p" (0) - </> (0). 



Therefore 



V"(0) 
V(0) 



= 24 2 



=1 (1 - q' y 



-1; 



and 



V(0) v:iO) , V:(0) 

" ,(0)   Sf3(0)  4(0) 



 2?l 00  ,2/1—1 CO  2W— 1 



8-2   2   + 2   

L  =i (1 + q' 'f .=1 (1 + q' '-'r .=1 (1 - ?''"-'/ 



= 8 



\  V 



+ 2 



- 2 



.=1 (1 + q 'T- .=1 (1 - q y n=i (1 - q y 
on combining the first two series and writing the third as the difference of 
two series. If we add corresponding terms of the first two series in the last 
line, we get at once 

V(0) 



 V(0) V10)\  Vi0) 2  



= 1 + 



 ,(0) %(0) %iO) n=i l~q'''y V(0) 



472 THE TRANSCENDENTAL FUNCTIONS [cHAP. XXI 

Utilising the differential equations of § 21-4, this may be written 

1 d%' (0 I t) 
V(0|t) dr 

\  1 d%(0\ T) 1 d% 0\ r) 1 d%(0\ r) 

~ 2(0|t) dr " 3(0|t) dr %(0\ t) dr 

Integrating with regard to r, we get 

V (0, q) = C% (0, q) % (0, q) % (0, q), 

where C is a constant (independent of q). To determine C, make q- 0; since 

\ imq- X = % \ imq-- % = 2 lim 3 = l, lim 4 = l, 

q O q O q -O (/ -O 

we see that = 1; and so 

which is the result stated. 

21-42. The value of the constant G. 

From the result just obtained, we can at once deduce the value of the 
constant G which was introduced in § 21-3. 
For, by the formulae of that section, 

 / = 0(0)= 2q  GU 1- q' % % = 2qiGU l + f  )  

M=l n=l 

 3 = G  n (1 + r''-')\ x = GU i- q ' -'f, 

and so, by | 21-41, we have 

00 O) OO CO 

n (1 - (f y = G  n (1 + q;"')' n (i + q '- y n (i - q  -y. 

Now all the products converge absolutely, since \ q\ < l, and so the 
following rearrangements are permissible : 

I n (1 - g "-') n (1 - ? 'ol • I fi (1 + (t-') n (1 +  'ol 
= n (i-9'O n (i + g* ) 

M=l W=l 

= n (1 - g * ), 

M = l 

the first step following from the consideration that all positive integers are 
comprised under the forms 2n — 1 and 2n. 
Hence the equation determining G is 

n (1 - (f'J = G\ 



n=\ 



andso G=+ n (1 - r/  



21*42 — 21-5] THE THETA FUNCTIONS 473 

To determine the ambiguity in sign, we observe that G is an analytic 
function of q (and consequently one-valued) throughout the domain \ q\ < \; 
and from the product for  3(2 ), we see that G- 1 as q—>0. Hence the 
plus sign must always be taken ; and so we have established the result 

G=Yl (l- 'O- 

Example 1. Shew that  1=22*0 . 

Example 2. Shew that 

Example 3. Shew that 

1 + 2 i y -= n  (i-(72 )(i+g2'i-i)2). 

)! = 1 n = l 

21  S. Connexion of the Sigma-f unction with the Theta- functions. 

It has been seen    20-421 example 3) that the function a- [z \ wx, M2), formed with 
the periods 2a)i, 2co2, is expressible in the form 

where (/=exp (ttiojo/wi). 

If we compare this result with the product of § 21 '4 for  1 [z \ t), we see at once that 

.(.) =  exp( Y ,-in(l-./ )-3i( |- ). 
TT \ 2coi/ 2  =i   \ 2coi I COj/ 

To express j i in terms of Theta-functious, take logarithms and differentiate twice, 
so that 



 < )=:i-(0— =(e,)-( .T 






.4>' 

where v =  7rzja)  and the function cf) is that defined in § 21"41. 

Expanding in ascending powers of z and equating the terms independent of z in this 
result, we get 



a>i 3 \ 2(oiJ \ 2(Ui/ (f) (0) 

and SO ,;=\ —— -— . 

Lzooi  i 

Consequently a-  z \ wi, wo) can be expressed in terms of Theta-functions by the 
formula 

'031 / I'-. l \ o / I  "2  



mi 



a( lo, co,)=- ,exp( --g J5,( v 

where v hivzlwi. 

Example. Prove that 

/n'-a-iBi" 7n'\ 

21"5. The expression of elliptic functions hy means of Theta-functions. 

It has just been seen that Theta-functions are substantially equivalent 
to Sigma-functions, and so, corresponding to the formulae of §§ 20'5-20"58, 
there will exist expressions for elliptic functions in terms of Theta-functions. 



474 THE TRANSCENDENTAL FUNCTIONS [CHAP. XXI 

From the theoretical point of view, the formulae of §§ 20"5-20"53 are the 
more important on account of their S3'mmetry in the periods, but in practice 
the Theta-function formulae have two advantages, (i) that Theta-functions 
are more readily computed than Sigma-functions, (ii) that the Theta- 
functions have a specially simple behaviour with respect to the real period, 
which is generally the significant period in applications of elliptic functions 
in Applied Mathematics. 

Let f z) be an elliptic function with periods 2wi, 2\&J2; let a fundamental 
set of zeros (aj, Oa, ... an) and poles (/3j, /3..., ... /3 ) be chosen, so that 



as in § 20o3. 

Then, by the methods of § 20"53, the reader will at once verify that 

TTZ — TTCUr 1 W.jX TTZ — TTySr t  Wg 

r=\ (.   

where A  is a constant ; and if 



f z) = A,\ \ \ X[— - -'~' \ \  %  .  

' V Iw, (jdJ \ 2\&), CO 



1)1). 

m = l 

be the principal part oi f z) at its pole  , then, by the methods of §20-')2, 

r=i (, =.! (m-1)! dz"" " V 2\&)i \&)i/J 
where J. 2 is a constant. 

This formula is important in connexion with the integration of elliptic 
functions. An example of an application of the formula to a dynamical 
problem will be found in § 22741. 

Example. Shew that 

  3  (2)\  \  \ \     - l' ( ) , - 3 - 3" 

5i2 (2) '    dz .9i (2)  i'3 ' 
and deduce that 

21'51. Jacohis imaginary transformation. 

If an elliptic function be constructed with periods 2\&)i, 2w2, such that 

/ (( 2/a)i) > 0, 
it might be convenient to regard the periods as being 'Iw , — 2\&)i : for these 
numbers are periods and, if I (co-i/coi) >0, then also /(— oji/wo)> 0. In the 
case of the elliptic functions which have been considered up to this point, 
the periods have appeared in a symmetrical manner and nothing is gained 
by this point of view. But in the case of the Theta-functions, which are 
only quasi-periodic, the behaviour of the function with respect to the real 
period tt is quite different from its behaviour with respect to the complex 
period ttt. Consequently, in view of the result of § 21"43, we may expect to 



21-51] THE THETA FUNCTIONS 475 

obtain transformations of Theta-functions in which the period-ratios of the 
two Theta-functions involved are respectively t and — l/r. 

The transformations of the four Theta-functions were first obtained by 
Jacobi*, who obtained them from the theory of elliptic functions ; but Poissonf 
had previously obtained a formida identical with one of the transformations 
and the other three transformations can be obtained from this one by ele- 
mentary algebra. A direct proof of the transformations is due to Landsberg, 
who used the methods of contour integration ij:. The investigation of Jacobi's 
formulae, which we shall now give, is based on Liouville's theorem ; the precise 
formula which we shall establish is 

where (- ir)'   is to be interpreted by the convention arg(- iV) <  '  

For brevity, we shall write - 1 /t = r', q — exp  ttW). 

The only zeros of ' 3  z \ t) and ' 3  t z j t) are simple zeros at the points 
at which 

1 1 / / ,,,1,1/ 

z = mir -  UTTT + 2 '"' + o '""''' TZ = mir + tiTTT 4-   vr 4- .3 ttt 

respectively, where m, n, m, n take all integer values; taking m =- )i — \, 
n = m, we see that the quotient 

is an integral function with no zeros. 

A 1 . / X . X (IZTTT + 7r-T-\ \  \  . , 

Also   z + ttt)  ylf z) = exp ( -. j  q  e -'  = 1, 

while yjr (z - 'tt) -  yjr ( z) = exp ( ; j x q~ e~-' '' = 1. 

Consequently -v/r (z) is a doubly-periodic function with no zeros or poles ; 
and so (§ 20'12) i/ ( ) must be a constant, A (independent of 2). 

Thus  3 (z ! t) = exp (tW/7r)  3 ( r \ r') ; 

and writing z +  -tt,   +   ttt,   -t- - tt -i-   ttt in turn for z, we easily get 
 , (z : t) = exp  irz-'/Tr) % (zr I r), 

A% (z\ t)= exp (itZ /tt) % (ZT I t), 

J.' i (z\ t)= — i exp  Wz-j-n) % (zt \ t'). 

* Journal fur Math. in. (1828), pp. 403-404 [Ges. Werke, i. (1881), pp. 264-265]. 

t Mem. dc VAcad. des Sci. vi. (1827), p. 592; the special case of the formula in which z-0 
had been given earlier by Poisson, Journal de VEcole polyteclmique, xii. (cahier xix), (1823), 
p. 420. 

+ This method is indicated in example 17 of Chapter vi, p. 124. See Landsberg, Journal fiir 
Math. CXI. (1893), pp. 234-253. 



47 G THE TRANSCENDENTAL FUNCTIONS [CHAP. XXI 

We still have to prove that A = — it)- ; to do so, differentiate the last 
equation and then put 2 = 0; we get 

 /(OJT) = -iVV(0 t'). 
But %' (0 t) =  , (0 j t)  3 (0 i t)  4 (0 1 t) 

and  / (0 : t') =  , (0 : r')  3 (0 \ r)  , (0 t') ; 

on dividing these results and substituting, we at once get A~'- = — W, and so 

A = ± -iT) -. 
To determine the ambiguity in sign, we observe that 
 3(0 t)= 3(0|t'), 

both the Theta-functions being analytic functions of t when 7 (t) > ; 
thus A is analytic and one-valued in the upper half r-plane. Since the 
Theta-functions are both positive when t is a pure imaginary, the plus sign 
must then be taken. Hence, by the theory of analytic continuation, we 
always have 

A = +  - ir)  ; 

this gives the transformation stated. 
It has thus been shewn that 

X 1 °  

5" on-Trir+tniz \  'V As -mr]- 1  1:17) 



7j= -oe 

Example 1. Shew that 

when TT = -.  

Example 2. Shew that 



B, Q\ t) \  %AO\ t') 
53(0|r) 53(0|r') 



MOPr + 1)  i  2iOJjr) 
53(0 I r + 1) 54(0|r)' 



Example 3. Shew that 

and shew that the plus sign should be taken. 

21"52. Landens type of transformation. 

A transformation of elliptic integrals (§ 227), which is of historical 
interest, is due to Landen (§ 22-42); this transformation follows at once 
from a transformation connecting Theta-functions with parameters t and 2t, 

namely 

X  z\ t)X  z I t)  3(0 It)  4(0 JT) 
 4(22|2t)  4(0|2t) 

which we shall now prove. 

The zeros of   z r) i z\ r) are simple zeros at the points where 

z = [m + -\ IT -  \ n + -A TTT and where z = imr +in + -Airr, where m and n 



21-52, 21-6] THE THETA FUNCTIONS 477 

take all integral values ; these are the points where 1z = mir + Ui Ar - ir .'Ir, 
which are the zeros of  4 (2  j 2t). Hence the quotient 

 4 (2  I 2t) 
has no zeros or poles. Moreover, associated with the periods it and ttt, it 
has multipliers 1 and  cf-  e"-'' )   - q-  e''  ) -   - q~~"' e- " ) = \ \ it is therefore 
a doubly-periodic function, and is consequen-tly (§ 20"12) a constant. The 
value of this constant may be obtained by putting z —   and we then have 
the result stated. 

If we write \ \   +;- TTT for z  we get a corresponding result for the other 

Theta-functions, namely 

 ,( |t) i( |t)  3 (0 1 )>4 (OK) 

 i(2 |2r) '"" 4(0i2T) 

21-6. The differential equations satisfied hy quotients of Theta-functions. 
From § 21-11 example 3, it is obvious that the function 

has periodicity factors - 1, + 1 associated with the periods tt, ttt respectively; 
and consequently its derivative 

[X  z)  4  z) - %' ( )  1 ( )l - V ( ) 
has the same periodicity factors. 

But it is easy to verify that  .(z) ,, z)/ J  (z) has periodicity f;xctors - 1, 
+ 1 ; and consequently, if <  (z) be defined as the quotient 

 X (z) % (z) - X ( )  1 i )] - [% (z) % (z) , 
then (f) (z) is doubly-periodic with periods tt and ttt ; and the only possible 
poles of (f) (z) are simple poles at points congruent to   tt and   vr -h .  ttt. 



Now consider cf) iz + I ttt] ; from the relations of § 21-11, namely 

% Z + l7rT =iq-h-''X z\ ' ,( z + l7rT'j = iq-ie-''%(z), 

%\ z+l7rT =q-ie'''X(z), x z- l'rrT =q~ e-''% z), 
we easily see that 

(  (  -i-   ttt) =  - X (z)  1 (z) + X ( )  4 ( )l - [% (z) % (z) . 
Hence   (z) is doubly-periodic with periods tt and -  ttt ; and, relative to 



these periods, the only possible poles of   z) are simple poles at points 
1 
2 



congruent to   vr 



478 THE TRANSCENDENTAL FUNCTIONS [CHAP. XXI 

Therefore ( 20-12), </)( ) is a constant; and making z- 0, we see that 
the value of this constant is [ i' ' 4  -  1 2' a  =  Z- 

We have therefore established the important result that 

writing   = * i ( )/ 4  2) and making use of the results of § 21 2, we see that 

(§)' =  -  ~  ' ''  ' "  '  '' ' 

This differential equation possesses the solution  i( )/ 4(2). It is not 
difficult to see that the general solution is ±% z + a)/%(z + a) where a 
is the constant of integration ; since this quotient changes sign when a is 
increased by tt, the negative sign may be suppressed without affecting the 
generality of the solution. 

Example 1. Shew that 

dz [Si  z;j 2 S4 (2)  4 (z) • 

Example 2. Shew thcat • 

lPiii)l=\ q2 ii?) Mi) 

21*61. The genesis of the Jacohian Ellip.tic function* snu. 
The differential equation 

(S)'   '' "  ' '  '' "  ' '' ' 
which was obtained in § 21-6, may be brought to a canonical form by a slight 
change of variable. 

Writet  %/% = y,  V =   ; 

then, if A:- be written in place of  2/ 3, the equation determining y in terms 
of M is 

(|y = a-/)(i-A.y). 

This differential equation has the particular solution 

The function of u on the right has multipliers -1,4-1 associated with 
the periods 7r%", irT -,;-; it is therefore a doubly-periodic function with 
periods 27r 3  ttt -. In any cell, it has two simple poles at the points 
congruent to hirr a- and tt -J  + iTTT j- ; and, on account of the nature of the 
quasi-periodicity of y, the residues at these points are equal and opj3osite in 
sign ; the zeros of the function are the points congruent to and tt sI 

* Jacobi and other early writers used the notation sin am in phice of sn. 

t Notice, from the formulae of § 21-3, that  2 + 0,  3 + when \ q\ < l, except when q = 0, in 
which case the Theta-functions degenerate; the substitutions are therefore legitimate. 



21-61, 21*62] THE THETA FUNCTIONS 479 

It is customary to regard y as depending on k rather than on q ; and to 
exhibit y as a function of u and k, we write 

2/ = sn  u, k), 
or simply y = sn u. 

It is now evident that sn (u, k) is an elliptic function of the second 
of the types described in § 20'13 ; when g— >0 (so that  '— >0), it is easy to see 
that sn(w, A")— >sin v. 

The constant k is called the modulus; if  ''  =  4/ 3, so that k  + k'' =l, 
k' is called the complementary modulus. The quasi-periods ir , ttt .  are 
usually written 2K, 2iK', so that sn (u, k) has periods 4iK, 2 K'. 

From § 21-51, we see that 2K' = 7r' . (0 \ r'), so that K' is the same 
function of t as K is of t, when tt' = — 1. 

Example 1. Shew that 

dzS, z) -"  S,(z) S, z)' 
and deduce that, if ?y = -   " ~. , and u = z\$ -, theu 

• •J - 4 ( ) 

Example 2. Shew that 

rf2 3. 



4(2) ''•' S, z)3, z)' 



d 3 ' (z) 
and deduce that, if j/ = -r  6"7 \ >  '  u zS , then 

 3  4 [z) 

Example 3. Obtain the following results : 

[These results are convenient for calculating /•, k\ A", A'' when q is given.] 

21'62. Jacohi's earlier notation*. The Theta-f unction ©(?<) and the 
Eta-fanction H ( ). 

The presence of the factors  3"- in the expression for sn  u, k) renders it 
sometimes desirable to use the notation which Jacobi employed in the 
Fvndamenta Nova, and subsequently discarded. The function which is of 
primary importance with this notation is © (u), defined by the equation 

Ch) (u) =  4  u%-' i t), 
so that the periods associated with © (u) are 2K and 2iK'. 

* This is the notation employed throughout the Fundamenta Nova. 



480 THE TRANSCENDENTAL FUNCTIONS [CHAP. XXI 

The function \& u + K) then replaces  3 (' ) ; and in place of  i(-?) we 
have the function H (u) defined by the equation 

H ( ) = - iq - ie'"" " <-' '  e (u + i7v") =    u%-'- , t), 
and  .  z) is replaced by H (?/ + A"). 

The reader will have no difficulty in translating the analysis of this 
chapter into Jacobi's earlier notation. 

Example 1. If e'  u)=- , , shew that the singularities of -t-t-t are simple poles 

at the points congruent to I'K' (mod 2A', -liK') ; and the residue at each singularity is 1. 

Example 2. Shew that 

H' (0) = W A'-i H ( A') e (0) e (A'). 

21"7. The problem of Inversion. 

Up to the present, the Jacobian elliptic function sn (i  k) has been 
implicitly regarded as depending on the parameter q rather than on the 
modulus k ; and it has been shewn that it satisfies the differential equation 

I — -  — I = (1 - sn- u) (1 — k- sn- u), 

where A:  = V (0, ?)/ V (0, 5). 

But, in those problems of Applied Mathematics in which elliptic functions 
occur, we have to deal with the solution of the differential equation 



 I)-('- '> '-'y-> 



in which the modulus k is given, and we have no a priori knowledge of the 
value of q\ and, to prove the existence of an analytic function sn(M, k) 
which satisfies this equation, we have to shew that a number t exists* such 
that 

When this number t has been shewn to exist, the function sn(w, k) can 
be constructed as a quotient of Theta-functions, satisfying the differential 
equation and possessing the properties of being doubly-periodic and analytic 
except at simple poles ; and also 

lim sn u, k)/i( = 1. 

That is to say, we can invert the integral 

\  [y dt 

 ~Jo  l-(')  l-k'i ) ' 
so as to obtain the equation y= sn (u, k). 

* The existence of a number r, for which / (t) > 0, involves t)ie existence of a number q such 
that I g I < 1. An alternative procedure would be to discuss tlie differential equation directly, 
after the manner of Chapter x. 



217, 2r7l] THE THETA FUNCTIONS 481 

The difficulty, of course, arises in shewing that the equation 

c=V(0|tW(0|t), 

(where c has been written for L-). has a solution. 

When* < c < 1, it is easy to shew that a solution exists. From the 
identity given in §21*2 corollary, it is evident that it is sufficient to prove 
the existence of a solution of the equation 

1-c = V(0|t)/V(0!t), 

CO /  \  Q2n-1\ 8 

which may be written 1 - c = 11  - ) . 

Now, as q increases from to 1, the product on the right is continuous 
and steadily decreases from 1 to ; and so (§ 3'63) it passes through the 
value 1 — c once and only once. Consequently a solution of the equation 
in T exists and the problem of inversion may be regarded as solved. 

21 "71. The problem of inversion for complex values of c. The modular functions 

f r),g T),h T). 

The i roblem of inversion may be regarded as a problem of Integral Calcuhis, and it 
may be j roved, by somewhat lengthy algebraical investigations involving a discussion of 

the behaviour of I (I - ;!-) ~ 2 (1 — k fi) ~ 2 dt, when y lies on a 'Eiemann surface,' that the 

J 
problem of inversion possesses a solution. For an exhaustive discussion of this aspect of 
the problem, the reader is referred to Hancock, Elliptic Functions, i. (New York, 1910). 

It is, however, more in accordance with the sjjirit of this work to prove by Cauchy's 
method (§ 6-.31) that the equation =  2* (  I ' V- s* (  1 '')    one root lying in a certain 
domain of the T-j)lane and that (subject to certain limitations) this root is an analytic 
function of c, when c is regarded as variable. It has been seen that the existence of this 
root yields the solution of the inversion problem, so that the existence of the Jacobian 
elliptic function with given modulus k will have been demonstrated. 

The method just indicated has the advantage of exhibiting the potentialities of what 
are known as modular /mictions. The general theory of these functions (which are of 
great importance in connexion with the Theories of Transformation of Elliptic Functions) 
has been considered in a treatise by Klein and Fricket. 

:Mnir ,s S./ 0\ t) 



Let   / (r) = We ir n \ —  , — r  \ 



IgV""   '" ' J  3 (0 

" =53-'(0 



..  \ g(2n-l) T 8 54*(0ir) 



h r)=-f r)lg r). 
Then, if tt = — 1, the functions just introduced possess the following properties : 
/(r + 2)=/(r), 5r(r + 2)=5r(r), f r)+g r) = \, 

f r +  )  h (r), / (r') =g (r), g (r') =/(r), 

by §§ 21 '2 corollary, 2r51 example 1. 

* This is the case which is of practical importance. 

t F. Klein, Vorlesungen uber die Theorie der clliptischen Modnlfunktionen (ausgearbeitet und 
vervoUstandigt von E. Fricke). (Leipzig, 1890.) 

W. M. A. 31 



482 



THE TRANSCENDENTAL FUNCTIONS [CHAP. XXI 



It is easy|\ to see that as /(r)- - + QO , the functions iV<*~""/('") = /i W and g (t) tend to 
unity, uniformly with resi)ect to R (r), when - 1   (r)   1 ; and the derivates of these two 
functions (with regard to r) tend uniformly to zero* in the same circumstances. 

21 'Til. The principal solution off (t) — r = 0. 

It has been seen in § 6*31 that, if /(t) is analytic inside and on any contour, iiri times 
the number of roots of the equation /(t) — c = inside the contour is equal to 

/• 1 df r) 
Ifir c ir ' 

taken ]round the contour in question. 

Take the contour ABCDEFE' D'C B' A shewn in the figure, it being supposed 
temporarily! that /(t) — c has no zero actually on the contour. 

E' . F E 




-1 1 

The contour is constructed in the following manner : 

FE is drawn parallel to the real axis, at a large distance from it. 

AB la the inverse of FE with respect to the circle | t | = 1. 

BC is the inverse of ED with respect to [ r | = 1, Z) being chosen so that D\=AO. 

By elementary geometry, it follows that, since C and D are inverse points and 1 is its 
own inverse, the circle on D\ as diameter passes through C ; and so the arc CD of this 
circle is the reflexion of the arc AB in the line R (r) =  . 

The left-hand half of the figure is the reflexion of the right-hand half in the line 
R t) = 0. 

* This follows from the expressions for the Tlieta-functions as power series in q, it being 
observed that [ 9 [ -  as I (t)  - -|- oo . 

t The values of/ T) at points on the contour are discussed in § 21'712. 



2r71l] THE THETA FUNCTIONS * 483 

It will now be shewn that, unless* c  1 or c O, the equation /(r) — c=0 has one, and 
only one, root inside the contour, provided that FE is sufficiently distant from the real 
axis. This root will be called the principal root of the equation. 

To establish the existence of this root, consider / -rr\ — --y  dr taken along; the 
various portions of the contour. 

Since/(r + 2)=/(r), we have , 

I j BE J ED- ) f (t) -C dr 

Also, as T describes BC and B'C", r'(= — l/r) describes E'D' and ED respectively; 
and so 

   BC j C-B'] f T)-C dr \ j BC Jc-B']g T)-C dr 

 ] ED' J DE) g r)-C dr 
= 0, 
because g (•r' + 2)= (r'), and consequently corresponding elements of the integrals cancel. 
Since / (r ± 1 ) = A (r), we have   

[j D'C jCD]f r)-C dr jB:ABh T)-c dr 

but, as T describes B'AB, r describes EE\ and so the integral round the complete contoui* 
reduces to 

/" f\ ] df r)   1 dh r')   1 dfiyiXdr 

jEE'\ f -r)-c dr h(T') — c dr f 'r') — c dr ] 

 i    dfjr) 1 dh r) 1 dlMidr 

]EE'\ f T)-C dr h r)\ \ -c.h r)] dr '  g  r) - C dr j ' 

Now as EE' moves off to infinityt, /(t) — c-*- -c=t=0,  (t)-c- -1 — c4=0, and so the 
limit of the integral is 

- lim f  —   -   log h (r)  dr 

J EE' l-C.A(r) dr  °   '  

 X m \ 1 r . logACr) \  rflog (r) ]   

.' EE  C.h r)\ dr dr j ' 

But 1- c.A(t) 1, fi(r)~ \,g, r)- l,   - 0,  - 0, and so the limit of the 

dr clr 

integral is 



I nidr = 2-i 

J E'E 



Now, if we choose EE' to be initially so far from the real axis that / (r) — c,  - c.h (r), 
g  r) — e have no zeros when r is above EE', then the contour will pass over no zeros 
of /(r) — c as EE' moves off to infinity and the radii of the arcs CD, D'C, B'AB diminish 
to zero ; and then the integral will not change as the contour is modified, and so the 
original contour integral will be Sttz, and the number of zeros oif r) — c inside the original 
contour will be precisely one. 

* It is shewn in § 21'71'2 that, if c l or c O, then/(T) -c has a zero on the contour. 
t It has been supposed temporarily that c=|=0 and 4=1. 

31—2 



484 THE TRANSCENDENTAL FUNCTIONS [CHAP. XXI 

21 '712. The values of the modular function f  t) on the contoitr considered. 

We now have to discuss the point mentioned at the beginning of  5 21-711, concerning 
the zeros of f r) — c on the lines* joining +1 to ±l + x t and on the semicircles of 
05C1,  - ) C'B'0. 

As T goes from 1 to 1 + x t or from — 1 to — 1 + x i, /(t) goes from - x to through 
real negative values. So, if c is negative, we make an indentation in DE and a corre- 
sponding indentation in D' E' ; and the integrals along the indentations cancel in virtue of 
the relation /(t + 2) +/(r). 

As r describes the semicircle 0 C1,t' goes from - 1 + x I'to — 1, and/(r) = 5r(T') = l — /(r), 
and goes from 1 to +x through real values ; it would be possible to make indentations in 
BC and B'C to avoid this difficulty, but we do not do so for the following reason : the 
ettect of changing the sign of the imaginary part of the number '• is to change the sign of the 
real part of r. Now, if < 7  (c) < 1 and I (c) be small, this merely make-s t cross OF by a 
short i>ath ; if R (c) < 0, t goes from DE to D' E' (or vice versa) and the value of q alters 
only slightly ; but if R c) > 1, r goes from BC to B'C, and so q is not a one-valued function 
of c so far as circuits round c = +1 are concerned ; to make q a one-valued function of c, 
we cut the c-plane from -l-l to -fx ; and then for values of c in the cut plane, q is 
determined as a one-valued analytic function of c, say q (c), by the formula q (c) = e' '"' ''' 

where 

. , 1 /• r df r), 
2771 j/(r)-C dr 

as may be seen from § 6'3, by using the method of § 5-22. 

If c describes a circuit not siu-rounding the point c=l, q c) is one-valued, but t c) is 
one-valued only if, in addition, the circuit does not surround the point c = 0. 

21 '72. The periods, regarded as functions of the modidus. 

Since K=\ 7rB:  (0, q) we see from 5  21-712 that K is a one-valued analytic function of 
c =k ) when a cut from 1 to -l-x is made in the c-plane; but since K'= -irK, we see 
that K' is not a one-valued function of c unless an additional cut is made from to — x ; 
it will appear later (§ 22-32) that the cut fi'om 1 to -fx which was necessary so far as 
K is concerned is not necessary c\ s regards K'. 

2173. The inversion-problem associated iv it h Weierstrassiari elliptic functions. 

It will now be shewn that, when invariants g.  and g  are given, such that g %lg , it 
is possible to construct the Weierstrassian elliptic function with these invariants ; that is 
to say, we shall shew that it is possible to construct \ periods 2(i)i, 2u>.2 such that the function 
fp  z ( oi, do) has invariants g  and g . 

The problem is solved if we can obtain a solution of the differential equation 



m 



   if -9- -93 

of the form   =   (2 i wi, <t>2)- 

"We proceed to effect the solution of the equation with the aid of Theta-functions. 
Let v = Az, where .4 is a constant to be determined presently. 

* We have seen that EE ' can be so chosen that f (t) -c lias uo zeros either on EE ' or on 
the small circular arcs. 

t On the actual calculation of the periods, see E. T. A. Innes, Proc. Edinburgh Royal Sue. 
XXVII. (1907), pp. 357-368. 



21*7 12-21-8] THE THETA FUNCTIONS 485 

By the methods of § 21 -e, it is easily seeu that 

and hence, using the results of § 21 "2, we have 

Now let ei, e. , e  be the roots of the equation ' y' —g-iy—gz =  j chosen in such an order 
that ( 1 — e'2)l ei — 63) is not* a real number greater than unity or negative. 

In these circumstances the equation 

61-63  3'(U|r) 

possesses a solution (§ 21 "712) such that /(r)>0; this eqvxation determines the parameter 
T of the Theta-functions, which has, up till now, been at our disposal. 

Choosing t in this manner, let A be next chosen so thatt 



Then the function 
satisfies the equation 



.V =  ' |! j 53MO I r) V (0 I r) + ei 
(   ) = 4 (y - ei)  y - 62)  y - 63). 



The periods of y, qua function of z, are ttJ, tttJA ; calling these 2(bi, 2W2 we have 

/(co2/<ai)>0. 
The function   z | wi, oa.  may be constructed with these periods, and it is easily 

seen that  (3)- %\ f-f-J  ,2(0 | t) V(0 | r)-ei is an elliptic function with no pole at 
the origin | ; it is therefore a constant, C, say. 

If 6*2 , 6*3 be the invariants of   (2 | wi , a>. , we have 

Aip ( z)-G.  z)-G, = f  z) =   ip z)-C-e,]   z)-C-e. l iJ z)-C-e, , 
and so, comparing coefiicients of powers of <  (z), we have 

= 12C, G2=g-2-l2C% G3=g3-g,C+iC  
Hence (7=0, G.2=g2, Gz g ; 

and so the function   z \ co , wo) with the required invariants has been constructed. 

21'8. The numerical Computation of elliptic functions. 

The series proceeding in ascending powers of q are convenient for 
calculating Theta-functions generally, even when |   | is as large as 0"9. But 
it usually happens in practice that the modulus k is given and the calculation 

* If  \ iZ: >i thenO<   <:l; andif   <0, then 1-  >1, and 

ei-e  ei-i'j e -e  Ci-e  

 j -ej   ri\  lZ 1" <l. 

The values 0, 1, qo of (ej - <'2)/( i -  3) a-i'e excluded since (12  4= 27(/3 . 

t The sign attached to   is a matter of indifference, since we deal exclusively with even 
functions of v and -. 

t The terms in z'"  cancel, and there is no term in z~i because the function is even. 



486 THE TRANSCENDENTAL FUNCTIONS [CHAP. XXI 

of K, K' and q is necessary. It  v ll be seen later (§§ 22'801, 22'32) that 
K, K' are expressible in terms of hypergeometric functions, by the equations 

but these series converge slowly except when | k \ and j k' \ respectively are 
quite small ; so that the series are never simultaneously suitable for numerical 
calculations. 

To obtain more convenient series for numerical work, we first calculate q 
as a root of the equation k =  o" (0, q)l 'i' (0, q), and then obtain K from the 

formula K= r ir' i (0, q) and K' from the formula 

K'=7r- K\ og, l!q). 

The equation k = V (0, ?)/ V (0, q) 

is equivalent to*  Jk' =  4 (0, 5)/ 3 (0, q). 

-i 111 
Writing 2e = —j, , (so that < e < ;j when 0<k< 1), we get 

\ %i O,q)-' AO, q)  %(0,q') 
 ' % 0,q) + % 0,q) % 0,q )- 

We have seen (§§ 21*71-21-712) that this equation in g  possesses a 

solution which is an analytic function of e* when I e < - ; and so q will be 

expansible in a Maclaurin series in powers of e in this domainf. 

It remains to determine the coefficients in this expansion from the 
equation 

g + (/ + r + ... 
  ~ 1 + 2 * + 25" + . . . ' 
which may be written 

q=6 + 2qU-q''+2q' €-q-'+ ...; 

the reader will easily verify by continually substituting e + 2q*e — q  + ... 
for q w herever q occurs on the right that the first two termsj are given by 

q = e + 2e' + 15e  + loOe'  + U''). 

It has just been seen that this series converges when j e , <   . 

[Note. The first two terms of this expansion usually suffice ; thus, even if k be as 
large as  (0-8704) = 0-933..., e = |, 2€-  = 0-0000609, 15f  = 0-0000002.] 

Example. Given k = k' = \ IJ2, calculate q, K, A" by means of the expansion just 
obtained, and also by observing that t=i, so that q = e~  . 

[y = 0-04.32 139,  = A" = 1-854075.] 

* In numerical work < A; < 1, and so q is positive and <  k' < 1. 

t The Theta-functions do not vanish when |5|<1 except at '  = 0, so this gives the only 
possible branch point. 

X This expansion was given by Weierstrass, Werke, ii. (1895), p. '276. 



21-9] 



THE THETA FUNCTIONS 



487 



21 'Q. The notations employed for the Theta-f unctions. 

The following scheme indicates the principal systems of notation which have been 
employed by various writers ; the symbols in any one column all denote the same 
function. 



 i(tj) 


9,(nz) 


9s (nz) 


S (nz) 


Jacobi 


 l( ) 


 2(2) 


h  ) 


Si z) 


Tannery and Molk 


(9i (0)2) 


62 (os) 


03  (CZ) 


6 m) 


Briot and Bouquet 


e, z) 


02   ) 


e iz) 


 0(2) 


Weierstrass, Halphen, Hancock 


6 z) 


6i z) 


Osiz) 


BoM 


Jordan, Harkness and Morley 



The notation employed by Hermite, H. J. S. Smith and some other mathematicians is 
expressed by the equation 



v=0, 1) 



6 , , x)= 2 ( - T" q  '+i )- e'""  " >  '  ; (/x = 0, 1 
tt = -   

with this notation the results of § 21*11 example 3 take the very concise form 
e , ,(a;+a) =  -y 6 , , (.r),  , , (.   + ar) = (-)"?-   e" ' *"* '  6 , , (.r). 

Cayley employs Jacobi's earlier notation (§ 21 •62). The advantage of the Weierstrassian 
notation is that unity (instead of tt) is the real period of 63  z) and  0 (2). 

Jordan's notation exhibits the analogy between the Theta-functions and the three 
Sigraa-functions defined in § 20'421. The reader will easily obtain relations, similar 
to that of § 21 •43, connecting 6  ( ) with 0-,.  iwxz) when r= 1, 2, 3. 

REFERENCES. 
L. EuLER, Opera Omnia, (1), xx. (Leipzig, 1912). 
C G. J. Jacobi, Fundamenta Nova* (Konigsberg, 1829) ; Ges. Math. Werke, i. 

pp. 497-538. 
C. Hermite, Oeuvres Mathematiqiies. (Paris, 1905-1917.) 
F. Klein, Vorlesungen iiher die Theorie der elliptischen Modulfunktionen (Ausgear- 

beitet und vervoUstandigt von R. Fricke). (Leipzig, 1890.) 
H. Weber, EUiptische Funktionen und algebraische Zahlen. (Brunswick, 1891.) 
J. Tannery et J. Molk, Fonctions Elliptiques. (Paris, 1893-1902.) 

Miscellaneous Examples. 

1. Obtain the addition-formulae 

9, y+z)9, y-z)9, =hHy)9o  z)-9.  2/)h  z) = S, y)9  z)-9  y)9,  z\ 

 2 (y + 2)   2  y -  ) V = 9  iy) s,' (z) - s,-  (y) V (2) = S-2' (.y)  4  (2) -  3  (y)  1' (2), 

 3 (y +  )  3 (y - 2) V =  4  (y )  3  (2) - 5 2 (y) V (,)   532 (3/) 5 2 (,) \   2 (3,)  2 (,)  

 4 (y + -')  4 iy - z) 9C  =  3  (i/)  3   z) - 5/ (y )  2   z) = V [y) 9i' (z) - 9,  (y) 5i2 (3). 

(Jacobi.) 
* Reprinted in his Ges. Math. Werke, i. (1881), pp. 49-239. 



488 



THE TRANSCENDENTAL FUNCTIONS 



[chap. XXI 



2. Obtain the addition-formulae 

•94(y+2)54Cy- )V=V(y) 2M2)+V(y)V(2)=V(3/)V(2)+V(y) 3'( ). 

and, by increasing  / by half periods, obtain the corresponding formulae for 

5r (i/ + z) Sr Ly - --)  2  and Br y- -z)Br 2f-z) V. 
where r=l, 2, 3. (Jacobi.) 

3. Obtain the formulae 

5l 0/ ± --)  2 (y +  ) h -94 = -9, Q/) \$2 (i/)  3 (2)  4 (2) ±  3 (y)  4 C )  1 C )  2  z), 
Sl(]/±z)Ss(l/ + z)  2 4 = • ! Cy)  3 (y)  2 (2)  4 C ) ±  2 i )  4 (y)  1 (z)  3 i l 
h l/±z) -34 (y + ') 5.2-93 = 5, Cy)  4 (y)  2 (2)  3 C ) ±  o (y)  3 (y) Si  z) Si (z), 

So i/±z)SsQ/ + z) 3. S  = So (y) S3 (v/) S., (2)  3 (z) + S   y) S   1/) S  (z) 3  (z), 
S, y±z) Si i/ + z) S.,Si=S, (j/) Si (y)  2  z) Si  z) + Si (j/)  3 (y)  1 (2)  3 (2), 
S3 i/±z)S, y + z)SsSi = S, y)Si 9/)S3 z)Siiz) + S,iy)S. i/)S, z)S2 z). 

(Jacobi.) 

4. Obtain the duplication-fomiulae 

5,  2y) S,Si  = 5,2 (y) V (3/) - 3i2 (3,) S 2 (\ y)  

53 (2y) 53 V =  3  (y) 54  0/) - - i' (y) -92'  (. ), 
Si  2y) Si' = S3* (1/) - S,* (y ) = Si* y)- S,* (y). 



Obtain the dupHcation-formula 

S, (2y) 52 3 4 = 2 1 (y) -9, (y)  3 (y)  4 (y). 

Obtain dupUcation-fomuilae from the results indicated in example 2. 

Shew that, with the notation of § 21 '22, 

[l]-[2] = [4]'-[3]', [l]-[3] = [l]'-[3]', [l]-[4] = [2]'-[3]', 
[2]-[3] = [l]'-[4]', [2]-[4] = [2]-[4]', [3]-[4] = [2]'-[l]'. 

Shew that 

2 [11 22] = [11 22]' + [2211]' -[4433]' + [3344]', 

2[1133] = [1133]' + [3311]'-[4422]' + [2244]', 
2[1144] = [1144]'+[4411]'-[3322]' + [2233]', 
2 [2233] = [2233]' + [3322]' - [441 1]' + [1 1 44]', 
2[2244] = [2244]' + [4422]'-[3311]' + [1133]', 
2 [3344] = [3344]' + [4433]' - [221 1 ]' + [1 1 22]'. 

Obtain the formulae 

2n- fa =-2qi U  (1 -?2")2 (1 -y2n-l)-2| . 



(Jacobi.) 
(Jacobi.) 



(Jacobi.) 



k k' 



i = 2fyi n  (l+?2H)2(l-f/ -i)- 



10. Deduce the following i-esults from example 9 : 
n (1 - j2n- 1)6 = 2 i > •'/(•->, 



n  l-f' f =27v-'q- M'K\ 



n (l+?2 -i) =22 (M')-i, 
1=1 

n (i+92 )  =lq -kk'- , 



n (i-j )  

n=l 



= An- q- k k'-'K\ 



n (1+9" 

n = l 



Iq- k k'-K 



(Jacobi.) 



THE THETA FUNCTIONS 489 

11. By considering I .* /* e "" dz taken along the contour formed by the parallelogram 

J °i (2) 

whose corners are —  tt, Itt,  Tr + Trr, —  tt + ttt, shew that 
and deduce that, when | I z) \ < I ttt), 

12. Obtain the following expansions : 

S3' z)\  * ( - )'   sin 2 3 

each expansion being valid in the strip of the s-plane in which the series involved is 

absolutely convergent. 

(Jacobi.) 

13. Shew that, if \ I y)\ < l (ttt-) and 1 / (2) | < 7 (ttt), then 

  ± = cot y + cot 2 + 4 i 2 q'"'  sin  2my + 2nz). 
 1 (y)  1 (2) m=l  =i 

(Math. Trip. 1908.) 

14. Shew that, if | /(s) j < \ I (ttt), then 

TT  ,(S)"2' 

where a  = 2 2 j('  +  )(2' +' + ). 

(Math. Trip. 1903.) 

[Obtain a reduction formula for    by considering    Si(z) -U-'' 'dz taken round the 
contour of example 11.] 

15. Shew that 

cos 22 + g*" J 



 --rT=i o+ 2 a cos2?i2, 



 1(2) L  =il-2j2ncos: 

is a doublv-periodic function of z with no singularities, and deduce that it is zero. 

Prove similarly that 

 2( )\  . A% g-" sin 22 

.92 (z)  !=i 1 + 2? " cos 22 + 2 " 

 3 (2) " ~  lil + 2g''' -icos22 + j4 -2' 

 4'.(g) . I g2n-l sii, 2g 

  4(2)  =i l-2g ''-icos22 + \$ "-2' 
16. Obtain the values of k, k', K, K' correct to six places of decimals when q= . 
[/ .•= 0-895769, F = 0-444518, 
 =2-262700, Zr' = 1-658414.] 



490 THE TRANSCENDENTAL FUNCTIONS [CHAP. XXI 

17. Shew that, if w+x + i/ + 2=0, then, with the notation of § 21-22, 

[3] + [l] = [2] + [4], 
[1234] + [3412] + [2143] + [4321] =0. 

18. Shew that 

 4(y)  4(2) hilZ + z) ' ' i(y)Si  )h(2/ +  y 

19. By putting x=y = z, w=3a; in Jacobi's fundamental formulae, obtain the following 

results : 

Si3 ( )  j (3 .)  .  3 ( .)   (ar) =  3 (2.1,.) 5  , 

 33  x) S3 (ar) - 3i  (x) \$i (ar) = B   2x) 5-2, 
So3 (.f) 3. (Sx) + 3 3 ( ) 5  (3.1;) =  33 (2 )  3 . 

20. Deduce from example 19 that 

 Si  (x) Si  3x) Si  + Si   x) Si  Sx) Si     +  Ss  (x) S3  3x) S  - V ( )  4 (3 )  2    

=  52=* ('*-•)  2 (3.f) S3- +  4  (x) Si  3x) .932   • 
(Trinity, 1882.) 


