\chapter{Asymptotic Expansions and Summable Series}

\Section{Simple example of an asymptotic expansion.}

Consider the function $f(x) = \int_{x}^{\infty} t^{-1} e^{x-t}
\dmeasure t$, where $x$ is real and positive, and the path of
integration is the real axis.

By repeated integrations by parts, we obtain
$$
f(x) = \frac{1}{x} - \frac{1}{x^{2}} + \frac{2!}{x^{3}} - \cdots +
\frac{ (-)^{n-1} (n-1)!}{x^{n}} + (-)^{n} n! \int_{x}^{\infty}
\frac{e^{x-t} \dmeasure t}{t^{n+1}}.
$$

In connexion with the function $f(x)$, we therefore consider the
expression
$$
u_{n-1} = \frac{ (-)^{n-1} (n-1)!}{x^{n}},
$$
and we shall write
$$
\sum_{m=0}^{n} u_{m} = \frac{1}{x} - \frac{1}{x^{2}} +
\frac{2!}{x^{3}} - \cdots + \frac{ (-)^{n} n!}{x^{n+1}} = S_{n}(x).
$$
Then we have $\absval{u_{m}/u_{m-1}} = mx^{-1} \rightarrow \infty$.
\emph{The series $\sum u_{m}$ is therefore divergent for all values of
$x$}. In spite of this, however, the series can be used for the
calculation of $f(x)$; this can be seen in the following way.

Take any fixed value for the number $n$, and calculate the value of
$S_{n}$. We have
$$
f(x) - S_{n}(x) = (-)^{n+1} (n+1)! \int_{x}^{\infty} \frac{e^{x-t}
\dmeasure t}{t^{n+2}},
$$
and therefore, since $e^{x-t} \leq 1$,
$$
\absval{ f(x) - S_{n}(x) } = (n+1)! \int_{x}^{\infty} \frac{e^{x-t}
\dmeasure t}{t^{n+2}} < (n+1)! \int_{x}^{\infty} \frac{\dmeasure
t}{t^{n+2}} = \frac{n!}{x^{n+1}}.
$$
For values of $x$ which are sufficiently large, the right-hand member
of this equation is very small. Thus, if we take $x \geq 2n$, we have
$$
\absval{ f(x) - S_{n}(x)} < \frac{1}{2^{n+1} n^{2}},
$$
which for large values of $n$ is very small. It follows therefore that
\emph{the value of the function $f(x)$ can he calculated with great
accuracy for large values of $x$, by taking the sum of a suitable
number of terms of the series $\sum u_{m}$}.

Taking even fairly small values of $x$ and $n$
$$
S_{5}(10) = 0.09152, \quad 0 < f(10) - S_{5}(10) < 0.00012.
$$
%
% 151
%

The series is on this account said to be an asymptotic expansion of
the function $f(x)$. The precise definition of an asymptotic expansion
will now be given.

\Section{Definition of an asymptotic expansion.} A divergent series
$$
A_{0} + \frac{A_{1}}{z} + \frac{a_{2}}{z^{2}} + \cdots +
\frac{A_{n}}{z^{n}} + \cdots,
$$
in which the sum of the first $(n + 1)$ terms is $S_{n}$, is said to
be an \emph{asymptotic expansion} of a function $f(z)$ for a given
range of values of $\arg$, if the expression $R_{n}(z) = z^{n}[f(z) -
S_{n}(z)]$ satisfies the condition
$$
\lim_{ \absval{z} \rightarrow \infty } R_{n}(z) = 0 \quad (\textrm{$n$
fixed}),
$$
even though
$$
\lim_{n \rightarrow \infty} \absval{R_{n}(z)} = \infty \quad
(\textrm{$z$ fixed}).
$$
When this is the case, we can make
$$
\absval{ z^{n} [f(z) - S_{n}(z)] } < \eps,
$$
where $\eps$ is arbitrarily small, by taking $\absval{z}$ sufficiently
large.

We denote the fact that the series is the asymptotic expansion of
$f(z)$ by writing
$$
f(z) \sim \sum_{n=0}^{\infty} A_{n} z^{-n}.
$$

The definition which has just been given is due to
\Poincare\footnote{TODO}. Special asymptotic expansions had, however,
been discovered and used in the eighteenth century by Stirling,
Maclaurin and Euler. Asymptotic expansions are of great importance in
the theory of Linear Differential Equations, and in Dynamical
Astronomy; some applications will be given in subsequent chapters of
the present work.

The example discussed in \hardsectionref{8}{1} clearly satisfies the
definition just given: for, when $x$ is positive, $\absval{x^{n} [f(x)
- S_{n}(x)]} < n! x^{-1} \rightarrow 0$ as $x \rightarrow \infty$.

%\begin{Remark} For the sake of simplicity, in this chapter we shall
for the most part consider asymptotic expansions only in connexion
with real positive values of the argument. The theory for complex
values of the argument may be discussed by an extension of the
analysis.

\Subsection{Another example of an asymptotic expansion.} As a second
example, consider the function $f(x)$, represented by the series
$$
f(x) = \sum_{k=1}^{\infty} \frac{c^{k}}{x+k},
$$
where $x > 0$ and $0 < c < 1$.

% 
% 152
%

The ratio of the $k$th term of this series to the $(k- l)$th is less
than $c$, and consequently the series converges for all positive
values of $x$. We shall confine our attention to positive values of
$x$. We have, when $x > k$,
$$
\frac{1}{x+k} = \frac{1}{x} - \frac{k}{x^{2}} + \frac{k^{2}}{x^{3}} -
\frac{k^{3}}{x^{4}} + \frac{k^{4}}{x^{5}} - \cdots.
$$

If, therefore, it were allowable\footnote{It is not allowable, since
$k>x$ for all terms of the series after some definite term.} to expand
each fraction $\frac{1}{x+k}$ in this way, and to rearrange the series
for $f(x)$ in descending powers of $x$, we should obtain the formal
series
$$
\frac{A_{1}}{x} + \frac{A_{2}}{x^{2}} + \cdots + \frac{A_{n}}{x^{n}} +
\cdots,
$$
where
$$
A_{n} = (-)^{n-1} \sum_{k=1}^{\infty} k^{n-1} c^{k}.
$$
But this procedure is not legitimate, and in fact $\sum_{n=1}^{\infty}
A_{n} x^{-n}$ diverges. We can, however, shew that it is an asymptotic
expansion of $f(x)$.

For let
$$
S_{n}(x) = \frac{A_{1}}{x} + \frac{A_{2}}{x^{2}} + \cdots +
\frac{A_{n}}{x^{n}} %TODO: verify correct subscript; book is
inconsistent
$$

Then S\^\{x)= i ('- -\^4 + \^lf + ... + izyi\^)

k = l\ \ xj J x + k'

so that TODO

Now TODO converges for any given value of n and is equal to C„, say;
and hence fc=i

TODO

Consequently TODO

Example. If TODO, where x is positive and the path of integration is
the

J X

real axis, prove that

TODO

[In fact, it was shewn by Stokes in 1857 that

TODO

the upper or lower sign is to be taken according as TODO. ]
%\end{Remark}
\Section{Multiplication of asymptotic expansions.}

We shall now shew that two asymptotic expansions, valid for a common
range of values of arg\^', can be multiplied together in the same way
as ordinary series, the result being a new asymptotic expansion.

TODO

For let TODO

% 
% 153
% 
and let Sn\{z) and Tn\{z) be the sums of their first (n + 1) terms; so
that, n being fixed,

f(z) - Sn (Z) = (Z-X <t> (Z) - Tn \{z) = (z\^).

Then, if C\^ = \^o\^m + \^iB,n-i + . . . + \^m\^o. it is obvious that*

SJz)Tn\{z)= i C,nZ--\^+0(z-).

But f\{z) Cf> (Z) = \{Sn (Z) + (Z-)] [Tn \{z) + (\^-»)\}

= Sn \{Z) Tn (Z) + (Z-\^)

TO =

This result being true for any fixed value of n, we see that

TODO

\Subsection{Integration of asymptotic expansions.}

We shall now shew that it is permissible to integrate an asymptotic
expansion term by term, the resulting series being the asymptotic
expansion of the integral of the function represented by the original
series.

For let TODO and let TODO.

TODO

Then, given any positive number e, we can find Xq such that TODO when
x>Xo,

and therefore

TODO

But TODO and therefore TODO.

On the other hand, it is not in general permissible t to
diflFerentiate an asymptotic expansion; this may be seen by
considering TODO.

\Subsection{Uniqueness of an asymptotic expansion.}

A question naturally suggests itself, as to whether a given series can
be

* See\hardsubsectionref{2}{1}{1}; we use o (2~") to denote any function i/- (z) such that
3" i/- (2) -»- as | 2, -*- x . t For a theorem concerning
differentiation of asymptotic expansions representing analytic
functions, see Kitt, Bull. American Math. Soc. xxiv. (1918), pp.
225-227.

%
% 154
%

the asymptotic expansion of several distinct functions. The answer to
this is in the affirmative. To shew this, we first observe that there
are functions L \{x) which are represented asymptotically by a series
all of whose terms are zero, i.e. functions such that lim x\^L \{x) =
for every fixed value of n. The

a; -*• 00

function e~\^ is such a function when x is positive. The asymptotic
expansion* of a function J\{x) is therefore also the asymptotic
expansion of

J\{x) + L(x).

On the other hand, a function cannot be represented by more than one
distinct asymptotic expansion over the whole of a given range of
values of z; for, if

TODO

then TODO

which can only be if TODO,

Important examples of asymptotic expansions will be discussed later,
in connexion with the Gamma-function (Chapter xii) and Bessel
functions (Chapter xvii).

\Section{Methods of 'summing' series.}

We have seen that it is possible to obtain a development of the form

f(x)= i A,,x~'" + R,,\{x),

m =

00

where Rn(x)-\^ co as ?i- > oo, and the series S) A.,nX~\^ does not
converge.

m =

We now consider what meaning, if any, can be attached to the ' sum '
of a non- convergent series. That is to say, given the numbers ao,
a\^, a\^, ..., we wish to formulate definite rules by which we can
obtain from them a

00 00

number 8 such that S = 'S, an if 2 a„ converges, and such that aS'
exists

TODO

when this series does not converge.

\Subsection{Borel's method of summation.} We have seen \hardsubsectionref{7}{8}{1}) that

00 /"»

2 anz"" = e-\^(f) (tz) dt,

n=0 J

where (tz) = X ", the equation certainly being true inside; the
circle

00

of convergence of S a\^z'\^. If the integral exists at points z
outside this circle, we define the ' Borel sum ' of S a,i2" to mean
the integral.

* It has been shewn that when the coefficients in the expansion
satisfy certain inequaUties, there is only one analytic function with
that asymptotic expansion. See PJiil. Trans. 213, a, (1911), pp.
279-313.

t Borel, Le\^o)is stir les Series Diveryentes (1901), pp. 97-115.

% 
% 155
% 

Thus, whenever R(z)< I, the ' Bore] sum ' of the series S 2''\^ is

»=o

1 e-*e''dt = \{l-z)-\

J

If the ' Borel sum ' exists we say that the series is ' summable (B).'

\Subsection{Elder's* method of summation.} A method, practically due
to Euler, is suggested by the theorem of\hardsectionref{3}{7}l;

00 00

the ' sum ' of S a,i may be defined as lim S ctn\^'\^ when this limit
exists.

n=0 2:\^-1-0 n=0

Thus the ' sum ' of the series 1 - 1 + 1 - 1 + ... would be lim (1 - X
+ cc- - ...) = lim (1 + x)~\^ = i.

\Subsection{Cesdro's-f method of summation.}

Let Sn = «! + ao + . . . + «n; then if S = lim - (s, + Sj + • • • +
\^n) exists, we say that S Un is 'summable (C'l),' and that its sum
(CI) is S. It is necessary to establish the 'condition of
consistency;!:,' namely that S= 2 a„ when this series is convergent.

00 n

To obtain the required result, let TODO, then we have

m = l m = l

to prove that TODO.

Given e, we can choose n such that

so TODO

Then, if i\^ > n, we have

n+p

% a,

m=n-rl

< e for all values of p, and

TODO

Since TODO is a positive decreasing sequence, it follows from Abel's
inequality \hardsubsubsectionref{2}{3}{0}{1}) that

TODO

Therefore

TODO

* Instit. Cale. Diff. (1755). See Borel, loc. cit. Introduction, t
Bulletin des Sciences Math. (2), xiv. (1890), p. 114. + See the end of\hardsectionref{8}{4}.

%
% 156
%
Making v-\^x, we see that, if S be any one of the limit points (§
2'21) of S\^, then

n I

S - % a„, \^ e. Therefore, since | s - 5,1 ! < e, we have

This inequality being true for every positive value of e we infer, as
in\hardsubsectionref{2}{2}{1}, that S =\^s; that is to say S\^ has the unique limit s;
this is the theorem which had to be proved.

Example 1. Frame a definition of 'uuiforui .•summability (Cl) of a
series of variable terms.'

Example 2. TODO

\Subsubsection{TODO:Cesdrds general method of summation.}

A series TODO is said to be 'summable (Cr)' if TODO exists, where

It follows from\hardsubsectionref{8}{4}{3} example 2 that the 'condition of consistency'
is satisfied; in fact it can be proved* that if a series is summable
(C/) it is also summable \{Cr) when r>r'; the condition of consistency
is the particular case of this result when r = 0.

\Subsection{The method of summation of Rieszi.}

A more extended method of ' summing ' a series than the preceding is
by means of

hm 2 ( 1 - \^- cin,

in which X,i is any real function of n which tends to infinity with n.
A series for which this limit exists is said to be 'summable \{Rr)
with sum-function X„.'

\Section{Hardy's convergence theorem.}

s summable ( a,i=0(l/?i),

Let S an he a series tuhick is summable \{G 1). Then if

w=l

the series TODO converges.

n=\

* Bromwich, Infinite Series, § 122. t Comptes Rendus, cxlix. (1910),
pp. 18-21.

X Proc. London Math. Sac. (2), viii. (1910), pp. 302-304. For the
proof here given, we are indebted to Mr Littlewood.

Let Sn = a I + a.\^ +. + cin; then since S Un is summable \{G 1), we
have

% 
% 157
% 

M = l

Si + So+ ... + Sn = n[s + (l)j,

where s is the sum ((71) of X «».

Let TODO and let

Sm-s = t,a, \{m = l, 2, ... n),

ti + to+ ... +tn = (Tn •

With this notation, it is sufficient to shew that, if j a„ | < Kn~\^,
where K is independent of n, and if On = n.o (1), then tn -> sis n ->
cc .

Suppose first that a\^, a\^, ... are real. Then, if tn does not tend
to zero, there is some positive number h such that there are an
unlimited number of the numbers t,i which satisfy either\^ (i) t\^ > h
or (ii) tn < -h. We shall shew that either of these hypotheses implies
a contradiction. Take the former*, and choose n so that tn > h.

Then, when r = 0, 1,2,

< K/n.

Now plot the points P,. whose coordinates are (r, tn+r) in a Cartesian
diagram. Since tn\^r+i-tn+r = an+r+i, the slope of the line PrPr+i is
less than = arc tan (K/n).

Therefore the points Pq, Pj, P.,, ... lie above the line y = h - xtan
6. Let Pk be the last uf the points P„, Pj, ... which lie on the left
of a;'= hcot 6, so that TODO.

Draw rectangles as shewn in the figure. The area of these rectangles
exceeds the area of the triangle bounded by y = h - x tan 6 and the
axes; that is to say

TODO

* The reader will see that the latter hypothesis involves a
contradiction by using arguments of a precisely similar character to
those which will be employed in dealing with the former hypothesis.

%
% 158
%

But I (Tn+k - 0"7i-i ! < I 0'7i+A; ! + I \^n-i \

= (n + k).o\{l) + \{n-l).o(l) = n.o\{l), since k \^ hnK~\^, and h, K
are independent of n.

Therefore, for a set of values of n tending to infinity,

\^h-K~\^n <n.o\{l), which is impossible since \^h'-K~\^ is )iot o (1)
as /i-> x .

This is the contradiction obtained on the hypothesis that lim tn> h >
0; therefore Urn tn \^ 0. Similarly, by taking the corresponding case
in which tn\^ - h, we arrive at the result lim tn \^ 0. Therefore
since lim tn > lim tn,

we have lim t\^ = lim tn = 0,

and so tn - > 0.

That is to say Sn -> s, and so 2 a,i is convergent and its sum is s.

If an be complex, we consider R (a„) and / (a\^) separately, and find

TODO

that S R\{an) and S /(c/„) converge by the theorem just proved, and so
TODO

The reader will see in Chapter ix that this result is of great
importance in the modern theory of Fourier series.

Corollary. If TODO be a function of TODO such that TODO

throughout a domain of values of TODO, and if TODO, where K is
independent of |,

2 a„ (\^) converges uniformly throughoxit the domain. n=i

For, retaining the notation of the preceding section, if \^n(\^) does
not tend to zero uniformly, we can find a positive number h
independent of n and | such that an infinite sequence of values of n
can be found for which t\^ (\^») > A or i„ (\^„) <-h for some point
TODO of the domain*; the value of \^„ depends on the value of n under
consideration.

We then find, as in the original theorem,

TODO

for a set of values of n tending to infinity. The contradiction
implied in the inequality shews that h does not exist, and so
\^rt(|)-9-0 uniformly.

* It is assumed that «„ (\^) is real; the extension to complex
variables cau be made as in the former theorem. If no such number h
existed, <„ (|) would tend to zero uniformly.

t It is essential to observe that the constants involved iu the
inequality do not depend on |,j. For if, say, K depended on \^„, K~\^
would really be a function of n and might be o (1) qua function of n,
and the inequality would not imply a contradiction.

%
% 159
%

REFERENCES.

H. PoiNCAR\^, Acta Mathematica, viii. (1886), pp. 295-344.

E. BoREL, Lecons sur les Series Divergentes (Paris, 1901).

T. J. Pa. Bromwich, Theory of Infinite Series (1908), Ch. xi.

E. W. Barnes, Phil. Trans, of the Royal Society, 206, a (1906), pp.
249-297.

G. H. Hardy and J. E. Littlewood, Proc. London Math. Soc. (2), xi.
(1913), pp. 1-16*.

G. N. Watson, Phil. Trans, of the Royal Society, 213, a (1911), pp.
279-313.

S. Chapman +, Proc. London Math. Soc. (2), ix. (1911), pp. 369-409.

Miscellaneous Examples.

/" e~\^ 1 2 ! 4 !

:; :idt~ - ~ + -\^ - ...

when X is real and positive.

2. Discuss the representation of the function

f(x)=\{'' \^<\^\{t)e\^-dt

(where x is supposed real and positive, and is a function subject to
certain general con- ditions) by means of the series

TODO

Shew that in certain cases (e.g. (\^(;) = e'") the series is
absolutely convergent, and represents TODO for large positive values
of $x$ but that in certain other cases the series is the asymptotic
expansion of $f(x)$.

3. Shew that

for large positive values of z.

TODO

(Legendre, Exerdces de Calc. Int. (1811), p. 340.) 4. Shew that if,
when x>0,

TODO

Shew also that/(.r) can be expanded into an absolutely convergent
series of the form

TODO \addexamplecitation{\Schlomilch.}

•'\^ ' A.=i(\^+l)(.r-i-2)...(\^ + \^-\}

5. Shew that if the series 1+0 + 0-1+04-1 + + 0-1 + ..., in which two
zeros precede each -1 and one zero precedes each +1, be 'summed' by
Ceskro's method, its sum is f. (Euler, Borel.)

6. Shew that the series 1-21 + 4! - ... cannot be summed by Borel's
method, but the series l+0-2! + + 4! + ... can be so summed.

* This paper contains many references to recent developments of the
subject. t A bibliography of the literature of summable series will be
found on p. 372 of this memoir.